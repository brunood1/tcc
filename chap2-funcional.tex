\chapter{Introdução à análise funcional}

\textcolor{red}{(introdução)}

\section{Espaços de Banach}

\textcolor{red}{(introdução)}

\begin{dbox}
    Seja $X$ um espaço vetorial sobre um corpo $\bK$. Uma função $\Vert \cdot \Vert : X \to \bR$ é dita ser uma norma se satisfaz
    \begin{itemize}[leftmargin=*]
        \item $\Vert x \Vert \geqslant 0$ para todo $x \in X$
        \item $\Vert x \Vert = 0 \iff x = 0$
        \item $\Vert \lambda x \Vert = |\lambda| \Vert x \Vert$ para todo $x \in X$ e $\lambda \in \bK$
        \item $\Vert x + y \Vert \leqslant \Vert x \Vert + \Vert y \Vert$ para todo $x,y \in X$
    \end{itemize}
\end{dbox}

\textcolor{red}{(definições iniciais)}

\begin{ex}
    O espaço euclidiano $\bR^n$ munido da norma
    \[
        \Vert x \Vert = \sqrt{x_1^2 + \cdots + x_n^2}
    \]
    onde $x = (x_1,\dots,x_n)$ é um espaço de Banach
\end{ex}

\begin{ex}
    O espaço
    \[
        \ell^p \equiv \ell^p(\bR) = \left\{ x = (x_n) \,; \sum_{i=1}^{\infty} |x_i|^p < \infty \right\}
    \]
    com $1 \leqslant p < \infty$ munido da norma
    \[
        \Vert x \Vert_p = \left( \sum_{i=1}^{\infty} |x_i|^p \right)^{\frac{1}{p}}
    \]
    é um espaço de Banach.
\end{ex}

\begin{ex}
    O espaço
    \[
        \ell^\infty \equiv \ell^\infty(\bR) = \left\{ x = (x_n) \,; \sup_{n \in \bN} |x_n| < \infty \right\}
    \]
    munido da norma
    \[
        \Vert x \Vert_\infty = \sup_{n \in \bN} |x_n|
    \]
    é um espaço de Banach.
\end{ex}

\begin{ex}
    O espaço
    \[
        \cC([a,b], \bK) = \{f : [a,b] \to \bK \,; f \text{ é contínua}\}
    \]
    munido da norma
    \[
        \Vert f \Vert_{\max} = \max_{t \in [a,b]} \{|f(t)|\}
    \]
    é um espaço de Banach
\end{ex}

\begin{ex}
    O espaço $\cC([0,1], \bR)$ munido da métrica
    \[
        d(f,g) = \int_0^1 |f(t) - g(t)| \, dt
    \]
    não é um espaço completo
\end{ex}