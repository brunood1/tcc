\documentclass[a4paper, 11pt]{article}
\usepackage[utf8]{inputenc}
\usepackage[T1]{fontenc}

\usepackage[margin=1in]{geometry}
\usepackage{amsmath, amsfonts, amsthm, amssymb, amsxtra}

\usepackage{graphicx}
\usepackage{float}

\usepackage[portuguese]{babel}

\usepackage[dvipsnames]{xcolor}

% \usepackage{cmbright}

\usepackage{tabularray}
\usepackage{enumitem}
\usepackage{multicol}

\usepackage{caption} 
\captionsetup{justification=centering} 

\usepackage{hyperref}
\hypersetup{hidelinks}

\usepackage{tikz}
\usetikzlibrary{intersections, angles, calc, positioning}
\usetikzlibrary{shapes.geometric, arrows.meta}
\usetikzlibrary{decorations.pathmorphing, decorations.pathreplacing}

\usepackage[explicit]{titlesec}
\usepackage[letterspace=100]{microtype}
\usepackage{fmtcount}

\titleformat{\chapter}[display]
{\bfseries\large\sf\lsstyle}
{\huge\filleft\mbox{\MakeUppercase{\chaptertitlename} \NUMBERstring{chapter}}}
{1.5ex}
{\titlerule
\vspace*{1.1ex}%
\MakeUppercase{#1}}
[\vspace*{1.5ex}%
\titlerule]
\titlespacing*{\chapter}{0pt}{-10pt}{25pt}

\titleformat{\section}{\Large\bfseries\sffamily\centering}{\thesection}{0.5em}{#1}

\usepackage{thmtools}
\usepackage{thm-restate}
\usepackage[framemethod=TikZ]{mdframed}
\mdfsetup{skipabove=1em,skipbelow=0em, 
innertopmargin=10pt, innerbottommargin=8pt,
innerleftmargin=8pt, innerrightmargin=8pt}

\renewcommand{\qedsymbol}{$\Box$}

\theoremstyle{definition}

\declaretheoremstyle[headfont=\bfseries\sffamily, bodyfont=\normalfont, mdframed={nobreak, backgroundcolor=orange!10, linecolor=orange!40!black}]{thmbox}
\declaretheorem[style=thmbox, name=Definição]{dbox}
\declaretheorem[style=thmbox, name=Teorema]{tbox}
\declaretheorem[style=thmbox, name=Proposição]{pbox}
\declaretheorem[style=thmbox, name=Corolário]{cbox}
\declaretheorem[style=thmbox, name=Lema]{lbox}

\declaretheoremstyle[headfont=\bfseries\sffamily, bodyfont=\normalfont]{exp}
\declaretheorem[style=exp, name=Exemplo]{ex}

\declaretheoremstyle[headfont=\sffamily\itshape, bodyfont=\normalfont]{pr}
\declaretheorem[numbered=no, style=pr, name=Demonstração, qed=\qedsymbol]{prf}

\setlength{\parindent}{0pt}
\setlength{\parskip}{5pt}

\newcommand{\medcup}{\mathsf{U}}

\newcommand{\bN}{\mathbb{N}}
\newcommand{\bZ}{\mathbb{Z}}
\newcommand{\bQ}{\mathbb{Q}}
\newcommand{\bR}{\mathbb{R}}
\newcommand{\bC}{\mathbb{C}}
\newcommand{\bK}{\mathbb{K}}

\newcommand{\bz}{\mathbf{0}}

\newcommand{\cC}{\mathcal{C}}
\newcommand{\cA}{\mathcal{A}}
\newcommand{\cL}{\mathcal{L}}
\newcommand{\cB}{\mathcal{B}}
\newcommand{\cM}{\mathcal{M}}

\begin{document}

\section*{Espaços $\cL^p$ (continuação)}

Agora, vmos provar que $(\cL^p, \Vert \cdot \Vert_p)$ é um espaço vetorial normado para $1 \leqslant p < \infty$.
\begin{pbox}
    A aplicação $\Vert \cdot \Vert_p : \cL^p \to \bR$ dada por
    \[
        \Vert f \Vert_p = \left( \int |f|^p \, d\mu \right)^{\frac{1}{p}}
    \]
    é uma norma
\end{pbox}
\begin{prf}
    Note que
    \begin{enumerate}
        \item $\displaystyle  \Vert f \Vert_p = \left(\int |f|^p \, d\mu \right)^{\frac{1}{p}} \geqslant 0$ pois $|f| \geqslant 0$.
        \item $\displaystyle \Vert \lambda f \Vert_p = \left( \int |\lambda f|^p \,d\mu \right)^{\frac{1}{p}} = \left( \int |\lambda|^p |f|^p \,d\mu \right)^{\frac{1}{p}} = \left(|\lambda|^p \int |f|^p \,d\mu \right)^{\frac{1}{p}} = |\lambda| \left( \int |f|^p \,d\mu \right)^{\frac{1}{p}} = |\lambda| \Vert f \Vert_p$
        \item $\Vert f + g \Vert_p \leqslant \Vert f \Vert_p + \Vert g \Vert_p$ pela Desigualdade de Minkowski.
    \end{enumerate}
    Portanto $\Vert \cdot \Vert_p$ é uma norma.
\end{prf}

Agora, nosso objetivo é mostrar que $\cL^p$ com $1 \leqslant p < \infty$ é um espaço de Banach, isto é, um espaço vetorial normado completo.
Para isso precisamos das seguintes definições

\begin{dbox}
    Seja $(f_n)$ uma sequência em $\cL^p$ com $1 \leqslant p < \infty$. Dizemos que $(f_n)$ é de Cauchy se dado $\varepsilon > 0$ existe $n_0 \in \bN$ tal que
    \[
        \Vert f_n - f_m \Vert < \varepsilon
    \]
    para todo $n,m \geqslant n_0$
\end{dbox}

\begin{dbox}
    Sejam $(f_n)$ uma sequência em $\cL^p$ e $f \in \cL^p$ com $1 \leqslant p < \infty$. Dizemos que $(f_n)$ é convergente e converge para $f$ se dado $\varepsilon > 0$, existe $n_0 \in \bN$ tal que
    \[
        \Vert f_n - f \Vert_p \leqslant \varepsilon
    \]
    para todo $n \geqslant n_0$. Equivalentemente
    \[
        \lim \Vert f_n - f \Vert_p = 0
    \]
\end{dbox}

\begin{dbox}
    Um espaço métrico $(X,d)$ é completo se toda sequência de Cauchy é convergente.
\end{dbox}

\begin{tbox}[Teorema de Riesz-Fischer]
    $\cL^p$ com $1 \leqslant p < \infty$ é um espaço de Banach.
\end{tbox}
\begin{prf}
    Seja $(f_n)$ uma sequência de Cauchy em $\cL^p$.
    Mostremos que $(f_n)$ é convergente. 
    Com efeito, sabemos que dado $\varepsilon > 0$, existe $n_0 \in \bN$ tal que
    \[
        \Vert f_n - f_m \Vert < \varepsilon
    \]
    para todo $n, m \geqslant n_0$.
    Esscolhendo $\varepsilon$ de forma adequada e passando a uma subsequência se necessário, temos que
    \begin{equation} \label{eq:1}
        \Vert f_{n+1} - f_n \Vert < 2^{=n}
    \end{equation}
    Defina $g : X \to \overline\bR$ por
    \[
        g(x) = |f_1(x)| + \sum_{n=1}^{\infty} |f_{n+1} (x) - f_n(x)|.
    \]
    Observe que $g \in \cM^+(X,\eth)$, pois $g \geqslant 0$ e
    \[
        g = |f_1| + \lim_{k\to\infty} \sum_{n=1}^{k} |f_{n+1} - f_n|
    \]
    isto é, $g$ é formado pela soma e pelo limite de funções mensuráveis ($f_n$ é integrável, em particular, mensurável).
    Queremos mostrar que $g \in \cL^p$. De fato
    \[
        \begin{aligned}
            \int |g|^p \,d\mu 
            &= \int \left( |f_1| + \lim_{k\to\infty} \sum_{n=1}^{k} |f_{n+1} - f_n| \right)^p \, d\mu\\
            &= \int \liminf_{k\to\infty} \left( |f_1| +  \sum_{n=1}^{k} |f_{n+1} - f_n| \right)^p \, d\mu\\
            &\leqslant \liminf_{k\to\infty} \textcolor{orange}{\int  \left( |f_1| +  \sum_{n=1}^{k} |f_{n+1} - f_n| \right)^p \, d\mu}\\
            &= \liminf_{k\to\infty} \textcolor{orange}{\left\Vert |f_1| + \sum_{n=1}^k |f_{n+1} - f_n| \right\Vert _p^p}\\
            &\leqslant \liminf_{k\to\infty} \left( \left\Vert f_1 \right\Vert + \sum_{n=1}^k\left\Vert  f_{n+1} - f_n \right\Vert _p \right)^p\\
            &\leqslant \left( \left\Vert f_1 \right\Vert _p + \sum_{n=1}^k\left\Vert  f_{n+1} - f_n \right\Vert _p \right)^p\\
            &\leqslant \left( \Vert f_1 \Vert_p + \sum_{n=1}^{\infty} 2^{-n}\right)\\
            &< \infty.
        \end{aligned}
    \]
    Portanto, $g \in \cL^p$.
    Agora seja, $E = \{x \in X \,; g(x) < \infty\} \in \eth$. Dito isso, $N = X \smallsetminus E = \{x \in X \,; g(x) = \infty\} \in \eth$. Mostremos que $N$ tem medida nula.
    Com efeito, suponha que $\mu(N) > 0$, dessa forma
    \[
        \int_X |g|^p \geqslant \int_{N} |g|^p = \infty \int d\mu = \infty \mu(N) = \infty,
    \]
    o que implicaria em
    \[
        \int |g|^p = \infty
    \]
    que é uma contradição pois $g \in \cL^p$.
    Dessa forma $\mu(N) = 0$, isto é, $g < \infty$ em quase toda parte em $X$.
    Sendo assim, defina $f : X \to \bR$ por
    \[
        f(x) = 
        \begin{cases}
            f_1(x) + \sum_{n=1}^{\infty} (f_{n+1}(x) - f_n(x)) &\text{se } x \in X\\
            0 &\text{se } x \not\in X.
        \end{cases}
    \]
    Mostremos que $f \in \cL^p$.
    Note que
    \[
        f(x) = \left( f_1(x) + \sum_{n=1}^{\infty} (f_{n+1}(x) - f_n(x)) \right) \chi_E.
    \]
    Daí
    \[
        |f| = \left| f_1 + \sum_{n=1}^{\infty} (f_{n+1} - f_n) \right| |\chi_E| \leqslant |f_1| + \sum_{n=1}^{\infty}|f_{n+1} - f_n| = g.
    \]
    Consequentemente, $|f|^p < g^p$.
    Logo
    \[
        \int |f|^p \,d\mu \leqslant \int g^p \,d\mu < \infty.
    \]
    Portanto, $f \in \cL^p$.
    Por outro lado, para todo $x \in E$
    \[
        \begin{aligned}
            f(x) &= f_1(x) + \sum_{n=1}^{\infty} (f_{n+1}(x) - f_n(x))\\
            &= f_1(x) + \lim_{k\to \infty}\sum_{n=1}^{k} (f_{n+1}(x) - f_n(x))\\
            &= \lim_{k\to \infty} \left( f_1(x) + f_2(x) - f_1(x) + f_3(x) - f_2(x) + \cdots + f_{k+1}(x) - f_k(x)\right)\\[5pt]
            &= \lim_{k\to \infty} f_{k+1}(x) = \lim _{k\to \infty} f_{k}(x).
        \end{aligned}
    \]
    Como $\mu(N) = 0$, então $\lim f_n = f$ em quase toda parte em $X$.
    É fácil ver que
    \begin{equation}
        |f_k| = \left| f_1 + \sum_{n=1}^{k-1} (f_{n+1} - f_n) \right| \leqslant |f_1| + \sum_{n=1}^{k-1} |f_{n+1} - f_n| \leqslant \sum_{n=1}^{\infty} |f_{n+1} - f_n| = g.
    \end{equation}
    Por isso
    \[
        |f_n - f|^p \leqslant (|f_n| + |f|)^p \leqslant (2g)^p = 2^p g^p
    \]
    para todo $n \in \bN$.
    Como $g \in \cL^p$, então $2^p g^p \in \cL^1$.
    Dessa forma, pelo Teorema da Convergência Dominada, chegamos a
    \[
        \lim \Vert f_n - f \Vert_p = \lim \left( \int |f_n - f|^p \, d\mu \right)^\frac{1}{p} = \int \lim |f_n - f|^p \, d\mu = 0
    \]
    Isto prova que $\cL^p$ é completo.
\end{prf}

\begin{dbox}
    Seja $(X,\eth,\mu)$ um espaço de medida. O espaço
    \[
        \cL^\infty = \cL^\infty(X,\eth,\mu) = \{f : X\to \bR \,; f \text{ é mensurável e limitada qtp em } X\}
    \]
    é chamado Espaço de Lebesgue $\cL^\infty$.
    Para cada $f \in \cL^\infty$, definimos
    \[
        \Vert f \Vert_\infty = \inf \{M \geqslant 0 \,; |f(x)| \leqslant M \text{ qtp em } X\}
    \]
    Por fim, dizemos que $f$ é uma função essencialmente limitada.
\end{dbox}

\textbf{\sffamily Observação:} Note que
\[
    \Vert f \Vert_\infty = \inf \{M \geqslant 0 \,; \mu(\{x \in X \,; |f(x)| > M\} = 0)\}.
\]
Isto segue da seguinte equivalência

\[
    |f(x)| \leqslant M \text{ qtp em } X \iff \mu(\{x \in X \,; |f(x)| > M\}) = 0.
\]
De fato, $|f(x)| \leqslant M$ em quase toda parte em $X$ se, e somente se, existe $N \in \eth$ tal que $\mu(N) = 0$ e $|f(x)| \leqslant M$ para todo $x \in N^\cC$.
Note que $\{ x \in X \,; |f(x)| > M\} \subseteq N$, dessa forma
\[
    \mu(\{x \in X\,; |f(x)| > M\}) \leqslant \mu(N) = 0
\]
Portanto, $\mu(\{x \in X\,; |f(x)| > M\}) = 0$.

Reciprocamente, se $\mu(\{x \in X\,; |f(x)| > M\}) = 0$, então $|f(x)| \leqslant M$ para todo $x \in \{x \in X\,; |f(x)| > M\}$, isto é, $|f(x)| \leqslant M$ em quase toda parte em $X$.

\begin{pbox}
    Seja $(X,\eth,\mu)$ um espaço de medida.
    Então
    \[
        |f(x)| \leqslant \Vert f \Vert_\infty \text{ qtp em } X
    \]
    para todo $f \in \cL^\infty$
\end{pbox}
\begin{prf}
    Se $f \in \cL^\infty$, então existe $M \geqslant 0$ tal que $|f(x)| \leqslant M$ em quase toda parte em $X$.
    Daí, como $\Vert f \Vert_\infty = \inf \{ M_0 \geqslant 0 \,; |f(x)| \leqslant M_0 \text{ qtp em } X\}$, temos que dado $\varepsilon > 0$ conseguimos encontrar $M_\varepsilon \geqslant 0$ tal que $|f(x)| \leqslant M_\varepsilon$ em quase toda parte em $X$,
    \begin{center}
        \begin{tikzpicture}
            \draw[-stealth] (0,0) to (4,0);
    
            \draw (1,0.1) to (1,-0.1) node[below, font=\small] {$\Vert f \Vert_\infty$};
            \draw (3,0.1) to (3,-0.1) node[below, font=\small] {$\Vert f \Vert_\infty + \varepsilon$};
            \draw (2,0.1) node[above, font=\small] {$M_\varepsilon$} to (2,-0.1);
        \end{tikzpicture}
    \end{center}
    como $M_\varepsilon < \Vert f \Vert_\infty + \varepsilon$, então
    \[
        |f(x)| \leqslant M_\varepsilon < \Vert f \Vert_\infty +  \varepsilon.
    \]
    Fazendo $\varepsilon \to 0$ chegamos a
    \[
        |f(x)| \leqslant \Vert f \Vert_\infty \text{ qtp em } X
    \]
\end{prf}

Agora mostremos que $\cL^\infty$ é um espaço vetorial normado

\begin{pbox}
    A aplicação $\Vert \cdot \Vert_\infty : \cL^\infty \to \bR$ dada por
    \[
        \Vert f \Vert_\infty = \inf \{M \geqslant 0 \,; |f(x)|\leqslant M \text{ qtp em } X\}
    \]
    é uma norma
\end{pbox}
\begin{prf}
    Note que
    \begin{enumerate}
        \item $\Vert f \Vert_\infty \geqslant 0$ pois $0$ é cota inferior de $\{M \geqslant 0 \,; |f(x)|\leqslant M \text{ qtp em } X\}$.
        \item $\Vert f \Vert_\infty = 0$, assim dado $\varepsilon > 0$ existe $M_\varepsilon \geqslant 0$ tal que $|f(x)| \leqslant M_\varepsilon$ em quase toda parte em $X$, com $M_\varepsilon < \varepsilon$.
        Daí, $|f(x)| < \varepsilon$ em quase toda parte em $X$.
        Fazendo $\varepsilon \to 0$, encontramos
        \[
            |f(x)| \leqslant 0 \text{ qtp em } X
        \]
        Dessa forma, $f(x) = 0$ em quase toda parte em $X$.

        Reciprocamente, $\Vert 0 \Vert_\infty = \inf \{M \geqslant 0 \,; 0 \leqslant M \text{ qtp em } X\} = \inf[0,\infty) = 0$

        \item (Desigualdade de Minkowski em $\cL^\infty$) Se $f, g \in \cL^\infty$ então as funções são limitadas em quase toda parte em $X$, dito isso, $f + g$ também é limitada em quase toda parte em $X$.
        Logo $f + g \in \cL^\infty$.

        Por outro lado, como $f, g \in \cL^\infty$, então existem $M,\hat M \in \eth$ tais que $\mu(M) = \mu(\hat M) = 0$ e $|f(x)| \leqslant \Vert f \Vert_\infty$ para todo $x \not\in M$ e $|g(x)| \leqslant \Vert g \Vert_\infty$ para todo $x \not\in \hat M$.
        Seja $N = M \cup \hat M \in \eth$.
        Daí $\mu(N) = \mu(M \cup \hat M) \leqslant \mu(M) + \mu(\hat M) = 0 + 0 = 0$.
        Além disso
        \[
            f(x) + g(x) \leqslant |f(x)| + |g(x)| \leqslant \Vert f \Vert_\infty + \Vert g \Vert_\infty \text{ qtp em } X
        \]
        para todo $x \not\in N$, com $\mu(N) = 0$.
        Dessa forma
        \[
            \Vert f + g \Vert_\infty \leqslant \Vert f \Vert_\infty + \Vert g \Vert_\infty.
        \]
    \end{enumerate}
    Portanto, $\Vert \cdot \Vert_\infty$ é uma norma.
\end{prf}

\begin{pbox}[Desigualdade de Hölder em $\cL^\infty$]
    Seja $(X,\eth,\mu)$ um espaço de medida. Se $f \in \cL^1$ e $g \in \cL^\infty$, então $fg \in \cL^1$ e
    \[
        \Vert fg \Vert_1 \leqslant \Vert f \Vert_1 \Vert g \Vert_\infty
    \]
\end{pbox}
\begin{prf}
    Note que se $g \in \cL^\infty$ então $|g| \leqslant \Vert g \Vert_\infty$ em quase toda parte em $X$.
    Consequentemente
    \[
        \Vert fg \Vert_1 = \int |fg| \,d\mu = \int |f|\,|g| \,d\mu \leqslant \int |f| \Vert g \Vert_\infty \,d\mu = \Vert g \Vert_\infty \int |f| \,d\mu = \Vert g \Vert_\infty \Vert f \Vert_1
    \]
\end{prf}
\end{document}