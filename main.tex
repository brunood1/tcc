\documentclass[a4paper, 11pt]{report}
\usepackage[T1]{fontenc}
\usepackage[utf8]{inputenc}

\usepackage[margin=1in]{geometry}
\usepackage{amsmath, amsfonts, amsthm, amssymb, amsxtra}

\usepackage{graphicx}
\usepackage{float}

\usepackage[portuguese]{babel}

\usepackage[dvipsnames]{xcolor}

% \usepackage{sansmathfonts}
% \usepackage[T1]{fontenc}
% \renewcommand*\familydefault{\sfdefault} %% Only if the base font of the document is to be sans serif

\usepackage{cmbright}

\usepackage{tabularray}
\usepackage{enumitem}
\usepackage{multicol}
\usepackage{setspace}

\usepackage{caption} 
\captionsetup{justification=centering} 

\usepackage{hyperref}
\hypersetup{hidelinks}

\usepackage{tikz}
\usetikzlibrary{intersections, angles, calc, positioning}
\usetikzlibrary{shapes.geometric, arrows.meta}
\usetikzlibrary{decorations.pathmorphing, decorations.pathreplacing}

\usepackage[explicit]{titlesec}
\usepackage[letterspace=100]{microtype}
\usepackage{fmtcount}

\titleformat{\chapter}[display]
{\bfseries\large\sf\lsstyle}
{\huge\filleft\mbox{\MakeUppercase{\chaptertitlename} \NUMBERstring{chapter}}}
{1.5ex}
{\titlerule
\vspace*{1.1ex}%
\MakeUppercase{#1}}
[\vspace*{1.5ex}%
\titlerule]
\titlespacing*{\chapter}{0pt}{-10pt}{25pt}

\titleformat{\section}{\Large\bfseries\sffamily\centering}{\thesection}{0.5em}{#1}

\usepackage{thmtools}
\usepackage{thm-restate}
\usepackage[framemethod=TikZ]{mdframed}
\mdfsetup{skipabove=1em,skipbelow=0em, 
innertopmargin=9pt, innerbottommargin=8pt,
innerleftmargin=8pt, innerrightmargin=8pt}

\renewcommand{\qedsymbol}{$\Box$}

\theoremstyle{definition}

\declaretheoremstyle[headfont=\bfseries\sffamily, bodyfont=\normalfont, 
mdframed={nobreak}
]{thmbox}
\declaretheorem[style=thmbox, name=Definição, numberwithin=chapter]{dbox}
\declaretheorem[style=thmbox, name=Teorema, numberwithin=chapter, sibling=dbox]{tbox}
\declaretheorem[style=thmbox, name=Proposição, numberwithin=chapter, sibling=dbox]{pbox}
\declaretheorem[style=thmbox, name=Corolário, numberwithin=chapter, sibling=dbox]{cbox}
\declaretheorem[style=thmbox, name=Lema, numberwithin=chapter, sibling=dbox]{lbox}

\declaretheoremstyle[headfont=\bfseries\sffamily, bodyfont=\normalfont]{exp}
\declaretheorem[style=exp, name=Exemplo, numberwithin=chapter, sibling=dbox]{ex}

\declaretheoremstyle[headfont=\sffamily\itshape, bodyfont=\normalfont]{pr}
\declaretheorem[numbered=no, style=pr, name=Demonstração, qed=\qedsymbol]{prf}

\newcommand{\obs}{\noindent{\textbf{\textcolor{black}{\sffamily Observação:}}}~}

\setlength{\parskip}{5pt}

\definecolor{tangerine}{RGB}{249, 163, 13}

\newcommand{\medcup}{\mathsf{U}}
\newcommand{\m}{\text{-}}

\newcommand{\bN}{\mathbb{N}}
\newcommand{\bZ}{\mathbb{Z}}
\newcommand{\bQ}{\mathbb{Q}}
\newcommand{\bR}{\mathbb{R}}
\newcommand{\bC}{\mathbb{C}}
\newcommand{\bK}{\mathbb{K}}

\newcommand{\bu}{\mathbf{u}}
\newcommand{\bv}{\mathbf{v}}
\newcommand{\bX}{\mathbf{X}}
\newcommand{\BQ}{\mathbf{Q}}

\newcommand{\cA}{\mathcal{A}}
\newcommand{\cB}{\mathcal{B}}
\newcommand{\cC}{\mathcal{C}}
\newcommand{\cF}{\mathcal{F}}
\newcommand{\cM}{\mathcal{M}}
\newcommand{\cH}{\mathcal{H}}
\newcommand{\cL}{\mathcal{L}}
\newcommand{\cT}{\mathcal{T}}
\newcommand{\cP}{\mathcal{P}}
\newcommand{\cW}{\mathcal{W}}

\newcommand{\supp}{\mathrm{supp}\,}
\newcommand{\esssup}{\mathrm{ess\,sup}\,}
\newcommand{\loc}{\mathrm{loc}}
\newcommand{\sgn}{\mathrm{sgn}}

\newcommand{\doublehookrightarrow}{\;\substack{\hookrightarrow \\ \hookrightarrow}\;}

\newcommand{\sfrac}[2]{{}^{#1}\!\!/\!_{#2}}
\newcommand{\sint}{-\!\!\!\!\!\!\int}


\usepackage{csquotes}
\usepackage[sortcites, backend=biber, maxbibnames=5, minbibnames=3, sorting=nty, url=false, isbn=false, doi=false, defernumbers=true]{biblatex}
\addbibresource{bibliography.bib}


\begin{document}


\begin{figure}[!h]
\centering
	\includegraphics[scale=0.2]{ufs_vertical_positiva.eps}
\end{figure}
\vspace{-0.5cm}
\begin{center}

\begin{singlespace}

{\large{Universidade Federal de Sergipe}}

{\large{Centro de Ciências Exatas e Tecnologia}}

{\large{Departamento de Matemática}}

{\large{Graduação em Matemática}}
\end{singlespace}



\vspace{3.0cm}


\LARGE{{\textbf{Decaimentos das Soluções de Leray e um Problema de Dirichlet via Espaços de Sobolev}}}

\vspace {3cm}

{\Large{Bruno Sant'Anna Donato de Moura}}

\vspace{5.0cm}


\begin{singlespace}
\large{São Cristóvão - SE}\\
\large{Junho de 2025}
\end{singlespace}

\end{center}

\pagebreak

%%%%%% CONTRA-CAPA %%%%%

\thispagestyle{empty}

\begin{center}
	
	{\large Bruno Sant'Anna Donato de Moura}
	
	\vspace{6.5cm}
	
	{\Large{Decaimentos das Soluções de Leray e um Problema de Dirichlet via Espaços de Sobolev}}
	
\end{center}

\vspace{1.5cm}

\vfill

{\large
	
	\hspace{7cm}{\parbox{8cm}{\small Trabalho de Conclusão de Curso apresentado ao Departamento de Matemática da Universidade Federal de Sergipe para obtenção do título de Bacharel em Matemática Aplicada e Computacional.}}
}
\vfill
{\large \hspace{7cm}{\parbox{8cm}{\small Orientador: Prof. Wilberclay Gonçalves Melo}}}

\vspace{1cm}
\begin{center}
		Junho, 2025
\end{center}

\pagebreak

\thispagestyle{empty}

\begin{center}
\Large{Decaimentos das Soluções de Leray e um Problema de Dirichlet via Espaços de Sobolev}


\vspace{0.5cm}

{por}

\vspace{1.0cm}

\href{http://lattes.cnpq.br/4171937682648273}
{\large{Bruno Sant'Anna Donato de Moura}}
\end{center}

\vspace{1.0cm}

\noindent Área de Conhecimento: Ciências Exatas e Tecnologia

\noindent Subárea de Conhecimento: Matemática

\noindent Especialidade de Conhecimento: Análise

\vspace{1.0cm}

\noindent Banca Examinadora:

\vspace{2.5cm}


\begin{center}
\rule{10cm}{.1mm} \\
{Prof. Dr. Wilberclay Gonçalves Melo } \\
(Orientador - DMA - UFS)
\vspace{0.8cm}


\rule{10cm}{.1mm} \\
{Profª. MSc. Natielle dos Santos Costa} \\
(Primeira Examinadora - Instituto Federal de Sergipe (IFS))
\vspace{0.8cm}


\rule{10cm}{.1mm} \\
{Prof. Dr. Gerson Cruz Araujo} \\
(Segundo Examinador - DMA - UFS)
\vspace{0.8cm}

\end{center}

\pagebreak

\thispagestyle{empty}

\mbox{}

\vspace{19cm}


\hfill{
\begin{minipage}[b]{6.5cm}
\slshape{\large{À meu avô, que sempre me ensinou matemática.
		} }

\end{minipage}}

\pagebreak

% \chapter*{Agradecimentos}
\chapter*{Resumo}

Neste trabalho, apresentamos uma introdução à teoria dos Espaços de Sobolev. Mais especificamente, estabelecemos as principais definiições, propriedades, desigualdades, bem como os mergulhos contínuos e compactos envolvendo esses espaços e os Espaços de Lebesgue. Além disso, com essas informações em mãos, utilizamos tais espaços como ambiente para conduzir um estudo sobre as soluções fracas de Leray para as Equações de Navier-Stokes, ademais, mostramos a existência de soluções fracas para um específico problema de Dirichlet.

\vfill

\noindent\textbf{Palavras-Chave: Equações Diferenciais Parciais; Espaços de Sobolev; Análise Funcional; Equações de Navier-Stokes; Problema de Dirichlet}.

\chapter*{Abstract}

Throughout this work, we present an introduction to Sobolev Spaces. Namely, we establish the main definitions, properties, innequalities, as well as continuous and compact embeddings involving these spaces and the Lebesgue spaces. Furthermore, with these concepts, we aim to study the Leray weak solutions to the Navier-Stokes equations, in addition, we also show the existance of weak solutions to a specific Dirichlet problem.

\vfill

\noindent\textbf{Keywords: Partial Differential Equations; Sobolev Spaces; Functional Analysis; Navier-Stokes Equtions; Dirichlet Problem}.
\tableofcontents
\chapter*{Introdução}

Neste trabalho, estamos interessados em dissertar sobre os famosos Espaços de Sobolev e suas utilidades nos estudos de um problema de Dirichlet e das Equações de Navier-Stokes. Sendo assim, antes de começarmos a introduzir o que
% precisamente 
pretendemos expor nesta monografia, permita-nos descrever brevemente algumas informações sobre os matemáticos mais relevantes que são frequentemente homenageados nestas teorias.

Sergei Lvovich Sobolev (1908--1989) foi um matemático soviético que contribuiu em diversas áreas da Matemática e ciências afins. Mais especificamente, foi pioneiro no estudo de soluções fracas para algumas equações diferenciais. Além disso, é importante destacar que  Sobolev também orientou Olga Ladyzhenskaya (1922--2004),  a qual foi a matemática russa responsável por obter uma solução global no tempo para as equações de Navier-Stokes em espaços Euclidianos bidimensionais. Vale enfatizar, por fim, que o conhecimento adquirido através da pesquisa de Sobolev eventualmente se tornou uma área própria da Análise Funcional e que apresentamos neste texto uma introdução aos Espaços de Sobolev com a meta de aplicá-la a um problema de Dirichlet e às Equações de Navier-Stokes. (Ver \cite{sobolev} para mais detalhes sobre Sobolev).

Claude Louis Marie Henri Navier (1785--1836) foi um matemático e engenheiro francês com contribuições extremamente importantes na dinâmica de fluidos. Em seu artigo ``Sur les lois des mouvements des fluides, en ayant égard à l’adhésion des molecules'', Navier mostrou as primeiras derivações das equações que futuramente carregariam seu nome. Mais precisamente, em $1820$, na École des Ponts et Chaussées, Navier iniciou a pesquisar a matemática envolvida no movimento de um fluido viscoso, e já nos dois anos decorrentes descreveu as, hoje denominadas, Equações de Navier-Stokes. Porém, o trabalho de Navier tinha um raciocínio imperfeito do ponto de vista matemático; mas, as equações encontradas descreviam corretamente tal movimento. Essa imprecisão na derivação do modelo inspirou Stokes a escrever de forma precisa como encontrar as famosas Equações de Navier-Stokes. (Para mais detalhes, ver \cite{navier}).

George Gabriel Stokes (1819--1903) foi um físico e matemático irlandês que trabalhou em diversas áreas; mais  especificamente, na dinâmica de fluidos, onde, com base na pesquisa de Navier, Stokes desenvolveu equações que modelam o movimento de fluidos, conhecidas nos dias atuais como Equações de Navier-Stokes. É importante lembrar que a existência de  soluções clássicas e globais no tempo para este sistema em espaços Euclidianos tridimensionais ainda é um problema em aberto e é considerado um dos questionamentos mais relevantes da Análise Não-linear. (Citamos \cite{stokes} para mais informações).

Jean Leray (1906--1998) foi um matemático francês que trabalhou em diversas áreas da Matemática como, por exemplo, Topologia e Álgebra Homológica.
Gostaríamos de ressaltar que seu principal trabalho foi em Equações Diferenciais Parciais, é intitulado ``Sur le mouvement d'un liquide visqueux emplassement l'espace'' e prova a existência de soluções fracas (ou turbulentas), denominadas hoje soluções de Leray, para as Equações de Navier-Stokes  (ver \cite{leray-fluid}). Deste modo, neste trabalho, apresentamos algumas estimativas de decaimento  para estas mesmas soluções em Espaços de Sobolev, quando o tempo de existência é arbitrariamente grande. (Para mais informações, ver \cite{leray}).

Johann Peter Gustav Lejeune Dirichlet (1805--1859) foi um matemático alemão e um dos pesquisadores mais importantes em diversas áreas da Matemática como, por exemplo, Teoria dos Números e Análise Matemática. Em Equações Diferencias Parciais, Dirichlet estudou diversos problemas de fronteira. Esta é a razão pela qual decidimos estudar um certo sistema de Equações Diferenciais Parciais, através da aplicação do Teorema de Lax-Milgram (em Espaços de Hilbert), com o intuito de comprovar a existência de soluções fracas para o problema apresentado em Espaços de Sobolev. (Para mais detalhes, ver \cite{dirichlet}).

Agora, permita-nos estabelecer, com um pouco mais de detalhes, o que vai ser estudado nos dois capítulos desta monografia. No primeiro capítulo, apresentamos um estudo básico para os Espaços de Sobolev. Mais precisamente, determinamos suas definições, juntamente com o conceito de derivada fraca, e suas propriedades elementares. Além disso, para poder demonstrar algumas desigualdades, conhecidas como Desigualdades de Sobolev, Morrey e Gagliardo-Nirenberg-Sobolev, adicionamos ao nosso texto o significado de Extensões e Traços de operadores lineares limitados. Com o objetivo de concluir uma teoria introdutória para os Espaços de Sobolev, determinamos alguns mergulhos contínuos e compactos, envolvendo alguns destes espaços e Espaços de Lebesgue. Para facilitar a compreensão do leitor, vamos listar alguns resultados que representam nosso estudo neste primeira parte do trabalho.
\begin{itemize}[label={\small$\bullet$}]
  \item Sejam $\Omega$ um aberto limitado, com fronteira de classe $\cC^1$ e $\Omega'$ um aberto tal que $\Omega \Subset \Omega'$. Então, existe um operador linear limitado $E : \mathcal{W}^{1,p}(\Omega) \to \mathcal{W}^{1,p}(\mathbb{R} ^n)$, com $1 \leq p < \infty$, tal que, para cada $u \in \mathcal{W}^{k,p}(\Omega)$, tem-se que
    \begin{enumerate}
        \item $Eu = u$ qtp em $\Omega$;
        \item $\supp Eu \subseteq \Omega'$;
        \item $\Vert Eu \Vert_{\mathcal{W}^{1,p}(\mathbb{R}^n)} \leqslant c \Vert u \Vert_{\mathcal{W}^{1,p}(\Omega)}$.
    \end{enumerate}
  \item Seja $\Omega$ um aberto limitado, com fronteira de classe $C^1$. Então, existe um operador linear limitado $T : \mathcal{W}^{1,p}(\Omega) \to \mathcal{L}^p(\partial \Omega)$, com $1 \leq p < \infty$, tal que
    \begin{enumerate}
        \item $Tu = u \big|_{\partial \Omega}$, se $u \in \mathcal{W}^{1,p}(\Omega) \cap C(\Omega)$;
        \item $\Vert Tu \Vert_{\mathcal{L}^p(\partial\Omega)} \leq c \Vert u \Vert_{\mathcal{W}^{1,p}(\Omega)}$.
    \end{enumerate}
  \item Sejam $\Omega$ um aberto limitado com fronteira de classe $\cC^1$ e $u \in \mathcal{W}^{k,p}(\Omega)$.
    Se $kp < n$, então $u \in \mathcal{L}^q(\Omega)$, onde
    \[
        \frac{1}{q} = \frac{1}{p} - \frac{k}{n}.
    \]
    Além disso, a desigualdade
    \[
        \Vert u \Vert_{\mathcal{L}^q(\Omega)} \leqslant c \Vert u \Vert_{\mathcal{W}^{k,p}(\Omega)}
    \]
    é valida.
  \item Seja $\Omega \subseteq \mathbb{R}^n$ um aberto limitado com fronteira de classe $\cC^1$.
  Então, $\mathcal{W}^{1,p}(\Omega)$ está mergulhado compactamente em  $\mathcal{L}^q(\Omega)$,
        com $1 \leq p < n$ e $1 \leq q < p^*$.
\end{itemize}

\textbf{Gostaríamos de enfatizar que a referência bibliográfica principal que foi usada neste primeiro capítulo é a seguinte: \cite{evans-pde}.}

O segundo capítulo é destinado a duas aplicações dos Espaços de Sobolev:
a primeira delas, visa estudar o decaimento de soluções fracas  de Leray para as Equações de Navier-Stokes nos Espaços de Sobolev $H^k(\mathbb{R}^3)$. Para este fim, é necessário obter algumas formas de estimar as normas de Lebesgue-$L^2(\mathbb{R}^3)$ destas mesmas soluções. Já a segunda aplicação diz respeito a mostrar a existência de uma única solução fraca no Espaço de Sobolev $H^1_0(\mathbb{R}^3)$ para um específico problema de Dirichlet, onde utilizamos como ferramenta o Teorema de Lax-Milgram. Vale lembrar que este resultado pode ser visto em um curso introdutório de Análise Funcional  (ver \cite{kreyszig-functional.analysis} para mais detalhes). Com a meta de facilitar o entendimento do leitor, vamos listar alguns resultados que representam as informações descritas logo acima.
\begin{itemize}[label={\small$\bullet$}]
\item Seja $\mathbf{u}(\cdot,t)$ uma solução fraca de Leray para as Equações de Navier-Stokes, então
    \[
        \Vert \mathbf{u}(\cdot,t) \Vert_{\mathcal{L}^2(\mathbb{R}^3)} \to 0,
    \]
    quando $t \to \infty$.
    É importante lembrar que este resultado foi conjecturado por Leray em seu artigo \cite{leray-fluid} e foi provado somente cinquenta anos depois por Kato \cite{kato-navier.stokes}.
\item Seja $\mathbf{u}(\cdot,t)$ uma solução fraca de Leray para as Equações de Navier-Stokes, então
    \[
        t^{\frac{k}{2}} \Vert D^k\mathbf{u}(\cdot,t) \Vert_{\mathcal{L}^2(\mathbb{R}^3)} = 0,
    \]
    quando $t \to \infty$,
    para todo $k \geq 0$ inteiro.
\item Considere o problema de Dirichlet
\begin{equation}\label{w1}
    \left\{
    \begin{aligned}
        -\Delta u + u = f, &\text{ em } \Omega;\\
        u = 0, &\text{ sobre } \partial\Omega,
    \end{aligned}
    \right.
\end{equation}
onde $\Omega \subseteq \mathbb{R}^n$ é um aberto limitado e $f \in \mathcal{L}^2(\Omega)$. Então,
existe uma única solução fraca $u\in H_0^1(\mathbb{R}^3)$ para o sistema (\ref{w1}).
\end{itemize}
\textbf{É relevante informar que as referências bibliográficas principais que foram usadas neste segundo capítulo são as seguintes: \cite{kreyszig-functional.analysis,zingano-edp}}

Permita-nos notificar o leitor que alguns preliminares são considerados conhecidos para o desenvolvimento deste trabalho. Mais especificamente, algumas definições e resultados importantes sobre cursos introdutórios de Teoria da Medida e Integração, Análise Funcional e Análise Matemática podem ser encontrados no decorrer do nosso texto ou nas referências \cite{axler-measure.theory, bartle-measure.theory, brezis-functional.analysis, folland-real.analysis,kreyszig-functional.analysis, elon-analise.rn, elon-espacos.metricos, munkres-analysis.on.manifolds}. Em adição, como já foi relatado anteriormente, as referêncais principais para este estudo são os textos \cite{adams-sobolev, evans-pde,zingano-edp}.

Por fim, é relevante ressaltar que, em toda a monografia, mesmo se o valores das constantes, que aparecem nas desigualdades ou igualdades, mudarem linha a linha, ainda continuamos denotando-as pelos mesmos símbolos.
% \input{chap1-medida.tex}
% \chapter{Introdução à análise funcional}

\textcolor{red}{(introdução)}

\section{Espaços de Banach}

\textcolor{red}{(introdução)}

\begin{dbox}
    Seja $X$ um espaço vetorial sobre um corpo $\bK$. Uma função $\Vert \cdot \Vert : X \to \bR$ é dita ser uma norma se satisfaz
    \begin{itemize}[leftmargin=*]
        \item $\Vert x \Vert \geqslant 0$ para todo $x \in X$
        \item $\Vert x \Vert = 0 \iff x = 0$
        \item $\Vert \lambda x \Vert = |\lambda| \Vert x \Vert$ para todo $x \in X$ e $\lambda \in \bK$
        \item $\Vert x + y \Vert \leqslant \Vert x \Vert + \Vert y \Vert$ para todo $x,y \in X$
    \end{itemize}
\end{dbox}

\textcolor{red}{(definições iniciais)}

\begin{ex}
    O espaço euclidiano $\bR^n$ munido da norma
    \[
        \Vert x \Vert = \sqrt{x_1^2 + \cdots + x_n^2}
    \]
    onde $x = (x_1,\dots,x_n)$ é um espaço de Banach
\end{ex}

\begin{ex}
    O espaço
    \[
        \ell^p \equiv \ell^p(\bR) = \left\{ x = (x_n) \,; \sum_{i=1}^{\infty} |x_i|^p < \infty \right\}
    \]
    com $1 \leqslant p < \infty$ munido da norma
    \[
        \Vert x \Vert_p = \left( \sum_{i=1}^{\infty} |x_i|^p \right)^{\frac{1}{p}}
    \]
    é um espaço de Banach.
\end{ex}

\begin{ex}
    O espaço
    \[
        \ell^\infty \equiv \ell^\infty(\bR) = \left\{ x = (x_n) \,; \sup_{n \in \bN} |x_n| < \infty \right\}
    \]
    munido da norma
    \[
        \Vert x \Vert_\infty = \sup_{n \in \bN} |x_n|
    \]
    é um espaço de Banach.
\end{ex}

\begin{ex}
    O espaço
    \[
        \cC([a,b], \bK) = \{f : [a,b] \to \bK \,; f \text{ é contínua}\}
    \]
    munido da norma
    \[
        \Vert f \Vert_{\max} = \max_{t \in [a,b]} \{|f(t)|\}
    \]
    é um espaço de Banach
\end{ex}

\begin{ex}
    O espaço $\cC([0,1], \bR)$ munido da métrica
    \[
        d(f,g) = \int_0^1 |f(t) - g(t)| \, dt
    \]
    não é um espaço completo
\end{ex}
\chapter{Espaços de Sobolev} \label{ch:sobolev}

Os espaços de Sobolev desempenham um papel fundamental na análise funcional e nas equações diferenciais parciais, oferecendo uma estrutura adequada para o estudo de problemas envolvendo funções que podem não ser diferenciáveis no sentido clássico. Introduzidos como uma extensão dos conceitos de derivada e integrabilidade, esses espaços permitem trabalhar com soluções generalizadas, chamadas de soluções fracas, ampliando o escopo de problemas que podem ser tratados matematicamente. Neste capítulo, serão apresentados os conceitos básicos dos espaços de Sobolev e suas principais propriedades elementares.

Vale ressaltar que, quando necessário, consideramos que as funções $\Omega \subseteq \bR^n \to \bR$ são definidas de forma que sejam nulas em $\bR^n \setminus \Omega$. 
\section{Preliminares}

Antes de começarmos de fato o estudo dos espaços de Sobolev, precisamos de algumas definições que serão usadas extensivamente nesse capítulo.

\begin{dbox} \label{def:suporte}
    Seja $\varphi : \Omega \subseteq \bR ^n \to \bR$ uma função qualquer. Definimos o suporte de $\varphi$ por
    \[
        \supp \varphi = \overline{\{x \in \Omega \,; \varphi(x) \neq 0\}}.
    \]
    Além disso, se $\supp\varphi$ é compacto, dizemos que $\varphi$ tem suporte compacto.
\end{dbox}

Note que $\{x \in \Omega \,; \varphi(x) \neq 0\} \subseteq \supp \varphi$; então\footnotemark, $(\supp \varphi)^\cC \subseteq \{x \in \Omega \,; \varphi(x) = 0\}$. Ou seja, se $x \not\in \supp \varphi$, então $\varphi(x) = 0$.
Por outro lado, $\supp \varphi \subseteq \Omega$, então, $\Omega^\cC \subseteq (\supp \varphi)^\cC$. \footnotetext{Utilizamos a notação $X^\cC$ para o complementar do conjunto $X$.}
Se $\Omega$ é aberto, então $\partial \Omega \subseteq \Omega^\cC$.
Portanto, se $\Omega$ é aberto, $\varphi$ se anula em $\partial\Omega$, já que
\[
    \partial \Omega \subseteq \Omega^\cC \subseteq (\supp \varphi)^\cC \subseteq \{x \in \Omega \,; \varphi(x) = 0\}.
\]

\begin{dbox}
    Seja $\Omega$ um aberto de $\bR^n$. O espaço $\cC^k(\Omega)$ é composto por todas funções $f : \Omega \to \bR$ contínuas em que suas derivadas parciais de ordem menor ou igual a $k$ também são contínuas.
    Se $f \in \cC^k(\Omega)$, dizemos que $f$ é de classe $\cC^k$.

    O conjunto das funções infinitamente diferenciáveis é definido por
    \[
        \cC^\infty(\Omega) = \bigcap_{k=0}^\infty \cC^k(\Omega).
    \]
\end{dbox}

A definição abaixo será amplamente utilizada nesse capítulo.

\begin{dbox}
    O conjunto das funções contínuas com suporte compacto em $\Omega$ é definido por
    \[
        \cC_c(\Omega) = \{f : \Omega \to \bR \text{ contínua} \,; \supp f \text{ é compacto}\}.
    \]
    Além disso, definimos
    \[
        \cC^k_c(\Omega) = \cC_c(\Omega) \cap \cC^k(\Omega),
    \]
    que é o conjunto das funções $f : \Omega \to \bR$ de classe $\cC^k$ com suporte compacto.
\end{dbox}

Nesse texto, quanto não estiver explicito o contrário, $\Omega \subseteq \bR^n$

\begin{dbox}[Espaços $\cL^p$ e $\cL^\infty$]
    O espaço $\cL^p = \cL^p(\Omega)$, com $1 \leqslant p < \infty$, é composto pelas funções $f : \bR^n \to \bR$, tais que
    \[
        \Vert f \Vert_{\cL^p(\Omega)} = \int_\Omega |f|^p \, dx < \infty.
    \]
    O espaço $\cL^\infty = \cL^\infty(\Omega)$ é composto pelas funções $f : \bR^n \to \bR$ mensuráveis e limitadas em quase toda parte em $\Omega$, e definimos sua norma por
    \[
        \Vert f \Vert_{\cL^\infty(\Omega)} = \mathrm{ess\,sup} \{|f(x)|; x \in \Omega\} =\inf \{M \geqslant 0 \,; |f(x)| \leqslant M \text{ qtp em } \Omega\}.
    \]
\end{dbox}

\begin{dbox}
    O conjunto das funções $f \in \cL^p(\Omega)$ localmente somáveis, isto é, integráveis em todo conjunto compacto de $\Omega$ é denotado por $\cL^p_{\mathrm{loc}}(\Omega)$.
\end{dbox}

Os resultados abaixos são de extrema importância no estudo de espaços de Sobolev e podem ser encontrados, em sua maioria, em um curso introdutório de medida e integração de Lebesgue (ver \cite{axler-measure.theory,bartle-measure.theory,folland-real.analysis}).

\begin{tbox}[Teorema da Convergência Dominada] \label{thm:teorema-da-convergencia-dominada}
    Seja $(f_n)$ uma sequência de funções integráveis que converge em quase toda parte para a função mensurável $f$.
    Se existe uma função integrável $g$ tal que $|f_n| \leqslant g$ em quase toda parte, para todo $n \in \bN$, então $f$ é integrável e
    \[
        \int_\Omega f \,d\mu = \lim \int_\Omega f_n \,d\mu.
    \]
\end{tbox}
\begin{prf}
    Ver \cite{bartle-measure.theory}, p.p. 44.
\end{prf}

\begin{tbox} \label{thm:lp-completo-pre}
    O espaço $(\cL^p(\Omega), \Vert \cdot \Vert_{\cL^p(\Omega)})$, com $1 \leqslant p \leqslant \infty$, é um espaço de Banach.\footnotemark
\end{tbox}
\begin{prf}
    Ver \cite{bartle-measure.theory}, p.p. 59 ($1 \leqslant p < \infty$) e p.p. 61 ($p = \infty$).
\end{prf}

\footnotetext{Espaço vetorial normado completo, i.e., toda sequência de Cauchy é convergente.}

\begin{tbox}[Desigualdade de Hölder] \label{thm:pre-desigualdade-de-holder}
    Sejam $f \in \cL^p(\Omega)$ e $g \in \cL^q(\Omega)$, onde $1 \leqslant p \leqslant \infty$, e $q \in \bR$ é tal que $p$ e $q$ são expoentes conjugados\footnotemark.
    Então, $fg \in \cL^1(\Omega)$ e
    \[
        \Vert fg \Vert_{\cL^1(\Omega)} \leqslant \Vert f \Vert_{\cL^p(\Omega)} \Vert g \Vert_{\cL^q(\Omega)}.
    \]
\end{tbox}
\begin{prf}
    Ver \cite{bartle-measure.theory}, p.p. 56.
\end{prf}

\footnotetext{$p$ e $q$ são ditos expoentes conjugados se $\frac{1}{p} + \frac{1}{q} = 1$.}

Uma generalização desse resultado será apresentada abaixo.

\begin{tbox}[Desigualdade de Hölder Generalizada] \label{thm:pre-desigualdade-de-holder-gen}
    Sejam $0 < p_1,\dots,p_N \leqslant \infty$ tais que $\frac{1}{p} = \frac{1}{p_1} + \cdots + \frac{1}{p_N}$ e $f = f_1 f_2 \cdots f_N$ onde $f_j \in \cL^{p_j}(\Omega)$, para todo $j = 1,\dots,N$. 
    Então, $f \in \cL^p$ e 
    \[
        \Vert f \Vert_p \leqslant \Vert f_1 \Vert_{\cL^{p_1}(\Omega)} \cdots \Vert f_N \Vert_{\cL^{p_N}(\Omega)}.
    \] 
\end{tbox}
\begin{prf}
    Ver \cite{adams-sobolev}, p.p. 25.
\end{prf}

\begin{tbox}[Desigualdade de Young] \label{thm:desigualdade-de-young}
    Sejm $A,B \geqslant 0$, $1 \leqslant p < \infty$ e $q \in \bR$ tal que $p$ e $q$ são expoentes conjuntados. Então
    \[
        AB \leqslant \frac{A^p}{p} + \frac{B^q}{q},
    \]
    onde a igualdade é válida se, e somente se, $A^p = B^q$.
\end{tbox}
\begin{prf}
    Ver \cite{bartle-measure.theory}, p.p. 56.
\end{prf}

\begin{tbox}[Desigualdade de Young para convluções] \label{thm:desigualde-de-young-para-convolucoes}
    Sejam $f \in \cL^p(\Omega)$ e $g \in \cL^q(\Omega)$, com $1 \leqslant p,q,r \leqslant \infty$ e
    \[
        \frac{1}{p} + \frac{1}{q} = \frac{1}{r} +1.
    \]
    Então
    \[
        \Vert f * g \Vert_{\cL^r(\Omega)} \leqslant \Vert f \Vert_{\cL^p(\Omega)} \Vert g \Vert_{\cL^q(\Omega)}.
    \]
\end{tbox}
\begin{prf}
    Ver \cite{brezis-functional.analysis}, p.p. 104.
\end{prf}

\begin{tbox} \label{thm:omega-limitado}
    Se $\Omega$ é limitado e, $1 \leqslant p < q \leqslant \infty$.
    Então, $\cL^q(\Omega) \subseteq \cL^p(\Omega)$ e 
    \[
        \Vert f \Vert_{\cL^p(\Omega)} \leqslant c \Vert f \Vert_{\cL^q(\Omega)},
    \]
    para toda $f \in \cL^q(\Omega)$ e $c > 0$ é uma constante que depende de $p$, $q$ e $\Omega$.
    Além disso, se $f \in \cL^\infty(\Omega)$, tem-se que
    \[
        \Vert f \Vert_{\cL^p(\Omega)} \to \Vert f  \Vert_{\cL^\infty(\Omega)},
    \]
    quando $p \to \infty$.
\end{tbox}
\begin{prf}
    Ver \cite{adams-sobolev}, p.p. 28.
\end{prf}

\begin{tbox}[Desigualdade de Interpolação] \label{thm:desigualdade-de-inteporlacao}
    Sejam $0 < p < q < r \leqslant \infty$.
    Então $\cL^p(\Omega) \cap \cL^r(\Omega) \subseteq \cL^q(\Omega)$ e
    \[
        \Vert f \Vert_{\cL^q(\Omega)} \leqslant \Vert f \Vert_{\cL^p(\Omega)}^\theta \Vert f \Vert_{\cL^r(\Omega)}^{1 - \theta},
    \]
    para toda $f \in \cL^p(\Omega)$,
    onde $\theta \in (0,1)$ e
    \begin{equation*} \label{eq:3}
        \frac{1}{q} = \frac{\theta}{p} + \frac{1 - \theta}{r} \quad \left( \text{i.e., } \theta = \frac{\frac{1}{q} - \frac{1}{r}}{\frac{1}{p} - \frac{1}{r}}\right).
    \end{equation*}
\end{tbox}
\begin{prf}
    Ver \cite{folland-real.analysis}, p.p. 185.
\end{prf}

\begin{tbox} \label{thm:densidadeC}
    O espaço das funções contínuas $\cC(\Omega)$ é denso em $\cL^p(\Omega)$.
\end{tbox}
\begin{prf}
    Ver \cite{adams-sobolev}, p.p. 31.
\end{prf}


\begin{tbox}[Integração por partes em $\bR ^n$] \label{thm:integracao-por-partes}
    Seja $\Omega \subseteq \bR^n$ um aberto e $u, v : \overline\Omega \to \bR$ de classe $\cC^1$. 
    Então,
    \[
        \int_\Omega u \dfrac{\partial v}{\partial x_i} \,dx = \int_{\partial\Omega} uv \nu_i \, dS- \int_\Omega v \dfrac{\partial u}{\partial x_i} \,dx,
    \]
    onde $\nu = (\nu_1,\dots,\nu_n)$ é o vetor normal unitário que aponta para fora em $\partial\Omega$.   
\end{tbox}
\begin{prf}
    Ver \Cite{evans-pde}, p.p. 628.
\end{prf}

\begin{tbox}[Coordenadas polares em $\bR^n$] \label{thm:coordenadas-polares}
    Seja $f : B(y,s) \subseteq \bR^n \to \bR$ uma função integrável, então
    \[
        \int_{B(y,s)} f \,dx = \int_0^s\int_{\partial B(y,r)} f \,dS dr,
    \]
    onde $B(y,s)$ é a bola aberta de raio $s > 0$ e centrada em $y$.
\end{tbox}
\begin{prf}
    Ver \cite{folland-real.analysis}, p.p. 78.
\end{prf}

\begin{tbox} \label{thm:teorema-do-brezis}
    Seja $(u_n)_{n=1}^\infty$ uma sequência em $\cL^p(\Omega)$ tal que $\Vert u_n - u \Vert_{\cL^p(\Omega)} \to 0$, quando $n \to \infty$, para alguma função $u \in \cL^p(\Omega)$.
    Então, existem uma subsequência $(u_{n_k})_{k=1}^\infty$ de $(u_n)_{n=1}^\infty$ e uma função $v \in \cL^p(\Omega)$ tais que
    \begin{enumerate}[label=\textbf{(\alph*)}, leftmargin=*]
        \item $u_{n_k}(x) \to u(x)$ qtp\footnotemark ~ em $\Omega$;
        \item $|u_{n_k}(x)| \leqslant v(x)$ para todo $k \in \bN$, qtp em $\Omega$.
    \end{enumerate}
\end{tbox}
\begin{prf}
    Ver \cite{brezis-functional.analysis}, p.p. 94.
\end{prf}

\footnotetext{Duas funções $u$ e $v$ são ditas iguais em quase toda parte (qtp) se $u(x) = v(x)$ para todo $x \in X$, onde $X^\cC$ tem medida nula.}

\begin{tbox}[Critério de compacidade de Arzelà-Ascoli] \label{thm:arzela}
    Seja $(u_k)$ uma sequência de funções de $\bR^n$ em $\bR$ tal que
    \[
        |u_k(x)| \leqslant M,
    \]
    para todo $x \in \bR^n$, onde $M >0$ é uma constante e $(u_k)$ é equicontínua\footnotemark.
    Então, existem uma subsequência $(u_{k_j})_{j =1}^\infty \subseteq (u_k)_{k=1}^\infty$ e uma função contínua $u$ tal que $u_{k_j} \to u$, quando $j \to \infty$, uniformemente em conjuntos compactos de $\bR^n$.
\end{tbox}
\begin{prf}
    Ver \cite{folland-real.analysis}, p.p 137.
\end{prf}

\footnotetext{Uma sequência de funções $(u_k)$ é dita equicontínua se dado $\varepsilon > 0$, existe $\delta > 0$ (independente de $x$ e $k$), tal que
\[
    \Vert x - y \Vert < \delta \implies |u_k(x) - u_k(y)| < \varepsilon,
\]
para todo $k \in \bN$.
}

\begin{tbox}[Teorema de Fubini] \label{thm:fubini}
    Sejam $(\Omega_1, \cM_1, \mu)$ e $(\Omega_2, \cM_2, \nu)$ espaços de medida.
    Então, se $f \in \cL^1(\Omega_1 \times \Omega_2)$, a igualdade abaixo é válida:
    \[
        \int_{\Omega_1 \times \Omega_2} f \, d\mu \times d\nu = \int_{\Omega_1} \left( \int_{\Omega_2} f(x,y) \,d \nu \right) \, d\mu = \int_{\Omega_2} \left( \int_{\Omega_1} f(x,y) \,d\mu \right) \,d\nu.
    \]
\end{tbox}
\begin{prf}
    Ver \cite{folland-real.analysis}, p.p. 68.
\end{prf}

Uma outra forma de enunciar o Teorema de Fubini é a seguinte.
\begin{tbox}
    Sejam $\Omega_1 \subseteq \bR^n$, $\Omega_2  \subseteq \bR^d$ e $f : \Omega_1 \times \Omega_2 \to \bR$ uma função integrável.
Então,
\[
    \int_{\Omega_1 \times \Omega_2} f \,dX = \int_{\Omega_1} \int_{\Omega_2} f \, dydx = \int_{\Omega_2} \int_{\Omega_1} f \,dxdy.
\]
\end{tbox}

Essa versão do Teorema de Fubini é provada para integrais de Riemann, porém como em alguns casos as integrais de Riemann e Lebesgue coincidem, essas versões são equivalentes.


\begin{tbox}[Mudança de variáveis] \label{thm:mudanca-de-variaveis}
    Seja $\Psi : A \to B$ um difeomorfismo entre abertos de $\bR^n$. Seja $f : B \to \bR$ uma função contínua. Então, $f$ é integrável sobre $B$ se, e somente se, $(f \circ \Psi) |\det D\Psi|$ é integrável sobre $A$. Neste caso,
    \[
        \int_B f \,dx = \int_A (f \circ \Psi) |\det D\Psi| \,dy.
    \]
\end{tbox}
\begin{prf}
    O capítulo 4 de \cite{munkres-analysis.on.manifolds} é destinado a demonstração desse teorema.
\end{prf}

\section{Motivação} \label{sec:motivacao}
Considere o problema de Dirichlet
\begin{equation} \label{eq:problema-dirichlet-motivacao}
    \left\{  
        \begin{aligned}
            -\Delta u + u = f &\text{ em } \Omega;\\
            u = 0 &\text{ sobre } \partial\Omega,
        \end{aligned}
    \right.
\end{equation}
onde $\Omega \subseteq \bR^n$ é um aberto limitado.
Uma solução clássica para o problema acima é uma função $u \in \cC^2(\Omega)$ satisfazendo (\ref{eq:problema-dirichlet-motivacao}). 
Por outro lado, multiplicando ambos os lados da primeira equação em (\ref{eq:problema-dirichlet-motivacao}) por uma função $\phi \in \cC^{\infty}_C(\Omega)$ e integrando sobre $\Omega$, obtemos
\[
    \int_\Omega - \phi \Delta u \,dx + \int_\Omega u \phi \,dx = \int_\Omega f \phi \,dx.
\]
Note que, utlizando integração por partes (ver Teorema \ref{thm:integracao-por-partes}), a primeira integral acima pode ser reescrita como
\[
    \int_\Omega -\phi \Delta u \,dx = -\sum_{i=1}^n \int_\Omega \phi \frac{\partial^2 u}{\partial x_i^2} \,dx = -\sum_{i=1}^n \left( \int_{\partial\Omega} \phi \frac{\partial u}{\partial x_i} \nu_i \,dS - \int_\Omega \frac{\partial \phi}{\partial x_i} \frac{\partial u}{\partial x_i} \,dx\right).
\]
Mas, como $\phi$ se anula sobre $\partial \Omega$, inferimos
\[
    \int_\Omega -\phi \Delta u \,dx = \sum_{i=1}^n \int_\Omega \frac{\partial \phi}{\partial x_i} \frac{\partial u}{\partial x_i} \,dx = \int_{\Omega} \nabla u \cdot \nabla \phi \,dx.
\]
Dito isso, dizemos que $u$ é uma solução fraca do problema de Dirichlet se
\begin{equation} \label{eq:sol-fraca-dirichlet}
    \int_{\Omega} \nabla u \cdot \nabla \phi \,dx + \int_\Omega u\phi \,dx = \int_\Omega f \phi \,dx,
\end{equation}
para toda função $\phi \in \cC^{\infty}_C(\Omega)$\footnote{Nomenclatura: $\phi$ é chamada função teste.}.
Observe agora que, não precisamos mais que $u$ seja de classe $\cC^2$ já que a segunda derivada de $u$ não é utilizada em (\ref{eq:sol-fraca-dirichlet}). Na verdade, não precisamos nem que $u$ seja contínua, apenas integrável em $\Omega$.
Na Seção \ref{sec:dirichlet}, voltaremos a essa motivação para mostrar que para qualquer função $f \in \cL^2(\Omega)$, existe uma única solução fraca para (\ref{eq:problema-dirichlet-motivacao}). 
Essa solução é uma função que pertece ao espaço de Sobolev $H^1_0(\Omega)$ (esse e outros espaços e suas propriedades serão estudadas ao longo desse trabalho).


\section{Espaços de Sobolev $\cW^{k,p}(\Omega)$}

\begin{figure}
    \centering
    \includegraphics[height=3cm]{sobolev.jpg}
    \caption{Sergei Lvovich Sobolev (1908 -- 1989)}
\end{figure}

Nosso objetivo agora, é generalizar a noção de derivada para funções que não são diferenciáveis em um aberto $\Omega$ do $\bR^n$ e explorar algumas propriedades elementares.

Seja $u \in \cC^1(\Omega)$, então se $\phi \in \cC^\infty_c(\Omega)$, utilizando integração por partes (ver Teorema \ref{thm:integracao-por-partes}), temos que
\[
    \int_\Omega u \dfrac{\partial \phi}{\partial x_i} \,dx = \int_{\partial\Omega} u\phi \nu_i \, dS- \int_\Omega \phi \dfrac{\partial u}{\partial x_i} \,dx,
\]
para todo $i = 1,\dots,n$. Como $\phi$ tem suporte compacto e $\Omega$ é um aberto, segue que $\phi$ se anula sobre $\partial\Omega$, como foi mostrado abaixo da definição de suporte. Portanto, a expressão acima se torna
\begin{equation} \label{eq:derivada-fraca-xi}
    \int_\Omega u \dfrac{\partial \phi}{\partial x_i} \,dx = - \int_\Omega \phi \dfrac{\partial u}{\partial x_i} \,dx,
\end{equation}
para todo $i = 1,\dots,n$.
Ademais, se agora $u$ for de classe $\cC^k$ em $\Omega$, com $k \in \bN$ e $\alpha = (\alpha_1,\dots,\alpha_n) \in \bN^n$ um multi-índice de ordem $|\alpha| = \alpha_1 + \cdots + \alpha_n = k$, então
\begin{equation} \label{eq:derivada-fraca-dalpha}
    \int_\Omega u D^\alpha \phi \,dx = (-1)^{|\alpha|} \int_\Omega \phi D^\alpha u \,dx.
\end{equation}
Essa expressão é válida, já que
\[
    D^\alpha = \dfrac{\partial^{\alpha_1} }{\partial x_1^{\alpha_1}} \cdots \dfrac{\partial^{\alpha_n} }{\partial x_n^{\alpha_n}} = \dfrac{\partial^{|\alpha|} }{\partial x_1^{\alpha_1} \cdots \partial x_n^{\alpha_n}},
\]
e podemos aplicar (\ref{eq:derivada-fraca-xi}) $|\alpha|$ vezes.

Queremos descobrir se existe uma classe de funções tal que (\ref{eq:derivada-fraca-dalpha}) ainda é válida, mesmo se $u$ não for de classe $\cC ^k$. Note que o lado esquerdo de (\ref{eq:derivada-fraca-dalpha}) está bem definido se $u \in \cL^1_{\mathrm{loc}}(\Omega)$.
O problema é que se $u$ não é necessáriamente uma função de classe $\cC^k$ então, o lado direito de (\ref{eq:derivada-fraca-dalpha}) não está bem definido. Para resolver isso, perguntamos se existe uma função $v \in \cL^1_{\mathrm{loc}}(\Omega)$ tal que (\ref{eq:derivada-fraca-dalpha}) é válida quando substituimos $D^\alpha u$ por $v$.

Essa pergunta motiva a definição abaixo.

\begin{dbox}[Derivada fraca]
    Sejam $u,v \in \cL^1_{\mathrm{loc}}(\Omega)$ e $\alpha$ um multi-índice. Dizemos que $v$ é a $\alpha$-ésima derivada parcial fraca de $u$, denotada por
    \[
        v = D^\alpha u,
    \]
    dado que
    \begin{equation} \label{eq:derivada-fraca}
        \int_\Omega u D^\alpha \phi \,dx = (-1)^{|\alpha|} \int_\Omega \phi D^\alpha u\,dx,
    \end{equation}
    para toda função $\phi \in \cC ^\infty_c(\Omega)$.
\end{dbox}

Isto é, se dada uma função $u$ e existe uma função $v$ que satisfaz (\ref{eq:derivada-fraca}) para toda $\phi$ função teste, dizemos que $D^\alpha u = v$ no sentido fraco.
Caso contrário, se não existir uma função $v$ que satisfaz (\ref{eq:derivada-fraca}), então $u$ não possui a $\alpha$-ésima derivada parcial fraca.

\obs Aqui, utlizamos a notação $dx$ ao invés de $d\mu$ na integral de Lebesgue como convenção para dar ênfase que estamos utilizando a medida de Lebesgue (e não uma medida qualquer).
Além disso, se uma função é integrável a Riemann e a Lebesgue (utlizando a medida de Lebesgue), suas integrais coincidem.

Agora, vejamos alguns exemplos sobre a derivada fraca.

\begin{ex} \label{ex:derivada-fraca-R}
    A função $u : \Omega  =(0,2) \to \bR$ dada por
    \begin{equation} \label{eq:uxexemplo1}
        u(x) = \left\{
            \begin{array}{rl}
                x, & \!\text{se }\; 0 < x \leqslant 1;\\
                1, & \!\text{se }\; 1 < x < 2,\\
            \end{array}
        \right.
    \end{equation}
    não é diferenciável no sentido usual, já que
    \[
        \lim_{h\to 0^-} \frac{u(1 + h) - u(1)}{h} = \lim_{h \to 0^-} \frac{1 + h -1}{h} = 1,
    \]
    mas
    \[
        \lim_{h\to 0^+} \frac{u(1 + h) - u(1)}{h} = \lim_{h \to 0^+} \frac{1 -1}{h} = 0.
    \]
    Porém, $u$ possui derivada fraca dada pela função
    \[
        v(x) = \left\{
            \begin{array}{rl}
                1, & \text{se }\; 0 < x \leqslant 1;\\
                0, & \text{se }\; 1 < x < 2.\\
            \end{array}
        \right.
    \]
    Com efeito, note que, para toda função $\phi \in \cC^\infty_c(\Omega)$, temos que
    \[
        \int_0^2 u \phi' \,dx = \int_0^1 x \phi' \,dx + \int_1^2 \phi' \,dx = x \phi \bigg|_0^1 - \int_0^1 \phi \,dx + \phi \bigg|_1^2.
    \]
    Como $\phi$ tem suporte compacto, $\phi(0) = \phi(2) = 0$. Assim,
    \[
        \int_0^2 u \phi' \,dx = \phi(1) - \int_0^1 \phi \,dx - \phi(1) = -\int_0^1 \phi \,dx.
    \]
    Por fim, basta escrever $0$ como uma integral para obter
    \[
        \int_0^2 u \phi' \,dx = - \left(  \int_0^1 1\phi \, dx + \int_1^2 0\phi \,dx  \right) = -\int_0^2 v \phi \,dx.
    \]
    Portanto, $v$ é a derivada de $u$ no sentido fraco.
\end{ex}

\begin{ex}
    A função $u : \Omega  = (0,2) \to \bR$ dada por
    \[
        u(x) = \left\{
            \begin{array}{rl}
                x, & \!\text{se }\; 0 < x \leqslant 1;\\
                2, & \!\text{se }\; 1 < x < 2,\\
            \end{array}
        \right.
    \]
    não possuí derivada fraca.
    Mostremos que $u'$ não existe no sentido fraco.
    Isto significa que não existe uma função $v \in \cL^1_\loc(\Omega)$ que satisfaz
    \[
        \int_\Omega u \phi' \, dx = -\int_\Omega v \phi \,dx,
    \]
    para toda função $\phi \in \cC_c^\infty(\Omega)$. 
    Com efeito, suponha, por absurdo, que existe tal $v$. Deste modo,
    \[
        -\int_0^2 v \phi \, dx = \int_0^2 u \phi' \,dx = \int_0^1 x \phi' \,dx + 2\int_1^2 \phi' \,dx =- \int_0^1 \phi \,dx - \phi(1).
    \]
    Seja $(\phi_n)$ uma sequência de funções suaves satisfazendo
    \[
        0 \leqslant \phi_n \leqslant 1, \quad \phi_n(1) = 1, \quad \phi_n(x) \to 0, \text{ quando } n\to\infty,\text{ se } x \neq 1.
    \]
    Isolando $\phi(1)$, substituindo $\phi$ por $\phi_n$ e fazendo $n \to \infty$, obtemos
    \[
        1 = \lim \phi_n(1) = \lim \left[ \int_0^2 v \phi_n \, dx- \int_0^1 \phi_n \,dx \right] = 0,
    \]
    pelo Teorema da Convergência Dominada (ver Teorema \ref{thm:teorema-da-convergencia-dominada}), pois $\phi_n \to 0$ qtp em $\Omega$, quando $n\to\infty$ e $|\phi_n(x)| \leqslant 1$, para todo todo $x \in \Omega$ e $v \in \cL^1_\loc(\Omega)$.
    Portanto, $u$ não possui derivada fraca.
\end{ex}

O primeiro resultado sobre a derivada fraca que queremos mostrar é sobre sua unicidade, para isso precisamos antes do seguinte lema.

\begin{lbox} \label{lm:lema}
    Sejam $u \in \cL^1_{\loc}(\Omega)$ e $\phi \in \cC^\infty(\Omega)$.
    Então,
    \[
        \int_\Omega u \phi = 0,
    \]
    se, e somente se $u \equiv 0$ qtp em $\Omega$.
\end{lbox}
\begin{prf}
    Ver \cite{brezis-functional.analysis}, p.p. 110.
\end{prf}

Com o lema acima em mente, temos todas as ferramentas para mostrar a unicidade da derivda fraca.

\begin{pbox} \label{pr:unicidade}
    Seja $\alpha$ um multi-índice. Se $v$ e $\tilde v$ são ambas $\alpha$-ésimas derivadas parciais fracas de uma função $u$.
    Então,
    \[
        v = \tilde v \text{ qtp em } \Omega.
    \]
\end{pbox}
\begin{prf}
    Sejam $v, \tilde v \in \cL^1_{\mathrm{loc}}(\Omega)$ tais que
    \[
        \int_\Omega u D^\alpha \phi \,dx = (-1)^{|\alpha|} \int_\Omega v \phi \,dx \;\text{ e }\; \int_\Omega u D^\alpha \phi \, dx= (-1)^{|\alpha|}\int_\Omega \tilde v \phi \,dx,
    \]
    para toda $\phi \in \cC^\infty_c(\Omega)$. Com isso, podemos escrever
    \[
        \int_\Omega (v - \tilde v) \phi \, dx = 0,
    \]
    para todo $\phi \in \cC^{\infty}_c(\Omega)$.
    Logo, pelo Lema \ref{lm:lema}, chegamos a
    \[
        v - \tilde v = 0 \text{ qtp em } \Omega.
    \]
    Portanto, $v = \tilde v$ qtp em $\Omega$.
\end{prf}

Vejamos um exemplo prático da Proposição \ref{pr:unicidade}.

\begin{ex}
    Considere a função $u$ do Exemplo \ref{ex:derivada-fraca-R}, vimos que
    \[
        v(x) = \left\{
            \begin{array}{rl}
                1, & \!\text{se }\; 0 < x \leqslant 1;\\
                0, & \!\text{se }\; 1 < x < 2,\\
            \end{array}
        \right.
    \]
    é a derivada de $u$ (dada em \ref{eq:uxexemplo1}) no sentido fraco.
    Porém, a função
    \[
        \tilde v(x) = \left\{
            \begin{array}{rl}
                1, & \text{se }\; 0 < x < 1;\\
                0, & \text{se }\; 1 \leqslant x < 2,\\
            \end{array}
        \right.
    \]
    também satisfaz a definição de derivada fraca de $u$.
    A primeira vista, temos a sensação de que essa função é um contra-exemplo para unicidade da derivada fraca, todavia $v$ e $\tilde v$ são iguais fora de um conjunto de medida nula.
    De fato,
    \[
        (v - \tilde v)(x) = \left\{
            \begin{array}{rl}
                1, & \text{se }\; x = 1;\\
                0, & \text{se }\; x \neq 1,\\
            \end{array}
        \right.
    \]
    Portanto, é verdade que
    \[
        v = \tilde v \text{ qtp em } (0,2),
    \]
    pois $\{1\}$ é finito, logo tem medida nula.\footnote{Todo conjunto enumerável (em particular finito) tem medida (de Lebesgue) nula.}
\end{ex}

Com a definição de derivada fraca estabelecida, podemos definir os espaços de Sobolev $\cW^{k,p}(\Omega)$.

\begin{dbox}
    Sejam $\Omega \subseteq \bR^n$ um aberto e $1 \leqslant p \leqslant \infty$. 
    Definimos o espaço de Sobolev $\cW^{1,p}(\Omega)$ por
    \[
        \cW^{1,p}(\Omega) = \{u \in \cL^p(\Omega) \,; D^{e_i} u \in \cL^p(\Omega), \text{ para todo } i = 1,\dots,n\},
    \]
    onde $e_i = (0,\dots,0,1,0,\dots,0)$ é o vetor com $1$ na $i$-ésima cordenada e $0$ caso contrário.
\end{dbox}

Existem duas formas de definir os espaços de Sobolev $\cW^{k,p}(\Omega)$, com $k \in \bN$, indutivamente, e pela derivada fraca.

\begin{dbox}
    Sejam $\Omega \subseteq \bR ^n$ um aberto, $k \in \bN$ e $1 \leqslant p \leqslant \infty$. Definimos o espaço de Sobolev $\cW^{k,p}(\Omega)$ por
    \[
        \cW^{k,p}(\Omega) = \left\{ u \in \cL^p(\Omega) \,; D^\alpha u \in \cL^p(\Omega), \text{ para todo multi-índice } \alpha \text{ com } |\alpha| \leqslant k\right\}.
    \]
    Ou, de outra forma,
    \[
        \cW^{k,p}(\Omega) = \left\{ u \in \cW^{k-1,p}(\Omega) \,; D^\alpha u \in \cW^{k-1,p}(\Omega) , \text{ para todo multi-índice } \alpha \text{ com } |\alpha| = 1\right\},
    \]
    onde $D^\alpha u$ é a $\alpha$-ésima derivada parcial de $u$ no sentido fraco.
\end{dbox}

\obs Definimos também o espaço $\cW^{k,p}_\loc(\Omega)$ por
\[
    \cW^{k,p}_\loc(\Omega) = \left\{ u \in \cL^p_\loc(\Omega) \,; D^\alpha u \in \cL^p_\loc(\Omega), \text{ para todo multi-índice } \alpha \text{ com } |\alpha| \leqslant k\right\}.
\]

\obs Quando $p = 2$, a notação $H^{k}(\Omega)$ é comumente utilizada para dar ênfase que o espaço $\cW^{k,2}(\Omega)$ é um espaço de Hilbert\footnote{Espaço vetorial com produto interno completo.}, munido do produto interno
\[
    \left\langle u, v\right\rangle _{H^k(\Omega)} = \sum_{|\alpha| \leqslant k} \left\langle D^\alpha u, D^\alpha v\right\rangle _{\cL^2(\Omega)},
\]
onde
\[
    \left\langle D^\alpha u, D^\alpha v\right\rangle _{\cL^2(\Omega)} = \int_\Omega D^\alpha u D^\alpha v \,dx.
\]
No próximo capítulo, estudaremos algumas aplicações dos espaços de Sobolev, e os espaços $H^k(\Omega)$ serão utilizados.

\begin{dbox}
    O espaço $\cW^{k,p}(\Omega)$ admite norma
    \[
        \Vert u \Vert_{\cW^{k,p}(\Omega)} = \left( \sum_{|\alpha| \leqslant k} \Vert D^\alpha u \Vert_{\cL^p(\Omega)}^p \right)^{\frac{1}{p}},
    \]
    para $1 \leqslant p < \infty$ e 
    \[
        \Vert u \Vert_{\cW^{k,\infty}(\Omega)} 
        % = \sum_{|\alpha| \leqslant k} \esssup\!_\Omega |D^\alpha u| 
        = \sum_{|\alpha| \leqslant k} \Vert D^{\alpha}u \Vert_{\cL^\infty(\Omega)}.
    \]
\end{dbox}

\obs Dizemos que uma sequência $(u_n)$ converge para $u$ em $\cW^{k,p}(\Omega)$ se
\[
    \lim \Vert u_n - u \Vert_{\cW^{k,p}(\Omega)} = 0,
\]
e denotamos por $u_n \to u \text{ em } \cW^{k,p}(\Omega)$ quando $n \to \infty$.
Além disso, dizemos que $(u_n)$ converge para $u$ em $\cW^{k,p}_\loc(\Omega)$ se $u_n$ converge para $u$ em $\cW^{k,p}(\Omega')$ para todo conjunto aberto $\Omega'$ compactamente contido em $\Omega$, isto é $\Omega' \subseteq \Omega$ e $\overline{\Omega'}$ é compacto. Essa inclusão será denotada por $\Omega' \Subset \Omega$.

Ainda não temos todas as ferramentas necessárias para provar que as normas da definição anterior são de fato normas em $\cW^{k,p}(\Omega)$, com $1 \leqslant p \leqslant \infty$. Isso será feito após o Teorema \ref{thm:propriedades-derivada-fraca} sobre as propriedades da derivada fraca, que são necessárias para verificar que $\Vert \cdot \Vert_{\cW^{k,p}(\Omega)}$ satisfaz a definição de norma.

Um outro espaço importante é o espaço $\cW^{k,p}_0(\Omega)$ ($H^k_0(\Omega)$ se $p = 2$) que é definido como o fecho de $\cC^{\infty}_c(\Omega)$ em $\cW^{k,p}(\Omega)$. Ou seja, dada uma função em $u \in \cW^{k,p}_0(\Omega)$, existe uma sequência $(u_n)$ em $\cC^{\infty}_c(\Omega)$ tal que $u_n \to u$ em $\cW^{k,p}(\Omega)$, quando $n \to \infty$.
O Teorema \ref{thm:traco-2} mostra uma equivalência para as funções nesse espaço.

\obs Essa não é a única forma de definir uma norma em $\cW^{k,p}(\Omega)$, a norma que definimos acima é equivalente, por exemplo, à norma
\[
    \sum_{|\alpha| \leqslant k} \Vert D^\alpha u \Vert_{\cL^p(\Omega)},
\]
com $1 \leqslant p < \infty$, e a norma $\Vert \cdot \Vert_{\cW^{k,\infty}(\Omega)}$ é equivalente a
\[
    \max_{|\alpha|\leqslant k} \Vert D^\alpha u \Vert_{\cL^\infty(\Omega)}.
\]

O próximo exemplo ilustra um caso em que uma função pode ou não possuir derivada fraca dependendo da dimensão $n$ do espaço Euclidiano e do expoente de integração $p$.

\begin{ex}
    Seja $\Omega = B(0,1) \subseteq \bR^n$ a bola aberta de raio 1 centrada na origem, e considere $u : \Omega \to \bR$ dada por
    \begin{equation} \label{eq:po}
        x \mapsto \Vert x \Vert^{-\alpha},
    \end{equation}
    onde $\alpha > 0$. Queremos verificar para quais valores de $\alpha > 0$, $n$ e $p$ a função $u$ pertence ao espaço $\cW^{1,p}(\Omega)$.
    Primeiramente, note que $u$ é suave fora de $0$ com
    \[
        \dfrac{\partial u}{\partial x_i} = \frac{-\alpha x_i}{\Vert x \Vert^{\alpha + 2}} \quad (x \neq 0),
    \]
    logo, como
    \[
        Du(x) = \left(\frac{-\alpha x_1}{\Vert x \Vert^{\alpha + 2}},\dots,\frac{-\alpha x_n}{\Vert x \Vert^{\alpha + 2}}  \right) \quad (x \neq 0),
    \]
    segue que
    \[
        \Vert Du(x) \Vert = \frac{|\alpha|}{\Vert x \Vert^{\alpha + 1}} \quad (x \neq 0).
    \]
    Seja $\phi \in \cC^\infty_c(\Omega)$ e fixe $\varepsilon > 0$. Por integração por partes (ver Teorema \ref{thm:integracao-por-partes}), temos que
    \begin{equation} \label{eq:op}
        \int_{\Omega \setminus B[0, \varepsilon]} u \dfrac{\partial \phi}{\partial x_i} \, dx = -\int_{\Omega \setminus B[0, \varepsilon]} \phi \dfrac{\partial u}{\partial x_i} \,dx + \int_{\partial B[0,\varepsilon]} u \phi \nu_i \,dS,
    \end{equation}
    onde $\nu = (\nu_1,\dots,\nu_n)$ denota o vetor normal unitário que aponta para fora em $\partial B[0,\varepsilon]$ (bola fechada de raio $\varepsilon > 0  $ centrada na origem).
    Agora se $\alpha + 1 < n$, $\Vert Du \Vert \in \cL^1(\Omega)$.
    De fato, integrando $\Vert Du \Vert$ sobre $\Omega$, concluímos que
    \[
        \int_\Omega \Vert Du \Vert \,dx = |\alpha|\int_{B(0,1)} \frac{1}{\Vert x \Vert^{\alpha+1}} \,dx.
    \]
    Transformando em coordenadas polares (ver Teorema \ref{thm:coordenadas-polares}), conseguimos simplificar essa integral da seguinte forma:
    \[
        \int_{B(0,1)} \frac{1}{\Vert x \Vert^{\alpha+1}} \,dx =  \int_0^1 \int_{\Vert x \Vert = r} \frac{1}{\Vert x \Vert^{\alpha + 1}} \,dS dr = \int_0^1 \int_{\Vert x \Vert= r} \frac{1}{r^{\alpha + 1}} \, dS dr = \int_0^1 \frac{1}{r^{\alpha+1}}  \int_{\Vert x \Vert = r} dS dr,
    \]
    onde a integral de superficie acima, é igual a área da esfera $n$-dimensional de raio $r$ dada por
    \[
        \frac{2\pi^{\frac{n}{2}}}{\Gamma(\frac{n}{2})}r^{n-1},
    \]
    onde $\Gamma$ e a função gama (ver \cite{artin-gamma} para mais detalhes), a qual, por simplicidade, vamos denotar por $\sigma(n) r^{n-1}$. Dessa forma, chegamos a
    \[
        \int_0^1 \frac{1}{r^{\alpha+1}}  \int_{\Vert x \Vert = r} dS dr = \sigma(n)\int_0^1 r^{n-\alpha-2} \,dr = \sigma(n)\left(\frac{1^{n-\alpha-1}}{n-\alpha -1} - \frac{0^{n-\alpha-1}}{n-\alpha-1}\right).
    \]
    Note que, se $n - \alpha - 1 < 0$ então $0^{n-\alpha-1} = \infty$. Sendo assim, deduzimos que
    \[
        \int_\Omega \Vert Du \Vert \, dx < \infty \iff \alpha + 1 < n.
    \]
    Portanto, $\Vert Du \Vert \in \cL^1(\Omega)$ desde que $\alpha + 1 < n$.
    Nesse caso, inferimos, por (\ref{eq:po}), que
    \[
        \left| \int_{\partial B[0,\varepsilon]} u \phi \nu_i \,dS \right| \leqslant 
        % \int_{\partial B[0,\varepsilon]} |u \phi \nu_i| \,dS = 
        \int_{\partial B[0,\varepsilon]} |u| |\phi| |\nu_i| \,dS \leqslant \Vert \phi \Vert_{\cL^\infty(\Omega)}\int_{\partial B[0,\varepsilon]} |u|dS \leqslant \Vert \phi \Vert_{\cL^\infty(\Omega)}\int_{\partial B[0,\varepsilon]} \varepsilon^{-\alpha} \,dS,
    \]
    pois $|\nu_i| \leqslant \Vert \nu \Vert = 1$. Essa última integral pode ser calculada por meio de coordenadas polares de forma análoga ao que foi feito anteriormente, resultando em
    \[
        \left| \int_{\partial B[0,\varepsilon]} u \phi \nu_i \,dS \right| \leqslant c \varepsilon^{n-1-\alpha} \to 0,
    \]
    quando $\varepsilon \to 0$. Portanto, por (\ref{eq:op}), deduzimos que
    \[
        \int_{\Omega} u \dfrac{\partial \phi}{\partial x_i} \, dx = -\int_{\Omega} \phi \dfrac{\partial u}{\partial x_i} \,dx,
    \]
    para toda $\phi \in \cC^\infty_c(\Omega)$, desde que $0 < \alpha < n-1$. 

    Além disso, $\Vert Du \Vert \in \cL^p(\Omega)$ se, e somente se, $(\alpha + 1)p < n$, esse cálculo é feito de forma análoga ao que foi feito para verificar quando $\Vert Du \Vert \in \cL^1(\Omega)$. Consequentemente, $u \in \cW^{1,p}(\Omega)$ se, e somente se $\alpha < \frac{n-p}{p}$.
    Em particular, $u \not\in \cW^{1,p}(\Omega)$ quando $p \geqslant n$.
\end{ex}

Agora, vamos explorar as propriedades elementares das derivadas fracas.

\begin{tbox}[Propriedades da derivada fraca] \label{thm:propriedades-derivada-fraca}
    Sejam $u, v \in \cW^{k,p}(\Omega)$ e $\alpha$ um multi-índice com $1 \leqslant |\alpha| \leqslant k$.
    Então,
    \begin{enumerate}[leftmargin=*, label=\textbf{(\alph*)}]
        \item $D^\beta(D^\alpha u) = D^\alpha (D^\beta u) = D^{\alpha + \beta} u$ para todos multi-índices $\alpha$ e $\beta$ com $|\alpha| + |\beta| \leqslant k$;
        \item $D^\alpha u \in \cW^{k - |\alpha|,p}(\Omega)$;
        \item \textbf{(Linearidade)} para todo $\lambda \in \bR$, tem-se que $\lambda u + v \in \cW^{k,p}(\Omega)$ e
        \[
            D^{\alpha}(\lambda u + v) = \lambda D^\alpha u + D^\alpha v;
        \]
        \item se $\Omega'$ é um aberto de $\Omega$, então $u \in \cW^{k,p}(\Omega')$;
        \item \textbf{(Regra de Leibniz)} se $\eta \in \cC^\infty_c(\Omega)$, então $\eta u \in \cW^{k,p}(\Omega)$ e
        \begin{equation} \label{eq:regra-de-leibniz}
            D^\alpha (\eta u) = \sum_{\sigma \leqslant \alpha} \binom{\alpha}{\sigma} D^{\sigma} \eta D^{\alpha - \sigma} u,
        \end{equation}
        onde
        \[
            \binom{\alpha}{\sigma} = \frac{\alpha!}{\sigma!(\alpha - \sigma)!}, \quad \alpha! = \alpha_1!\cdots \alpha_n!
        \]
        e $\sigma \leqslant \alpha$ significa $\sigma_j \leqslant \alpha_j$, para todo $j = 1,\dots,n$.
    \end{enumerate}
\end{tbox}
\begin{prf}
    ~

    \textbf{(a)} Mostremos que $D^{\beta}D^{\alpha} u = D^{\alpha + \beta}u$ (a demonstração para $D^{\alpha}D^{\beta} u$ é análoga).
    Com efeito, é verdade que
    \[
        \int_\Omega D^{\alpha} u D^{\beta} \phi \, dx = (-1)^{|\alpha|} \int_\Omega u D^{\alpha} D^{\beta} \phi \, dx = (-1)^{|\alpha|} \int_\Omega u D^{\alpha + \beta} \phi \,dx,
    \]
    para toda $\phi \in \cC^{\infty}_c(\Omega)$. Note que a última igualdade é válida pelo fato de $\phi$ ser uma função infinitamente diferenciável e $D^\alpha$ e $D^\beta$ serem derivadas parciais usuais.
    Dessa forma $D^{\beta}D^{\alpha} \phi = D^{\alpha + \beta} \phi$.
    Dito isso, utilizando a definição de derivada fraca, obtemos
    \[
        \int_\Omega D^{\alpha} u D^{\beta} \phi \, dx = (-1)^{|\alpha|}(-1)^{|\alpha| + |\beta|} \int_\Omega \phi D^{\alpha+\beta} u \,dx = (-1)^{|\beta|} \int_\Omega \phi D^{\alpha + \beta} u \,dx.
    \]
    Portanto, $D^{\beta} D^{\alpha} u = D^{\alpha + \beta} u$ no sentido fraco.

    \textbf{(b)} Suponha que $D^\alpha u \not\in \cW^{k-|\alpha|,p}(\Omega)$, então existe um multi-índice $\beta$, com $|\beta| \leqslant k - |\alpha|$, tal que $D^{\beta}(D^{\alpha}u) \not\in \cL^p(\Omega)$.
    Pelo item anterior temos que $D^{\alpha+\beta}u \not\in \cL^p(\Omega)$, o que é uma contradição, pois por hipótese $D^{\gamma}u \in \cL^p(\Omega)$ para todo multi-índice $\gamma$ com $|\gamma| \leqslant k$ e, como $|\beta| \leqslant k - |\alpha|$, tem-se $|\gamma| = |\alpha| + |\beta| \leqslant k$ (para $\gamma = \alpha + \beta$).
    Portanto, $D^{\alpha}u \in \cW^{k-|\alpha|,p}(\Omega)$.

    \textbf{(c)} Note que
    \[
        \int_\Omega (\lambda u + v) D^{\alpha}\phi \,dx = \int_\Omega \lambda u D^\alpha \phi \, dx + \int_\Omega v D^\alpha \phi \, dx = \lambda \int_\Omega u D^{\alpha} \phi \,dx + \int_\Omega v D^{\alpha} \phi \, dx.
    \]
    Utilizando a definição de derivada fraca nas duas útlimas integrais acima, obtemos
    {\small
    \[
        \lambda \int_\Omega u D^{\alpha} \phi \,dx + \int_\Omega v D^{\alpha} \phi \, dx = \lambda (-1)^{|\alpha|} \int_\Omega \phi D^{\alpha} u \, dx + (-1)^{|\alpha|}\int_\Omega \phi D^{\alpha} v \, dx = \int_\Omega (\lambda D^{\alpha}u + D^{\alpha}v )\phi \,dx,
    \]}\!
    para todo $\phi \in \cC^{\infty}_c(\Omega)$. Portanto, $D^{\alpha}(\lambda u + v) = \lambda D^\alpha u + D^\alpha v$ no sentido fraco.

    \textbf{(d)} Seja $\Omega' \subseteq \Omega$ um aberto, queremos verificar que $u \in \cW^{k,p}(\Omega')$.
    De fato, basta verificar que as integrais
    \[
        \int_{\Omega'} |u|^p \, dx \;\text{ e }\; \int_{\Omega'} |D^{\alpha}u|^p \, dx
    \]
    são finitas, para todo $\alpha$ multi-índice com $|\alpha| \leqslant k$. De fato, é verdade que
    \[
        \int_{\Omega'} |u|^p \, dx \leqslant \int_{\Omega} |u|^p \,dx < \infty,
    \]
    e
    \[
        \int_{\Omega'} |D^\alpha u|^p \, dx \leqslant \int_{\Omega} |D^\alpha u|^p \,dx < \infty,
    \]
    para todo $\alpha$ multi-índice com $|\alpha| \leqslant k$, ambas pelo fato de $u \in \cW^{k,p}(\Omega)$.
    Assim, $u \in \cW^{k,p}(\Omega')$.

    \textbf{(e)} Para mostrar que (\ref{eq:regra-de-leibniz}) é válida, utilizaremos indução sobre $|\alpha|$. 
    
    Com efeito, para $|\alpha|= 1$, como $\eta,\phi \in \cC^{\infty}_c(\Omega)$, temos que
    \[
        D^{\alpha} (\eta \phi) = \phi D^{\alpha}\eta + \eta D^{\alpha} \phi.
    \]
    Dessa forma,
    \[
        \int_\Omega \eta u D^{\alpha} \phi \, dx = \int_\Omega u D^{\alpha} (\eta \phi) - \int_\Omega u\phi D^{\alpha} \eta \, dx.
    \]
    Como $\eta$ e $\phi$ têm suporte compacto, então $\eta\phi$ também tem.
    De fato, é fácil ver que
    \[
        \{x \in \Omega \,; \eta(x)\phi(x) \neq 0\} \subseteq \{x \in \Omega \,; \eta(x) \neq 0\} \cap \{x \in \Omega \,; \phi(x) \neq 0\},
    \] 
    o que implica em
    \[
        \supp(\eta\phi) \subseteq \supp \eta \cap \supp \phi.
    \]
    Portanto $\supp(\eta\phi)$ é comapcto, desde que $\supp \eta$ e $\supp \phi$ tambem sejam.
    Dito isso, utilizando a definição de derivada fraca apenas na primeira integral do lado direito da igualdade acima, chegamos a
    \[
        \int_\Omega \eta u D^{\alpha} \phi \, dx = -\int_\Omega \eta \phi D^{\alpha} u \, dx - \int_\Omega u\phi D^{\alpha} \eta \, dx = -\int_\Omega (\eta D^{\alpha}u + uD^{\alpha}\eta) \phi \,dx.
    \]
    Portanto, $D^{\alpha}(\eta u) = \eta D^{\alpha}u + uD^{\alpha}\eta$ como queriamos mostrar.

    Agora seja $m \in \bN$ tal que $m < k$ e suponha que $(\ref{eq:regra-de-leibniz})$ é válida para todo multi-índice $\alpha$ com $|\alpha| \leqslant m$ e toda função teste $\eta$.
    Considere um multi-índice $\alpha$ com $|\alpha| = m + 1$.
    Então, $\alpha$ é da forma $\alpha = \beta + \gamma$ com $|\beta| = m$ e $|\gamma| = 1$. 
    Deste modo, podemos escrever por \textbf{(a)} que
    \[
        \int_\Omega \eta u D^{\alpha}\phi \, dx = \int_\Omega \eta u D^{\beta + \gamma}\phi \, dx = \int_\Omega \eta u D^\beta(D^\gamma \phi) \, dx = (-1)^{|\beta|} \int_\Omega D^{\beta} (\eta u) D^{\gamma}\phi \,dx,
    \]
    para todo $\phi \in \cC^{\infty}_c(\Omega)$.
    Como $|\beta| = m$ podemos utilizar a hipótese de indução em $D^{\beta}(\eta u)$ e a definição da $\gamma$-ésima derivada fraca. Assim, por \textbf{(c)}, deduzimos que
    {\small
    \[
        \int_\Omega \eta u D^{\alpha}\phi \, dx = (-1)^{|\beta|} \int_\Omega \sum_{\sigma \leqslant \beta} \binom{\beta}{\sigma} D^{\sigma} \eta D^{\beta - \sigma} u D^{\gamma}\phi \, dx = (-1)^{|\beta| + |\gamma|} \int_\Omega \bigg[\sum_{\sigma \leqslant \beta} \binom{\beta}{\sigma} D^{\gamma}(D^{\sigma} \eta D^{\beta - \sigma} u)\bigg] \phi \, dx.
    \]}\!
    Além disso, como $|\gamma| = 1$, podemos aplicar a regra de Leibniz novamente, obtendo
    \begin{equation} \label{eq:integral-etau}
        \int_\Omega \eta u D^{\alpha}\phi \, dx = (-1)^{|\alpha|} \int_\Omega \bigg[\sum_{\sigma \leqslant \beta} \binom{\beta}{\sigma} \left(D^{\rho}\eta D^{\alpha - \rho} u + D^{\sigma} \eta D^{\alpha - \sigma} u\right)\bigg] \phi \,dx,
    \end{equation}
    onde $\rho = \sigma + \gamma$. 
    Note que podemos escrever o somatório acima da seguinte forma:
    \begin{equation*} \label{eq:somatorio}
        \sum_{\gamma \leqslant \rho \leqslant \alpha} \binom{\alpha - \gamma}{\rho - \gamma} D^{\rho}\eta D^{\alpha - \rho} u + \sum_{0 \leqslant\sigma \leqslant \beta} \binom{\alpha - \gamma}{\sigma} D^{\sigma} \eta D^{\alpha - \sigma} u,
    \end{equation*}
    o qual ainda pode ser expandido em quatro somatórios como abaixo:
    \begin{equation} \label{eq:somatorio-2}
        \begin{aligned}
            \sum_{\gamma \leqslant \rho \leqslant \beta} \binom{\alpha - \gamma}{\rho - \gamma} D^{\rho}\eta D^{\alpha - \rho} u &+ \sum_{\beta < \rho \leqslant \alpha} \binom{\alpha - \gamma}{\rho - \gamma} D^{\rho}\eta D^{\alpha - \rho} u \\
            &+ \sum_{0 \leqslant \sigma < \gamma} \binom{\alpha - \gamma}{\sigma} D^{\sigma} \eta D^{\alpha - \sigma} u + \sum_{\gamma \leqslant \sigma \leqslant \beta} \binom{\alpha - \gamma}{\sigma} D^{\sigma} \eta D^{\alpha - \sigma} u.
           \end{aligned}
    \end{equation}
    Porém, note que $0 \leqslant \sigma < \gamma$ implica em $\sigma = 0$. Com efeito, $0 \leqslant \sigma$ significa que $0 \leqslant \sigma_i$ para todo $i = 1,\dots,n$ e $\sigma < \gamma$ significa que existe um $j = 1,\dots,n$ tal que $\sigma_j < \gamma_j$. Como $|\gamma| = 1$, suponha, sem perda de generalidade, que $\gamma = e_1 = (1,0,\dots,0)$.
    Dessa forma, para $j = 1$, concluímos que $0 \leqslant \sigma_1 < 1$.
    Como $\sigma_1 \in \bN$ segue que $\sigma_1 = 0$. Por outro lado, para $i = 2,\dots,n$, vale $0 \leqslant \sigma_i < 0$,
    que implica em $\sigma_i = 0$. Portanto $\sigma = 0$.
    Da mesma forma, $\beta < \rho \leqslant \alpha$ implica em $\rho =\alpha$.
    Assim, (\ref{eq:somatorio-2}) pode ser escrito da seguinte forma:
    \begin{equation} \label{eq:somatorio-3}
        % \binom{\alpha - \gamma}{0}
        \eta D^{\alpha} u + \sum_{\gamma \leqslant \rho \leqslant \beta} \binom{\alpha - \gamma}{\rho - \gamma} D^{\rho}\eta D^{\alpha - \rho} u + \sum_{\gamma \leqslant \sigma \leqslant \beta} \binom{\alpha - \gamma}{\sigma} D^{\sigma} \eta D^{\alpha - \sigma} u + 
        % \binom{\alpha - \gamma}{\alpha - \gamma}
        u D^{\alpha}\eta.
    \end{equation}
    Por fim, a menos de uma mudança de variaveis, escrevemos (\ref{eq:somatorio-3}) como
    \[
        \sum_{\sigma \leqslant \alpha} \binom{\alpha}{\sigma} D^{\sigma} \eta D^{\alpha - \sigma} u,
    \]
    pois
    \[
        \binom{\alpha - \gamma}{\sigma} + \binom{\alpha - \gamma}{\sigma - \gamma} = \binom{\alpha}{\sigma}
    \]
    e
    \[
        \binom{\alpha}{0} = 1 = \binom{\alpha}{\alpha}.
    \]
    Sendo assim, voltando para a igualdade (\ref{eq:integral-etau}), temos que
    \[
        \int_\Omega \eta u D^{\alpha}\phi \, dx = (-1)^{|\alpha|} \int_\Omega \bigg[\sum_{\sigma \leqslant \alpha} \binom{\alpha}{\sigma} D^{\sigma} \eta D^{\alpha - \sigma} u\bigg] \phi \,dx,
    \]
    para todo $\phi \in \cC^{\infty}_c(\Omega)$,
    Portanto, inferimos que
    \[
        D^\alpha (\eta u) = \sum_{\sigma \leqslant \alpha} \binom{\alpha}{\sigma} D^{\sigma} \eta D^{\alpha - \sigma} u,
    \]
    como queriamos mostrar.
\end{prf}

Com os resultados obtidos, agora é possível verificar que os espaços de Sobolev $W^{k,p}(\Omega)$ são espaços de Banach. Inicialmente, verificamos que $\Vert \cdot \Vert_{\cW^{k,p}(\Omega)}$ é uma norma em $\cW^{k,p}(\Omega)$.

\begin{tbox}
    $(\cW^{k,p}(\Omega), \Vert \cdot \Vert_{\cW^{k,p}(\Omega)})$, com $1 \leqslant p \leqslant \infty$ é um espaço vetorial normado.
\end{tbox}
\begin{prf}
    Inicialmente, considere que $1 \leqslant p < \infty$.
    \begin{enumerate}[leftmargin=*]
        \item Seja $u \in \cW^{k,p}(\Omega)$. Com isso, é verdade que
        \[
            \Vert u \Vert_{\cW^{k,p}(\Omega)} = 0 \iff \sum_{|\alpha| \leqslant k} \Vert D^\alpha u \Vert_{\cL^p(\Omega)}^p = 0 \iff \Vert D^\alpha u \Vert_{\cL^p(\Omega)}^p = 0 \iff D^\alpha u = 0,
        \]
        para todo multi-índice $\alpha$ com $|\alpha| \leqslant k$.
        Em particular, se $\alpha = (0,\dots,0)$, $u = D^\alpha u = 0$.

        Por outro lado, se $u = 0$, $D^\alpha u = 0$ para todo multi-índice $\alpha$. Sendo assim, $\Vert u \Vert_{\cW^{k,p}(\Omega)} = 0$.

        \item Sejam $\lambda \in \bR$ e $u \in \cW^{k,p}(\Omega)$. Sendo assim, pelo Teorema \ref{thm:propriedades-derivada-fraca}, podemos escrever
        \[
            \begin{aligned}
                \Vert \lambda u \Vert_{\cW^{k,p}(\Omega)}^p = \sum_{|\alpha| \leqslant k} \Vert D^\alpha (\lambda u) \Vert_{\cL^p(\Omega)}^p &= \sum_{|\alpha| \leqslant k} \Vert \lambda D^\alpha u \Vert_{\cL^p(\Omega)}^p\\ 
                &= |\lambda|^p\sum_{|\alpha| \leqslant k} \Vert D^\alpha u \Vert_{\cL^p(\Omega)}^p = |\lambda|^p\Vert u \Vert_{\cW^{k,p}(\Omega)}^p.
            \end{aligned}
        \]
        Portanto, $\Vert \lambda u \Vert_{\cW^{k,p}(\Omega)} = |\lambda|\Vert u \Vert_{\cW^{k,p}(\Omega)}$.

        \item Sejam $u, v \in \cW^{k,p}(\Omega)$ Pelos Teoremas \ref{thm:lp-completo-pre} e \ref{thm:propriedades-derivada-fraca}, segue que
        {\small
        \[
            \begin{aligned}
                \Vert u + v \Vert_{\cW^{k,p}(\Omega)} = \left(\sum_{|\alpha| \leqslant k} \Vert D^\alpha u + D^\alpha v \Vert_{\cL^p(\Omega)}^p\right)^{\!\!\frac{1}{p}} &\leqslant \left(\sum_{|\alpha| \leqslant k} \left(\Vert D^\alpha u \Vert_{\cL^p(\Omega)} + \Vert D^\alpha v \Vert_{\cL^p(\Omega)}\right)^p \right)^{\!\!\frac{1}{p}}.\\
            &\leqslant \left( \sum_{|\alpha| \leqslant k} \Vert D^\alpha u \Vert_{\cL^p(\Omega)}^p \right)^{\!\!\frac{1}{p}} + \left( \sum_{|\alpha| \leqslant k} \Vert D^\alpha v \Vert_{\cL^p(\Omega)}^p \right)^{\!\!\frac{1}{p}}.
            \end{aligned}
        \]}\!
        Ou seja,
        \[
            \Vert u + v \Vert_{\cW^{k,p}(\Omega)} \leqslant \Vert u \Vert_{\cW^{k,p}(\Omega)} + \Vert v \Vert_{\cW^{k,p}(\Omega)}.
        \]
    \end{enumerate}
    Portanto, $\Vert \cdot \Vert_{\cW^{k,p}(\Omega)}$ é uma norma em $\cW^{k,p}(\Omega)$, com $1 \leqslant p < \infty$.

    Para o caso $p = \infty$, a demonstração é análoga.
    \begin{enumerate}[leftmargin=*]
        \item Seja $u \in \cW^{k,\infty}(\Omega)$. Com isso, é verdade que
        \[
            \Vert u \Vert_{\cW^{k,\infty}(\Omega)} = 0 \iff \sum_{|\alpha| \leqslant k} \Vert D^\alpha u \Vert_{\cL^\infty(\Omega)} = 0 \iff \Vert D^\alpha u \Vert_{\cL^\infty(\Omega)} = 0 \iff D^\alpha u = 0,
        \]
        para todo multi-índice $\alpha$ com $|\alpha| \leqslant k$.
        Em particular, se $\alpha = (0,\dots,0)$, $u = D^\alpha u = 0$.

        Por outro lado, se $u = 0$, $D^\alpha u = 0$ para todo multi-índice $\alpha$. Sendo assim, $\Vert u \Vert_{\cW^{k,\infty}(\Omega)} = 0$.

        \item Sejam $\lambda \in \bR$ e $u \in \cW^{k,\infty}(\Omega)$. Sendo assim, pelos Teoremas \ref{thm:lp-completo-pre}, \ref{thm:propriedades-derivada-fraca}, podemos escrever
        \[
            \begin{aligned}
                \Vert \lambda u \Vert_{\cW^{k,\infty}(\Omega)} = \sum_{|\alpha| \leqslant k} \Vert D^\alpha (\lambda u) \Vert_{\cL^\infty(\Omega)} &= \sum_{|\alpha| \leqslant k} \Vert \lambda D^\alpha u \Vert_{\cL^\infty(\Omega)}\\ 
                &= |\lambda|\sum_{|\alpha| \leqslant k} \Vert D^\alpha u \Vert_{\cL^\infty(\Omega)} = |\lambda\Vert u \Vert_{\cW^{k,\infty}(\Omega)}.
            \end{aligned}
        \]
        Portanto, $\Vert \lambda u \Vert_{\cW^{k,\infty}(\Omega)} = |\lambda|\Vert u \Vert_{\cW^{k,\infty}(\Omega)}$.

        \item Sejam $u, v \in \cW^{k,\infty}(\Omega)$. Daí, pelo Teorema \ref{thm:propriedades-derivada-fraca}, segue que
        \[
            \begin{aligned}
                \Vert u + v \Vert_{\cW^{k,\infty}(\Omega)} &= \sum_{|\alpha| \leqslant k} \Vert D^\alpha u + D^\alpha v \Vert_{\cL^\infty(\Omega)}\\ &\leqslant \sum_{|\alpha| \leqslant k} \Big( \Vert D^\alpha u \Vert_{\cL^\infty(\Omega)} + \Vert D^\alpha v \Vert_{\cL^\infty(\Omega)} \Big)\\
                &= \sum_{|\alpha| \leqslant k} \Vert D^\alpha u \Vert_{\cL^\infty(\Omega)} + \sum_{|\alpha| \leqslant k}\Vert D^\alpha v \Vert_{\cL^\infty(\Omega)} = \Vert u \Vert_{\cW^{k,\infty}(\Omega)} + \Vert v \Vert_{\cW^{k,\infty}(\Omega)}.
            \end{aligned}
        \]
    \end{enumerate}
    Portanto, $\Vert \cdot \Vert_{\cW^{k,\infty}(\Omega)}$ é uma norma em $\cW^{k,\infty}(\Omega)$.
\end{prf}

Agora, mostremos a completude do espaço de Sobolev $\cW^{k,p}(\Omega)$.

\begin{tbox} \label{thm:sobolev-completo}
    $(\cW^{k,p}(\Omega), \Vert \cdot \Vert_{\cW^{k,p}(\Omega)})$, com $1 \leqslant p \leqslant \infty$, é completo.
\end{tbox}
\begin{prf}
    Seja $(u_n)$ uma sequência de Cauchy em $\cW^{k,p}(\Omega)$, com $1 \leqslant p < \infty$. Isto é, dado $\varepsilon > 0$ existe $n_0 \in \bN$ tal que
    \[
        \Vert u_n - u_m \Vert_{\cW^{k,p}(\Omega)} < \varepsilon,
    \]
    para todo $n,m > n_0$.
    Note que, se $1 \leqslant p < \infty$, tem-se
    % \[
    % \begin{aligned}
    %     \Vert D^\alpha u_n - D^\alpha u_m \Vert_{\cL^p(\Omega)}^p &\leqslant \sum_{|\alpha| \leqslant k} \Vert D^\alpha u_n - D^\alpha u_m \Vert_{\cL^p(\Omega)}^p\\ 
    %     &= \sum_{|\alpha| \leqslant k} \Vert D^\alpha (u_n - u_m) \Vert_{\cL^p(\Omega)}^p = \Vert u_n - u_m \Vert_{\cW^{k,p}(\Omega)}^p < \varepsilon^p
    % \end{aligned}
    % \]
    \[
        \Vert D^\alpha u_n - D^\alpha u_m \Vert_{\cL^p(\Omega)}^p \leqslant \sum_{|\alpha| \leqslant k} \Vert D^\alpha u_n - D^\alpha u_m \Vert_{\cL^p(\Omega)}^p = \Vert u_n - u_m \Vert_{\cW^{k,p}(\Omega)}^p < \varepsilon^p,
    \]
    e se $p = \infty$,
    \[
        \Vert D^\alpha u_n - D^\alpha u_m \Vert_{\cL^\infty(\Omega)} \leqslant \sum_{|\alpha| \leqslant k} \Vert D^\alpha u_n - D^\alpha u_m \Vert_{\cL^\infty(\Omega)} = \Vert u_n - u_m \Vert_{\cW^{k,\infty}(\Omega)} < \varepsilon,
    \]
    para todo $n,m > n_0$ e $\alpha$ multi-índice com $|\alpha| \leqslant k$. Ou seja, $(D^\alpha u_n)$ é uma sequência de Cauchy em $\cL^p(\Omega)$, com $1 \leqslant p \leqslant \infty$, para cada $\alpha$ multi-índice com $|\alpha| \leqslant k$. Lembrando que $\cL^p(\Omega)$, com $1 \leqslant p \leqslant \infty$, é um espaço completo (ver Teorema \ref{thm:lp-completo-pre}), podemos escrever
    \begin{equation} \label{eq:Dalfaun}
        D^\alpha u_n \to u_\alpha \text{ em } \cL^p(\Omega).
    \end{equation}
    Em particular, se $\alpha = (0,\dots,0)$, denotamos $D^\alpha u_n$ por $u_n$ e $u_\alpha$ por $u$.
    Por fim, precisamos mostrar que
    \[
        D^\alpha u = u_\alpha.
    \]
Com efeito, pelo Teorema \ref{thm:teorema-do-brezis}, utlizando a definição de derivada fraca e passando a uma subsequência (se necessário), obtemos
    \[
        \begin{aligned}
            \int_\Omega u D^\alpha \phi \, dx 
            = \int_\Omega (\lim u_n) D^\alpha \phi 
            = \lim\! \int_\Omega u_n D^\alpha \phi \,dx
            &= \lim\, (-1)^{|\alpha|}\!\int_\Omega  D^\alpha u_n \phi\,dx \\
            &= (-1)^{|\alpha|}\!\int_\Omega \lim D^\alpha u_n \phi \,dx 
            = (-1)^{|\alpha|}\!\int_\Omega u_\alpha \phi \,dx,
        \end{aligned}
    \]
    para todo $\phi \in \cC^{\infty}_c(\Omega)$ e mult-índice $\alpha$ com $|\alpha| \leqslant k$.
    Portanto, $D^\alpha u = u_\alpha$.
    Por fim, se $1 \leqslant p < \infty$,
    \[
        \Vert u_n - u \Vert_{\cW^{k,p}(\Omega)}^p = \sum_{|\alpha| \leqslant k} \Vert D^\alpha u_n - D^\alpha u \Vert_{\cL^p(\Omega)}^p = \sum_{|\alpha| \leqslant k} \Vert D^\alpha u_n - u_\alpha \Vert_{\cL^p(\Omega)}^p \to 0,
    \]
    e se $p = \infty$,
    \[
        \Vert u_n - u \Vert_{\cW^{k,\infty}(\Omega)} = \sum_{|\alpha| \leqslant k} \Vert D^\alpha u_n - D^\alpha u \Vert_{\cL^\infty(\Omega)} = \sum_{|\alpha| \leqslant k} \Vert D^\alpha u_n - u_\alpha \Vert_{\cL^\infty(\Omega)} \to 0,
    \]
    quando $n \to \infty$ por (\ref{eq:Dalfaun}).
    Consequentemente, $u_n \to u$ em $\cW^{k,p}(\Omega)$ quando $n \to \infty$, com $1 \leqslant p \leqslant \infty$.
\end{prf}

\section{Aproximações} \label{sec:aproximacoes}

% Não é ideal ficar voltando o tempo todo à definição de derivadas fracas. Para explorar as propriedades mais profundas dos espaços de Sobolev, precisamos de métodos sistemáticos para aproximar funções nesses espaços por funções suaves. A técnica de molificação, que envolve a convolução de uma função com uma função conhecida como função molificadora que é suave e tem suporte compacto, oferece essa ferramenta. Esse processo resulta em uma sequência de funções suaves que convergem para a função original no espaço de Sobolev, permitindo a aproximação de funções sem derivada bem definida por funções suaves. As funções molificadoras são essenciais na teoria dos espaços de Sobolev, possibilitando o estudo de soluções fracas para equações diferenciais parciais e estabelecendo resultados importantes de densidade.

A definição de derivada fraca, em muitos casos, não é suficiente para mostrar propriedades mais profundas dos espaços de Sobolev.
Uma forma de contornar esse problema é procurando formas de aproximar em $\cW^{k,p}$ por uma sequência de funções suaves.
Esse processo é conhecido como molificação ou regularização e é feito por meio de uma convolução da função a ser aproximada e uma função especial chamada função molificadora. 
Nesse capítulo, apresentaremos alguns resultados e exemplos da teoria de aproximação, que é de extrema importância no estudo de equações diferenciais parciais.

Um exemplo de função molificadora é
\begin{equation} \label{eq:molificador-friedrich}
    \eta(x) =
    \left\{
        \begin{array}{lr}
            c \exp \left( \frac{1}{|x|^2 - 1} \right), &\! \text{se }\; |x| < 1;\\
            0, &\!\text{se }\; |x| \geqslant 1,
        \end{array}
    \right.
\end{equation}
conhecida como molificador de Friedrich, onde $c > 0$ é uma constante tal que
\[
    \int_{\bR^n} \eta \,dx = 1.
\]
De forma geral, uma função molificadora é uma função $\eta$ de classe $\cC^\infty$ com suporte compacto satifazendo:
\begin{itemize}[leftmargin=*, label=\small{$\bullet$}]
    \item $\displaystyle \int_{\bR^n} \eta \,dx = 1$;
    \item $\displaystyle \lim_{\varepsilon \to 0} \varepsilon^{-n}\eta(x/\varepsilon) = \delta(x)$, onde $\delta$ é a função delta de Dirac\footnote{i.e., $\delta(x) = 
    \left\{  
        \begin{array}{ll}
            0 &\text{se } x \neq 0\\
            \infty &\text{se } x = 0.
        \end{array}
    \right.
    $}.
\end{itemize}
Dada uma função molificadora, para cada $\varepsilon > 0$, definimos
\begin{equation} \label{eq:eta-epsilon}
    \eta_\varepsilon(x) = \frac{1}{\varepsilon^n} \eta\left( \frac{x}{\varepsilon} \right), 
\end{equation}
para todo $x \in \bR^n$. Essa função $\eta_\varepsilon$ é de classe $\cC^\infty$, $\supp \eta_\varepsilon \subseteq B[0,\varepsilon]$ e, por mudança de variáveis
\begin{equation} \label{eq:integral1}
    \int_{\bR^n} \eta_\varepsilon \,dx = 1.
\end{equation}
Essa aplicação será utilizada para realizar as convoluções que aproximam as funções em $\cW^{k,p}(\Omega)$. Se $u$ é uma função locamente integrável, definimos a molificação de $u$ por $u^\varepsilon = \eta_\varepsilon * u$, isto é,
\[
    u^\varepsilon(x) = \int_\Omega \eta_\varepsilon(x-y) u(y) = \int_{B[0,\varepsilon]} \eta_\varepsilon(y) u(x-y) \,dy,
\]
para todo $x \in \bR^n$.

O primeiro teorema que vamos estudar demonstra algumas propriedades importantes sobre essas aproximações.

\begin{tbox}[Aproximação local por funções suaves] \label{thm:aprox1}
    Seja $u \in \cW^{k,p}(\Omega)$, com $1 \leqslant p < \infty$, e defina
    \[
        u^{\varepsilon} = \eta_{\varepsilon} * u \;\text{ em } \Omega_\varepsilon,
    \]
    onde
    \[
        \Omega_\varepsilon = \{ x \in \Omega \,; d(x,\partial\Omega) > \varepsilon\}.
    \]
    Então,
    \begin{enumerate}[leftmargin=*, label=\textbf{(\alph*)}]
        \item $u^\varepsilon \in \cC^{\infty}(\Omega_\varepsilon)$, para cada $\varepsilon > 0$;
        \item $u^\varepsilon \to u$ em $\cW^{k,p}_{\mathrm{loc}}(\Omega)$, quando $\varepsilon \to 0$.
    \end{enumerate}
\end{tbox}
\begin{prf}
    ~

    \textbf{(a)} Seja $g(x) = (x-y)/\varepsilon $. Logo, pela regra da cadeia, chegamos a
    \begin{align}
        \dfrac{\partial}{\partial x_i} \big[\eta_\varepsilon(x-y)\big] &= \frac{1}{\varepsilon^n}\dfrac{\partial }{\partial x_i}\left[\eta\left( \frac{x - y}{\varepsilon} \right) \right]\label{eq:l} \\
        &=\frac{1}{\varepsilon^n} \frac{\partial}{\partial x_i} \big[ \eta \circ g(x) \big] = \frac{1}{\varepsilon^n} \sum_{k =1}^n \dfrac{\partial \eta}{\partial x_k}\left( \frac{x - y}{\varepsilon} \right) \dfrac{\partial g_k}{\partial x_i}(x) = \frac{1}{\varepsilon^{n+1}} \dfrac{\partial \eta}{\partial x_i}\left( \frac{x - y}{\varepsilon} \right), \nonumber
    \end{align}
    para todo $i = 1,\dots,n$. Por outro lado, sejam $x \in \Omega_\varepsilon$, $i = 1,\dots,n$ e $h$ de forma que $x + he_i \in \Omega_\varepsilon$. Deste modo,
    \begin{equation} \label{eq:m}
        \frac{u^\varepsilon(x + he_i) - u^\varepsilon(x)}{h} 
            = \frac{1}{\varepsilon^n} \int_\Omega \frac{1}{h}\left[  \eta\left( \frac{x + he_i - y}{\varepsilon} \right) - \eta\left( \frac{x -y}{\varepsilon} \right) \right] u(y) \,dy.
    \end{equation} 
    Novamente utilizando a regra da cadeia e (\ref{eq:l}), segue que
    \begin{equation} \label{eq:n}
        \begin{aligned}
            \frac{1}{h}\left[  \eta\left( \frac{x + he_i - y}{\varepsilon} \right) - \eta\left( \frac{x -y}{\varepsilon} \right) \right] &=
            \frac{\varepsilon^n}{h} \bigg[ \eta_\varepsilon (x-y + he_i) - \eta_\varepsilon(x-y) \bigg]\\
            &\to
            \varepsilon^n \frac{\partial\eta_\varepsilon}{\partial x_i} (x-y) =
            \frac{1}{\varepsilon}\dfrac{\partial \eta}{\partial x_i}\left( \frac{x - y}{\varepsilon} \right),
        \end{aligned}
    \end{equation}
    quando $h \to 0$. Consequentemente, por (\ref{eq:l}), (\ref{eq:m}), (\ref{eq:n}) e pelo Teorema da converência dominada (ver Teorema \ref{thm:teorema-da-convergencia-dominada}), temos que
    \[
        \dfrac{\partial u^\varepsilon}{\partial x_i}(x) = \int_\Omega \frac{1}{\varepsilon^{n+1}} \dfrac{\partial \eta}{\partial x_i}\left( \frac{x - y}{\varepsilon} \right) u(y) \,dy = \int_\Omega \dfrac{\partial \eta_\varepsilon}{\partial x_i} (x-y) u(y) \,dy = \frac{\partial \eta_\varepsilon}{\partial x_i} * u(x).
    \]
    Sendo assim,
    indutivamente, podemos mostrar que $D^\alpha u^\varepsilon$ existe e
    \[
        D^\alpha u^\varepsilon = D^\alpha\eta_\varepsilon * u,
    \]
    para todo multi-índice $\alpha$
    O fato de $u^\varepsilon$ ser de classe $\cC^\infty$ segue do fato de $\eta_\varepsilon$ ser de classe $\cC^\infty$ (por definição) e $u \in \cW^{k,p}(\Omega)$.

    \begin{figure}
        \centering
        \input{mollifiers/omegaepsilon.tex}
        \caption{Representação visual dos conjuntos $\Omega_\varepsilon$.\\Fonte: Autoral.}
    \end{figure}

    \textbf{(b)} Afirmamos que a $\alpha$-ésima derivada parcial de $u^\varepsilon$ no sentido usual é igual a convulução de $\eta_\varepsilon$ com a $\alpha$-ésima derivada parcial fraca de $u$ para todo $\alpha$ com $|\alpha| \leqslant k$, isto é,
    \[
        D^\alpha u^\varepsilon = \eta_\varepsilon * D^\alpha u.
    \]
    Com efeito, no item \textbf{(a)} vimos que $D^\alpha u^\varepsilon = D^\alpha \eta_\varepsilon * u$. Primeiramente, se $g(x) = x - y$, pela regra da cadeia, temos que
    \[
        D^{e_i}_x \eta_\varepsilon (x -y) = \dfrac{\partial}{\partial x_i} (\eta_\varepsilon \circ g)(x) = \sum_{k=1}^n \dfrac{\partial \eta_\varepsilon}{\partial x_k} (x - y) \dfrac{\partial g_k}{\partial x_i}(x) = \dfrac{\partial \eta_\varepsilon}{\partial x_i}(x-y).
    \]
    Por outro lado, se $h(y) = x - y$, obtemos, pela regra da cadeia novamente, que
    \[
        D^{e_i}_y \eta_\varepsilon (x-y) = \dfrac{\partial}{\partial y_i} (\eta_\varepsilon \circ h)(y) = \sum_{k=1}^n \dfrac{\partial \eta_\varepsilon}{\partial y_k} (x - y) \dfrac{\partial h_k}{\partial y_i}(y) = -\dfrac{\partial \eta_\varepsilon}{\partial y_i}(x-y).
    \]
    Dessa forma, ao menos de uma mudança de notação, chegamos a
    \[
        D^{e_i}_x \eta_\varepsilon (x-y) = -D^{e_i}_y \eta_\varepsilon (x-y).
    \]
    Repetindo esse cálculo $|\alpha|$ vezes, obtemos
    \[
        D^\alpha_x \eta_\varepsilon (x-y) = (-1)^{|\alpha|} D^\alpha_y \eta_\varepsilon (x-y).
    \]
    Deste modo, podemos escrever
    \[
        D^{\alpha} u^\varepsilon (x) = \int_\Omega D^\alpha_x \eta_\varepsilon (x-y) u(y) \,dy = (-1)^{|\alpha|} \int_\Omega D^{\alpha}_y \eta_\varepsilon (x-y) u(y) \,dy.
    \]
    Fixando $x \in \Omega_\varepsilon$, a função $\phi_x(y) =\eta_\varepsilon(x-y)$ é suave e tem suporte compacto em $\Omega$.
    Aplicando a definição de derivada fraca com função teste $\eta_\varepsilon(x-y)$, segue que
    \[
        D^\alpha u^\varepsilon(x) = (-1)^{|\alpha| + |\alpha|}  \int_\Omega \eta_\varepsilon(x-y) D^{\alpha} u(y) \,dy = \int_\Omega \eta_\varepsilon(x-y) D^\alpha u(y).
    \]
    Portanto, deduzimos que
    \begin{equation}
        D^\alpha u^\varepsilon = \eta_\varepsilon * D^\alpha u = [D^\alpha u]^\varepsilon.
    \end{equation}

    Além disso, afirmamos que dados abertos $V, W$ tais que $V \Subset W \Subset \Omega$, uma função $v \in \cL^p_\loc(\Omega)$ e $\varepsilon > 0$ suficientemente pequeno, inferimos
    \begin{equation} \label{eq:assss}
        \Vert v^\varepsilon \Vert_{\cL^p(V)} \leqslant \Vert v \Vert_{\cL^p(W)}.
    \end{equation}
    Com efeito, se $p = 1$, note que
    \[
        |v^\varepsilon(x)| =  \left| \int_{B[x,\varepsilon]} \eta_\varepsilon (x-y) v(y) \,dy \right| \leqslant \int_{B[x,\varepsilon]} |\eta_\varepsilon(x-y)| | v(y)| \,dy.
    \]
    Integrando sobre $V$ e utilizando o Teorema de Fubini (ver Teorema \ref{thm:fubini}), concluímos que
    \[
        \int_V |v^\varepsilon(x)| \,dx \leqslant \int_V  \int_{B[x,\varepsilon]} |v(y)||\eta_\varepsilon(x-y)| \, dydx \leqslant \int_W |v(y)| \int_{B[y,\varepsilon]} |\eta_\varepsilon (x-y)| dxdy.
    \]
    Porém, $\int_{B[y,\varepsilon]} \eta_\varepsilon(x- y) \,dx = 1$ (ver \ref{eq:integral1}); sendo assim, obtemos
    \[
        \Vert v^\varepsilon \Vert_{\cL^1(V)} \leqslant \Vert v \Vert_{\cL^1(W)}.
    \]
    Se $1 < p < \infty$, observe que
    \begin{equation} \label{eq:nn1}
        |v^\varepsilon(x)| = \left| \int_{B[x,\varepsilon]} \eta_\varepsilon(x-y) v(y) \,dy \right| \leqslant \int_{B[x,\varepsilon]} \left| \eta_\varepsilon^{1 - \frac{1}{p}} (x-y) \eta_\varepsilon^{\frac{1}{p}} (x-y) v(y) \right| \,dy.
    \end{equation}
    Utilizando a desigualdade de Hölder (ver Teorema \ref{thm:pre-desigualdade-de-holder}) na última integral acima, segue que
    {\small
        \begin{equation} \label{eq:nn2}
            \int_{B[x,\varepsilon]} \left| \eta_\varepsilon^{1 - \frac{1}{p}} (x-y) \eta_\varepsilon^{\frac{1}{p}} (x-y) v(y) \right| \,dy \leqslant \left( \int_{B[x,\varepsilon]} \eta_\varepsilon(x-y) dy \right)^{1 - \frac{1}{p}} \left( \int_{B[x,\varepsilon]} \eta_\varepsilon(x-y) |v(y)|^p \,dy \right)^{\frac{1}{p}}.
        \end{equation}
    }\!\!
    Lembrando que, $\int_{B[x,\varepsilon]} \eta_\varepsilon(x-y) \,dy = 1$ (ver (\ref{eq:integral1})), integrando  as desigualdades (\ref{eq:nn1}) e (\ref{eq:nn2}) sobre $V$ e utilizando o Teorema de Fubini (ver Teorema \ref{thm:fubini}), obtemos
    \[
        \int_V |v^\varepsilon(x)|^p \,dx \leqslant \int_V \int_{B[x,\varepsilon]} \eta_\varepsilon(x-y) |v(y)|^p \,dy dx \leqslant \int_W |v(y)|^p \int_{B[y,\varepsilon]} \eta_\varepsilon(x-y) \,dxdy.
    \]
    Isto é,
    \begin{equation*}
        \Vert v^\varepsilon \Vert_{\cL^p(V)} \leqslant \Vert v \Vert_{\cL^p(W)}.
    \end{equation*}
    Isto prova \ref{eq:assss}.

    Por fim, seja $V \Subset \Omega$ um aberto. Afirmamos que
    \begin{equation}
        D ^\alpha u^\varepsilon \to D^\alpha u \text{ em } \cL^p(V),
    \end{equation}
    quando $\varepsilon \to 0$, para todo multi-índice $\alpha$ com $|\alpha| \leqslant k$.
    De fato, sejam $W$ um aberto de forma que $V \Subset W \Subset \Omega$ e $\delta > 0$. Utilizando (\ref{eq:assss}) com $v ^\varepsilon = D^\alpha u^\varepsilon$ e $v = D^\alpha u$, e escolhendo $w \in \cC(W)$ tal que
    \[
        \Vert D^\alpha u - w \Vert_{\cL^p(W)} < \delta,
    \]
    (ver Teorema \ref{thm:densidadeC}), temos que
    \[
        \begin{aligned}
            \Vert D^\alpha u^\varepsilon - D^\alpha u \Vert_{\cL^p(V)} 
            &\leqslant \Vert D^\alpha u^\varepsilon - w^\varepsilon \Vert_{\cL^p(V)} + \Vert w^\varepsilon - w \Vert_{\cL^p(V)} + \Vert w - D^\alpha u \Vert_{\cL^p(V)}\\
            &\leqslant \Vert D^\alpha u - w \Vert_{\cL^p(W)} + \Vert w^\varepsilon - w \Vert_{\cL^p(V)} + \Vert w - D^\alpha u \Vert_{\cL^p(W)}\\ 
            &\leqslant 2\delta + \Vert w^\varepsilon - w \Vert_{\cL^p(V)},
        \end{aligned}
    \]
    para todo $\alpha$ multi-índice com $|\alpha| \leqslant k$.
    Como $w \in \cC(W)$, então $w^\varepsilon \to w$ uniformemente\footnote{Uma sequência de funções $(w_k)$ converge uniformemente para $w$, se, dado $\varepsilon > 0$, existe $K \in \bN$ (que não pode depender de $x$) tal que
    \[
        | w_k(x) - w(x) | < \varepsilon,
    \]
    para todo $x$ e $k > K$.
    } em $V$ quando $\varepsilon \to 0$. 
    Portanto, $D^\alpha u^\varepsilon \to D^\alpha u$ em $\cL^p(V)$, quando $\varepsilon \to 0$, para todo $\alpha$ multi-índice com $|\alpha| \leqslant k$ (basta utilizar o Teorema da Convergência Dominada).
    Dessa forma,
    \[
        \Vert u^\varepsilon - u \Vert^p_{\cW^{k,p}(V)} = \sum_{|\alpha| \leqslant k} \Vert D^\alpha u^\varepsilon - D^\alpha u \Vert^p_{\cL^p(V)} \to 0,
    \]
    quando $\varepsilon \to 0$. Portanto, $u^\varepsilon \to u$ em $\cW^{k,p}(V)$, quando $\varepsilon \to 0$, para todo $V \Subset \Omega$. Isto é,
    \[
        u^\varepsilon \to u \text{ em } \cW^{k,p}_\loc(\Omega),
    \]
    quando $\varepsilon \to 0$.
\end{prf}

Agora, apresentaremos dois exemplos em que podemos visualizar essas aproximações graficamente.

\begin{ex}
    A função $u(x) = |x|$ definida no intervalo aberto $\Omega = (-1,1) \subseteq \bR$ é um exemplo clássico de função que não é diferenciável no sentido usual. É fácil verificar que $u \in \cW^{1,p}(\Omega)$. Com efeito, dada uma função teste $\phi \in \cC^\infty_c(\Omega)$ temos que
    \[
        \int_\Omega u \phi' \,dx = \int_{\text{-}1}^1 u \phi' \,dx = \int_{\text{-}1}^0 u \phi' \,dx + \int_0^1 u \phi' \,dx.
    \]
    Utlizando integração por partes (ver Teorema \ref{thm:integracao-por-partes}), obtemos
    \[
        \int_\Omega u \phi'\,dx =  - \int_{\m1}^0 -\phi \,dx  - \int_0^1 \phi \,dx -x \phi \bigg|^{0}_{\m1}+ x\phi \bigg|_0^1 = -\int_{\m1}^{0} - \phi \,dx -\int_0^1 \phi \,dx - \phi(-1) + \phi(1).
    \]
    Porém, $\phi$ tem suporte compacto, logo; se anula em $\partial\Omega = \{-1,1\}$. Dito isso, podemos escrever
    \[
        \int_\Omega u \phi'\,dx = - \int_{\text{-}1}^0 -\phi \,dx - \int_0^1 \phi \,dx = -\int_\Omega \sgn(x)\phi \,dx.
    \]
    Portanto, é verdade que
    \[
        u'= \sgn
    \]
    no sentido fraco, onde
    \[
        \sgn(x) = 
        \left\{ 
            \begin{array}{rr}
                1, &\!\text{se }\; x > 0;\\
                0, &\!\text{se }\; x = 0;\\
                -1,&\!\text{se }\; x < 0.
            \end{array}
        \right.
    \]
    Além disso, é verdade que
    \[
        \int_\Omega |\sgn(x)|^p \,dx = \int_{\text{-}1}^1 1^p \,dx =  \mu((-1,1)) = 2 < \infty,
    \]
    onde $\mu$ é a medida de Lebesgue (ver \cite{axler-measure.theory} para mais detalhes). Assim, $u'\in \cL^p(\Omega)$ e, portanto, $u \in \cW^{1,p}(\Omega)$ para $1 \leqslant p < \infty$.
    Vamos utiizar a convolução para encontrar uma aproximação suave de $u$. 
    Seja $\eta : \bR \to \bR$ dada por
    \[
        \eta(x) = \left\{ 
            \begin{array}{lr}
                c \exp\left(\frac{1}{x^2 - 1} \right), & \text{se }\; |x| \leqslant 1;\\
                0, & \text{se }\; |x| > 1,
            \end{array}
        \right.
    \]
    (ver Figura \ref{fig:eta})
    onde $c$ é determinado de forma que
    \[
        \int_{\bR} \eta \,dx = 1,
    \]
    isto é,
    \[
        c = \left( \int_{[-1,1]} \exp \left(\frac{1}{x^2 - 1} \right) \, dx\right)^{-1}.
    \]
    Infelizmente, a função $\eta$ não tem primitiva que pode ser expressa por meio de funções elementares, então é necessário utilizar um método numérico, ou expansão em Taylor para calcular a constante $c$.
    Também, definimos a função molificadora $\eta_\varepsilon : \bR \to \bR$ por
    \[
        \eta_\varepsilon(x) = \frac{1}{\varepsilon} \eta\left( \frac{x}{\varepsilon} \right).
    \]
    para todo $x \in \bR^n$. Note que
    \[
        \int_\bR \eta_\varepsilon \, dx = 1 \;\text{ e }\; \supp \eta = [-\varepsilon,\varepsilon].
    \]
    Portanto, podemos utilizar essa função para aproximar $u$. Com efeito, basta realizar a convolução $\eta_\varepsilon * u$, isto é,
    \[
        u^\varepsilon(x) = \int_{[-\varepsilon,\varepsilon]} \eta_\varepsilon(x) u(x-y) \,dy.
    \]
    A Figura \ref{fig:convolucao} foi feita utilizando um método numérico para resolver essa integral para diferentes valores de $\varepsilon$.

    \begin{figure}
        \centering
        \begin{tikzpicture}
    \draw[thin, black!30] (-4,-4) grid[step=1] (4,4);
    \draw[thick, black!60, -stealth] (-4,0) to (4,0); 
    \draw[thick, black!60, -stealth] (0,-4) to (0,4);

    \node[align=left, anchor=south west] at (-3.9,-3.9) {$\textcolor{ProcessBlue!90!black}{\blacksquare\!\blacksquare \;\; \varepsilon =1}$ \\ $\textcolor{ForestGreen}{\blacksquare\!\blacksquare \;\; \varepsilon = 0.5}$ \\ $\textcolor{Aquamarine}{\blacksquare\!\blacksquare \;\; \varepsilon = 0.25}$};
    
    \coordinate (01-1) at (-3.000, 3.000);
    \coordinate (02-1) at (-2.940, 2.940);
    \coordinate (03-1) at (-2.880, 2.880);
    \coordinate (04-1) at (-2.817, 2.817);
    \coordinate (05-1) at (-2.757, 2.757);
    \coordinate (06-1) at (-2.697, 2.697);
    \coordinate (07-1) at (-2.637, 2.637);
    \coordinate (08-1) at (-2.577, 2.577);
    \coordinate (09-1) at (-2.514, 2.517);
    \coordinate (10-1) at (-2.454, 2.454);
    \coordinate (11-1) at (-2.394, 2.394);
    \coordinate (12-1) at (-2.334, 2.337);
    \coordinate (13-1) at (-2.274, 2.277);
    \coordinate (14-1) at (-2.211, 2.217);
    \coordinate (15-1) at (-2.151, 2.160);
    \coordinate (16-1) at (-2.091, 2.103);
    \coordinate (17-1) at (-2.031, 2.046);
    \coordinate (18-1) at (-1.971, 1.992);
    \coordinate (19-1) at (-1.908, 1.935);
    \coordinate (20-1) at (-1.848, 1.884);
    \coordinate (21-1) at (-1.788, 1.830);
    \coordinate (22-1) at (-1.728, 1.779);
    \coordinate (23-1) at (-1.668, 1.728);
    \coordinate (24-1) at (-1.605, 1.680);
    \coordinate (25-1) at (-1.545, 1.632);
    \coordinate (26-1) at (-1.485, 1.587);
    \coordinate (27-1) at (-1.425, 1.542);
    \coordinate (28-1) at (-1.365, 1.497);
    \coordinate (29-1) at (-1.302, 1.458);
    \coordinate (30-1) at (-1.242, 1.416);
    \coordinate (31-1) at (-1.182, 1.380);
    \coordinate (32-1) at (-1.122, 1.341);
    \coordinate (33-1) at (-1.062, 1.308);
    \coordinate (34-1) at (-0.999, 1.275);
    \coordinate (35-1) at (-0.939, 1.242);
    \coordinate (36-1) at (-0.879, 1.215);
    \coordinate (37-1) at (-0.819, 1.185);
    \coordinate (38-1) at (-0.759, 1.161);
    \coordinate (39-1) at (-0.696, 1.137);
    \coordinate (40-1) at (-0.636, 1.113);
    \coordinate (41-1) at (-0.576, 1.095);
    \coordinate (42-1) at (-0.516, 1.077);
    \coordinate (43-1) at (-0.456, 1.059);
    \coordinate (44-1) at (-0.393, 1.047);
    \coordinate (45-1) at (-0.333, 1.035);
    \coordinate (46-1) at (-0.273, 1.023);
    \coordinate (47-1) at (-0.213, 1.017);
    \coordinate (48-1) at (-0.153, 1.011);
    \coordinate (49-1) at (-0.090, 1.005);
    \coordinate (50-1) at (-0.030, 1.005);
    \coordinate (51-1) at (0.030, 1.005);
    \coordinate (52-1) at (0.090, 1.005);
    \coordinate (53-1) at (0.153, 1.011);
    \coordinate (54-1) at (0.213, 1.017);
    \coordinate (55-1) at (0.273, 1.023);
    \coordinate (56-1) at (0.333, 1.035);
    \coordinate (57-1) at (0.393, 1.047);
    \coordinate (58-1) at (0.456, 1.059);
    \coordinate (59-1) at (0.516, 1.077);
    \coordinate (60-1) at (0.576, 1.095);
    \coordinate (61-1) at (0.636, 1.113);
    \coordinate (62-1) at (0.696, 1.137);
    \coordinate (63-1) at (0.759, 1.161);
    \coordinate (64-1) at (0.819, 1.185);
    \coordinate (65-1) at (0.879, 1.215);
    \coordinate (66-1) at (0.939, 1.242);
    \coordinate (67-1) at (0.999, 1.275);
    \coordinate (68-1) at (1.062, 1.308);
    \coordinate (69-1) at (1.122, 1.341);
    \coordinate (70-1) at (1.182, 1.380);
    \coordinate (71-1) at (1.242, 1.416);
    \coordinate (72-1) at (1.302, 1.458);
    \coordinate (73-1) at (1.365, 1.497);
    \coordinate (74-1) at (1.425, 1.542);
    \coordinate (75-1) at (1.485, 1.587);
    \coordinate (76-1) at (1.545, 1.632);
    \coordinate (77-1) at (1.605, 1.680);
    \coordinate (78-1) at (1.668, 1.728);
    \coordinate (79-1) at (1.728, 1.779);
    \coordinate (80-1) at (1.788, 1.830);
    \coordinate (81-1) at (1.848, 1.884);
    \coordinate (82-1) at (1.908, 1.935);
    \coordinate (83-1) at (1.971, 1.992);
    \coordinate (84-1) at (2.031, 2.046);
    \coordinate (85-1) at (2.091, 2.103);
    \coordinate (86-1) at (2.151, 2.160);
    \coordinate (87-1) at (2.211, 2.217);
    \coordinate (88-1) at (2.274, 2.277);
    \coordinate (89-1) at (2.334, 2.337);
    \coordinate (90-1) at (2.394, 2.394);
    \coordinate (91-1) at (2.454, 2.454);
    \coordinate (92-1) at (2.514, 2.517);
    \coordinate (93-1) at (2.577, 2.577);
    \coordinate (94-1) at (2.637, 2.637);
    \coordinate (95-1) at (2.697, 2.697);
    \coordinate (96-1) at (2.757, 2.757);
    \coordinate (97-1) at (2.817, 2.817);
    \coordinate (98-1) at (2.880, 2.880);
    \coordinate (99-1) at (2.940, 2.940);
    \coordinate (00-1) at (3.000, 3.000);

    \coordinate (01-2) at (-3.000, 3.000);
    \coordinate (02-2) at (-2.940, 2.940);
    \coordinate (03-2) at (-2.880, 2.880);
    \coordinate (04-2) at (-2.817, 2.817);
    \coordinate (05-2) at (-2.757, 2.757);
    \coordinate (06-2) at (-2.697, 2.697);
    \coordinate (07-2) at (-2.637, 2.637);
    \coordinate (08-2) at (-2.577, 2.577);
    \coordinate (09-2) at (-2.514, 2.514);
    \coordinate (10-2) at (-2.454, 2.454);
    \coordinate (11-2) at (-2.394, 2.394);
    \coordinate (12-2) at (-2.334, 2.334);
    \coordinate (13-2) at (-2.274, 2.274);
    \coordinate (14-2) at (-2.211, 2.211);
    \coordinate (15-2) at (-2.151, 2.151);
    \coordinate (16-2) at (-2.091, 2.091);
    \coordinate (17-2) at (-2.031, 2.031);
    \coordinate (18-2) at (-1.971, 1.971);
    \coordinate (19-2) at (-1.908, 1.908);
    \coordinate (20-2) at (-1.848, 1.848);
    \coordinate (21-2) at (-1.788, 1.788);
    \coordinate (22-2) at (-1.728, 1.728);
    \coordinate (23-2) at (-1.668, 1.668);
    \coordinate (24-2) at (-1.605, 1.605);
    \coordinate (25-2) at (-1.545, 1.545);
    \coordinate (26-2) at (-1.485, 1.485);
    \coordinate (27-2) at (-1.425, 1.425);
    \coordinate (28-2) at (-1.365, 1.365);
    \coordinate (29-2) at (-1.302, 1.302);
    \coordinate (30-2) at (-1.242, 1.242);
    \coordinate (31-2) at (-1.182, 1.182);
    \coordinate (32-2) at (-1.122, 1.125);
    \coordinate (33-2) at (-1.062, 1.065);
    \coordinate (34-2) at (-0.999, 1.008);
    \coordinate (35-2) at (-0.939, 0.954);
    \coordinate (36-2) at (-0.879, 0.903);
    \coordinate (37-2) at (-0.819, 0.852);
    \coordinate (38-2) at (-0.759, 0.804);
    \coordinate (39-2) at (-0.696, 0.759);
    \coordinate (40-2) at (-0.636, 0.717);
    \coordinate (41-2) at (-0.576, 0.681);
    \coordinate (42-2) at (-0.516, 0.645);
    \coordinate (43-2) at (-0.456, 0.615);
    \coordinate (44-2) at (-0.393, 0.585);
    \coordinate (45-2) at (-0.333, 0.564);
    \coordinate (46-2) at (-0.273, 0.543);
    \coordinate (47-2) at (-0.213, 0.525);
    \coordinate (48-2) at (-0.153, 0.513);
    \coordinate (49-2) at (-0.090, 0.507);
    \coordinate (50-2) at (-0.030, 0.501);
    \coordinate (51-2) at (0.030, 0.501);
    \coordinate (52-2) at (0.090, 0.507);
    \coordinate (53-2) at (0.153, 0.513);
    \coordinate (54-2) at (0.213, 0.525);
    \coordinate (55-2) at (0.273, 0.543);
    \coordinate (56-2) at (0.333, 0.564);
    \coordinate (57-2) at (0.393, 0.585);
    \coordinate (58-2) at (0.456, 0.615);
    \coordinate (59-2) at (0.516, 0.645);
    \coordinate (60-2) at (0.576, 0.681);
    \coordinate (61-2) at (0.636, 0.717);
    \coordinate (62-2) at (0.696, 0.759);
    \coordinate (63-2) at (0.759, 0.804);
    \coordinate (64-2) at (0.819, 0.852);
    \coordinate (65-2) at (0.879, 0.903);
    \coordinate (66-2) at (0.939, 0.954);
    \coordinate (67-2) at (0.999, 1.008);
    \coordinate (68-2) at (1.062, 1.065);
    \coordinate (69-2) at (1.122, 1.125);
    \coordinate (70-2) at (1.182, 1.182);
    \coordinate (71-2) at (1.242, 1.242);
    \coordinate (72-2) at (1.302, 1.302);
    \coordinate (73-2) at (1.365, 1.365);
    \coordinate (74-2) at (1.425, 1.425);
    \coordinate (75-2) at (1.485, 1.485);
    \coordinate (76-2) at (1.545, 1.545);
    \coordinate (77-2) at (1.605, 1.605);
    \coordinate (78-2) at (1.668, 1.668);
    \coordinate (79-2) at (1.728, 1.728);
    \coordinate (80-2) at (1.788, 1.788);
    \coordinate (81-2) at (1.848, 1.848);
    \coordinate (82-2) at (1.908, 1.908);
    \coordinate (83-2) at (1.971, 1.971);
    \coordinate (84-2) at (2.031, 2.031);
    \coordinate (85-2) at (2.091, 2.091);
    \coordinate (86-2) at (2.151, 2.151);
    \coordinate (87-2) at (2.211, 2.211);
    \coordinate (88-2) at (2.274, 2.274);
    \coordinate (89-2) at (2.334, 2.334);
    \coordinate (90-2) at (2.394, 2.394);
    \coordinate (91-2) at (2.454, 2.454);
    \coordinate (92-2) at (2.514, 2.514);
    \coordinate (93-2) at (2.577, 2.577);
    \coordinate (94-2) at (2.637, 2.637);
    \coordinate (95-2) at (2.697, 2.697);
    \coordinate (96-2) at (2.757, 2.757);
    \coordinate (97-2) at (2.817, 2.817);
    \coordinate (98-2) at (2.880, 2.880);
    \coordinate (99-2) at (2.940, 2.940);
    \coordinate (00-2) at (3.000, 3.000);

    \coordinate (01-3) at (-3.000, 3.000);
    \coordinate (02-3) at (-2.940, 2.940);
    \coordinate (03-3) at (-2.880, 2.880);
    \coordinate (04-3) at (-2.817, 2.817);
    \coordinate (05-3) at (-2.757, 2.757);
    \coordinate (06-3) at (-2.697, 2.697);
    \coordinate (07-3) at (-2.637, 2.637);
    \coordinate (08-3) at (-2.577, 2.577);
    \coordinate (09-3) at (-2.514, 2.514);
    \coordinate (10-3) at (-2.454, 2.454);
    \coordinate (11-3) at (-2.394, 2.394);
    \coordinate (12-3) at (-2.334, 2.334);
    \coordinate (13-3) at (-2.274, 2.274);
    \coordinate (14-3) at (-2.211, 2.211);
    \coordinate (15-3) at (-2.151, 2.151);
    \coordinate (16-3) at (-2.091, 2.091);
    \coordinate (17-3) at (-2.031, 2.031);
    \coordinate (18-3) at (-1.971, 1.971);
    \coordinate (19-3) at (-1.908, 1.908);
    \coordinate (20-3) at (-1.848, 1.848);
    \coordinate (21-3) at (-1.788, 1.788);
    \coordinate (22-3) at (-1.728, 1.728);
    \coordinate (23-3) at (-1.668, 1.668);
    \coordinate (24-3) at (-1.605, 1.605);
    \coordinate (25-3) at (-1.545, 1.545);
    \coordinate (26-3) at (-1.485, 1.485);
    \coordinate (27-3) at (-1.425, 1.425);
    \coordinate (28-3) at (-1.365, 1.365);
    \coordinate (29-3) at (-1.302, 1.302);
    \coordinate (30-3) at (-1.242, 1.242);
    \coordinate (31-3) at (-1.182, 1.182);
    \coordinate (32-3) at (-1.122, 1.122);
    \coordinate (33-3) at (-1.062, 1.062);
    \coordinate (34-3) at (-0.999, 0.999);
    \coordinate (35-3) at (-0.939, 0.939);
    \coordinate (36-3) at (-0.879, 0.879);
    \coordinate (37-3) at (-0.819, 0.819);
    \coordinate (38-3) at (-0.759, 0.759);
    \coordinate (39-3) at (-0.696, 0.696);
    \coordinate (40-3) at (-0.636, 0.636);
    \coordinate (41-3) at (-0.576, 0.576);
    \coordinate (42-3) at (-0.516, 0.519);
    \coordinate (43-3) at (-0.456, 0.465);
    \coordinate (44-3) at (-0.393, 0.414);
    \coordinate (45-3) at (-0.333, 0.369);
    \coordinate (46-3) at (-0.273, 0.330);
    \coordinate (47-3) at (-0.213, 0.300);
    \coordinate (48-3) at (-0.153, 0.276);
    \coordinate (49-3) at (-0.090, 0.261);
    \coordinate (50-3) at (-0.030, 0.252);
    \coordinate (51-3) at (0.030, 0.252);
    \coordinate (52-3) at (0.090, 0.261);
    \coordinate (53-3) at (0.153, 0.276);
    \coordinate (54-3) at (0.213, 0.300);
    \coordinate (55-3) at (0.273, 0.330);
    \coordinate (56-3) at (0.333, 0.369);
    \coordinate (57-3) at (0.393, 0.414);
    \coordinate (58-3) at (0.456, 0.465);
    \coordinate (59-3) at (0.516, 0.519);
    \coordinate (60-3) at (0.576, 0.576);
    \coordinate (61-3) at (0.636, 0.636);
    \coordinate (62-3) at (0.696, 0.696);
    \coordinate (63-3) at (0.759, 0.759);
    \coordinate (64-3) at (0.819, 0.819);
    \coordinate (65-3) at (0.879, 0.879);
    \coordinate (66-3) at (0.939, 0.939);
    \coordinate (67-3) at (0.999, 0.999);
    \coordinate (68-3) at (1.062, 1.062);
    \coordinate (69-3) at (1.122, 1.122);
    \coordinate (70-3) at (1.182, 1.182);
    \coordinate (71-3) at (1.242, 1.242);
    \coordinate (72-3) at (1.302, 1.302);
    \coordinate (73-3) at (1.365, 1.365);
    \coordinate (74-3) at (1.425, 1.425);
    \coordinate (75-3) at (1.485, 1.485);
    \coordinate (76-3) at (1.545, 1.545);
    \coordinate (77-3) at (1.605, 1.605);
    \coordinate (78-3) at (1.668, 1.668);
    \coordinate (79-3) at (1.728, 1.728);
    \coordinate (80-3) at (1.788, 1.788);
    \coordinate (81-3) at (1.848, 1.848);
    \coordinate (82-3) at (1.908, 1.908);
    \coordinate (83-3) at (1.971, 1.971);
    \coordinate (84-3) at (2.031, 2.031);
    \coordinate (85-3) at (2.091, 2.091);
    \coordinate (86-3) at (2.151, 2.151);
    \coordinate (87-3) at (2.211, 2.211);
    \coordinate (88-3) at (2.274, 2.274);
    \coordinate (89-3) at (2.334, 2.334);
    \coordinate (90-3) at (2.394, 2.394);
    \coordinate (91-3) at (2.454, 2.454);
    \coordinate (92-3) at (2.514, 2.514);
    \coordinate (93-3) at (2.577, 2.577);
    \coordinate (94-3) at (2.637, 2.637);
    \coordinate (95-3) at (2.697, 2.697);
    \coordinate (96-3) at (2.757, 2.757);
    \coordinate (97-3) at (2.817, 2.817);
    \coordinate (98-3) at (2.880, 2.880);
    \coordinate (99-3) at (2.940, 2.940);
    \coordinate (00-3) at (3.000, 3.000);

    \draw[very thick, dashed] (-3,3) -- (0,0) -- (3,3);

    \draw[very thick, ProcessBlue!90!black] 
       (01-1) -- (02-1) -- (03-1) -- (04-1) -- (05-1) -- (06-1) -- (07-1) -- (08-1) -- (09-1) -- (10-1)
    -- (11-1) -- (12-1) -- (13-1) -- (14-1) -- (15-1) -- (16-1) -- (17-1) -- (18-1) -- (19-1) -- (20-1)
    -- (21-1) -- (22-1) -- (23-1) -- (24-1) -- (25-1) -- (26-1) -- (27-1) -- (28-1) -- (29-1) -- (30-1)
    -- (31-1) -- (32-1) -- (33-1) -- (34-1) -- (35-1) -- (36-1) -- (37-1) -- (38-1) -- (39-1) -- (40-1)
    -- (41-1) -- (42-1) -- (43-1) -- (44-1) -- (45-1) -- (46-1) -- (47-1) -- (48-1) -- (49-1) -- (50-1)
    -- (51-1) -- (52-1) -- (53-1) -- (54-1) -- (55-1) -- (56-1) -- (57-1) -- (58-1) -- (59-1) -- (60-1)
    -- (61-1) -- (62-1) -- (63-1) -- (64-1) -- (65-1) -- (66-1) -- (67-1) -- (68-1) -- (69-1) -- (70-1)
    -- (71-1) -- (72-1) -- (73-1) -- (74-1) -- (75-1) -- (76-1) -- (77-1) -- (78-1) -- (79-1) -- (80-1)
    -- (81-1) -- (82-1) -- (83-1) -- (84-1) -- (85-1) -- (86-1) -- (87-1) -- (88-1) -- (89-1) -- (90-1)
    -- (91-1) -- (92-1) -- (93-1) -- (94-1) -- (95-1) -- (96-1) -- (97-1) -- (98-1) -- (99-1) -- (00-1);

    \draw[very thick, ForestGreen] 
       (01-2) -- (02-2) -- (03-2) -- (04-2) -- (05-2) -- (06-2) -- (07-2) -- (08-2) -- (09-2) -- (10-2)
    -- (11-2) -- (12-2) -- (13-2) -- (14-2) -- (15-2) -- (16-2) -- (17-2) -- (18-2) -- (19-2) -- (20-2)
    -- (21-2) -- (22-2) -- (23-2) -- (24-2) -- (25-2) -- (26-2) -- (27-2) -- (28-2) -- (29-2) -- (30-2)
    -- (31-2) -- (32-2) -- (33-2) -- (34-2) -- (35-2) -- (36-2) -- (37-2) -- (38-2) -- (39-2) -- (40-2)
    -- (41-2) -- (42-2) -- (43-2) -- (44-2) -- (45-2) -- (46-2) -- (47-2) -- (48-2) -- (49-2) -- (50-2)
    -- (51-2) -- (52-2) -- (53-2) -- (54-2) -- (55-2) -- (56-2) -- (57-2) -- (58-2) -- (59-2) -- (60-2)
    -- (61-2) -- (62-2) -- (63-2) -- (64-2) -- (65-2) -- (66-2) -- (67-2) -- (68-2) -- (69-2) -- (70-2)
    -- (71-2) -- (72-2) -- (73-2) -- (74-2) -- (75-2) -- (76-2) -- (77-2) -- (78-2) -- (79-2) -- (80-2)
    -- (81-2) -- (82-2) -- (83-2) -- (84-2) -- (85-2) -- (86-2) -- (87-2) -- (88-2) -- (89-2) -- (90-2)
    -- (91-2) -- (92-2) -- (93-2) -- (94-2) -- (95-2) -- (96-2) -- (97-2) -- (98-2) -- (99-2) -- (00-2);

    \draw[very thick, Aquamarine] 
       (01-3) -- (02-3) -- (03-3) -- (04-3) -- (05-3) -- (06-3) -- (07-3) -- (08-3) -- (09-3) -- (10-3)
    -- (11-3) -- (12-3) -- (13-3) -- (14-3) -- (15-3) -- (16-3) -- (17-3) -- (18-3) -- (19-3) -- (20-3)
    -- (21-3) -- (22-3) -- (23-3) -- (24-3) -- (25-3) -- (26-3) -- (27-3) -- (28-3) -- (29-3) -- (30-3)
    -- (31-3) -- (32-3) -- (33-3) -- (34-3) -- (35-3) -- (36-3) -- (37-3) -- (38-3) -- (39-3) -- (40-3)
    -- (41-3) -- (42-3) -- (43-3) -- (44-3) -- (45-3) -- (46-3) -- (47-3) -- (48-3) -- (49-3) -- (50-3)
    -- (51-3) -- (52-3) -- (53-3) -- (54-3) -- (55-3) -- (56-3) -- (57-3) -- (58-3) -- (59-3) -- (60-3)
    -- (61-3) -- (62-3) -- (63-3) -- (64-3) -- (65-3) -- (66-3) -- (67-3) -- (68-3) -- (69-3) -- (70-3)
    -- (71-3) -- (72-3) -- (73-3) -- (74-3) -- (75-3) -- (76-3) -- (77-3) -- (78-3) -- (79-3) -- (80-3)
    -- (81-3) -- (82-3) -- (83-3) -- (84-3) -- (85-3) -- (86-3) -- (87-3) -- (88-3) -- (89-3) -- (90-3)
    -- (91-3) -- (92-3) -- (93-3) -- (94-3) -- (95-3) -- (96-3) -- (97-3) -- (98-3) -- (99-3) -- (00-3);
\end{tikzpicture}

        \caption{Aproximações suaves da função $|x|$ (em preto) por meio da convolução com uma função molificadora $\eta_\varepsilon$ com $\varepsilon = 1, 0.5, 0.25$.\\Fonte: Autoral.}
        \label{fig:convolucao}
    \end{figure}

    \begin{figure} 
        \centering
        \begin{tikzpicture}
            \begin{scope}[shift={(-4,0)}, scale=0.8]
                \draw[black!30] (-4,-4) grid (4,4);
                \draw[thick, black!60, -stealth] (-4,0) to (4,0); 
                \draw[thick, black!60, -stealth] (0,-4) to (0,4); 
                \draw[domain=-0.9999:0.9999, variable=\x, samples=50, smooth, ultra thick, tangerine!90!black] plot ({\x}, {2.257 * exp(-1 / (1 - (\x * \x)))});
                \draw[ultra thick, tangerine!90!black] (-1,0) -- (-3,0);
                \draw[ultra thick, tangerine!90!black] (1,0) -- (3,0);
            \end{scope}

            \begin{scope}[shift={(4,0)}, scale=0.8]
                \draw[black!30] (-4,-4) grid (4,4);
                \draw[thick, black!60, -stealth] (-4,0) to (4,0); 
                \draw[thick, black!60, -stealth] (0,-4) to (0,4); 
                \draw[domain=-0.299:0.299, variable=\x, samples=50, smooth, ultra thick, tangerine!90!black] plot ({\x}, {(1/0.3) * 2.257 * exp(-1 / (1 - ((\x / 0.3) * (\x / 0.3))))});
                \draw[ultra thick, tangerine!90!black] (-0.3,0) -- (-3,0);
                \draw[ultra thick, tangerine!90!black] (0.3,0) -- (3,0);
            \end{scope}
        \end{tikzpicture}
        \caption{Funções $\eta$ e $\eta_{\varepsilon}$ com $\varepsilon = 0.3$.\\ Fonte: Autoral.}
        \label{fig:eta}
    \end{figure}
\end{ex}

\begin{ex}
    Seja $\Omega = (-1,1) \times (-1,1) \subseteq \bR^2$. A função $u : \Omega \to \bR$ dada por
    \[
        u(x_1,x_2) = |x_1|^{\frac{1}{2}} + |x_2|^{\frac{1}{2}},
    \]
    não possui derivada no sentido usual pelo fato de $|\,\cdot\,|$ não ser diferenciável.
    Por outro lado, $u$ possui derivadas parciais fracas em $\cL^p(\Omega)$, quando $0 < p < 2$, dada por
    \[
        D^{e_i}u(x_1,x_2) = \frac{1}{2}\sgn(x_i) |x_i|^{-\frac{1}{2}},
    \]
    para todo $i = 1,2$.
    Com efeito, vamos calcular a derivada parcial fraca em relação a $i$-ésima coordenada. Utilizando integração por partes (ver Teorema \ref{thm:integracao-por-partes}), obtemos
    \begin{equation} \label{eq:derivada-fraca-exemplo-legal}
        \int_\Omega u D^{e_i} \phi \,dx = \int_{\partial\Omega} u \phi \nu^i \,ds - \int_\Omega \phi D^{e_i}u \,dx,
    \end{equation}
    onde $\phi \in \cC^\infty_c(\Omega)$ e $\nu = (\nu_1, \nu_2)$ é o vetor normal unitário que aponta para fora em $\partial \Omega$.
    Para calcular $D^{e_i} u$ precisamos dividir o domínio $\Omega$ em $\Omega_1$, $\Omega_2$, $\Omega_3$ e $\Omega_4$, onde $\Omega_i$ é a restrição ao $i$-ésimo quadrante (ver Figura \ref{fig:dominio}).
    \begin{figure}[H]
        \centering
        \begin{tikzpicture}
            \filldraw[tangerine!30] (2,2) rectangle (0,0);
            \filldraw[tangerine!30] (-2,-2) rectangle (0,0);
            \filldraw[tangerine!40] (-2,2) rectangle (0,0);
            \filldraw[tangerine!40] (2,-2) rectangle (0,0);
            
            \draw[thick] (2,2) rectangle (-2,-2);
            \draw[very thick, -stealth] (-3,0) to (3,0) node [right] {$x_1$};
            \draw[very thick, -stealth] (0,-3) to (0,3) node [above] {$x_2$};
        
            \node[align=center] at (1,1) {
                $\Omega_1$\\
                $x_1 > 0$\\
                $x_2 > 0$
            };
        
            \node[align=center] at (-1,1) {
                $\Omega_2$\\
                $x_1 < 0$\\
                $x_2 > 0$
            };
        
            \node[align=center] at (1,-1) {
                $\Omega_4$\\
                $x_1 > 0$\\
                $x_2 < 0$
            };
        
            \node[align=center] at (-1,-1) {
                $\Omega_3$\\
                $x_1 < 0$\\
                $x_2 < 0$
            };
        
            \filldraw (0,2) circle (0.05);
            \filldraw (0,-2) circle (0.05);
            \filldraw (2,0) circle (0.05);
            \filldraw (-2,0) circle (0.05);
        
            \node[below right] at (2,0) {$1$};
            \node[above right] at (0,2) {$1$};
            \node[above left] at (-2,0) {$-1$};
            \node[below left] at (0,-2) {$-1$};
        \end{tikzpicture}
        \caption{Domínio da função $u(x_1,x_2) = |x_1|^{\frac{1}{2}} + |x_2|^{\frac{1}{2}}$.\\
        Fonte: Autoral.}
        \label{fig:dominio}
    \end{figure}
    % Note que em $\Omega_1 = (0,1) \times (0,1)$, $u(x_1,x_2) = x_1^{\frac{1}{2}} + x_2^{\frac{1}{2}}$. Logo, nesse conjunto $D^{e_i} u$ existe no sentido usual e é dada por
    % \[
    %     D^{e_i} u = \frac{1}{2} x_i^{-\frac{1}{2}}.
    % \]
    % De forma análoga, deduzimos que
    % \[
    %     \begin{array}{ll}
    %         D^{e_i} u(x_1,x_2) = \frac{1}{2} (-x_i)^{-\frac{1}{2}}, &\text{ em } \Omega_2 \text{ e } \Omega_3, \\
    %         D^{e_i} u(x_1,x_2) = \frac{1}{2} x_i^{-\frac{1}{2}}, &\text{ em } \Omega_4.
    %     \end{array}
    % \]
    Note que, podemos escrever
    \begin{equation}
        u(x_1,x_2) =
        \left\{
            \begin{array}{lr}
                x_1^{\frac{1}{2}} + x_2^{\frac{1}{2}}, &\;\text{ em } \Omega_1;\\
                (-x_1)^{\frac{1}{2}} + x_2^{\frac{1}{2}}, &\;\text{ em } \Omega_2;\\
                (-x_1)^{\frac{1}{2}} + (-x_2)^{\frac{1}{2}}, &\;\text{ em } \Omega_3;\\
                x_1^{\frac{1}{2}} + (-x_2)^{\frac{1}{2}}, &\;\text{ em } \Omega_4.\\
            \end{array}
        \right.
    \end{equation}
    Além disso, em cada $\Omega_i$, a derivada parcial $D^{e_1}u$ existe no sentido usual, e é dada por
    \begin{equation}
        D^{e_1}u(x_1,x_2) =
        \left\{
            \begin{array}{lr}
                \frac{1}{2}x_1^{-\frac{1}{2}}, &\;\text{ em } \Omega_1;\\
                -\frac{1}{2}(-x_1)^{-\frac{1}{2}}, &\;\text{ em } \Omega_2;\\
                -\frac{1}{2}(-x_1)^{-\frac{1}{2}}, &\;\text{ em } \Omega_3;\\
                \frac{1}{2}x_1^{-\frac{1}{2}}, &\;\text{ em } \Omega_4.\\
            \end{array}
        \right.
    \end{equation}
    Dito isso, concluímos que
    \[
        \int_\Omega \phi D^{e_1}u \,dx = \int_{\Omega_1}  \frac{1}{2} x_1^{-\frac{1}{2}}\phi \,dx - \int_{\Omega_2} \frac{1}{2} (-x_1)^{-\frac{1}{2}} \phi \,dx - \int_{\Omega_3} \frac{1}{2} (-x_1)^{-\frac{1}{2}} \phi \,dx + \int_{\Omega_4} \frac{1}{2} x_1^{-\frac{1}{2}} \phi \,dx,
    \]
    que podemos escrever como
    \[
        \int_\Omega \phi D^{e_1} u \,dx = \int_{\Omega_1 \cup \Omega_4}  \frac{1}{2} x_1^{-\frac{1}{2}}\phi \,dx - \int_{\Omega_2 \cup \Omega_3} \frac{1}{2} (-x_1)^{-\frac{1}{2}} \phi \,dx = \frac{1}{2}\int_\Omega \sgn(x_1) |x_1|^{-\frac{1}{2}} \phi \,dx.
    \]
    De forma análoga, inferimos que
    \begin{equation}
        D^{e_2}u(x_1,x_2) =
        \left\{
            \begin{array}{lr}
                \frac{1}{2}x_2^{-\frac{1}{2}}, &\;\text{ em } \Omega_1;\\
                \frac{1}{2}x_2^{-\frac{1}{2}}, &\;\text{ em } \Omega_2;\\
                -\frac{1}{2}(-x_2)^{-\frac{1}{2}}, &\;\text{ em } \Omega_3;\\
                -\frac{1}{2}(-x_2)^{-\frac{1}{2}}, &\;\text{ em } \Omega_4.\\
            \end{array}
        \right.
    \end{equation}
    Sendo assim, deduzimos que
    \[
        \int_\Omega \phi D^{e_2} u \,dx = \int_{\Omega_1 \cup \Omega_2}  \frac{1}{2} x_2^{-\frac{1}{2}}\phi \,dx - \int_{\Omega_3 \cup \Omega_4} \frac{1}{2} (-x_2)^{-\frac{1}{2}} \phi \,dx = \frac{1}{2}\int_\Omega \sgn(x_2) |x_2|^{-\frac{1}{2}} \phi \,dx.
    \]
    Por fim, como $\phi$ tem suporte compacto em $\Omega$ (e $\Omega$ é aberto), $\phi$ se anula em $\partial\Omega$.
    Dessa forma, inferimos que
    \[
        \int_{\partial\Omega} u \phi \nu^i \,ds = 0.
    \]
    Portanto, por (\ref{eq:derivada-fraca-exemplo-legal}), chegamos a
    \[
        \int_\Omega  u D^{e_i} \phi \,dx = - \int_\Omega \frac{1}{2}\sgn(x_i) |x_i|^{-\frac{1}{2}} \phi \,dx,
    \]
    para todo $i=1,2$.
    Isto é,
    \[
        D^{e_i} u(x_1,x_2) = \frac{1}{2}\sgn(x_i) |x_i|^{-\frac{1}{2}},
    \]
    para todo $i = 1,2$, como queriamos mostrar.
    Além disso, podemos escrever
    \[
        \int_\Omega |D^{e_i} u (x_1,x_2)|^p \, dx = \int_{\text{-}1}^1 \int_{\text{-}1}^1 \left| \frac{1}{2} \sgn(x_i) |x_i|^{-\frac{1}{2}} \right|^p \, dx_i dx_j = \frac{1}{2^p}\int_{\text{-}1}^1 \int_{\text{-}1}^1 |\sgn(x_i)|^p |x_i|^{-\frac{p}{2}} \,dx_i dx_j.
    \]
    Utilizando o Teorema de Fubini (ver Teorema \ref{thm:fubini}), encontramos
    {\small
    \[
        \int_\Omega |D^{e_i} u (x_1,x_2)|^p \, dx = \frac{1}{2^p}\int_{\text{-}1}^1 dx_j \int_{\text{-}1}^1 |\sgn(x_i)|^p |x_i|^{-\frac{p}{2}} \,dx_i = \frac{1}{2^{p-1}} \int_{\text{-}1}^1 |\sgn(x_i)|^p |x_i|^{-\frac{p}{2}} \,dx_i.
    \]}\!
    Dito isso, precisamos ver para quais valores de $p$ essa integral é finita. Sendo assim
    \[
        \int_\Omega |D^{e_i} u (x_1,x_2)|^p \, dx = \frac{1}{2^{p-2}}\int_0^1 x_i^{-\frac{p}{2}} \,dx = \frac{1}{2^{p-2}} \left[ -\frac{1^{-\frac{p}{2}+1} - 0^{-\frac{p}{2}+1}}{\tfrac{p}{2} - 1} \right] = \frac{1}{2^{p-2}\left( 1 - \frac{p}{2} \right)}.
    \]
    Note que, a igualdade acima está bem definida quando $p < 2$. Portanto, $u \in \cW^{1,p}(\Omega)$ desde que $0 < p < 2$.

    Agora defina $\eta: \bR^2 \to \bR$ por
    \[
        \eta(x) = 
        \left\{ 
            \begin{array}{lr}
                c \exp\left(\frac{1}{\Vert x \Vert^2 - 1} \right), & \text{se } \Vert x \Vert < 1;\\
                0, & \text{se } \Vert x \Vert \geqslant 1,
            \end{array}
        \right.
    \]
    (ver Figura \ref{fig:eta-R2}) e $\eta_\varepsilon$ da mesma forma que foi feita no exemplo anterior. Novamente utilizaremos a convolução para encontrar uma aproximação suave para $u$, dada por
    \[
        u^{\varepsilon}(x_1,x_2) = \int_\Omega \eta(x_1,x_2) u(x_1 - y_1, x_2 - y_2) \,dy.
    \]

    Utilizando um método numérico para integrais duplas, podemos encontrar uma solução aproximada para $u^\varepsilon$.
    A Figura \ref{fig:aproximacao-suave-R2} mostra o gráfico de $u$, onde é possível ver os pontos onde a função não é diferenciável, e a sua aproximação suave $u^\varepsilon$.

    \begin{figure}
        \centering
        \includegraphics[width=0.4\textwidth]{2η.pdf}
        \hspace{10mm}
        \includegraphics[width=0.4\textwidth]{2ηε.pdf}
        \caption{$\eta$ e $\eta_\varepsilon$ com $\varepsilon = 0.5$.\\Fonte: Autoral.}
        \label{fig:eta-R2}
    \end{figure}

    \begin{figure}
        \centering
        \includegraphics[width=0.4\textwidth]{u.pdf}
        \hspace{10mm}
        \includegraphics[width=0.4\textwidth]{uε2.pdf}
        \caption{À esquerda, a função $u(x_1,x_2) = |x_1|^{\frac{1}{2}} + |x_2|^{\frac{1}{2}}$ e à direita sua aproximação\\suave $u^\varepsilon$ com $\varepsilon = 0.25$.\\Fonte: Autoral.}
        \label{fig:aproximacao-suave-R2}
    \end{figure}
\end{ex}

Agora, mostraremos que $\cC^{\infty}(\Omega) \cap \cW^{k,p}(\Omega)$ é um conjunto denso em $\cW^{k,p}(\Omega)$, sempre que 

\begin{tbox}[Meyers-Serrin] \label{thm:aprox-2}
    Sejam $\Omega$ um aberto limitado e $u \in \cW^{k,p}(\Omega)$, com $1 \leqslant p < \infty$.
    Então, existe uma sequência $(u_n) \subseteq \cC^{\infty}(\Omega) \cap \cW^{k,p}(\Omega)$ tal que
    \[
        u_n \to u \;\text{ em } \cW^{k,p}(\Omega),
    \] 
    quando $n \to \infty$.
\end{tbox}
\begin{prf}
    Primeiramente, temos que
    \begin{equation} \label{eq:omegai}
        \Omega = \bigcup_{i=1}^\infty \Omega_i,
    \end{equation}
    onde $\Omega_i = \{x \in \Omega \,; d(x, \partial\Omega) > \tfrac{1}{i}\}$ (ver Figura \ref{fig:omegai}). De fato,
    se $x \in \bigcup_{i=1}^\infty \Omega_i$, então $x \in \Omega_{i_0}$ para algum $i_0 \in \bN$, em particular, $x \in \Omega$ pois $\Omega_i \subseteq \Omega$ para todo $i \in \bN$.
    Por outro lado, se $x \in \Omega$, então $d(x,\partial \Omega) > 0$ (pois $\partial\Omega$ é fechado e $x \not\in \partial\Omega$), então existe algum $i_0 \in \bN$ tal que $d(x, \partial \Omega) > \frac{1}{i_0}$; logo, $x \in \Omega_{i_0}$. Dessa forma, $x \in \bigcup_{i=1}^\infty \Omega_i$.
    Portanto, vale (\ref{eq:omegai}). 
    \begin{figure}
        \centering
        \input{mollifiers/omega_i.tex}
        \caption{Representação visual dos conjuntos $\Omega_i$.\\Fonte: Autoral.}
        \label{fig:omegai}
    \end{figure}
    
    Defina $\Omega'_i = \Omega_{i+3} \setminus \overline{\Omega_{i+1}}$ (ver Figura \ref{fig:omegaii}).
    Além disso, escolha qualquer aberto $\Omega'_0 \Subset \Omega$ de forma que
    \[
        \Omega = \bigcup_{i=0}^\infty \Omega'_i
    \]
    \begin{figure}
        \centering
        \begin{tikzpicture}[scale=0.75]
    % Original curve coordinates
\coordinate (A-0) at (-3.7400,0.0000);
\coordinate (A-1) at (-3.7118,0.1210);
\coordinate (A-2) at (-3.6782,0.2403);
\coordinate (A-3) at (-3.6394,0.3578);
\coordinate (A-4) at (-3.5954,0.4735);
\coordinate (A-5) at (-3.5463,0.5871);
\coordinate (A-6) at (-3.4923,0.6987);
\coordinate (A-7) at (-3.4335,0.8081);
\coordinate (A-8) at (-3.3700,0.9153);
\coordinate (A-9) at (-3.3018,1.0201);
\coordinate (A-10) at (-3.2291,1.1224);
\coordinate (A-11) at (-3.1521,1.2222);
\coordinate (A-12) at (-3.0708,1.3194);
\coordinate (A-13) at (-2.9853,1.4138);
\coordinate (A-14) at (-2.8957,1.5053);
\coordinate (A-15) at (-2.8022,1.5940);
\coordinate (A-16) at (-2.7049,1.6796);
\coordinate (A-17) at (-2.6039,1.7621);
\coordinate (A-18) at (-2.4992,1.8414);
\coordinate (A-19) at (-2.3910,1.9174);
\coordinate (A-20) at (-2.2795,1.9899);
\coordinate (A-21) at (-2.1647,2.0590);
\coordinate (A-22) at (-2.0467,2.1244);
\coordinate (A-23) at (-1.9256,2.1862);
\coordinate (A-24) at (-1.8016,2.2442);
\coordinate (A-25) at (-1.6748,2.2983);
\coordinate (A-26) at (-1.5452,2.3484);
\coordinate (A-27) at (-1.4130,2.3945);
\coordinate (A-28) at (-1.2783,2.4364);
\coordinate (A-29) at (-1.1413,2.4740);
\coordinate (A-30) at (-1.0019,2.5073);
\coordinate (A-31) at (-0.8604,2.5361);
\coordinate (A-32) at (-0.7168,2.5604);
\coordinate (A-33) at (-0.5713,2.5800);
\coordinate (A-34) at (-0.4239,2.5949);
\coordinate (A-35) at (-0.2748,2.6049);
\coordinate (A-36) at (-0.1240,2.6101);
\coordinate (A-37) at (0.0282,2.6102);
\coordinate (A-38) at (0.1818,2.6052);
\coordinate (A-39) at (0.3367,2.5949);
\coordinate (A-40) at (0.4928,2.5794);
\coordinate (A-41) at (0.6500,2.5584);
\coordinate (A-42) at (0.8081,2.5320);
\coordinate (A-43) at (0.9670,2.5000);
\coordinate (A-44) at (1.1267,2.4622);
\coordinate (A-45) at (1.2870,2.4187);
\coordinate (A-46) at (1.4478,2.3693);
\coordinate (A-47) at (1.6090,2.3139);
\coordinate (A-48) at (1.7704,2.2525);
\coordinate (A-49) at (1.9321,2.1849);
\coordinate (A-50) at (2.0938,2.1110);
\coordinate (A-51) at (2.2555,2.0308);
\coordinate (A-52) at (2.4170,1.9441);
\coordinate (A-53) at (2.5783,1.8509);
\coordinate (A-54) at (2.7392,1.7510);
\coordinate (A-55) at (2.8996,1.6444);
\coordinate (A-56) at (3.0594,1.5310);
\coordinate (A-57) at (3.2185,1.4107);
\coordinate (A-58) at (3.3767,1.2833);
\coordinate (A-59) at (3.5341,1.1488);
\coordinate (A-60) at (3.6904,1.0071);
\coordinate (A-61) at (3.8455,0.8580);
\coordinate (A-62) at (3.9994,0.7016);
\coordinate (A-63) at (4.1519,0.5377);
\coordinate (A-64) at (4.3029,0.3662);
\coordinate (A-65) at (4.4523,0.1870);
\coordinate (A-66) at (4.6000,0.0000);
\coordinate (A-67) at (4.6805,-0.1065);
\coordinate (A-68) at (4.7573,-0.2122);
\coordinate (A-69) at (4.8305,-0.3171);
\coordinate (A-70) at (4.9000,-0.4212);
\coordinate (A-71) at (4.9660,-0.5244);
\coordinate (A-72) at (5.0283,-0.6268);
\coordinate (A-73) at (5.0870,-0.7283);
\coordinate (A-74) at (5.1421,-0.8288);
\coordinate (A-75) at (5.1935,-0.9283);
\coordinate (A-76) at (5.2414,-1.0268);
\coordinate (A-77) at (5.2856,-1.1243);
\coordinate (A-78) at (5.3263,-1.2207);
\coordinate (A-79) at (5.3633,-1.3160);
\coordinate (A-80) at (5.3968,-1.4101);
\coordinate (A-81) at (5.4266,-1.5030);
\coordinate (A-82) at (5.4529,-1.5948);
\coordinate (A-83) at (5.4756,-1.6852);
\coordinate (A-84) at (5.4947,-1.7744);
\coordinate (A-85) at (5.5102,-1.8623);
\coordinate (A-86) at (5.5221,-1.9489);
\coordinate (A-87) at (5.5304,-2.0340);
\coordinate (A-88) at (5.5352,-2.1178);
\coordinate (A-89) at (5.5364,-2.2001);
\coordinate (A-90) at (5.5340,-2.2809);
\coordinate (A-91) at (5.5281,-2.3602);
\coordinate (A-92) at (5.5186,-2.4380);
\coordinate (A-93) at (5.5055,-2.5142);
\coordinate (A-94) at (5.4889,-2.5887);
\coordinate (A-95) at (5.4687,-2.6617);
\coordinate (A-96) at (5.4450,-2.7329);
\coordinate (A-97) at (5.4177,-2.8024);
\coordinate (A-98) at (5.3869,-2.8702);
\coordinate (A-99) at (5.3525,-2.9362);
\coordinate (A-100) at (5.3146,-3.0005);
\coordinate (A-101) at (5.2732,-3.0628);
\coordinate (A-102) at (5.2282,-3.1233);
\coordinate (A-103) at (5.1797,-3.1819);
\coordinate (A-104) at (5.1276,-3.2385);
\coordinate (A-105) at (5.0721,-3.2931);
\coordinate (A-106) at (5.0130,-3.3458);
\coordinate (A-107) at (4.9504,-3.3964);
\coordinate (A-108) at (4.8843,-3.4449);
\coordinate (A-109) at (4.8146,-3.4913);
\coordinate (A-110) at (4.7415,-3.5356);
\coordinate (A-111) at (4.6648,-3.5776);
\coordinate (A-112) at (4.5847,-3.6175);
\coordinate (A-113) at (4.5010,-3.6552);
\coordinate (A-114) at (4.4139,-3.6905);
\coordinate (A-115) at (4.3232,-3.7236);
\coordinate (A-116) at (4.2291,-3.7543);
\coordinate (A-117) at (4.1314,-3.7826);
\coordinate (A-118) at (4.0303,-3.8085);
\coordinate (A-119) at (3.9257,-3.8320);
\coordinate (A-120) at (3.8176,-3.8530);
\coordinate (A-121) at (3.7060,-3.8715);
\coordinate (A-122) at (3.5910,-3.8875);
\coordinate (A-123) at (3.4725,-3.9009);
\coordinate (A-124) at (3.3505,-3.9116);
\coordinate (A-125) at (3.2250,-3.9198);
\coordinate (A-126) at (3.0961,-3.9252);
\coordinate (A-127) at (2.9637,-3.9280);
\coordinate (A-128) at (2.8279,-3.9280);
\coordinate (A-129) at (2.6886,-3.9253);
\coordinate (A-130) at (2.5458,-3.9197);
\coordinate (A-131) at (2.3996,-3.9113);
\coordinate (A-132) at (2.2500,-3.9000);
\coordinate (A-133) at (2.0718,-3.8846);
\coordinate (A-134) at (1.8956,-3.8683);
\coordinate (A-135) at (1.7212,-3.8511);
\coordinate (A-136) at (1.5488,-3.8331);
\coordinate (A-137) at (1.3784,-3.8141);
\coordinate (A-138) at (1.2100,-3.7941);
\coordinate (A-139) at (1.0437,-3.7732);
\coordinate (A-140) at (0.8795,-3.7513);
\coordinate (A-141) at (0.7175,-3.7283);
\coordinate (A-142) at (0.5576,-3.7043);
\coordinate (A-143) at (0.4000,-3.6793);
\coordinate (A-144) at (0.2446,-3.6532);
\coordinate (A-145) at (0.0915,-3.6259);
\coordinate (A-146) at (-0.0592,-3.5976);
\coordinate (A-147) at (-0.2076,-3.5680);
\coordinate (A-148) at (-0.3535,-3.5373);
\coordinate (A-149) at (-0.4971,-3.5054);
\coordinate (A-150) at (-0.6381,-3.4723);
\coordinate (A-151) at (-0.7766,-3.4379);
\coordinate (A-152) at (-0.9125,-3.4023);
\coordinate (A-153) at (-1.0458,-3.3654);
\coordinate (A-154) at (-1.1765,-3.3271);
\coordinate (A-155) at (-1.3045,-3.2875);
\coordinate (A-156) at (-1.4298,-3.2466);
\coordinate (A-157) at (-1.5524,-3.2043);
\coordinate (A-158) at (-1.6721,-3.1605);
\coordinate (A-159) at (-1.7891,-3.1154);
\coordinate (A-160) at (-1.9031,-3.0688);
\coordinate (A-161) at (-2.0143,-3.0207);
\coordinate (A-162) at (-2.1225,-2.9711);
\coordinate (A-163) at (-2.2277,-2.9201);
\coordinate (A-164) at (-2.3300,-2.8674);
\coordinate (A-165) at (-2.4291,-2.8133);
\coordinate (A-166) at (-2.5252,-2.7575);
\coordinate (A-167) at (-2.6181,-2.7001);
\coordinate (A-168) at (-2.7079,-2.6411);
\coordinate (A-169) at (-2.7945,-2.5804);
\coordinate (A-170) at (-2.8778,-2.5181);
\coordinate (A-171) at (-2.9578,-2.4540);
\coordinate (A-172) at (-3.0346,-2.3883);
\coordinate (A-173) at (-3.1079,-2.3208);
\coordinate (A-174) at (-3.1779,-2.2515);
\coordinate (A-175) at (-3.2444,-2.1804);
\coordinate (A-176) at (-3.3075,-2.1076);
\coordinate (A-177) at (-3.3670,-2.0328);
\coordinate (A-178) at (-3.4230,-1.9563);
\coordinate (A-179) at (-3.4755,-1.8778);
\coordinate (A-180) at (-3.5243,-1.7975);
\coordinate (A-181) at (-3.5694,-1.7152);
\coordinate (A-182) at (-3.6108,-1.6310);
\coordinate (A-183) at (-3.6485,-1.5448);
\coordinate (A-184) at (-3.6825,-1.4566);
\coordinate (A-185) at (-3.7126,-1.3664);
\coordinate (A-186) at (-3.7388,-1.2742);
\coordinate (A-187) at (-3.7612,-1.1799);
\coordinate (A-188) at (-3.7797,-1.0835);
\coordinate (A-189) at (-3.7942,-0.9850);
\coordinate (A-190) at (-3.8047,-0.8844);
\coordinate (A-191) at (-3.8111,-0.7816);
\coordinate (A-192) at (-3.8135,-0.6766);
\coordinate (A-193) at (-3.8117,-0.5695);
\coordinate (A-194) at (-3.8059,-0.4601);
\coordinate (A-195) at (-3.7958,-0.3485);
\coordinate (A-196) at (-3.7815,-0.2346);
\coordinate (A-197) at (-3.7629,-0.1185);
\coordinate (A-198) at (-3.7400,0.0000);

% Offset 1 curve coordinates
\coordinate (B-0) at (-3.1041,-0.1354);
\coordinate (B-1) at (-3.0821,-0.0408);
\coordinate (B-2) at (-3.0565,0.0502);
\coordinate (B-3) at (-3.0268,0.1401);
\coordinate (B-4) at (-2.9930,0.2290);
\coordinate (B-5) at (-2.9551,0.3166);
\coordinate (B-6) at (-2.9133,0.4031);
\coordinate (B-7) at (-2.8675,0.4883);
\coordinate (B-8) at (-2.8177,0.5722);
\coordinate (B-9) at (-2.7641,0.6546);
\coordinate (B-10) at (-2.7067,0.7355);
\coordinate (B-11) at (-2.6454,0.8148);
\coordinate (B-12) at (-2.5805,0.8924);
\coordinate (B-13) at (-2.5118,0.9682);
\coordinate (B-14) at (-2.4396,1.0421);
\coordinate (B-15) at (-2.3638,1.1140);
\coordinate (B-16) at (-2.2846,1.1837);
\coordinate (B-17) at (-2.2019,1.2511);
\coordinate (B-18) at (-2.1160,1.3162);
\coordinate (B-19) at (-2.0269,1.3788);
\coordinate (B-20) at (-1.9347,1.4388);
\coordinate (B-21) at (-1.8394,1.4961);
\coordinate (B-22) at (-1.7412,1.5506);
\coordinate (B-23) at (-1.6401,1.6022);
\coordinate (B-24) at (-1.5363,1.6507);
\coordinate (B-25) at (-1.4299,1.6961);
\coordinate (B-26) at (-1.3209,1.7383);
\coordinate (B-27) at (-1.2095,1.7771);
\coordinate (B-28) at (-1.0957,1.8125);
\coordinate (B-29) at (-0.9797,1.8443);
\coordinate (B-30) at (-0.8616,1.8725);
\coordinate (B-31) at (-0.7413,1.8970);
\coordinate (B-32) at (-0.6192,1.9177);
\coordinate (B-33) at (-0.4951,1.9344);
\coordinate (B-34) at (-0.3693,1.9471);
\coordinate (B-35) at (-0.2418,1.9557);
\coordinate (B-36) at (-0.1128,1.9601);
\coordinate (B-37) at (0.0178,1.9602);
\coordinate (B-38) at (0.1498,1.9559);
\coordinate (B-39) at (0.2831,1.9470);
\coordinate (B-40) at (0.4177,1.9336);
\coordinate (B-41) at (0.5534,1.9156);
\coordinate (B-42) at (0.6902,1.8927);
\coordinate (B-43) at (0.8280,1.8649);
\coordinate (B-44) at (0.9667,1.8321);
\coordinate (B-45) at (1.1063,1.7942);
\coordinate (B-46) at (1.2467,1.7511);
\coordinate (B-47) at (1.3877,1.7026);
\coordinate (B-48) at (1.5294,1.6487);
\coordinate (B-49) at (1.6716,1.5893);
\coordinate (B-50) at (1.8143,1.5241);
\coordinate (B-51) at (1.9573,1.4531);
\coordinate (B-52) at (2.1006,1.3762);
\coordinate (B-53) at (2.2442,1.2933);
\coordinate (B-54) at (2.3878,1.2041);
\coordinate (B-55) at (2.5315,1.1086);
\coordinate (B-56) at (2.6751,1.0067);
\coordinate (B-57) at (2.8185,0.8982);
\coordinate (B-58) at (2.9617,0.7829);
\coordinate (B-59) at (3.1045,0.6609);
\coordinate (B-60) at (3.2468,0.5318);
\coordinate (B-61) at (3.3886,0.3957);
\coordinate (B-62) at (3.5296,0.2522);
\coordinate (B-63) at (3.6699,0.1015);
\coordinate (B-64) at (3.8092,-0.0568);
\coordinate (B-65) at (3.9476,-0.2227);
\coordinate (B-66) at (4.0856,-0.3975);
\coordinate (B-67) at (4.1582,-0.4935);
\coordinate (B-68) at (4.2278,-0.5892);
\coordinate (B-69) at (4.2936,-0.6836);
\coordinate (B-70) at (4.3558,-0.7767);
\coordinate (B-71) at (4.4144,-0.8684);
\coordinate (B-72) at (4.4693,-0.9586);
\coordinate (B-73) at (4.5206,-1.0473);
\coordinate (B-74) at (4.5683,-1.1343);
\coordinate (B-75) at (4.6124,-1.2197);
\coordinate (B-76) at (4.6530,-1.3033);
\coordinate (B-77) at (4.6902,-1.3850);
\coordinate (B-78) at (4.7238,-1.4648);
\coordinate (B-79) at (4.7541,-1.5426);
\coordinate (B-80) at (4.7810,-1.6183);
\coordinate (B-81) at (4.8046,-1.6919);
\coordinate (B-82) at (4.8250,-1.7632);
\coordinate (B-83) at (4.8423,-1.8323);
\coordinate (B-84) at (4.8566,-1.8989);
\coordinate (B-85) at (4.8679,-1.9632);
\coordinate (B-86) at (4.8764,-2.0249);
\coordinate (B-87) at (4.8822,-2.0842);
\coordinate (B-88) at (4.8855,-2.1410);
\coordinate (B-89) at (4.8862,-2.1952);
\coordinate (B-90) at (4.8847,-2.2471);
\coordinate (B-91) at (4.8810,-2.2965);
\coordinate (B-92) at (4.8753,-2.3436);
\coordinate (B-93) at (4.8676,-2.3884);
\coordinate (B-94) at (4.8580,-2.4313);
\coordinate (B-95) at (4.8467,-2.4722);
\coordinate (B-96) at (4.8336,-2.5114);
\coordinate (B-97) at (4.8189,-2.5491);
\coordinate (B-98) at (4.8023,-2.5854);
\coordinate (B-99) at (4.7840,-2.6207);
\coordinate (B-100) at (4.7637,-2.6551);
\coordinate (B-101) at (4.7413,-2.6887);
\coordinate (B-102) at (4.7167,-2.7218);
\coordinate (B-103) at (4.6897,-2.7544);
\coordinate (B-104) at (4.6601,-2.7866);
\coordinate (B-105) at (4.6277,-2.8185);
\coordinate (B-106) at (4.5923,-2.8500);
\coordinate (B-107) at (4.5536,-2.8813);
\coordinate (B-108) at (4.5116,-2.9121);
\coordinate (B-109) at (4.4660,-2.9425);
\coordinate (B-110) at (4.4167,-2.9724);
\coordinate (B-111) at (4.3635,-3.0015);
\coordinate (B-112) at (4.3065,-3.0299);
\coordinate (B-113) at (4.2454,-3.0574);
\coordinate (B-114) at (4.1803,-3.0838);
\coordinate (B-115) at (4.1110,-3.1091);
\coordinate (B-116) at (4.0376,-3.1330);
\coordinate (B-117) at (3.9601,-3.1555);
\coordinate (B-118) at (3.8784,-3.1765);
\coordinate (B-119) at (3.7925,-3.1957);
\coordinate (B-120) at (3.7024,-3.2133);
\coordinate (B-121) at (3.6082,-3.2289);
\coordinate (B-122) at (3.5098,-3.2425);
\coordinate (B-123) at (3.4074,-3.2541);
\coordinate (B-124) at (3.3008,-3.2635);
\coordinate (B-125) at (3.1902,-3.2707);
\coordinate (B-126) at (3.0756,-3.2755);
\coordinate (B-127) at (2.9569,-3.2780);
\coordinate (B-128) at (2.8343,-3.2780);
\coordinate (B-129) at (2.7077,-3.2755);
\coordinate (B-130) at (2.5772,-3.2704);
\coordinate (B-131) at (2.4427,-3.2627);
\coordinate (B-132) at (2.3025,-3.2521);
\coordinate (B-133) at (2.1298,-3.2372);
\coordinate (B-134) at (1.9573,-3.2212);
\coordinate (B-135) at (1.7869,-3.2045);
\coordinate (B-136) at (1.6187,-3.1868);
\coordinate (B-137) at (1.4526,-3.1683);
\coordinate (B-138) at (1.2888,-3.1489);
\coordinate (B-139) at (1.1273,-3.1286);
\coordinate (B-140) at (0.9681,-3.1073);
\coordinate (B-141) at (0.8112,-3.0851);
\coordinate (B-142) at (0.6568,-3.0620);
\coordinate (B-143) at (0.5049,-3.0378);
\coordinate (B-144) at (0.3554,-3.0127);
\coordinate (B-145) at (0.2086,-2.9866);
\coordinate (B-146) at (0.0643,-2.9594);
\coordinate (B-147) at (-0.0772,-2.9312);
\coordinate (B-148) at (-0.2161,-2.9020);
\coordinate (B-149) at (-0.3522,-2.8718);
\coordinate (B-150) at (-0.4855,-2.8405);
\coordinate (B-151) at (-0.6158,-2.8081);
\coordinate (B-152) at (-0.7433,-2.7747);
\coordinate (B-153) at (-0.8678,-2.7402);
\coordinate (B-154) at (-0.9892,-2.7047);
\coordinate (B-155) at (-1.1076,-2.6681);
\coordinate (B-156) at (-1.2228,-2.6304);
\coordinate (B-157) at (-1.3348,-2.5917);
\coordinate (B-158) at (-1.4436,-2.5520);
\coordinate (B-159) at (-1.5490,-2.5113);
\coordinate (B-160) at (-1.6512,-2.4696);
\coordinate (B-161) at (-1.7499,-2.4269);
\coordinate (B-162) at (-1.8452,-2.3832);
\coordinate (B-163) at (-1.9370,-2.3387);
\coordinate (B-164) at (-2.0253,-2.2932);
\coordinate (B-165) at (-2.1101,-2.2469);
\coordinate (B-166) at (-2.1912,-2.1998);
\coordinate (B-167) at (-2.2688,-2.1519);
\coordinate (B-168) at (-2.3428,-2.1032);
\coordinate (B-169) at (-2.4132,-2.0539);
\coordinate (B-170) at (-2.4800,-2.0039);
\coordinate (B-171) at (-2.5432,-1.9534);
\coordinate (B-172) at (-2.6028,-1.9022);
\coordinate (B-173) at (-2.6590,-1.8505);
\coordinate (B-174) at (-2.7118,-1.7983);
\coordinate (B-175) at (-2.7612,-1.7455);
\coordinate (B-176) at (-2.8074,-1.6921);
\coordinate (B-177) at (-2.8504,-1.6382);
\coordinate (B-178) at (-2.8903,-1.5837);
\coordinate (B-179) at (-2.9272,-1.5284);
\coordinate (B-180) at (-2.9613,-1.4723);
\coordinate (B-181) at (-2.9926,-1.4153);
\coordinate (B-182) at (-3.0212,-1.3571);
\coordinate (B-183) at (-3.0471,-1.2977);
\coordinate (B-184) at (-3.0706,-1.2369);
\coordinate (B-185) at (-3.0914,-1.1744);
\coordinate (B-186) at (-3.1097,-1.1100);
\coordinate (B-187) at (-3.1255,-1.0436);
\coordinate (B-188) at (-3.1386,-0.9750);
\coordinate (B-189) at (-3.1491,-0.9039);
\coordinate (B-190) at (-3.1568,-0.8303);
\coordinate (B-191) at (-3.1616,-0.7539);
\coordinate (B-192) at (-3.1634,-0.6746);
\coordinate (B-193) at (-3.1620,-0.5923);
\coordinate (B-194) at (-3.1574,-0.5069);
\coordinate (B-195) at (-3.1494,-0.4183);
\coordinate (B-196) at (-3.1379,-0.3265);
\coordinate (B-197) at (-3.1227,-0.2315);
\coordinate (B-198) at (-3.1041,-0.1354);

% Offset 2 curve coordinates
\coordinate (C-0) at (-2.4683,-0.2708);
\coordinate (C-1) at (-2.4524,-0.2027);
\coordinate (C-2) at (-2.4348,-0.1399);
\coordinate (C-3) at (-2.4141,-0.0775);
\coordinate (C-4) at (-2.3905,-0.0155);
\coordinate (C-5) at (-2.3639,0.0461);
\coordinate (C-6) at (-2.3342,0.1075);
\coordinate (C-7) at (-2.3014,0.1685);
\coordinate (C-8) at (-2.2655,0.2290);
\coordinate (C-9) at (-2.2265,0.2891);
\coordinate (C-10) at (-2.1842,0.3486);
\coordinate (C-11) at (-2.1388,0.4074);
\coordinate (C-12) at (-2.0902,0.4655);
\coordinate (C-13) at (-2.0384,0.5227);
\coordinate (C-14) at (-1.9834,0.5789);
\coordinate (C-15) at (-1.9254,0.6339);
\coordinate (C-16) at (-1.8642,0.6877);
\coordinate (C-17) at (-1.8000,0.7401);
\coordinate (C-18) at (-1.7329,0.7910);
\coordinate (C-19) at (-1.6628,0.8402);
\coordinate (C-20) at (-1.5898,0.8877);
\coordinate (C-21) at (-1.5141,0.9332);
\coordinate (C-22) at (-1.4357,0.9767);
\coordinate (C-23) at (-1.3546,1.0181);
\coordinate (C-24) at (-1.2711,1.0572);
\coordinate (C-25) at (-1.1850,1.0939);
\coordinate (C-26) at (-1.0967,1.1281);
\coordinate (C-27) at (-1.0060,1.1597);
\coordinate (C-28) at (-0.9131,1.1886);
\coordinate (C-29) at (-0.8182,1.2146);
\coordinate (C-30) at (-0.7212,1.2378);
\coordinate (C-31) at (-0.6223,1.2579);
\coordinate (C-32) at (-0.5215,1.2749);
\coordinate (C-33) at (-0.4190,1.2888);
\coordinate (C-34) at (-0.3148,1.2993);
\coordinate (C-35) at (-0.2089,1.3064);
\coordinate (C-36) at (-0.1015,1.3101);
\coordinate (C-37) at (0.0074,1.3102);
\coordinate (C-38) at (0.1177,1.3066);
\coordinate (C-39) at (0.2295,1.2992);
\coordinate (C-40) at (0.3425,1.2879);
\coordinate (C-41) at (0.4568,1.2727);
\coordinate (C-42) at (0.5723,1.2534);
\coordinate (C-43) at (0.6890,1.2298);
\coordinate (C-44) at (0.8068,1.2020);
\coordinate (C-45) at (0.9257,1.1697);
\coordinate (C-46) at (1.0456,1.1329);
\coordinate (C-47) at (1.1665,1.0914);
\coordinate (C-48) at (1.2884,1.0450);
\coordinate (C-49) at (1.4111,0.9937);
\coordinate (C-50) at (1.5347,0.9372);
\coordinate (C-51) at (1.6591,0.8755);
\coordinate (C-52) at (1.7842,0.8084);
\coordinate (C-53) at (1.9100,0.7356);
\coordinate (C-54) at (2.0364,0.6572);
\coordinate (C-55) at (2.1634,0.5728);
\coordinate (C-56) at (2.2908,0.4824);
\coordinate (C-57) at (2.4186,0.3857);
\coordinate (C-58) at (2.5466,0.2826);
\coordinate (C-59) at (2.6749,0.1730);
\coordinate (C-60) at (2.8033,0.0566);
\coordinate (C-61) at (2.9316,-0.0667);
\coordinate (C-62) at (3.0599,-0.1971);
\coordinate (C-63) at (3.1879,-0.3347);
\coordinate (C-64) at (3.3156,-0.4798);
\coordinate (C-65) at (3.4428,-0.6323);
\coordinate (C-66) at (3.5713,-0.7949);
\coordinate (C-67) at (3.6360,-0.8806);
\coordinate (C-68) at (3.6983,-0.9662);
\coordinate (C-69) at (3.7568,-1.0502);
\coordinate (C-70) at (3.8116,-1.1322);
\coordinate (C-71) at (3.8628,-1.2123);
\coordinate (C-72) at (3.9103,-1.2904);
\coordinate (C-73) at (3.9542,-1.3662);
\coordinate (C-74) at (3.9945,-1.4398);
\coordinate (C-75) at (4.0313,-1.5110);
\coordinate (C-76) at (4.0647,-1.5797);
\coordinate (C-77) at (4.0947,-1.6457);
\coordinate (C-78) at (4.1213,-1.7089);
\coordinate (C-79) at (4.1448,-1.7693);
\coordinate (C-80) at (4.1652,-1.8266);
\coordinate (C-81) at (4.1826,-1.8808);
\coordinate (C-82) at (4.1971,-1.9317);
\coordinate (C-83) at (4.2091,-1.9793);
\coordinate (C-84) at (4.2185,-2.0234);
\coordinate (C-85) at (4.2257,-2.0640);
\coordinate (C-86) at (4.2308,-2.1010);
\coordinate (C-87) at (4.2340,-2.1344);
\coordinate (C-88) at (4.2357,-2.1642);
\coordinate (C-89) at (4.2361,-2.1904);
\coordinate (C-90) at (4.2354,-2.2132);
\coordinate (C-91) at (4.2340,-2.2327);
\coordinate (C-92) at (4.2320,-2.2491);
\coordinate (C-93) at (4.2296,-2.2627);
\coordinate (C-94) at (4.2272,-2.2738);
\coordinate (C-95) at (4.2247,-2.2827);
\coordinate (C-96) at (4.2223,-2.2899);
\coordinate (C-97) at (4.2200,-2.2957);
\coordinate (C-98) at (4.2178,-2.3007);
\coordinate (C-99) at (4.2154,-2.3052);
\coordinate (C-100) at (4.2127,-2.3097);
\coordinate (C-101) at (4.2095,-2.3146);
\coordinate (C-102) at (4.2053,-2.3203);
\coordinate (C-103) at (4.1998,-2.3269);
\coordinate (C-104) at (4.1926,-2.3347);
\coordinate (C-105) at (4.1834,-2.3438);
\coordinate (C-106) at (4.1716,-2.3543);
\coordinate (C-107) at (4.1568,-2.3662);
\coordinate (C-108) at (4.1389,-2.3794);
\coordinate (C-109) at (4.1173,-2.3938);
\coordinate (C-110) at (4.0918,-2.4092);
\coordinate (C-111) at (4.0622,-2.4255);
\coordinate (C-112) at (4.0283,-2.4423);
\coordinate (C-113) at (3.9898,-2.4596);
\coordinate (C-114) at (3.9467,-2.4771);
\coordinate (C-115) at (3.8989,-2.4946);
\coordinate (C-116) at (3.8462,-2.5117);
\coordinate (C-117) at (3.7888,-2.5284);
\coordinate (C-118) at (3.7264,-2.5444);
\coordinate (C-119) at (3.6592,-2.5595);
\coordinate (C-120) at (3.5872,-2.5735);
\coordinate (C-121) at (3.5103,-2.5862);
\coordinate (C-122) at (3.4287,-2.5975);
\coordinate (C-123) at (3.3423,-2.6073);
\coordinate (C-124) at (3.2512,-2.6154);
\coordinate (C-125) at (3.1554,-2.6216);
\coordinate (C-126) at (3.0551,-2.6258);
\coordinate (C-127) at (2.9501,-2.6280);
\coordinate (C-128) at (2.8407,-2.6280);
\coordinate (C-129) at (2.7268,-2.6258);
\coordinate (C-130) at (2.6085,-2.6211);
\coordinate (C-131) at (2.4858,-2.6141);
\coordinate (C-132) at (2.3549,-2.6042);
\coordinate (C-133) at (2.1877,-2.5897);
\coordinate (C-134) at (2.0190,-2.5742);
\coordinate (C-135) at (1.8526,-2.5578);
\coordinate (C-136) at (1.6886,-2.5406);
\coordinate (C-137) at (1.5269,-2.5226);
\coordinate (C-138) at (1.3676,-2.5037);
\coordinate (C-139) at (1.2109,-2.4840);
\coordinate (C-140) at (1.0566,-2.4634);
\coordinate (C-141) at (0.9050,-2.4419);
\coordinate (C-142) at (0.7560,-2.4196);
\coordinate (C-143) at (0.6098,-2.3963);
\coordinate (C-144) at (0.4663,-2.3722);
\coordinate (C-145) at (0.3256,-2.3472);
\coordinate (C-146) at (0.1879,-2.3212);
\coordinate (C-147) at (0.0531,-2.2944);
\coordinate (C-148) at (-0.0787,-2.2667);
\coordinate (C-149) at (-0.2073,-2.2381);
\coordinate (C-150) at (-0.3328,-2.2086);
\coordinate (C-151) at (-0.4551,-2.1783);
\coordinate (C-152) at (-0.5741,-2.1471);
\coordinate (C-153) at (-0.6897,-2.1150);
\coordinate (C-154) at (-0.8019,-2.0822);
\coordinate (C-155) at (-0.9106,-2.0486);
\coordinate (C-156) at (-1.0157,-2.0143);
\coordinate (C-157) at (-1.1172,-1.9792);
\coordinate (C-158) at (-1.2150,-1.9435);
\coordinate (C-159) at (-1.3090,-1.9072);
\coordinate (C-160) at (-1.3992,-1.8704);
\coordinate (C-161) at (-1.4855,-1.8330);
\coordinate (C-162) at (-1.5679,-1.7953);
\coordinate (C-163) at (-1.6463,-1.7572);
\coordinate (C-164) at (-1.7207,-1.7190);
\coordinate (C-165) at (-1.7910,-1.6805);
\coordinate (C-166) at (-1.8573,-1.6421);
\coordinate (C-167) at (-1.9195,-1.6036);
\coordinate (C-168) at (-1.9777,-1.5654);
\coordinate (C-169) at (-2.0319,-1.5274);
\coordinate (C-170) at (-2.0821,-1.4898);
\coordinate (C-171) at (-2.1285,-1.4527);
\coordinate (C-172) at (-2.1711,-1.4161);
\coordinate (C-173) at (-2.2102,-1.3802);
\coordinate (C-174) at (-2.2457,-1.3450);
\coordinate (C-175) at (-2.2780,-1.3105);
\coordinate (C-176) at (-2.3073,-1.2767);
\coordinate (C-177) at (-2.3337,-1.2436);
\coordinate (C-178) at (-2.3575,-1.2111);
\coordinate (C-179) at (-2.3789,-1.1790);
\coordinate (C-180) at (-2.3983,-1.1471);
\coordinate (C-181) at (-2.4157,-1.1153);
\coordinate (C-182) at (-2.4315,-1.0833);
\coordinate (C-183) at (-2.4458,-1.0507);
\coordinate (C-184) at (-2.4586,-1.0171);
\coordinate (C-185) at (-2.4703,-0.9823);
\coordinate (C-186) at (-2.4806,-0.9459);
\coordinate (C-187) at (-2.4898,-0.9074);
\coordinate (C-188) at (-2.4976,-0.8665);
\coordinate (C-189) at (-2.5040,-0.8229);
\coordinate (C-190) at (-2.5089,-0.7762);
\coordinate (C-191) at (-2.5120,-0.7261);
\coordinate (C-192) at (-2.5132,-0.6725);
\coordinate (C-193) at (-2.5123,-0.6150);
\coordinate (C-194) at (-2.5090,-0.5536);
\coordinate (C-195) at (-2.5031,-0.4881);
\coordinate (C-196) at (-2.4943,-0.4184);
\coordinate (C-197) at (-2.4825,-0.3444);
\coordinate (C-198) at (-2.4683,-0.2708);

\draw[thick] (A-0)--(A-1)--(A-2)--(A-3)--(A-4)--(A-5)--(A-6)--(A-7)--(A-8)--(A-9)--(A-10)--(A-11)--(A-12)--(A-13)--(A-14)--(A-15)--(A-16)--(A-17)--(A-18)--(A-19)--(A-20)--(A-21)--(A-22)--(A-23)--(A-24)--(A-25)--(A-26)--(A-27)--(A-28)--(A-29)--(A-30)--(A-31)--(A-32)--(A-33)--(A-34)--(A-35)--(A-36)--(A-37)--(A-38)--(A-39)--(A-40)--(A-41)--(A-42)--(A-43)--(A-44)--(A-45)--(A-46)--(A-47)--(A-48)--(A-49)--(A-50)--(A-51)--(A-52)--(A-53)--(A-54)--(A-55)--(A-56)--(A-57)--(A-58)--(A-59)--(A-60)--(A-61)--(A-62)--(A-63)--(A-64)--(A-65)--(A-66)--(A-67)--(A-68)--(A-69)--(A-70)--(A-71)--(A-72)--(A-73)--(A-74)--(A-75)--(A-76)--(A-77)--(A-78)--(A-79)--(A-80)--(A-81)--(A-82)--(A-83)--(A-84)--(A-85)--(A-86)--(A-87)--(A-88)--(A-89)--(A-90)--(A-91)--(A-92)--(A-93)--(A-94)--(A-95)--(A-96)--(A-97)--(A-98)--(A-99)--(A-100)--(A-101)--(A-102)--(A-103)--(A-104)--(A-105)--(A-106)--(A-107)--(A-108)--(A-109)--(A-110)--(A-111)--(A-112)--(A-113)--(A-114)--(A-115)--(A-116)--(A-117)--(A-118)--(A-119)--(A-120)--(A-121)--(A-122)--(A-123)--(A-124)--(A-125)--(A-126)--(A-127)--(A-128)--(A-129)--(A-130)--(A-131)--(A-132)--(A-133)--(A-134)--(A-135)--(A-136)--(A-137)--(A-138)--(A-139)--(A-140)--(A-141)--(A-142)--(A-143)--(A-144)--(A-145)--(A-146)--(A-147)--(A-148)--(A-149)--(A-150)--(A-151)--(A-152)--(A-153)--(A-154)--(A-155)--(A-156)--(A-157)--(A-158)--(A-159)--(A-160)--(A-161)--(A-162)--(A-163)--(A-164)--(A-165)--(A-166)--(A-167)--(A-168)--(A-169)--(A-170)--(A-171)--(A-172)--(A-173)--(A-174)--(A-175)--(A-176)--(A-177)--(A-178)--(A-179)--(A-180)--(A-181)--(A-182)--(A-183)--(A-184)--(A-185)--(A-186)--(A-187)--(A-188)--(A-189)--(A-190)--(A-191)--(A-192)--(A-193)--(A-194)--(A-195)--(A-196)--(A-197)--(A-198) --cycle;

\draw[fill=tangerine!80, thick, dashed] (B-0)--(B-1)--(B-2)--(B-3)--(B-4)--(B-5)--(B-6)--(B-7)--(B-8)--(B-9)--(B-10)--(B-11)--(B-12)--(B-13)--(B-14)--(B-15)--(B-16)--(B-17)--(B-18)--(B-19)--(B-20)--(B-21)--(B-22)--(B-23)--(B-24)--(B-25)--(B-26)--(B-27)--(B-28)--(B-29)--(B-30)--(B-31)--(B-32)--(B-33)--(B-34)--(B-35)--(B-36)--(B-37)--(B-38)--(B-39)--(B-40)--(B-41)--(B-42)--(B-43)--(B-44)--(B-45)--(B-46)--(B-47)--(B-48)--(B-49)--(B-50)--(B-51)--(B-52)--(B-53)--(B-54)--(B-55)--(B-56)--(B-57)--(B-58)--(B-59)--(B-60)--(B-61)--(B-62)--(B-63)--(B-64)--(B-65)--(B-66)--(B-67)--(B-68)--(B-69)--(B-70)--(B-71)--(B-72)--(B-73)--(B-74)--(B-75)--(B-76)--(B-77)--(B-78)--(B-79)--(B-80)--(B-81)--(B-82)--(B-83)--(B-84)--(B-85)--(B-86)--(B-87)--(B-88)--(B-89)--(B-90)--(B-91)--(B-92)--(B-93)--(B-94)--(B-95)--(B-96)--(B-97)--(B-98)--(B-99)--(B-100)--(B-101)--(B-102)--(B-103)--(B-104)--(B-105)--(B-106)--(B-107)--(B-108)--(B-109)--(B-110)--(B-111)--(B-112)--(B-113)--(B-114)--(B-115)--(B-116)--(B-117)--(B-118)--(B-119)--(B-120)--(B-121)--(B-122)--(B-123)--(B-124)--(B-125)--(B-126)--(B-127)--(B-128)--(B-129)--(B-130)--(B-131)--(B-132)--(B-133)--(B-134)--(B-135)--(B-136)--(B-137)--(B-138)--(B-139)--(B-140)--(B-141)--(B-142)--(B-143)--(B-144)--(B-145)--(B-146)--(B-147)--(B-148)--(B-149)--(B-150)--(B-151)--(B-152)--(B-153)--(B-154)--(B-155)--(B-156)--(B-157)--(B-158)--(B-159)--(B-160)--(B-161)--(B-162)--(B-163)--(B-164)--(B-165)--(B-166)--(B-167)--(B-168)--(B-169)--(B-170)--(B-171)--(B-172)--(B-173)--(B-174)--(B-175)--(B-176)--(B-177)--(B-178)--(B-179)--(B-180)--(B-181)--(B-182)--(B-183)--(B-184)--(B-185)--(B-186)--(B-187)--(B-188)--(B-189)--(B-190)--(B-191)--(B-192)--(B-193)--(B-194)--(B-195)--(B-196)--(B-197)--(B-198) --cycle;

\draw[fill=white, thick] (C-0)--(C-1)--(C-2)--(C-3)--(C-4)--(C-5)--(C-6)--(C-7)--(C-8)--(C-9)--(C-10)--(C-11)--(C-12)--(C-13)--(C-14)--(C-15)--(C-16)--(C-17)--(C-18)--(C-19)--(C-20)--(C-21)--(C-22)--(C-23)--(C-24)--(C-25)--(C-26)--(C-27)--(C-28)--(C-29)--(C-30)--(C-31)--(C-32)--(C-33)--(C-34)--(C-35)--(C-36)--(C-37)--(C-38)--(C-39)--(C-40)--(C-41)--(C-42)--(C-43)--(C-44)--(C-45)--(C-46)--(C-47)--(C-48)--(C-49)--(C-50)--(C-51)--(C-52)--(C-53)--(C-54)--(C-55)--(C-56)--(C-57)--(C-58)--(C-59)--(C-60)--(C-61)--(C-62)--(C-63)--(C-64)--(C-65)--(C-66)--(C-67)--(C-68)--(C-69)--(C-70)--(C-71)--(C-72)--(C-73)--(C-74)--(C-75)--(C-76)--(C-77)--(C-78)--(C-79)--(C-80)--(C-81)--(C-82)--(C-83)--(C-84)--(C-85)--(C-86)--(C-87)--(C-88)--(C-89)--(C-90)--(C-91)--(C-92)--(C-93)--(C-94)--(C-95)--(C-96)--(C-97)--(C-98)--(C-99)--(C-100)--(C-101)--(C-102)--(C-103)--(C-104)--(C-105)--(C-106)--(C-107)--(C-108)--(C-109)--(C-110)--(C-111)--(C-112)--(C-113)--(C-114)--(C-115)--(C-116)--(C-117)--(C-118)--(C-119)--(C-120)--(C-121)--(C-122)--(C-123)--(C-124)--(C-125)--(C-126)--(C-127)--(C-128)--(C-129)--(C-130)--(C-131)--(C-132)--(C-133)--(C-134)--(C-135)--(C-136)--(C-137)--(C-138)--(C-139)--(C-140)--(C-141)--(C-142)--(C-143)--(C-144)--(C-145)--(C-146)--(C-147)--(C-148)--(C-149)--(C-150)--(C-151)--(C-152)--(C-153)--(C-154)--(C-155)--(C-156)--(C-157)--(C-158)--(C-159)--(C-160)--(C-161)--(C-162)--(C-163)--(C-164)--(C-165)--(C-166)--(C-167)--(C-168)--(C-169)--(C-170)--(C-171)--(C-172)--(C-173)--(C-174)--(C-175)--(C-176)--(C-177)--(C-178)--(C-179)--(C-180)--(C-181)--(C-182)--(C-183)--(C-184)--(C-185)--(C-186)--(C-187)--(C-188)--(C-189)--(C-190)--(C-191)--(C-192)--(C-193)--(C-194)--(C-195)--(C-196)--(C-197)--(C-198) --cycle;


\node[below=-2pt, scale=0.8] at (0,1) {$\overline{\Omega_{i+1}}$};
\node[above=7pt, scale=0.8] at (0,1) {$\Omega_{i+3}$};

\end{tikzpicture}
        \caption{Representação visual (em azul) dos conjuntos $\Omega'_i$.\\Fonte: Autoral.}
        \label{fig:omegaii}
    \end{figure}
    e seja $\{\phi_i\}_{i=0}^\infty$ uma partição da unidade suave subordinada à cobertura aberta\footnote{i.e., Uma cobertura de um conjunto $X$ é uma família $\{Y_\lambda\}_{\lambda\in L}$, tal que $X \subseteq \bigcup_{\lambda \in L} Y_\lambda$. $\{Y_\lambda\}$ é dita uma cobertura aberta se $Y_\lambda$ é aberto para todo $\lambda \in L$.} $\{\Omega'_i\}_{i=0}^\infty$, isto é,
    \[
       0 \leqslant \phi_i \leqslant 1 \text{ com } \phi_i \in \cC^\infty_c(\Omega'_i) \;\text{ e }\; \sum_{i=0}^\infty \phi_i = 1 \text{ em } \Omega.
    \]
    Como, por hipótese, $u \in \cW^{k,p}(\Omega)$, temos, pelo Teorema \ref{thm:propriedades-derivada-fraca}, que
    \[
        \phi_i u \in \cW^{k,p}(\Omega).
    \]
    Além disso, $\supp(\phi_i u) \subseteq \Omega'_i$.
    Fixando $\delta > 0$, escolha um $\varepsilon_i > 0$ de forma que $u^i := \eta_{\varepsilon_i} * \phi_i u \in \cC^{\infty}(\Omega) \cap \cW^{k,p}(\Omega)$ satisfaça
    \[
        \Vert u^i - \phi_i u \Vert_{\cW^{k,p}(\Omega)} \leqslant \frac{\delta}{2^{i+1}} \;\text{ e }\; \supp u^i \subseteq \Omega''_i,
    \]
    (ver Teorema \ref{thm:aprox1}),
    com $\Omega''_i = \Omega_{i+4} \setminus \overline{\Omega_{i}} \supseteq \Omega'_i$ .

    Seja $v = \sum_{i=1}^\infty u^i \in \cC^{\infty}(\Omega) \cap \cW^{k,p}(\Omega)$.
    Como 
    \[
        u = u \sum_{i=1}^\infty \phi_i = \sum_{i=1}^n \phi_i u,
    \]
    então, para cada $V \Subset \Omega$, inferimos que
    \[
        \Vert v - u \Vert_{\cW^{k,p}(V)} \leqslant \sum_{i=1}^\infty \Vert u^i - \phi_i u \Vert_{\cW^{k,p}(\Omega)} \leqslant \sum_{i=1}^\infty \frac{\delta}{2^{i+1}} = \frac{\delta}{2}.
    \]
    Passando ao supremo sobre os conjuntos $V \Subset \Omega$, obtemos
    \[
        \Vert v - u \Vert_{\cW^{k,p}(\Omega)} \leqslant \frac{\delta}{2} < \delta.
    \]
    Isto mostra que $u$ pertence ao fecho de $\cC^\infty(\Omega) \cap \cW^{k,p}(\Omega)$ em $\cW^{k,p}(\Omega)$. Logo é equivalente a existir uma sequência $(u_n) \subseteq \cC^\infty(\Omega) \cap \cW^{k,p}(\Omega)$ tal que $u_n \to u$ em $\cW^{k,p}(\Omega)$, quando $n \to \infty$.
\end{prf}

A definição abaixo é necessária para o próximo resultado de densidade.

\begin{dbox} \label{def:fronteira-ck}
    Sejam $\Omega \subseteq \bR^n$ um aberto limitado e $k \in \bN$. 
    Dizemos que sua fronteira $\partial \Omega$ é de classe $\cC^k$ se para cada ponto $\tilde x \in \partial \Omega$ existe um raio $r > 0$ e uma função de classe $\cC^k$ $\gamma : \bR^{n-1} \to \bR$ tal que, fazendo uma mudança de coordenadas se necessário, obtemos
    \[
        \Omega \cap B[\tilde x,r] = \{x \in B[\tilde x,r] \,; x_n > \gamma(x_1,\dots,x_{n-1})\}.
    \]
    De forma análoga, $\partial\Omega$ é de classe $\cC^\infty$ se é de classe $\cC^k$ para todo $k \in \bN$.
\end{dbox}

\begin{figure}
    \centering
    \begin{tikzpicture}[scale=1.5]
        \draw[thick, dashed, tangerine!30!black, fill=tangerine!40] (-2.076,1.929) .. controls (-1.374,-0.882) and (-0.462,2.304) .. (0.81,1.5) .. controls (2.112,0.549) and (2.121,1.836) .. (2.136,1.941) to (2.136,1.929);
    
        \draw[thick, -stealth, draw=black!70] (-2.2,0) to (2.2,0) node[right] {$\mathbb R^{n-1}$};
        \draw[thick, -stealth, draw=black!70] (0,-1) to (0,2.1) node[above] {$x_n$};
    
        \draw[stealth-stealth, thick, tangerine!50!black, font=\small] (0.5616,0) to node[right] {$\gamma(x_1,\dots,x_{n-1})$} (0.5616,1.6102);
        \node[below=2pt,font=\footnotesize, fill=white] at (0.5616,0) {$(x_1,\dots,x_{n-1})$};
        \filldraw (0.5616,0) circle (0.03);

        \node[tangerine!50!black] at (-1.419,1.524) {$\Omega$};
        \node[tangerine!50!black] at (-0.5,0.8) {$\partial\Omega$};
    \end{tikzpicture}
    \caption{Função $\gamma$ da Definição \ref{def:fronteira-ck}.\\
    Fonte: Autoral. Baseada em \cite{evans-pde} p.p. 626.}
\end{figure}

Com essa definição, conseguimos mostrar que $\cC^{\infty}(\overline{\Omega})$ é denso em $\cW^{k,p}(\Omega)$.

\begin{tbox} \label{thm:aprox3}
    Sejam $\Omega$ um aberto limitado com fronteira de classe $\cC^1$ e $u \in \cW^{k,p}(\Omega)$, com $1 \leqslant p < \infty$.
    Então, existe uma sequência $(u_n) \subseteq \cC^\infty(\overline\Omega)$ tal que
    \[
        u_n \to u \;\text{ em } \cW^{k,p}(\Omega),
    \]
    quando $n \to \infty$.
\end{tbox}
\begin{prf}
    Seja $\tilde x \in \partial \Omega$, como $\Omega$ tem fronteira de classe $\cC^1$, existe um raio $r > 0$ e uma função $\gamma : \bR^{n-1} \to \bR$ de classe $\cC^1$ tal que
    \begin{equation} \label{eq:E}
        \Omega \cap B[\tilde x, r] = \{x \in B[\tilde x, r] \,; x_n > \gamma(x_1,\dots,x_{n-1})\}.
    \end{equation}
    Definimos $V = \Omega \cap B[\tilde x, \sfrac{r}{2}]$ (ver Figura \ref{fig:conjunto-demonstracao-aprox-3}).
    Além disso, definimos para cada $\varepsilon > 0$, $\lambda > 0$ e $x \in V$
    \begin{equation} \label{eq:D}
        x^\varepsilon = x + \lambda \varepsilon e_n = (x_1,\dots,x_{n-1}, x_n + \lambda\varepsilon),
    \end{equation}
    onde $x = (x_1,\dots,x_n)$.
    Observe que, para um $\lambda > 0$ suficientemente grande e $\varepsilon > 0$ suficientemente pequeno, a bola $B[x^\varepsilon\!,\varepsilon]$ está contída em $\Omega \cap B[\tilde x,r]$ para todo $x \in V$.
    De fato, por definição, dado $x \in V$ temos que $x \in \Omega$ e $\Vert x - \tilde x \Vert \leqslant \sfrac{r}{2}$.
    Note que $x^\varepsilon \in \Omega \cap B(\tilde x, r)$ para todo $\varepsilon > 0$ suficientemente pequeno e $\lambda > 0$ suficientemente grande.
    Com efeito, como $x \in V$, então, $x \in \Omega \cap B[\tilde x, r]$, por \ref{eq:E}. Assim,
    \[
        x^\varepsilon_n = x_n + \lambda \varepsilon > x_n >  \gamma(x_{1},\dots,x_{n-1}),
    \]
    e, por (\ref{eq:D}), deduzimos que
    \[
        \Vert x^\varepsilon - \tilde x \Vert = \Vert x + \lambda \varepsilon e_n - \tilde x \Vert \leqslant \Vert x - \tilde x \Vert + \Vert \lambda \varepsilon e_n  \Vert \leqslant \frac{r}{2} + \lambda\varepsilon < r,
    \]
    fornecido que $0 <\varepsilon < \sfrac{r}{2\lambda}$.
    Logo, $x^\varepsilon \in \Omega \cap B(\tilde x, r)$ para todo $\varepsilon > 0$ suficientemente pequeno (pois $\Omega$ é aberto e $x \in \Omega$).
    Por isso,
    $B[x^\varepsilon\!, \varepsilon] \subseteq \Omega \cap B[\tilde x,r]$, diminuindo $\varepsilon > 0$ se necessário (já que $\Omega \cap B(\tilde x, r)$ é aberto).
    \begin{figure}
        \centering
        \begin{tikzpicture}
    \coordinate (1-001) at (3.000, 0.000);
    \coordinate (1-002) at (2.994, 0.190);
    \coordinate (1-003) at (2.976, 0.380);
    \coordinate (1-004) at (2.946, 0.568);
    \coordinate (1-005) at (2.904, 0.753);
    \coordinate (1-006) at (2.850, 0.936);
    \coordinate (1-007) at (2.785, 1.115);
    \coordinate (1-008) at (2.709, 1.289);
    \coordinate (1-009) at (2.622, 1.459);
    \coordinate (1-010) at (2.524, 1.622);
    \coordinate (1-011) at (2.416, 1.779);
    \coordinate (1-012) at (2.298, 1.928);
    \coordinate (1-013) at (2.171, 2.070);
    \coordinate (1-014) at (2.036, 2.204);
    \coordinate (1-015) at (1.892, 2.328);
    \coordinate (1-016) at (1.740, 2.444);
    \coordinate (1-017) at (1.582, 2.549);
    \coordinate (1-018) at (1.417, 2.644);
    \coordinate (1-019) at (1.246, 2.729);
    \coordinate (1-020) at (1.071, 2.802);
    \coordinate (1-021) at (0.891, 2.865);
    \coordinate (1-022) at (0.707, 2.915);
    \coordinate (1-023) at (0.521, 2.954);
    \coordinate (1-024) at (0.333, 2.982);
    \coordinate (1-025) at (0.143, 2.997);
    \coordinate (1-026) at (-0.048, 3.000);
    \coordinate (1-027) at (-0.238, 2.991);
    \coordinate (1-028) at (-0.427, 2.969);
    \coordinate (1-029) at (-0.614, 2.936);
    \coordinate (1-030) at (-0.799, 2.892);
    \coordinate (1-031) at (-0.981, 2.835);
    \coordinate (1-032) at (-1.159, 2.767);
    \coordinate (1-033) at (-1.332, 2.688);
    \coordinate (1-034) at (-1.500, 2.598);
    \coordinate (1-035) at (-1.662, 2.498);
    \coordinate (1-036) at (-1.817, 2.387);
    \coordinate (1-037) at (-1.965, 2.267);
    \coordinate (1-038) at (-2.104, 2.138);
    \coordinate (1-039) at (-2.236, 2.000);
    \coordinate (1-040) at (-2.358, 1.854);
    \coordinate (1-041) at (-2.471, 1.701);
    \coordinate (1-042) at (-2.574, 1.541);
    \coordinate (1-043) at (-2.667, 1.375);
    \coordinate (1-044) at (-2.748, 1.203);
    \coordinate (1-045) at (-2.819, 1.026);
    \coordinate (1-046) at (-2.878, 0.845);
    \coordinate (1-047) at (-2.926, 0.661);
    \coordinate (1-048) at (-2.962, 0.474);
    \coordinate (1-049) at (-2.986, 0.285);
    \coordinate (1-050) at (-2.998, 0.095);
    \coordinate (1-051) at (-2.998, -0.095);
    \coordinate (1-052) at (-2.986, -0.285);
    \coordinate (1-053) at (-2.962, -0.474);
    \coordinate (1-054) at (-2.926, -0.661);
    \coordinate (1-055) at (-2.878, -0.845);
    \coordinate (1-056) at (-2.819, -1.026);
    \coordinate (1-057) at (-2.748, -1.203);
    \coordinate (1-058) at (-2.667, -1.375);
    \coordinate (1-059) at (-2.574, -1.541);
    \coordinate (1-060) at (-2.471, -1.701);
    \coordinate (1-061) at (-2.358, -1.854);
    \coordinate (1-062) at (-2.236, -2.000);
    \coordinate (1-063) at (-2.104, -2.138);
    \coordinate (1-064) at (-1.965, -2.267);
    \coordinate (1-065) at (-1.817, -2.387);
    \coordinate (1-066) at (-1.662, -2.498);
    \coordinate (1-067) at (-1.500, -2.598);
    \coordinate (1-068) at (-1.332, -2.688);
    \coordinate (1-069) at (-1.159, -2.767);
    \coordinate (1-070) at (-0.981, -2.835);
    \coordinate (1-071) at (-0.799, -2.892);
    \coordinate (1-072) at (-0.614, -2.936);
    \coordinate (1-073) at (-0.427, -2.969);
    \coordinate (1-074) at (-0.238, -2.991);
    \coordinate (1-075) at (-0.048, -3.000);
    \coordinate (1-076) at (0.143, -2.997);
    \coordinate (1-077) at (0.333, -2.982);
    \coordinate (1-078) at (0.521, -2.954);
    \coordinate (1-079) at (0.707, -2.915);
    \coordinate (1-080) at (0.891, -2.865);
    \coordinate (1-081) at (1.071, -2.802);
    \coordinate (1-082) at (1.246, -2.729);
    \coordinate (1-083) at (1.417, -2.644);
    \coordinate (1-084) at (1.582, -2.549);
    \coordinate (1-085) at (1.740, -2.444);
    \coordinate (1-086) at (1.892, -2.328);
    \coordinate (1-087) at (2.036, -2.204);
    \coordinate (1-088) at (2.171, -2.070);
    \coordinate (1-089) at (2.298, -1.928);
    \coordinate (1-090) at (2.416, -1.779);
    \coordinate (1-091) at (2.524, -1.622);
    \coordinate (1-092) at (2.622, -1.459);
    \coordinate (1-093) at (2.709, -1.289);
    \coordinate (1-094) at (2.785, -1.115);
    \coordinate (1-095) at (2.850, -0.936);
    \coordinate (1-096) at (2.904, -0.753);
    \coordinate (1-097) at (2.946, -0.568);
    \coordinate (1-098) at (2.976, -0.380);
    \coordinate (1-099) at (2.994, -0.190);
    \coordinate (1-100) at (3.000, -0.000);

    \coordinate (2-001) at (-4.720, -0.850);
    \coordinate (2-002) at (-4.543, -0.936);
    \coordinate (2-003) at (-4.370, -1.015);
    \coordinate (2-004) at (-4.202, -1.088);
    \coordinate (2-005) at (-4.038, -1.154);
    \coordinate (2-006) at (-3.878, -1.213);
    \coordinate (2-007) at (-3.723, -1.267);
    \coordinate (2-008) at (-3.571, -1.314);
    \coordinate (2-009) at (-3.424, -1.356);
    \coordinate (2-010) at (-3.280, -1.391);
    \coordinate (2-011) at (-3.141, -1.421);
    \coordinate (2-012) at (-3.005, -1.446);
    \coordinate (2-013) at (-2.872, -1.465);
    \coordinate (2-014) at (-2.744, -1.479);
    \coordinate (2-015) at (-2.619, -1.488);
    \coordinate (2-016) at (-2.497, -1.491);
    \coordinate (2-017) at (-2.378, -1.490);
    \coordinate (2-018) at (-2.263, -1.485);
    \coordinate (2-019) at (-2.151, -1.474);
    \coordinate (2-020) at (-2.042, -1.460);
    \coordinate (2-021) at (-1.937, -1.441);
    \coordinate (2-022) at (-1.834, -1.418);
    \coordinate (2-023) at (-1.734, -1.390);
    \coordinate (2-024) at (-1.636, -1.359);
    \coordinate (2-025) at (-1.542, -1.325);
    \coordinate (2-026) at (-1.450, -1.287);
    \coordinate (2-027) at (-1.361, -1.245);
    \coordinate (2-028) at (-1.274, -1.200);
    \coordinate (2-029) at (-1.189, -1.152);
    \coordinate (2-030) at (-1.107, -1.100);
    \coordinate (2-031) at (-1.027, -1.046);
    \coordinate (2-032) at (-0.949, -0.989);
    \coordinate (2-033) at (-0.873, -0.930);
    \coordinate (2-034) at (-0.800, -0.868);
    \coordinate (2-035) at (-0.728, -0.804);
    \coordinate (2-036) at (-0.658, -0.737);
    \coordinate (2-037) at (-0.589, -0.669);
    \coordinate (2-038) at (-0.522, -0.599);
    \coordinate (2-039) at (-0.457, -0.526);
    \coordinate (2-040) at (-0.394, -0.453);
    \coordinate (2-041) at (-0.331, -0.377);
    \coordinate (2-042) at (-0.271, -0.301);
    \coordinate (2-043) at (-0.211, -0.223);
    \coordinate (2-044) at (-0.152, -0.144);
    \coordinate (2-045) at (-0.095, -0.064);
    \coordinate (2-046) at (-0.039, 0.017);
    \coordinate (2-047) at (0.017, 0.098);
    \coordinate (2-048) at (0.072, 0.180);
    \coordinate (2-049) at (0.125, 0.262);
    \coordinate (2-050) at (0.179, 0.345);
    \coordinate (2-051) at (0.231, 0.428);
    \coordinate (2-052) at (0.283, 0.510);
    \coordinate (2-053) at (0.335, 0.593);
    \coordinate (2-054) at (0.386, 0.675);
    \coordinate (2-055) at (0.437, 0.756);
    \coordinate (2-056) at (0.488, 0.838);
    \coordinate (2-057) at (0.539, 0.918);
    \coordinate (2-058) at (0.589, 0.997);
    \coordinate (2-059) at (0.640, 1.076);
    \coordinate (2-060) at (0.691, 1.153);
    \coordinate (2-061) at (0.742, 1.230);
    \coordinate (2-062) at (0.793, 1.304);
    \coordinate (2-063) at (0.844, 1.377);
    \coordinate (2-064) at (0.897, 1.449);
    \coordinate (2-065) at (0.949, 1.518);
    \coordinate (2-066) at (1.002, 1.586);
    \coordinate (2-067) at (1.056, 1.651);
    \coordinate (2-068) at (1.111, 1.715);
    \coordinate (2-069) at (1.166, 1.776);
    \coordinate (2-070) at (1.223, 1.834);
    \coordinate (2-071) at (1.280, 1.890);
    \coordinate (2-072) at (1.339, 1.943);
    \coordinate (2-073) at (1.399, 1.992);
    \coordinate (2-074) at (1.460, 2.039);
    \coordinate (2-075) at (1.522, 2.083);
    \coordinate (2-076) at (1.586, 2.123);
    \coordinate (2-077) at (1.651, 2.160);
    \coordinate (2-078) at (1.718, 2.193);
    \coordinate (2-079) at (1.786, 2.222);
    \coordinate (2-080) at (1.856, 2.248);
    \coordinate (2-081) at (1.928, 2.269);
    \coordinate (2-082) at (2.002, 2.286);
    \coordinate (2-083) at (2.078, 2.299);
    \coordinate (2-084) at (2.156, 2.308);
    \coordinate (2-085) at (2.236, 2.311);
    \coordinate (2-086) at (2.319, 2.310);
    \coordinate (2-087) at (2.403, 2.304);
    \coordinate (2-088) at (2.490, 2.294);
    \coordinate (2-089) at (2.580, 2.277);
    \coordinate (2-090) at (2.672, 2.256);
    \coordinate (2-091) at (2.767, 2.229);
    \coordinate (2-092) at (2.864, 2.197);
    \coordinate (2-093) at (2.964, 2.159);
    \coordinate (2-094) at (3.067, 2.115);
    \coordinate (2-095) at (3.173, 2.065);
    \coordinate (2-096) at (3.282, 2.009);
    \coordinate (2-097) at (3.394, 1.946);
    \coordinate (2-098) at (3.509, 1.877);
    \coordinate (2-099) at (3.628, 1.802);
    \coordinate (2-100) at (3.750, 1.720);

    \draw[thick] (0,0) circle (3);

    \draw[fill=tangerine!30] (1.9546,2.2777) -- (2-081) -- (2-080) -- (2-079) -- (2-078) -- (2-077) -- (2-076) -- (2-075) -- (2-074) -- (2-073) -- (2-072) -- (2-071) -- (2-070) -- (2-069) -- (2-068) -- (2-067) -- (2-066) -- (2-065) -- (2-064) -- (2-063) -- (2-062) -- (2-061) -- (2-060) -- (2-059) -- (2-058) -- (2-057) -- (2-056) -- (2-055) -- (2-054) -- (2-053) -- (2-052) -- (2-051) -- (2-050) -- (2-049) -- (2-048) -- (2-047) -- (2-046) -- (2-045) -- (2-044) -- (2-043) -- (2-042) -- (2-041) -- (2-040) -- (2-039) -- (2-038) -- (2-037) -- (2-036) -- (2-035) -- (2-034) -- (2-033) -- (2-032) -- (2-031) -- (2-030) -- (2-029) -- (2-028) -- (2-027) -- (2-026) -- (2-025) -- (2-024) -- (2-023) -- (2-022) -- (2-021) -- (2-020) -- (2-019) -- (2-018) -- (2-017) -- (2-016) -- (2-015) -- (-2.6047,-1.4883) -- (1-058) -- (1-057) -- (1-056) -- (1-055) -- (1-054) -- (1-053) -- (1-052) -- (1-051) -- (1-050) -- (1-049) -- (1-048) -- (1-047) -- (1-046) -- (1-045) -- (1-044) -- (1-043) -- (1-042) -- (1-041) -- (1-040) -- (1-039) -- (1-038) -- (1-037) -- (1-036) -- (1-035) -- (1-034) -- (1-033) -- (1-032) -- (1-031) -- (1-030) -- (1-029) -- (1-028) -- (1-027) -- (1-026) -- (1-025) -- (1-024) -- (1-023) -- (1-022) -- (1-021) -- (1-020) -- (1-019) -- (1-018) -- (1-017) -- (1-016) -- (1-015) --cycle;
    
    \draw[very thick] (2-005) -- (2-006) -- (2-007) -- (2-008) -- (2-009) -- (2-010) -- (2-011) -- (2-012) -- (2-013) -- (2-014) -- (2-015) -- (2-016) -- (2-017) -- (2-018) -- (2-019) -- (2-020) -- (2-021) -- (2-022) -- (2-023) -- (2-024) -- (2-025) -- (2-026) -- (2-027) -- (2-028) -- (2-029) -- (2-030) -- (2-031) -- (2-032) -- (2-033) -- (2-034) -- (2-035) -- (2-036) -- (2-037) -- (2-038) -- (2-039) -- (2-040) -- (2-041) -- (2-042) -- (2-043) -- (2-044) -- (2-045) -- (2-046) -- (2-047) -- (2-048) -- (2-049) -- (2-050) -- (2-051) -- (2-052) -- (2-053) -- (2-054) -- (2-055) -- (2-056) -- (2-057) -- (2-058) -- (2-059) -- (2-060) -- (2-061) -- (2-062) -- (2-063) -- (2-064) -- (2-065) -- (2-066) -- (2-067) -- (2-068) -- (2-069) -- (2-070) -- (2-071) -- (2-072) -- (2-073) -- (2-074) -- (2-075) -- (2-076) -- (2-077) -- (2-078) -- (2-079) -- (2-080) -- (2-081) -- (2-082) -- (2-083) -- (2-084) -- (2-085) -- (2-086) -- (2-087) -- (2-088) -- (2-089) -- (2-090) -- (2-091) -- (2-092) -- (2-093) -- (2-094) -- (2-095) -- (2-096) -- (2-097) -- (2-098) -- (2-099) -- (2-100);

    \filldraw (0,0) circle (0.06) node[below=2pt, scale=1.2] {$\tilde x$};
    \node[above=3pt] at (2-007) {$\Omega$};

    \filldraw (-1.5,-0.5) circle (0.06) node[below=2pt, scale=1.2] {$x$};

    \draw[fill=tangerine!50, thick] (-1.5,1.25) circle (0.75);
    \filldraw (-1.5,1.25) circle (0.06) node[below=1pt] {$x^\varepsilon$};

    \draw[thick, -stealth] (-3,2.5) node[above] {$B[x^\varepsilon\!,\varepsilon]$} to[bend right] (-2.25,1.6);

    \draw[thick, -stealth] (1.7,3) node[above] {$\Omega \cap B[x^\varepsilon\!,r]$} to[bend left] (0.85,2.1);

    \node[anchor=north west] at (-45:3) {$B[\tilde x, r]$};
\end{tikzpicture}
        \caption{Fonte: Autoral. Baseada em \cite{evans-pde} p.p. 253}
        \label{fig:conjunto-demonstracao-aprox-3}
    \end{figure}

    Agora, definimos $u_\varepsilon(x) = u(x^\varepsilon)$, para todo $x \in V$, e $v^\varepsilon = \eta_\varepsilon * u_\varepsilon$. Assim, pelo Teorema \ref{thm:aprox1}, $v^\varepsilon \in \cC^{\infty}(\overline{V})$ (lembre que $\lambda > 0$ e grande e $\varepsilon > 0$ é pequeno). 
    Dito isso, afirmamos que
    \[
        \Vert v^\varepsilon - u \Vert_{\cW^{k,p}(V)} \to 0,
    \]
    quando $\varepsilon \to 0$.
    De fato, seja $\alpha $ um multi-índice com $|\alpha| \leqslant k$, então
    \[
        \Vert D^\alpha v^\varepsilon - D^\alpha u \Vert_{\cL^p(V)} \leqslant \Vert D^\alpha v^\varepsilon - D^\alpha u_\varepsilon \Vert_{\cL^p(V)} + \Vert D^\alpha u_\varepsilon - D^\alpha u \Vert_{\cL^p(V)}.
    \]
   A segunda norma do lado direito da desigualdade acima vaí a $0$, quando $\varepsilon \to 0$, pois a translação é contínua na norma do espaço $\cL^p$ e
    \[
        \Vert D^\alpha v^\varepsilon - D^\alpha u_\varepsilon \Vert_{\cL^p(V)} = \Vert D^\alpha (\eta_\varepsilon * u_\varepsilon) - D^\alpha u_\varepsilon \Vert_{\cL^p(V)} \to 0,
    \]
    quando $\varepsilon \to 0$ (ver Teorema \ref{thm:aprox1}). Ou seja, é verdade que
    \[
        \begin{aligned}
            \Vert v^\varepsilon - u \Vert_{\cW^{k,p}(V)} &= \sum_{|\alpha| \leqslant k} \Vert D^\alpha v^\varepsilon - D^\alpha u \Vert_{\cL^p(V)} \\
            &\leqslant \sum_{|\alpha| \leqslant k} \Vert D^\alpha v^\varepsilon - D^\alpha u_\varepsilon \Vert_{\cL^p(V)} + \sum_{|\alpha| \leqslant k} \Vert D^\alpha u_\varepsilon - D^\alpha u \Vert_{\cL^p(V)} \to 0,
        \end{aligned}
    \]
    quando $\varepsilon \to 0$.

    Note que todos os cálculos foram feitos em uma vizinhança de um ponto $\tilde x \in \partial\Omega$. Dito isso, como $\partial \Omega$ é compacto (pois, $\Omega$ é limitado), pelo Teorema de Heine-Borel\footnote{Toda cobertura aberta de um conjunto compacto admite uma subcobertura finita.}, podemos encontrar uma quantidade finita de pontos $\tilde x_i \in \partial \Omega$, raios $r_i > 0$, conjuntos $V_i = \Omega \cap B[\tilde x_i,\sfrac{r_i\,}{2}]$ e funções $v_i^\varepsilon$, com $i = 1,\dots,N \in \bN$ tal que $\Vert v_i^\varepsilon - u \Vert_{\cW^{k,p}(V_i)} \to 0$, quando $\varepsilon \to 0$, e 
    \begin{equation} \label{eq:cobertura}
        \partial\Omega \subseteq \bigcup_{i=1}^N B(\tilde x_i, \sfrac{r_i}{2}).
    \end{equation}
    Além disso, considere um aberto $V_0 \Subset \Omega$ tal que $\Omega \subseteq \bigcup_{i=0}^N V_i$ (ver Figura \ref{fig:coberturafronteirav0}), e pelo Teorema \ref{thm:aprox-2}, encontramos $v_0^\varepsilon \in \cC^{\infty}(V_0) \cap \cW^{k,p}(V_0)$ com $\Vert v_0^\varepsilon - u \Vert_{\cW^{k,p}(V_0)} \to 0$, quando $\varepsilon \to 0$.
    \begin{figure}
        \centering
        \input{mollifiers/cobertura1.tex}
        \caption{Representação visual dos conjuntos em (\ref{eq:cobertura}).Note que $\{V_0\} \cup \{B (\tilde x_i, \sfrac{r_i}{2})\}_{i=1}^N$ forma uma cobertura para $\Omega$. \\Fonte: Autoral.}
        \label{fig:coberturafronteirav0}
    \end{figure}
    Seja $\{\phi_i\}_{i=0}^N$ uma partição da unidade subordinada aos conjuntos $\{V_i\}_{i=0}^N$ em $\Omega$.
    Defina
    \[
        v^\varepsilon = \sum_{i=0}^N \phi_i v_i^\varepsilon \in \cC^{\infty}(\overline\Omega),
    \]
    e observando que $u = \sum_{i=0}^N \phi_i u$ (pois $\sum_{i=0}^N \phi_i = 1$), obtemos
    \[
        \Vert D^\alpha v^\varepsilon - D^\alpha u \Vert_{\cL^p(\Omega)} \leqslant \sum_{i=0}^N \Vert D^\alpha (\phi_i v_i^\varepsilon) - D^\alpha(\phi_i u) \Vert_{\cL^p(\Omega)}.
    \]
    Utilizando a regra de Leibniz (ver Teorema \ref{thm:propriedades-derivada-fraca}), segue que
    \[
        \begin{aligned}
            \Vert D^\alpha v^\varepsilon - D^\alpha u \Vert_{\cL^p(\Omega)} &\leqslant \sum_{i=0}^N\sum_{\sigma \leqslant \alpha} \binom{\alpha}{\sigma} \left\Vert  D^{\sigma} \phi_i \left[ D^{\alpha - \sigma} \left( v_i^\varepsilon - u \right) \right] \right\Vert _{\cL^p(V_i)}\\ 
            &\leqslant c\sum_{i=0}^N\sum_{\sigma \leqslant \alpha} \binom{\alpha}{\sigma} \Vert D^{\alpha-\sigma}(v^\varepsilon_i - u) \Vert_{\cL^p(V_i)},
        \end{aligned}
    \]
    onde utilizamos o fato de $\phi_i$ (e consequentemente $D^\sigma \phi_i$) ter suporte compacto e ser suave na última desigualdade acima. 
    Ademais, como $|\alpha - \sigma| \leqslant k$, temos que
    \[
        \Vert D^\alpha v^\varepsilon - D^\alpha u \Vert_{\cL^p(\Omega)} \leqslant c \sum_{i=0}^N\sum_{\sigma \leqslant \alpha} \binom{\alpha}{\sigma} \Vert v_i^\varepsilon - u \Vert_{\cW^{k,p}(V_i)} \leqslant c\sum_{i=0}^N \Vert v_i^\varepsilon - u \Vert_{\cW^{k,p}(V_i)} \to 0 ,
    \]
    quando $\varepsilon \to 0$.
    Por fim, definindo $u_n := v^{\frac{1}{n}} \in \cC^{\infty}(\overline\Omega)$, para todo $n \in \bN$, chegamos a 
    \[
        \Vert u_n - u \Vert_{\cW^{k,p}(\Omega)} \to 0,
    \]
    quando $n \to \infty$, como era desejado.
\end{prf}

\section{Extensões}

Em alguns casos é mais viável trabalhar com funções definidas no espaço Euclidiano inteiro ao invés de funções definida em um aberto específico.
Nessa seção, veremos uma forma de estender funções em $\cW^{1,p}(\Omega)$ para aplicações em $\cW^{1,p}(\bR^n)$ por meio de um operador linear.

\begin{tbox} \label{thm:extensao}
        Sejam $\Omega$ um aberto limitado, com fronteira de classe $\cC^1$ e $\Omega'$ um aberto tal que $\Omega \Subset \Omega'$. Então, existe um operador linear limitado $E : \cW^{1,p}(\Omega) \to \cW^{1,p}(\bR ^n)$, com $1 \leqslant p < \infty$, tal que para cada $u \in \cW^{k,p}(\Omega)$, tem-se que
    \begin{enumerate}[leftmargin=*, label=\textbf{(\alph*)}]
        \item $Eu = u$ qtp em $\Omega$;
        \item $\supp Eu \subseteq \Omega'$;
        \item $\Vert Eu \Vert_{\cW^{1,p}(\bR^n)} \leqslant c \Vert u \Vert_{\cW^{1,p}(\Omega)}$, onde a constante $c$ depende apenas de $p$, $\Omega$ e $\Omega'$.
    \end{enumerate}
\end{tbox}
\begin{prf}
    Seja $\tilde x \in \partial\Omega$ e considere inicialmente que $\partial\Omega$ esteja contido no plano $\{x_n = 0\}$ perto de $\tilde x$.
    Dessa forma, podemos supor que existe uma bola $B = B(\tilde x, r)$ tal que
    \begin{equation} \label{eq:reflexao}
        \begin{aligned}
            B^+ &= B \cap \{x_n > 0\} \subseteq \overline\Omega;\\
            B^- &= B \cap \{x_n \leqslant 0\} \subseteq \bR^n \setminus \Omega.
        \end{aligned}
    \end{equation}
    \begin{figure}
        \centering
        \begin{tikzpicture}[scale=1.25]
    \coordinate (p0) at (-3.7400, 0.0000);
    \coordinate (p1) at (-3.7073, 0.1283);
    \coordinate (p2) at (-3.6706, 0.2501);
    \coordinate (p3) at (-3.6299, 0.3656);
    \coordinate (p4) at (-3.5853, 0.4749);
    \coordinate (p5) at (-3.5371, 0.5780);
    \coordinate (p6) at (-3.4853, 0.6752);
    \coordinate (p7) at (-3.4301, 0.7664);
    \coordinate (p8) at (-3.3716, 0.8519);
    \coordinate (p9) at (-3.3098, 0.9316);
    \coordinate (p10) at (-3.2451, 1.0058);
    \coordinate (p11) at (-3.1773, 1.0745);
    \coordinate (p12) at (-3.1068, 1.1377);
    \coordinate (p13) at (-3.0336, 1.1958);
    \coordinate (p14) at (-2.9578, 1.2486);
    \coordinate (p15) at (-2.8797, 1.2964);
    \coordinate (p16) at (-2.7992, 1.3392);
    \coordinate (p17) at (-2.7165, 1.3772);
    \coordinate (p18) at (-2.6318, 1.4104);
    \coordinate (p19) at (-2.5452, 1.4390);
    \coordinate (p20) at (-2.4567, 1.4630);
    \coordinate (p21) at (-2.3666, 1.4826);
    \coordinate (p22) at (-2.2750, 1.4979);
    \coordinate (p23) at (-2.1820, 1.5089);
    \coordinate (p24) at (-2.0876, 1.5159);
    \coordinate (p25) at (-1.9921, 1.5188);
    \coordinate (p26) at (-1.8956, 1.5178);
    \coordinate (p27) at (-1.7981, 1.5130);
    \coordinate (p28) at (-1.6999, 1.5045);
    \coordinate (p29) at (-1.6010, 1.4925);
    \coordinate (p30) at (-1.5016, 1.4769);
    \coordinate (p31) at (-1.4018, 1.4580);
    \coordinate (p32) at (-1.3017, 1.4358);
    \coordinate (p33) at (-1.2015, 1.4104);
    \coordinate (p34) at (-1.1012, 1.3819);
    \coordinate (p35) at (-1.0011, 1.3505);
    \coordinate (p36) at (-0.9011, 1.3162);
    \coordinate (p37) at (-0.8016, 1.2792);
    \coordinate (p38) at (-0.7025, 1.2396);
    \coordinate (p39) at (-0.6040, 1.1974);
    \coordinate (p40) at (-0.5063, 1.1527);
    \coordinate (p41) at (-0.4094, 1.1058);
    \coordinate (p42) at (-0.3135, 1.0566);
    \coordinate (p43) at (-0.2188, 1.0053);
    \coordinate (p44) at (-0.1252, 0.9520);
    \coordinate (p45) at (-0.0331, 0.8968);
    \coordinate (p46) at (0.0576, 0.8397);
    \coordinate (p47) at (0.1466, 0.7810);
    \coordinate (p48) at (0.2339, 0.7207);
    \coordinate (p49) at (0.3193, 0.6589);
    \coordinate (p50) at (0.4027, 0.5957);
    \coordinate (p51) at (0.4840, 0.5313);
    \coordinate (p52) at (0.5630, 0.4657);
    \coordinate (p53) at (0.6396, 0.3990);
    \coordinate (p54) at (0.7137, 0.3313);
    \coordinate (p55) at (0.7852, 0.2628);
    \coordinate (p56) at (0.8540, 0.1936);
    \coordinate (p57) at (0.9198, 0.1237);
    \coordinate (p58) at (0.9826, 0.0532);
    \coordinate (p59) at (1.0423, -0.0176);
    \coordinate (p60) at (1.0987, -0.0888);
    \coordinate (p61) at (1.1517, -0.1603);
    \coordinate (p62) at (1.2012, -0.2318);
    \coordinate (p63) at (1.2471, -0.3034);
    \coordinate (p64) at (1.2892, -0.3748);
    \coordinate (p65) at (1.3273, -0.4461);
    \coordinate (p66) at (1.3615, -0.5170);
    \coordinate (p67) at (1.3915, -0.5876);
    \coordinate (p68) at (1.4172, -0.6576);
    \coordinate (p69) at (1.4385, -0.7271);
    \coordinate (p70) at (1.4552, -0.7957);
    \coordinate (p71) at (1.4673, -0.8636);
    \coordinate (p72) at (1.4747, -0.9305);
    \coordinate (p73) at (1.4771, -0.9964);
    \coordinate (p74) at (1.4744, -1.0612);
    \coordinate (p75) at (1.4666, -1.1247);
    \coordinate (p76) at (1.4535, -1.1868);
    \coordinate (p77) at (1.4350, -1.2475);
    \coordinate (p78) at (1.4109, -1.3066);
    \coordinate (p79) at (1.3812, -1.3641);
    \coordinate (p80) at (1.3457, -1.4198);
    \coordinate (p81) at (1.3042, -1.4735);
    \coordinate (p82) at (1.2567, -1.5253);
    \coordinate (p83) at (1.2030, -1.5750);
    \coordinate (p84) at (1.1430, -1.6226);
    \coordinate (p85) at (1.0765, -1.6678);
    \coordinate (p86) at (1.0035, -1.7106);
    \coordinate (p87) at (0.9238, -1.7509);
    \coordinate (p88) at (0.8373, -1.7885);
    \coordinate (p89) at (0.7439, -1.8235);
    \coordinate (p90) at (0.6434, -1.8556);
    \coordinate (p91) at (0.5357, -1.8848);
    \coordinate (p92) at (0.4207, -1.9109);
    \coordinate (p93) at (0.2982, -1.9339);
    \coordinate (p94) at (0.1682, -1.9537);
    \coordinate (p95) at (0.0305, -1.9701);
    \coordinate (p96) at (-0.1151, -1.9830);
    \coordinate (p97) at (-0.2686, -1.9924);
    \coordinate (p98) at (-0.4302, -1.9981);
    \coordinate (p99) at (-0.6000, -2.0000);
    \coordinate (p100) at (-0.6000, -2.0000);
    \coordinate (p101) at (-0.7021, -2.0000);
    \coordinate (p102) at (-0.8022, -2.0000);
    \coordinate (p103) at (-0.9003, -1.9999);
    \coordinate (p104) at (-0.9966, -1.9999);
    \coordinate (p105) at (-1.0910, -1.9997);
    \coordinate (p106) at (-1.1835, -1.9996);
    \coordinate (p107) at (-1.2741, -1.9993);
    \coordinate (p108) at (-1.3629, -1.9989);
    \coordinate (p109) at (-1.4498, -1.9985);
    \coordinate (p110) at (-1.5350, -1.9979);
    \coordinate (p111) at (-1.6183, -1.9973);
    \coordinate (p112) at (-1.6998, -1.9964);
    \coordinate (p113) at (-1.7796, -1.9955);
    \coordinate (p114) at (-1.8576, -1.9943);
    \coordinate (p115) at (-1.9339, -1.9930);
    \coordinate (p116) at (-2.0085, -1.9916);
    \coordinate (p117) at (-2.0813, -1.9899);
    \coordinate (p118) at (-2.1525, -1.9880);
    \coordinate (p119) at (-2.2219, -1.9859);
    \coordinate (p120) at (-2.2898, -1.9835);
    \coordinate (p121) at (-2.3559, -1.9809);
    \coordinate (p122) at (-2.4205, -1.9781);
    \coordinate (p123) at (-2.4834, -1.9749);
    \coordinate (p124) at (-2.5447, -1.9715);
    \coordinate (p125) at (-2.6045, -1.9678);
    \coordinate (p126) at (-2.6627, -1.9638);
    \coordinate (p127) at (-2.7193, -1.9594);
    \coordinate (p128) at (-2.7744, -1.9548);
    \coordinate (p129) at (-2.8280, -1.9497);
    \coordinate (p130) at (-2.8800, -1.9443);
    \coordinate (p131) at (-2.9306, -1.9386);
    \coordinate (p132) at (-2.9797, -1.9325);
    \coordinate (p133) at (-3.0274, -1.9259);
    \coordinate (p134) at (-3.0736, -1.9190);
    \coordinate (p135) at (-3.1184, -1.9116);
    \coordinate (p136) at (-3.1618, -1.9038);
    \coordinate (p137) at (-3.2038, -1.8956);
    \coordinate (p138) at (-3.2444, -1.8869);
    \coordinate (p139) at (-3.2837, -1.8777);
    \coordinate (p140) at (-3.3216, -1.8681);
    \coordinate (p141) at (-3.3581, -1.8579);
    \coordinate (p142) at (-3.3934, -1.8473);
    \coordinate (p143) at (-3.4274, -1.8361);
    \coordinate (p144) at (-3.4600, -1.8244);
    \coordinate (p145) at (-3.4914, -1.8122);
    \coordinate (p146) at (-3.5216, -1.7994);
    \coordinate (p147) at (-3.5505, -1.7860);
    \coordinate (p148) at (-3.5782, -1.7720);
    \coordinate (p149) at (-3.6047, -1.7575);
    \coordinate (p150) at (-3.6300, -1.7423);
    \coordinate (p151) at (-3.6541, -1.7266);
    \coordinate (p152) at (-3.6771, -1.7102);
    \coordinate (p153) at (-3.6989, -1.6931);
    \coordinate (p154) at (-3.7196, -1.6754);
    \coordinate (p155) at (-3.7392, -1.6571);
    \coordinate (p156) at (-3.7577, -1.6380);
    \coordinate (p157) at (-3.7751, -1.6183);
    \coordinate (p158) at (-3.7915, -1.5978);
    \coordinate (p159) at (-3.8068, -1.5767);
    \coordinate (p160) at (-3.8211, -1.5548);
    \coordinate (p161) at (-3.8343, -1.5321);
    \coordinate (p162) at (-3.8466, -1.5088);
    \coordinate (p163) at (-3.8579, -1.4846);
    \coordinate (p164) at (-3.8682, -1.4597);
    \coordinate (p165) at (-3.8775, -1.4339);
    \coordinate (p166) at (-3.8859, -1.4074);
    \coordinate (p167) at (-3.8934, -1.3801);
    \coordinate (p168) at (-3.9000, -1.3519);
    \coordinate (p169) at (-3.9057, -1.3229);
    \coordinate (p170) at (-3.9105, -1.2930);
    \coordinate (p171) at (-3.9145, -1.2623);
    \coordinate (p172) at (-3.9176, -1.2307);
    \coordinate (p173) at (-3.9198, -1.1982);
    \coordinate (p174) at (-3.9213, -1.1647);
    \coordinate (p175) at (-3.9220, -1.1304);
    \coordinate (p176) at (-3.9219, -1.0952);
    \coordinate (p177) at (-3.9210, -1.0590);
    \coordinate (p178) at (-3.9194, -1.0218);
    \coordinate (p179) at (-3.9170, -0.9837);
    \coordinate (p180) at (-3.9140, -0.9447);
    \coordinate (p181) at (-3.9102, -0.9046);
    \coordinate (p182) at (-3.9057, -0.8635);
    \coordinate (p183) at (-3.9006, -0.8214);
    \coordinate (p184) at (-3.8948, -0.7783);
    \coordinate (p185) at (-3.8884, -0.7342);
    \coordinate (p186) at (-3.8813, -0.6889);
    \coordinate (p187) at (-3.8737, -0.6427);
    \coordinate (p188) at (-3.8654, -0.5953);
    \coordinate (p189) at (-3.8566, -0.5469);
    \coordinate (p190) at (-3.8472, -0.4974);
    \coordinate (p191) at (-3.8372, -0.4467);
    \coordinate (p192) at (-3.8268, -0.3950);
    \coordinate (p193) at (-3.8158, -0.3420);
    \coordinate (p194) at (-3.8043, -0.2880);
    \coordinate (p195) at (-3.7924, -0.2328);
    \coordinate (p196) at (-3.7799, -0.1764);
    \coordinate (p197) at (-3.7670, -0.1188);
    \coordinate (p198) at (-3.7537, -0.0600);
    \coordinate (p199) at (-3.7400, 0.0000);

    \draw[thick, tangerine, fill=tangerine, fill opacity=0.1, dashed] (p0) -- (p1) -- (p2) -- (p3) -- (p4) -- (p5) -- (p6) -- (p7) -- (p8) -- (p9) -- (p10) -- (p11) -- (p12) -- (p13) -- (p14) -- (p15) -- (p16) -- (p17) -- (p18) -- (p19) -- (p20) -- (p21) -- (p22) -- (p23) -- (p24) -- (p25) -- (p26) -- (p27) -- (p28) -- (p29) -- (p30) -- (p31) -- (p32) -- (p33) -- (p34) -- (p35) -- (p36) -- (p37) -- (p38) -- (p39) -- (p40) -- (p41) -- (p42) -- (p43) -- (p44) -- (p45) -- (p46) -- (p47) -- (p48) -- (p49) -- (p50) -- (p51) -- (p52) -- (p53) -- (p54) -- (p55) -- (p56) -- (p57) -- (p58) -- (p59) -- (p60) -- (p61) -- (p62) -- (p63) -- (p64) -- (p65) -- (p66) -- (p67) -- (p68) -- (p69) -- (p70) -- (p71) -- (p72) -- (p73) -- (p74) -- (p75) -- (p76) -- (p77) -- (p78) -- (p79) -- (p80) -- (p81) -- (p82) -- (p83) -- (p84) -- (p85) -- (p86) -- (p87) -- (p88) -- (p89) -- (p90) -- (p91) -- (p92) -- (p93) -- (p94) -- (p95) -- (p96) -- (p97) -- (p98) -- (p99) -- (p100) -- (p101) -- (p102) -- (p103) -- (p104) -- (p105) -- (p106) -- (p107) -- (p108) -- (p109) -- (p110) -- (p111) -- (p112) -- (p113) -- (p114) -- (p115) -- (p116) -- (p117) -- (p118) -- (p119) -- (p120) -- (p121) -- (p122) -- (p123) -- (p124) -- (p125) -- (p126) -- (p127) -- (p128) -- (p129) -- (p130) -- (p131) -- (p132) -- (p133) -- (p134) -- (p135) -- (p136) -- (p137) -- (p138) -- (p139) -- (p140) -- (p141) -- (p142) -- (p143) -- (p144) -- (p145) -- (p146) -- (p147) -- (p148) -- (p149) -- (p150) -- (p151) -- (p152) -- (p153) -- (p154) -- (p155) -- (p156) -- (p157) -- (p158) -- (p159) -- (p160) -- (p161) -- (p162) -- (p163) -- (p164) -- (p165) -- (p166) -- (p167) -- (p168) -- (p169) -- (p170) -- (p171) -- (p172) -- (p173) -- (p174) -- (p175) -- (p176) -- (p177) -- (p178) -- (p179) -- (p180) -- (p181) -- (p182) -- (p183) -- (p184) -- (p185) -- (p186) -- (p187) -- (p188) -- (p189) -- (p190) -- (p191) -- (p192) -- (p193) -- (p194) -- (p195) -- (p196) -- (p197) -- (p198) -- (p199);            

    \draw (-1.1,-2) circle (0.78);

    \filldraw[fill=tangerine!30, draw=tangerine!60!black] 
    (-1.1, -2) ++(-0.78, 0) % move to leftmost point of the circle
    arc[start angle=180, end angle=0, radius=0.78];
    % Close the path by going back to the center
    \draw (-1.1,-2) -- ++(-0.78,0);
    \draw (-1.1,-2) -- ++(0.78,0);

    \node[scale=1.25] at (-3,0.6) {$\Omega$};

    \filldraw (-1.1,-2) circle (0.035);
    \node[below] at (-1.1,-2) {$\tilde x$};
    \node[below=11pt] at (-1.1,-2) {$B^-$};
    \node[above=11pt] at (-1.1,-2) {$B^+$};

\end{tikzpicture}
        \caption{Representação visual de (\ref{eq:reflexao}). Note que perto de $\tilde x$, a fronteira de $\Omega$ é plana.\\Fonte: Autoral.}
        \label{fig:BmaisBmenos}
    \end{figure}
    (ver Figura \ref{fig:BmaisBmenos}).
    Além disso, assuma que $u \in \cC^\infty(\overline\Omega)$ e defina
    \begin{equation} \label{eq:barux}
        \bar u(x) =
        \left\{ 
            \begin{array}{ll}
                u(x), & \text{se } x\in B^+;\\
                -3u(x_1,\dots,x_{n-1}, -x_n) + 4u(x_1,\dots,x_{n-1},-\frac{x_n}{2}), & \text{se } x \in B^-,
            \end{array}
        \right.
    \end{equation}
    a qual chamamos de reflexão de ordem superior da função $u$ de $B^+$ a $B^-$.
    Afirmamos que $\bar u \in \cC^1(B)$.
    Com efeito, denotando $u^- = \bar u \big|_{B^-}$, $u^+ = \bar u \big|_{B^+}$, podemos ver que
    \[
        \dfrac{\partial u^-}{\partial x_n} = \dfrac{\partial u^+}{\partial x_n} \;\text{ em }\; \{x_n = 0\}.
    \]
    De fato, pela regra da cadeia, podemos escrever
    \[
        \dfrac{\partial u^-}{\partial x_n}(x_1,\dots,x_n) = 3 \frac{\partial u}{\partial x_n}(x_1,\dots,x_{n-1}, -x_n)  - 2\frac{\partial u}{\partial x_n}(x_1,\dots,x_{n-1},\tfrac{x_n}{2}),
    \]
    quando $x_n = 0$, obtemos
    \[
        \begin{aligned}
            \dfrac{\partial u^-}{\partial x_n}(x_1,\dots,x_{n-1},0)
         &= 3\frac{\partial u}{\partial x_n}(x_1,\dots,x_{n-1},0) - 2\frac{\partial u}{\partial x_n}(x_1,\dots,x_{n-1},0) = \frac{\partial u}{\partial x_n}(x_1,\dots,x_{n-1},0)\\[5pt] 
         &= \dfrac{\partial u^+}{\partial x_n}(x_1,\dots,x_{n-1},0).
        \end{aligned}
    \]
    Também é verdade que
    \[
        u^+ = u^- \,\text{ e }\, \dfrac{\partial u^-}{\partial x_i} = \dfrac{\partial u^+}{\partial x_i}  \;\text{ em }\; \{x_n = 0\},
    \]
    para todo $i = 1,2,\dots,n-1$ (ver \ref{eq:reflexao}). Portanto, $D^\alpha u^- = D^\alpha u^+$ em $\{x_n = 0\}$ com $|\alpha| \leqslant 1$. Sendo assim, $\bar u \in \cC^1 (B)$, pois fora de $\{x_n = 0\}$, as componentes de $\bar u$ já eram de classe $\cC^\infty$, então apenas restava verificar que em $B^+ \cap B^- = B \cap \{x_n = 0\}$ as componentes em $B^+$ e $B^-$ se igualavam, implicando a continuidade $\bar u$ e suas derivadas.

    Agora, desejamos mostrar que 
    \begin{equation} \label{eq:desigualdade-B-Bmais}
        \Vert \bar u \Vert_{\cW^{1,p}(B)} \leqslant c \Vert u \Vert_{\cW^{1,p}(B^+)},
    \end{equation}
    onde $c$ é uma constante positiva que não depende de $u$.
    De fato, sabemos que
    \[
        \Vert \bar u \Vert_{\cW^{1,p}(B)}^p = \sum_{|\alpha| \leqslant 1} \Vert D^\alpha\bar u \Vert_{\cL^P(B)}^p = \sum_{|\alpha| \leqslant 1} \left[\int_B |D^\alpha \bar u| ^p \,dx\right].
    \]
    Como $B = B^+ \cup B^-$, e denotando $(x_1,\dots,x_{n-1})$ por $x'$ podemos reescrever o último somatório acima da seguinte forma:
    % \small{
    % \[
    %     \begin{aligned}
    %         \sum_{|\alpha| \leqslant 1} \left[\int_B |D^\alpha \bar u| ^p \,dx\right] &= \sum_{|\alpha| \leqslant 1} \left[ \int_{B^+} |D^\alpha u(x)|^p \,dx + \int_{B^-} |4D^\alpha u(x',-\tfrac{x_n}{2}) - 3D^\alpha u(x',-x_n) |^p \,dx \right]\\
    %         &\leqslant \sum_{|a| \leqslant 1} \left[ \int_{B^+} |D^\alpha u(x)|^p \,dx + 3 \cdot 2^p\int_{B^-} |D^\alpha u(x',-x_n)|^p \,dx + 4 \cdot 2^p\int_{B^-} |D^\alpha u(x',-x_n)|^p \,dx   \right]
    %     \end{aligned}
    % \]}
    \[
        \sum_{|\alpha| \leqslant 1} \left[\int_B |D^\alpha \bar u| ^p \,dx\right] = \sum_{|\alpha| \leqslant 1} \left[ \int_{B^+} |D^\alpha u(x)|^p \,dx + \int_{B^-} |4D^\alpha [u(x',-\tfrac{x_n}{2})] - 3D^\alpha [u(x',-x_n)] |^p \,dx \right]
    \]
    \[
        \leqslant \sum_{|\alpha| \leqslant 1} \left[ \int_{B^+} |D^\alpha u(x)|^p \,dx + 3 \cdot 2^p\int_{B^-} |D^\alpha u(x',-x_n)|^p \,dx + 4 \cdot 2^p\int_{B^-} |D^\alpha u(x',-x_n)|^p \,dx   \right], 
    \]
    onde usamos o fato de que
    \[
        (a + b)^p \leqslant (2 \max\{a,b\})^p = 2^p \max\{a,b\}^p \leqslant 2^p (a^p + b^p),
    \]
    para todo $a,b \geqslant 0$. Porém, $-x_n, -\tfrac{x_n}{2} \geqslant 0$ em $B^-$, então podemos considerar, através de uma mudança de variáveis ($y_i = x_i$ para todo $i = 1,\dots,n-1$ e $y_n = -x_n$), que as integrais sobre $B^-$ são integrais sobre $B^+$, para encontrar
    \[
        \Vert \bar u \Vert_{\cW^{1,p}(B)}^p = \sum_{|\alpha| \leqslant 1} \left[\int_B |D^\alpha \bar u| ^p \,dx\right] \leqslant c\sum_{|\alpha| \leqslant 1} \left[ \int_{B^+}|D^\alpha u|^p \,dx \right] = c \Vert u \Vert_{\cW^{1,p}(B^+)}^p.
    \]
    Portanto,
    \[
        \Vert \bar u \Vert_{\cW^{1,p}(B)} \leqslant c \Vert u \Vert_{\cW^{1,p}(B^+)}.
    \]

    Por outro lado, se $\partial\Omega$ não está necessáriamente contido no plano $\{x_n = 0\}$ perto de $\tilde x$, temos que existe um homeomorfismo $\Phi$, com inversa $\Psi$, que planifica $\partial \Omega$ perto de $\tilde x$ (ver Figura \ref{fig:homeomorfismo}),  basta usar a função $\gamma$ de classe $\cC^1$ da Definição \ref{def:fronteira-ck} e definir $\Phi$ por
    \begin{equation} \label{eq:Phi}
        \Phi(x) = (x_1,x_2,...,x_{n-1}, x_n - \gamma(x_1,\dots,x_{n-1})).
    \end{equation}
    De forma análoga, definimos $\Psi$ por
    \begin{equation} \label{eq:Psi}
        \Psi(y) = (y_1,y_2,\dots,y_{n-1},y_n + \gamma(y_1,\dots,y_{n-1})).
    \end{equation}
    \begin{figure}
        \centering
        \begin{tikzpicture}
    \begin{scope}[shift={(-3,0)}]
        \coordinate (1-001) at (1.500, 0.000);
        \coordinate (1-002) at (1.497, 0.095);
        \coordinate (1-003) at (1.488, 0.190);
        \coordinate (1-004) at (1.473, 0.284);
        \coordinate (1-005) at (1.452, 0.377);
        \coordinate (1-006) at (1.425, 0.468);
        \coordinate (1-007) at (1.393, 0.557);
        \coordinate (1-008) at (1.354, 0.645);
        \coordinate (1-009) at (1.311, 0.729);
        \coordinate (1-010) at (1.262, 0.811);
        \coordinate (1-011) at (1.208, 0.889);
        \coordinate (1-012) at (1.149, 0.964);
        \coordinate (1-013) at (1.086, 1.035);
        \coordinate (1-014) at (1.018, 1.102);
        \coordinate (1-015) at (0.946, 1.164);
        \coordinate (1-016) at (0.870, 1.222);
        \coordinate (1-017) at (0.791, 1.275);
        \coordinate (1-018) at (0.708, 1.322);
        \coordinate (1-019) at (0.623, 1.364);
        \coordinate (1-020) at (0.535, 1.401);
        \coordinate (1-021) at (0.445, 1.432);
        \coordinate (1-022) at (0.354, 1.458);
        \coordinate (1-023) at (0.260, 1.477);
        \coordinate (1-024) at (0.166, 1.491);
        \coordinate (1-025) at (0.071, 1.498);
        \coordinate (1-026) at (-0.024, 1.500);
        \coordinate (1-027) at (-0.119, 1.495);
        \coordinate (1-028) at (-0.213, 1.485);
        \coordinate (1-029) at (-0.307, 1.468);
        \coordinate (1-030) at (-0.400, 1.446);
        \coordinate (1-031) at (-0.491, 1.418);
        \coordinate (1-032) at (-0.580, 1.384);
        \coordinate (1-033) at (-0.666, 1.344);
        \coordinate (1-034) at (-0.750, 1.299);
        \coordinate (1-035) at (-0.831, 1.249);
        \coordinate (1-036) at (-0.908, 1.194);
        \coordinate (1-037) at (-0.982, 1.134);
        \coordinate (1-038) at (-1.052, 1.069);
        \coordinate (1-039) at (-1.118, 1.000);
        \coordinate (1-040) at (-1.179, 0.927);
        \coordinate (1-041) at (-1.236, 0.851);
        \coordinate (1-042) at (-1.287, 0.771);
        \coordinate (1-043) at (-1.333, 0.687);
        \coordinate (1-044) at (-1.374, 0.601);
        \coordinate (1-045) at (-1.410, 0.513);
        \coordinate (1-046) at (-1.439, 0.423);
        \coordinate (1-047) at (-1.463, 0.330);
        \coordinate (1-048) at (-1.481, 0.237);
        \coordinate (1-049) at (-1.493, 0.143);
        \coordinate (1-050) at (-1.499, 0.048);
        \coordinate (1-051) at (-1.499, -0.048);
        \coordinate (1-052) at (-1.493, -0.143);
        \coordinate (1-053) at (-1.481, -0.237);
        \coordinate (1-054) at (-1.463, -0.330);
        \coordinate (1-055) at (-1.439, -0.423);
        \coordinate (1-056) at (-1.410, -0.513);
        \coordinate (1-057) at (-1.374, -0.601);
        \coordinate (1-058) at (-1.333, -0.687);
        \coordinate (1-059) at (-1.287, -0.771);
        \coordinate (1-060) at (-1.236, -0.851);
        \coordinate (1-061) at (-1.179, -0.927);
        \coordinate (1-062) at (-1.118, -1.000);
        \coordinate (1-063) at (-1.052, -1.069);
        \coordinate (1-064) at (-0.982, -1.134);
        \coordinate (1-065) at (-0.908, -1.194);
        \coordinate (1-066) at (-0.831, -1.249);
        \coordinate (1-067) at (-0.750, -1.299);
        \coordinate (1-068) at (-0.666, -1.344);
        \coordinate (1-069) at (-0.580, -1.384);
        \coordinate (1-070) at (-0.491, -1.418);
        \coordinate (1-071) at (-0.400, -1.446);
        \coordinate (1-072) at (-0.307, -1.468);
        \coordinate (1-073) at (-0.213, -1.485);
        \coordinate (1-074) at (-0.119, -1.495);
        \coordinate (1-075) at (-0.024, -1.500);
        \coordinate (1-076) at (0.071, -1.498);
        \coordinate (1-077) at (0.166, -1.491);
        \coordinate (1-078) at (0.260, -1.477);
        \coordinate (1-079) at (0.354, -1.458);
        \coordinate (1-080) at (0.445, -1.432);
        \coordinate (1-081) at (0.535, -1.401);
        \coordinate (1-082) at (0.623, -1.364);
        \coordinate (1-083) at (0.708, -1.322);
        \coordinate (1-084) at (0.791, -1.275);
        \coordinate (1-085) at (0.870, -1.222);
        \coordinate (1-086) at (0.946, -1.164);
        \coordinate (1-087) at (1.018, -1.102);
        \coordinate (1-088) at (1.086, -1.035);
        \coordinate (1-089) at (1.149, -0.964);
        \coordinate (1-090) at (1.208, -0.889);
        \coordinate (1-091) at (1.262, -0.811);
        \coordinate (1-092) at (1.311, -0.729);
        \coordinate (1-093) at (1.354, -0.645);
        \coordinate (1-094) at (1.393, -0.557);
        \coordinate (1-095) at (1.425, -0.468);
        \coordinate (1-096) at (1.452, -0.377);
        \coordinate (1-097) at (1.473, -0.284);
        \coordinate (1-098) at (1.488, -0.190);
        \coordinate (1-099) at (1.497, -0.095);
        \coordinate (1-100) at (1.500, -0.000);

        \coordinate (2-001) at (-2.360, -0.425);
        \coordinate (2-002) at (-2.271, -0.468);
        \coordinate (2-003) at (-2.185, -0.508);
        \coordinate (2-004) at (-2.101, -0.544);
        \coordinate (2-005) at (-2.019, -0.577);
        \coordinate (2-006) at (-1.939, -0.607);
        \coordinate (2-007) at (-1.861, -0.633);
        \coordinate (2-008) at (-1.786, -0.657);
        \coordinate (2-009) at (-1.712, -0.678);
        \coordinate (2-010) at (-1.640, -0.696);
        \coordinate (2-011) at (-1.570, -0.711);
        \coordinate (2-012) at (-1.502, -0.723);
        \coordinate (2-013) at (-1.436, -0.733);
        \coordinate (2-014) at (-1.372, -0.739);
        \coordinate (2-015) at (-1.309, -0.744);
        \coordinate (2-016) at (-1.248, -0.746);
        \coordinate (2-017) at (-1.189, -0.745);
        \coordinate (2-018) at (-1.132, -0.742);
        \coordinate (2-019) at (-1.076, -0.737);
        \coordinate (2-020) at (-1.021, -0.730);
        \coordinate (2-021) at (-0.968, -0.720);
        \coordinate (2-022) at (-0.917, -0.709);
        \coordinate (2-023) at (-0.867, -0.695);
        \coordinate (2-024) at (-0.818, -0.680);
        \coordinate (2-025) at (-0.771, -0.662);
        \coordinate (2-026) at (-0.725, -0.643);
        \coordinate (2-027) at (-0.680, -0.622);
        \coordinate (2-028) at (-0.637, -0.600);
        \coordinate (2-029) at (-0.595, -0.576);
        \coordinate (2-030) at (-0.554, -0.550);
        \coordinate (2-031) at (-0.514, -0.523);
        \coordinate (2-032) at (-0.475, -0.495);
        \coordinate (2-033) at (-0.437, -0.465);
        \coordinate (2-034) at (-0.400, -0.434);
        \coordinate (2-035) at (-0.364, -0.402);
        \coordinate (2-036) at (-0.329, -0.369);
        \coordinate (2-037) at (-0.295, -0.334);
        \coordinate (2-038) at (-0.261, -0.299);
        \coordinate (2-039) at (-0.229, -0.263);
        \coordinate (2-040) at (-0.197, -0.226);
        \coordinate (2-041) at (-0.166, -0.189);
        \coordinate (2-042) at (-0.135, -0.150);
        \coordinate (2-043) at (-0.105, -0.111);
        \coordinate (2-044) at (-0.076, -0.072);
        \coordinate (2-045) at (-0.047, -0.032);
        \coordinate (2-046) at (-0.019, 0.008);
        \coordinate (2-047) at (0.008, 0.049);
        \coordinate (2-048) at (0.036, 0.090);
        \coordinate (2-049) at (0.063, 0.131);
        \coordinate (2-050) at (0.089, 0.172);
        \coordinate (2-051) at (0.116, 0.214);
        \coordinate (2-052) at (0.142, 0.255);
        \coordinate (2-053) at (0.167, 0.296);
        \coordinate (2-054) at (0.193, 0.337);
        \coordinate (2-055) at (0.219, 0.378);
        \coordinate (2-056) at (0.244, 0.419);
        \coordinate (2-057) at (0.269, 0.459);
        \coordinate (2-058) at (0.295, 0.499);
        \coordinate (2-059) at (0.320, 0.538);
        \coordinate (2-060) at (0.345, 0.577);
        \coordinate (2-061) at (0.371, 0.615);
        \coordinate (2-062) at (0.396, 0.652);
        \coordinate (2-063) at (0.422, 0.689);
        \coordinate (2-064) at (0.448, 0.724);
        \coordinate (2-065) at (0.475, 0.759);
        \coordinate (2-066) at (0.501, 0.793);
        \coordinate (2-067) at (0.528, 0.826);
        \coordinate (2-068) at (0.555, 0.857);
        \coordinate (2-069) at (0.583, 0.888);
        \coordinate (2-070) at (0.611, 0.917);
        \coordinate (2-071) at (0.640, 0.945);
        \coordinate (2-072) at (0.669, 0.971);
        \coordinate (2-073) at (0.699, 0.996);
        \coordinate (2-074) at (0.730, 1.020);
        \coordinate (2-075) at (0.761, 1.041);
        \coordinate (2-076) at (0.793, 1.062);
        \coordinate (2-077) at (0.825, 1.080);
        \coordinate (2-078) at (0.859, 1.097);
        \coordinate (2-079) at (0.893, 1.111);
        \coordinate (2-080) at (0.928, 1.124);
        \coordinate (2-081) at (0.964, 1.135);
        \coordinate (2-082) at (1.001, 1.143);
        \coordinate (2-083) at (1.039, 1.150);
        \coordinate (2-084) at (1.078, 1.154);
        \coordinate (2-085) at (1.118, 1.156);
        \coordinate (2-086) at (1.159, 1.155);
        \coordinate (2-087) at (1.202, 1.152);
        \coordinate (2-088) at (1.245, 1.147);
        \coordinate (2-089) at (1.290, 1.139);
        \coordinate (2-090) at (1.336, 1.128);
        \coordinate (2-091) at (1.383, 1.115);
        \coordinate (2-092) at (1.432, 1.098);
        \coordinate (2-093) at (1.482, 1.079);
        \coordinate (2-094) at (1.534, 1.057);
        \coordinate (2-095) at (1.587, 1.032);
        \coordinate (2-096) at (1.641, 1.004);
        \coordinate (2-097) at (1.697, 0.973);
        \coordinate (2-098) at (1.755, 0.939);
        \coordinate (2-099) at (1.814, 0.901);
        \coordinate (2-100) at (1.875, 0.860);

        \draw[thick] (0,0) circle (1.5);

        \draw[fill=tangerine!40] (0.9773,1.1388) -- (2-081) -- (2-080) -- (2-079) -- (2-078) -- (2-077) -- (2-076) -- (2-075) -- (2-074) -- (2-073) -- (2-072) -- (2-071) -- (2-070) -- (2-069) -- (2-068) -- (2-067) -- (2-066) -- (2-065) -- (2-064) -- (2-063) -- (2-062) -- (2-061) -- (2-060) -- (2-059) -- (2-058) -- (2-057) -- (2-056) -- (2-055) -- (2-054) -- (2-053) -- (2-052) -- (2-051) -- (2-050) -- (2-049) -- (2-048) -- (2-047) -- (2-046) -- (2-045) -- (2-044) -- (2-043) -- (2-042) -- (2-041) -- (2-040) -- (2-039) -- (2-038) -- (2-037) -- (2-036) -- (2-035) -- (2-034) -- (2-033) -- (2-032) -- (2-031) -- (2-030) -- (2-029) -- (2-028) -- (2-027) -- (2-026) -- (2-025) -- (2-024) -- (2-023) -- (2-022) -- (2-021) -- (2-020) -- (2-019) -- (2-018) -- (2-017) -- (2-016) -- (2-015) -- (-1.3023,-0.7441) -- (1-058) -- (1-057) -- (1-056) -- (1-055) -- (1-054) -- (1-053) -- (1-052) -- (1-051) -- (1-050) -- (1-049) -- (1-048) -- (1-047) -- (1-046) -- (1-045) -- (1-044) -- (1-043) -- (1-042) -- (1-041) -- (1-040) -- (1-039) -- (1-038) -- (1-037) -- (1-036) -- (1-035) -- (1-034) -- (1-033) -- (1-032) -- (1-031) -- (1-030) -- (1-029) -- (1-028) -- (1-027) -- (1-026) -- (1-025) -- (1-024) -- (1-023) -- (1-022) -- (1-021) -- (1-020) -- (1-019) -- (1-018) -- (1-017) -- (1-016) -- (1-015) --cycle;

        \draw[very thick] (2-005) -- (2-006) -- (2-007) -- (2-008) -- (2-009) -- (2-010) -- (2-011) -- (2-012) -- (2-013) -- (2-014) -- (2-015) -- (2-016) -- (2-017) -- (2-018) -- (2-019) -- (2-020) -- (2-021) -- (2-022) -- (2-023) -- (2-024) -- (2-025) -- (2-026) -- (2-027) -- (2-028) -- (2-029) -- (2-030) -- (2-031) -- (2-032) -- (2-033) -- (2-034) -- (2-035) -- (2-036) -- (2-037) -- (2-038) -- (2-039) -- (2-040) -- (2-041) -- (2-042) -- (2-043) -- (2-044) -- (2-045) -- (2-046) -- (2-047) -- (2-048) -- (2-049) -- (2-050) -- (2-051) -- (2-052) -- (2-053) -- (2-054) -- (2-055) -- (2-056) -- (2-057) -- (2-058) -- (2-059) -- (2-060) -- (2-061) -- (2-062) -- (2-063) -- (2-064) -- (2-065) -- (2-066) -- (2-067) -- (2-068) -- (2-069) -- (2-070) -- (2-071) -- (2-072) -- (2-073) -- (2-074) -- (2-075) -- (2-076) -- (2-077) -- (2-078) -- (2-079) -- (2-080) -- (2-081) -- (2-082) -- (2-083) -- (2-084) -- (2-085) -- (2-086) -- (2-087) -- (2-088) -- (2-089) -- (2-090) -- (2-091) -- (2-092) -- (2-093) -- (2-094) -- (2-095) -- (2-096) -- (2-097) -- (2-098) -- (2-099) -- (2-100);

        \filldraw (0,0) circle (0.06) node[below=2pt, scale=1.2] {$\tilde x$};
        \node[above=15pt] at (2-007) {$\Omega$};
        \node[above left=6pt and 0pt] at (2-100) {$\partial\Omega$};
        \node[below] at (0,-1.5) {$B[\tilde x, r]$};
    \end{scope}

    \draw[-stealth, thick] (-2,1.8) to[bend left] node[above] {$\Phi$} (1.75,1.8);
    \draw[-stealth, thick] (1.75,-1.8) to[bend left] node[below] {$\Psi$} (-2,-1.8);

    \begin{scope}[shift={(3,0)}, scale=0.97]
        \draw[thick, fill=tangerine!40] (-1.87, 0) .. controls (-1.119,3.235) and (2.29,0.83) .. (2.3,0);
        \draw[thick] (2.3, 0) .. controls (2.295, -1.115) and (-2.17,-2.085) .. (-1.87,0);

        \draw[very thick] (-2.5,0) to (2.75,0);
        \filldraw (0,0) circle (0.06) node[below=2pt, scale=1.2] {$\tilde y = \Phi(\tilde x)$};

        \node[left] at (-2.5,0) {$\Phi(\partial \Omega)$};
        \node[right, align=center] at (2.75,0) {$\Phi(x', \gamma(x'))$\\$= (x',0)$};
        \node[above] at (2,1) {$\Phi(\Omega)$};
        \node[below] at (0,-1.5) {$\Phi(B[\tilde x, r])$};
    \end{scope}
\end{tikzpicture}
        \caption{Representação gráfica do homemorfismo $\Phi$.\\Fonte: Autoral. Baseada em \cite{evans-pde} p.p. 256.}
        \label{fig:homeomorfismo}
    \end{figure}
    \!\!Deste modo, é fácil ver que $\Psi^{-1} = \Phi$ e que $\Phi$ e $\Psi$ são de classe $\cC^1$ (por definição). Sendo assim, seja $y = \Phi(x)$ (ou seja, $x = \Psi(y)$) e definimos $u' \equiv u \circ \Psi$. Logo, como foi feito anteriormente ($u'$ é de classe $\cC^1$), podemos escolher uma bola $B = B(\tilde y, r)$ e definimos $\bar u'$ de forma que $\bar u' \in \cC^1(B)$ e
    \begin{equation} \label{eq:BBBB}
        \Vert \bar u' \Vert_{\cW^{1,p}(B)} \leqslant c \Vert u' \Vert_{\cW^{1,p}(B^+)}.
    \end{equation}
    Seja $B' = \Psi(B)$, assim conseguimos obter uma extensão $\bar u$ de $u$ para $B'$ com
    \[
        \Vert \bar u \Vert_{\cW^{1,p}(B')} \leqslant c \Vert u \Vert_{\cW^{1,p}(\Omega)}.
    \]
    De fato, para $\bar u \equiv \bar u' \circ \Phi$ (i.e., $\bar u' = \bar u \circ \Psi$), obtemos, utilizando o Teorema de Mudança de Variáveis (ver Teorema \ref{thm:mudanca-de-variaveis}), que
    \[
        \Vert D^\alpha \bar u' \Vert_{\cL^p(B)}^p = \int_B |D^\alpha \bar u'(y)|^p \,dy = \int_B |D^\alpha \bar u (\Psi (y))|^p \,dy = \int_{B'} |D^\alpha \bar u (x)|^p \,dx = \Vert D^\alpha \bar u \Vert_{\cL^p(B')}^p.
    \]
    Dessa forma, passando ao somatório, quando $|\alpha| \leqslant 1$, chegamos a
    \begin{equation} \label{eq:normaBigualnormaW}
        \Vert \bar u' \Vert_{\cW^{1,p}(B)} = \Vert  \bar u \Vert_{\cW^{1,p}(B')}.
    \end{equation}
    Além disso, é verdade que
    \[
        \begin{aligned}
            \Vert D^\alpha u' \Vert_{\cL^p(B^+)}^p = \int_{B^+} |D^\alpha u'(y) |^p \,dy &= \int_{B^+} |D^\alpha u (\Psi(y))|^p \,dy\\ &= \int_{\Psi(B^+)} |D^\alpha u(x)|^p \,dx \leqslant \int_{\Omega} |D^\alpha u(x)|^p \,dx = \Vert D^\alpha u \Vert_{\cL^p(\Omega)}^p,
        \end{aligned}
    \]
    para todo multi-índice $\alpha$, com $|\alpha| \leqslant 1$.
    Consequentemente, podemos escrever
    \begin{equation} \label{eq:normadesigualdadeBmais}
        \Vert u' \Vert_{\cW^{1,p}(B^+)} \leqslant \Vert u \Vert_{\cW^{1,p}(\Omega)}.
    \end{equation}
    Portanto, por (\ref{eq:BBBB}), (\ref{eq:normaBigualnormaW}) e (\ref{eq:normadesigualdadeBmais}), obtemos
    \begin{equation} \label{eq:desigualdadeWO}
        \Vert \bar u \Vert_{\cW^{1,p}(B')} = \Vert \bar u' \Vert_{\cW^{1,p}(B)} \leqslant c \Vert u' \Vert_{\cW^{1,p}(B^+)} \leqslant c \Vert u \Vert_{\cW^{1,p}(\Omega)},
    \end{equation}

    Como $\partial\Omega$ é compacto (pois $\Omega$ é limitado) e $\partial \Omega \subseteq \bigcup_{x \in \partial\Omega} B'_x$,
    onde $B'_x$ é aberto para cada $x \in \partial\Omega$, pois $B'_x = \Psi(B(\tilde y, r))$, e a imagem de um conjunto aberto por um homeomorfismo também é aberto. Pelo Teorema de Heine-Borel, existem pontos $x_i \in \partial\Omega$, abertos $B'_i$ e extensões $\bar u_i$ de $u$ em $B'_i$, com $i = 1,\dots,N$, de forma que $\partial\Omega \subseteq \bigcup_{i=1}^N B'_i$ (ver Figura \ref{fig:coberturafronteirab0}).
    Por outro lado, considere um aberto $B'_0 \Subset \Omega$ tal que
    \begin{equation} \label{eq:cobertura2}
        \Omega \subseteq \bigcup_{i=0}^N B'_i.
    \end{equation}
    \begin{figure}
        \centering
        \begin{tikzpicture}
    \coordinate (p0) at (-3.7400, 0.0000);
    \coordinate (p1) at (-3.7073, 0.1283);
    \coordinate (p2) at (-3.6706, 0.2501);
    \coordinate (p3) at (-3.6299, 0.3656);
    \coordinate (p4) at (-3.5853, 0.4749);
    \coordinate (p5) at (-3.5371, 0.5780);
    \coordinate (p6) at (-3.4853, 0.6752);
    \coordinate (p7) at (-3.4301, 0.7664);
    \coordinate (p8) at (-3.3716, 0.8519);
    \coordinate (p9) at (-3.3098, 0.9316);
    \coordinate (p10) at (-3.2451, 1.0058);
    \coordinate (p11) at (-3.1773, 1.0745);
    \coordinate (p12) at (-3.1068, 1.1377);
    \coordinate (p13) at (-3.0336, 1.1958);
    \coordinate (p14) at (-2.9578, 1.2486);
    \coordinate (p15) at (-2.8797, 1.2964);
    \coordinate (p16) at (-2.7992, 1.3392);
    \coordinate (p17) at (-2.7165, 1.3772);
    \coordinate (p18) at (-2.6318, 1.4104);
    \coordinate (p19) at (-2.5452, 1.4390);
    \coordinate (p20) at (-2.4567, 1.4630);
    \coordinate (p21) at (-2.3666, 1.4826);
    \coordinate (p22) at (-2.2750, 1.4979);
    \coordinate (p23) at (-2.1820, 1.5089);
    \coordinate (p24) at (-2.0876, 1.5159);
    \coordinate (p25) at (-1.9921, 1.5188);
    \coordinate (p26) at (-1.8956, 1.5178);
    \coordinate (p27) at (-1.7981, 1.5130);
    \coordinate (p28) at (-1.6999, 1.5045);
    \coordinate (p29) at (-1.6010, 1.4925);
    \coordinate (p30) at (-1.5016, 1.4769);
    \coordinate (p31) at (-1.4018, 1.4580);
    \coordinate (p32) at (-1.3017, 1.4358);
    \coordinate (p33) at (-1.2015, 1.4104);
    \coordinate (p34) at (-1.1012, 1.3819);
    \coordinate (p35) at (-1.0011, 1.3505);
    \coordinate (p36) at (-0.9011, 1.3162);
    \coordinate (p37) at (-0.8016, 1.2792);
    \coordinate (p38) at (-0.7025, 1.2396);
    \coordinate (p39) at (-0.6040, 1.1974);
    \coordinate (p40) at (-0.5063, 1.1527);
    \coordinate (p41) at (-0.4094, 1.1058);
    \coordinate (p42) at (-0.3135, 1.0566);
    \coordinate (p43) at (-0.2188, 1.0053);
    \coordinate (p44) at (-0.1252, 0.9520);
    \coordinate (p45) at (-0.0331, 0.8968);
    \coordinate (p46) at (0.0576, 0.8397);
    \coordinate (p47) at (0.1466, 0.7810);
    \coordinate (p48) at (0.2339, 0.7207);
    \coordinate (p49) at (0.3193, 0.6589);
    \coordinate (p50) at (0.4027, 0.5957);
    \coordinate (p51) at (0.4840, 0.5313);
    \coordinate (p52) at (0.5630, 0.4657);
    \coordinate (p53) at (0.6396, 0.3990);
    \coordinate (p54) at (0.7137, 0.3313);
    \coordinate (p55) at (0.7852, 0.2628);
    \coordinate (p56) at (0.8540, 0.1936);
    \coordinate (p57) at (0.9198, 0.1237);
    \coordinate (p58) at (0.9826, 0.0532);
    \coordinate (p59) at (1.0423, -0.0176);
    \coordinate (p60) at (1.0987, -0.0888);
    \coordinate (p61) at (1.1517, -0.1603);
    \coordinate (p62) at (1.2012, -0.2318);
    \coordinate (p63) at (1.2471, -0.3034);
    \coordinate (p64) at (1.2892, -0.3748);
    \coordinate (p65) at (1.3273, -0.4461);
    \coordinate (p66) at (1.3615, -0.5170);
    \coordinate (p67) at (1.3915, -0.5876);
    \coordinate (p68) at (1.4172, -0.6576);
    \coordinate (p69) at (1.4385, -0.7271);
    \coordinate (p70) at (1.4552, -0.7957);
    \coordinate (p71) at (1.4673, -0.8636);
    \coordinate (p72) at (1.4747, -0.9305);
    \coordinate (p73) at (1.4771, -0.9964);
    \coordinate (p74) at (1.4744, -1.0612);
    \coordinate (p75) at (1.4666, -1.1247);
    \coordinate (p76) at (1.4535, -1.1868);
    \coordinate (p77) at (1.4350, -1.2475);
    \coordinate (p78) at (1.4109, -1.3066);
    \coordinate (p79) at (1.3812, -1.3641);
    \coordinate (p80) at (1.3457, -1.4198);
    \coordinate (p81) at (1.3042, -1.4735);
    \coordinate (p82) at (1.2567, -1.5253);
    \coordinate (p83) at (1.2030, -1.5750);
    \coordinate (p84) at (1.1430, -1.6226);
    \coordinate (p85) at (1.0765, -1.6678);
    \coordinate (p86) at (1.0035, -1.7106);
    \coordinate (p87) at (0.9238, -1.7509);
    \coordinate (p88) at (0.8373, -1.7885);
    \coordinate (p89) at (0.7439, -1.8235);
    \coordinate (p90) at (0.6434, -1.8556);
    \coordinate (p91) at (0.5357, -1.8848);
    \coordinate (p92) at (0.4207, -1.9109);
    \coordinate (p93) at (0.2982, -1.9339);
    \coordinate (p94) at (0.1682, -1.9537);
    \coordinate (p95) at (0.0305, -1.9701);
    \coordinate (p96) at (-0.1151, -1.9830);
    \coordinate (p97) at (-0.2686, -1.9924);
    \coordinate (p98) at (-0.4302, -1.9981);
    \coordinate (p99) at (-0.6000, -2.0000);
    \coordinate (p100) at (-0.6000, -2.0000);
    \coordinate (p101) at (-0.7021, -2.0000);
    \coordinate (p102) at (-0.8022, -2.0000);
    \coordinate (p103) at (-0.9003, -1.9999);
    \coordinate (p104) at (-0.9966, -1.9999);
    \coordinate (p105) at (-1.0910, -1.9997);
    \coordinate (p106) at (-1.1835, -1.9996);
    \coordinate (p107) at (-1.2741, -1.9993);
    \coordinate (p108) at (-1.3629, -1.9989);
    \coordinate (p109) at (-1.4498, -1.9985);
    \coordinate (p110) at (-1.5350, -1.9979);
    \coordinate (p111) at (-1.6183, -1.9973);
    \coordinate (p112) at (-1.6998, -1.9964);
    \coordinate (p113) at (-1.7796, -1.9955);
    \coordinate (p114) at (-1.8576, -1.9943);
    \coordinate (p115) at (-1.9339, -1.9930);
    \coordinate (p116) at (-2.0085, -1.9916);
    \coordinate (p117) at (-2.0813, -1.9899);
    \coordinate (p118) at (-2.1525, -1.9880);
    \coordinate (p119) at (-2.2219, -1.9859);
    \coordinate (p120) at (-2.2898, -1.9835);
    \coordinate (p121) at (-2.3559, -1.9809);
    \coordinate (p122) at (-2.4205, -1.9781);
    \coordinate (p123) at (-2.4834, -1.9749);
    \coordinate (p124) at (-2.5447, -1.9715);
    \coordinate (p125) at (-2.6045, -1.9678);
    \coordinate (p126) at (-2.6627, -1.9638);
    \coordinate (p127) at (-2.7193, -1.9594);
    \coordinate (p128) at (-2.7744, -1.9548);
    \coordinate (p129) at (-2.8280, -1.9497);
    \coordinate (p130) at (-2.8800, -1.9443);
    \coordinate (p131) at (-2.9306, -1.9386);
    \coordinate (p132) at (-2.9797, -1.9325);
    \coordinate (p133) at (-3.0274, -1.9259);
    \coordinate (p134) at (-3.0736, -1.9190);
    \coordinate (p135) at (-3.1184, -1.9116);
    \coordinate (p136) at (-3.1618, -1.9038);
    \coordinate (p137) at (-3.2038, -1.8956);
    \coordinate (p138) at (-3.2444, -1.8869);
    \coordinate (p139) at (-3.2837, -1.8777);
    \coordinate (p140) at (-3.3216, -1.8681);
    \coordinate (p141) at (-3.3581, -1.8579);
    \coordinate (p142) at (-3.3934, -1.8473);
    \coordinate (p143) at (-3.4274, -1.8361);
    \coordinate (p144) at (-3.4600, -1.8244);
    \coordinate (p145) at (-3.4914, -1.8122);
    \coordinate (p146) at (-3.5216, -1.7994);
    \coordinate (p147) at (-3.5505, -1.7860);
    \coordinate (p148) at (-3.5782, -1.7720);
    \coordinate (p149) at (-3.6047, -1.7575);
    \coordinate (p150) at (-3.6300, -1.7423);
    \coordinate (p151) at (-3.6541, -1.7266);
    \coordinate (p152) at (-3.6771, -1.7102);
    \coordinate (p153) at (-3.6989, -1.6931);
    \coordinate (p154) at (-3.7196, -1.6754);
    \coordinate (p155) at (-3.7392, -1.6571);
    \coordinate (p156) at (-3.7577, -1.6380);
    \coordinate (p157) at (-3.7751, -1.6183);
    \coordinate (p158) at (-3.7915, -1.5978);
    \coordinate (p159) at (-3.8068, -1.5767);
    \coordinate (p160) at (-3.8211, -1.5548);
    \coordinate (p161) at (-3.8343, -1.5321);
    \coordinate (p162) at (-3.8466, -1.5088);
    \coordinate (p163) at (-3.8579, -1.4846);
    \coordinate (p164) at (-3.8682, -1.4597);
    \coordinate (p165) at (-3.8775, -1.4339);
    \coordinate (p166) at (-3.8859, -1.4074);
    \coordinate (p167) at (-3.8934, -1.3801);
    \coordinate (p168) at (-3.9000, -1.3519);
    \coordinate (p169) at (-3.9057, -1.3229);
    \coordinate (p170) at (-3.9105, -1.2930);
    \coordinate (p171) at (-3.9145, -1.2623);
    \coordinate (p172) at (-3.9176, -1.2307);
    \coordinate (p173) at (-3.9198, -1.1982);
    \coordinate (p174) at (-3.9213, -1.1647);
    \coordinate (p175) at (-3.9220, -1.1304);
    \coordinate (p176) at (-3.9219, -1.0952);
    \coordinate (p177) at (-3.9210, -1.0590);
    \coordinate (p178) at (-3.9194, -1.0218);
    \coordinate (p179) at (-3.9170, -0.9837);
    \coordinate (p180) at (-3.9140, -0.9447);
    \coordinate (p181) at (-3.9102, -0.9046);
    \coordinate (p182) at (-3.9057, -0.8635);
    \coordinate (p183) at (-3.9006, -0.8214);
    \coordinate (p184) at (-3.8948, -0.7783);
    \coordinate (p185) at (-3.8884, -0.7342);
    \coordinate (p186) at (-3.8813, -0.6889);
    \coordinate (p187) at (-3.8737, -0.6427);
    \coordinate (p188) at (-3.8654, -0.5953);
    \coordinate (p189) at (-3.8566, -0.5469);
    \coordinate (p190) at (-3.8472, -0.4974);
    \coordinate (p191) at (-3.8372, -0.4467);
    \coordinate (p192) at (-3.8268, -0.3950);
    \coordinate (p193) at (-3.8158, -0.3420);
    \coordinate (p194) at (-3.8043, -0.2880);
    \coordinate (p195) at (-3.7924, -0.2328);
    \coordinate (p196) at (-3.7799, -0.1764);
    \coordinate (p197) at (-3.7670, -0.1188);
    \coordinate (p198) at (-3.7537, -0.0600);
    \coordinate (p199) at (-3.7400, 0.0000);

\coordinate (q0) at (-3.3892, -0.1000);
\coordinate (q1) at (-3.3621, 0.0065);
\coordinate (q2) at (-3.3316, 0.1076);
\coordinate (q3) at (-3.2978, 0.2034);
\coordinate (q4) at (-3.2608, 0.2941);
\coordinate (q5) at (-3.2208, 0.3798);
\coordinate (q6) at (-3.1778, 0.4604);
\coordinate (q7) at (-3.1320, 0.5361);
\coordinate (q8) at (-3.0834, 0.6071);
\coordinate (q9) at (-3.0322, 0.6733);
\coordinate (q10) at (-2.9784, 0.7348);
\coordinate (q11) at (-2.9222, 0.7918);
\coordinate (q12) at (-2.8637, 0.8443);
\coordinate (q13) at (-2.8029, 0.8925);
\coordinate (q14) at (-2.7400, 0.9363);
\coordinate (q15) at (-2.6751, 0.9760);
\coordinate (q16) at (-2.6083, 1.0115);
\coordinate (q17) at (-2.5397, 1.0431);
\coordinate (q18) at (-2.4694, 1.0706);
\coordinate (q19) at (-2.3975, 1.0943);
\coordinate (q20) at (-2.3241, 1.1143);
\coordinate (q21) at (-2.2493, 1.1306);
\coordinate (q22) at (-2.1733, 1.1432);
\coordinate (q23) at (-2.0960, 1.1524);
\coordinate (q24) at (-2.0177, 1.1582);
\coordinate (q25) at (-1.9385, 1.1606);
\coordinate (q26) at (-1.8583, 1.1598);
\coordinate (q27) at (-1.7775, 1.1558);
\coordinate (q28) at (-1.6959, 1.1488);
\coordinate (q29) at (-1.6139, 1.1388);
\coordinate (q30) at (-1.5313, 1.1258);
\coordinate (q31) at (-1.4485, 1.1101);
\coordinate (q32) at (-1.3654, 1.0917);
\coordinate (q33) at (-1.2822, 1.0706);
\coordinate (q34) at (-1.1990, 1.0470);
\coordinate (q35) at (-1.1159, 1.0209);
\coordinate (q36) at (-1.0329, 0.9925);
\coordinate (q37) at (-0.9503, 0.9618);
\coordinate (q38) at (-0.8681, 0.9288);
\coordinate (q39) at (-0.7863, 0.8938);
\coordinate (q40) at (-0.7052, 0.8568);
\coordinate (q41) at (-0.6248, 0.8178);
\coordinate (q42) at (-0.5452, 0.7770);
\coordinate (q43) at (-0.4666, 0.7344);
\coordinate (q44) at (-0.3889, 0.6902);
\coordinate (q45) at (-0.3125, 0.6443);
\coordinate (q46) at (-0.2372, 0.5970);
\coordinate (q47) at (-0.1633, 0.5483);
\coordinate (q48) at (-0.0909, 0.4982);
\coordinate (q49) at (-0.0200, 0.4469);
\coordinate (q50) at (0.0492, 0.3945);
\coordinate (q51) at (0.1167, 0.3410);
\coordinate (q52) at (0.1823, 0.2865);
\coordinate (q53) at (0.2459, 0.2312);
\coordinate (q54) at (0.3074, 0.1750);
\coordinate (q55) at (0.3667, 0.1181);
\coordinate (q56) at (0.4238, 0.0607);
\coordinate (q57) at (0.4784, 0.0026);
\coordinate (q58) at (0.5306, -0.0558);
\coordinate (q59) at (0.5801, -0.1146);
\coordinate (q60) at (0.6269, -0.1737);
\coordinate (q61) at (0.6710, -0.2330);
\coordinate (q62) at (0.7120, -0.2924);
\coordinate (q63) at (0.7501, -0.3518);
\coordinate (q64) at (0.7850, -0.4111);
\coordinate (q65) at (0.8167, -0.4702);
\coordinate (q66) at (0.8450, -0.5291);
\coordinate (q67) at (0.8699, -0.5877);
\coordinate (q68) at (0.8913, -0.6458);
\coordinate (q69) at (0.9089, -0.7035);
\coordinate (q70) at (0.9229, -0.7605);
\coordinate (q71) at (0.9329, -0.8168);
\coordinate (q72) at (0.9390, -0.8724);
\coordinate (q73) at (0.9410, -0.9270);
\coordinate (q74) at (0.9388, -0.9808);
\coordinate (q75) at (0.9323, -1.0335);
\coordinate (q76) at (0.9214, -1.0851);
\coordinate (q77) at (0.9061, -1.1354);
\coordinate (q78) at (0.8861, -1.1845);
\coordinate (q79) at (0.8614, -1.2322);
\coordinate (q80) at (0.8319, -1.2784);
\coordinate (q81) at (0.7975, -1.3230);
\coordinate (q82) at (0.7580, -1.3660);
\coordinate (q83) at (0.7135, -1.4073);
\coordinate (q84) at (0.6637, -1.4467);
\coordinate (q85) at (0.6085, -1.4843);
\coordinate (q86) at (0.5479, -1.5198);
\coordinate (q87) at (0.4818, -1.5532);
\coordinate (q88) at (0.4100, -1.5845);
\coordinate (q89) at (0.3324, -1.6135);
\coordinate (q90) at (0.2490, -1.6401);
\coordinate (q91) at (0.1596, -1.6644);
\coordinate (q92) at (0.0642, -1.6861);
\coordinate (q93) at (-0.0375, -1.7052);
\coordinate (q94) at (-0.1454, -1.7216);
\coordinate (q95) at (-0.2597, -1.7352);
\coordinate (q96) at (-0.3805, -1.7459);
\coordinate (q97) at (-0.5080, -1.7537);
\coordinate (q98) at (-0.6421, -1.7584);
\coordinate (q99) at (-0.7830, -1.7600);
\coordinate (q100) at (-0.7830, -1.7600);
\coordinate (q101) at (-0.8677, -1.7600);
\coordinate (q102) at (-0.9508, -1.7600);
\coordinate (q103) at (-1.0323, -1.7600);
\coordinate (q104) at (-1.1122, -1.7599);
\coordinate (q105) at (-1.1905, -1.7598);
\coordinate (q106) at (-1.2673, -1.7596);
\coordinate (q107) at (-1.3425, -1.7594);
\coordinate (q108) at (-1.4162, -1.7591);
\coordinate (q109) at (-1.4884, -1.7588);
\coordinate (q110) at (-1.5590, -1.7583);
\coordinate (q111) at (-1.6282, -1.7577);
\coordinate (q112) at (-1.6959, -1.7570);
\coordinate (q113) at (-1.7621, -1.7562);
\coordinate (q114) at (-1.8268, -1.7553);
\coordinate (q115) at (-1.8901, -1.7542);
\coordinate (q116) at (-1.9520, -1.7530);
\coordinate (q117) at (-2.0125, -1.7516);
\coordinate (q118) at (-2.0715, -1.7500);
\coordinate (q119) at (-2.1292, -1.7483);
\coordinate (q120) at (-2.1855, -1.7463);
\coordinate (q121) at (-2.2404, -1.7442);
\coordinate (q122) at (-2.2940, -1.7418);
\coordinate (q123) at (-2.3462, -1.7392);
\coordinate (q124) at (-2.3971, -1.7363);
\coordinate (q125) at (-2.4467, -1.7333);
\coordinate (q126) at (-2.4950, -1.7299);
\coordinate (q127) at (-2.5420, -1.7263);
\coordinate (q128) at (-2.5877, -1.7224);
\coordinate (q129) at (-2.6322, -1.7183);
\coordinate (q130) at (-2.6754, -1.7138);
\coordinate (q131) at (-2.7174, -1.7090);
\coordinate (q132) at (-2.7582, -1.7039);
\coordinate (q133) at (-2.7977, -1.6985);
\coordinate (q134) at (-2.8361, -1.6928);
\coordinate (q135) at (-2.8733, -1.6866);
\coordinate (q136) at (-2.9093, -1.6802);
\coordinate (q137) at (-2.9441, -1.6733);
\coordinate (q138) at (-2.9779, -1.6661);
\coordinate (q139) at (-3.0104, -1.6585);
\coordinate (q140) at (-3.0419, -1.6505);
\coordinate (q141) at (-3.0723, -1.6421);
\coordinate (q142) at (-3.1015, -1.6332);
\coordinate (q143) at (-3.1297, -1.6240);
\coordinate (q144) at (-3.1568, -1.6143);
\coordinate (q145) at (-3.1829, -1.6041);
\coordinate (q146) at (-3.2079, -1.5935);
\coordinate (q147) at (-3.2319, -1.5824);
\coordinate (q148) at (-3.2549, -1.5708);
\coordinate (q149) at (-3.2769, -1.5587);
\coordinate (q150) at (-3.2979, -1.5461);
\coordinate (q151) at (-3.3179, -1.5331);
\coordinate (q152) at (-3.3370, -1.5194);
\coordinate (q153) at (-3.3551, -1.5053);
\coordinate (q154) at (-3.3723, -1.4906);
\coordinate (q155) at (-3.3886, -1.4754);
\coordinate (q156) at (-3.4039, -1.4596);
\coordinate (q157) at (-3.4184, -1.4432);
\coordinate (q158) at (-3.4319, -1.4262);
\coordinate (q159) at (-3.4447, -1.4086);
\coordinate (q160) at (-3.4565, -1.3905);
\coordinate (q161) at (-3.4675, -1.3717);
\coordinate (q162) at (-3.4777, -1.3523);
\coordinate (q163) at (-3.4870, -1.3322);
\coordinate (q164) at (-3.4956, -1.3115);
\coordinate (q165) at (-3.5033, -1.2902);
\coordinate (q166) at (-3.5103, -1.2681);
\coordinate (q167) at (-3.5165, -1.2455);
\coordinate (q168) at (-3.5220, -1.2221);
\coordinate (q169) at (-3.5267, -1.1980);
\coordinate (q170) at (-3.5307, -1.1732);
\coordinate (q171) at (-3.5340, -1.1477);
\coordinate (q172) at (-3.5366, -1.1214);
\coordinate (q173) at (-3.5385, -1.0945);
\coordinate (q174) at (-3.5397, -1.0667);
\coordinate (q175) at (-3.5403, -1.0383);
\coordinate (q176) at (-3.5402, -1.0090);
\coordinate (q177) at (-3.5394, -0.9790);
\coordinate (q178) at (-3.5381, -0.9481);
\coordinate (q179) at (-3.5361, -0.9165);
\coordinate (q180) at (-3.5336, -0.8841);
\coordinate (q181) at (-3.5305, -0.8508);
\coordinate (q182) at (-3.5268, -0.8167);
\coordinate (q183) at (-3.5225, -0.7818);
\coordinate (q184) at (-3.5177, -0.7460);
\coordinate (q185) at (-3.5123, -0.7093);
\coordinate (q186) at (-3.5065, -0.6718);
\coordinate (q187) at (-3.5001, -0.6334);
\coordinate (q188) at (-3.4933, -0.5941);
\coordinate (q189) at (-3.4860, -0.5539);
\coordinate (q190) at (-3.4782, -0.5128);
\coordinate (q191) at (-3.4699, -0.4708);
\coordinate (q192) at (-3.4612, -0.4278);
\coordinate (q193) at (-3.4521, -0.3839);
\coordinate (q194) at (-3.4426, -0.3390);
\coordinate (q195) at (-3.4327, -0.2932);
\coordinate (q196) at (-3.4223, -0.2464);
\coordinate (q197) at (-3.4117, -0.1986);
\coordinate (q198) at (-3.4006, -0.1498);
\coordinate (q199) at (-3.3892, -0.1000);





    \filldraw[ForestGreen!50!Black] (-3.740, 0.000) circle (0.07);    
    \filldraw[ForestGreen!50!Black] (-3.187, 1.064) circle (0.07);   
    \filldraw[ForestGreen!50!Black] (-2.088, 1.516) circle (0.07);   
    \filldraw[ForestGreen!50!Black] (-0.897, 1.315) circle (0.07);   
    \filldraw[ForestGreen!50!Black] (0.180, 0.758) circle (0.07);    
    \filldraw[ForestGreen!50!Black] (1.075, -0.059) circle (0.07);   
    \filldraw[ForestGreen!50!Black] (1.457, -1.172) circle (0.07);   
    \filldraw[ForestGreen!50!Black] (0.546, -1.882) circle (0.07);   
    \filldraw[ForestGreen!50!Black] (-0.661, -2.000) circle (0.07);  
    \filldraw[ForestGreen!50!Black] (-1.876, -1.994) circle (0.07);  
    \filldraw[ForestGreen!50!Black] (-3.089, -1.917) circle (0.07);  
    \filldraw[ForestGreen!50!Black] (-3.920, -1.199) circle (0.07);  
    \filldraw[ForestGreen!50!Black] (-3.740, 0.000) circle (0.07);   


    \draw[ForestGreen!50, dashed, fill=OliveGreen!50, fill opacity=0.3] (-3.740, 0.000) circle (0.71);    
    \draw[ForestGreen!50, dashed, fill=OliveGreen!50, fill opacity=0.3] (-3.187, 1.064) circle (0.73);   
    \draw[ForestGreen!50, dashed, fill=OliveGreen!50, fill opacity=0.3] (-2.088, 1.516) circle (0.74);   
    \draw[ForestGreen!50, dashed, fill=OliveGreen!50, fill opacity=0.3] (-0.897, 1.315) circle (0.67);   
    \draw[ForestGreen!50, dashed, fill=OliveGreen!50, fill opacity=0.3] (0.180, 0.758) circle (0.73);    
    \draw[ForestGreen!50, dashed, fill=OliveGreen!50, fill opacity=0.3] (1.075, -0.059) circle (0.71);   
    \draw[ForestGreen!50, dashed, fill=OliveGreen!50, fill opacity=0.3] (1.457, -1.172) circle (0.68);   
    \draw[ForestGreen!50, dashed, fill=OliveGreen!50, fill opacity=0.3] (0.546, -1.882) circle (0.74);   
    \draw[ForestGreen!50, dashed, fill=OliveGreen!50, fill opacity=0.3] (-0.661, -2.000) circle (0.71);  
    \draw[ForestGreen!50, dashed, fill=OliveGreen!50, fill opacity=0.3] (-1.876, -1.994) circle (0.75);  
    \draw[ForestGreen!50, dashed, fill=OliveGreen!50, fill opacity=0.3] (-3.089, -1.917) circle (0.70);  
    \draw[ForestGreen!50, dashed, fill=OliveGreen!50, fill opacity=0.3] (-3.920, -1.199) circle (0.68);  

    \draw[thick, ForestGreen!50!Black, fill=OliveGreen!50, fill opacity=0.3, dashed]  (q0) -- (q1) -- (q2) -- (q3) -- (q4) -- (q5) -- (q6) -- (q7) -- (q8) -- (q9) -- (q10) -- (q11) -- (q12) -- (q13) -- (q14) -- (q15) -- (q16) -- (q17) -- (q18) -- (q19) -- (q20) -- (q21) -- (q22) -- (q23) -- (q24) -- (q25) -- (q26) -- (q27) -- (q28) -- (q29) -- (q30) -- (q31) -- (q32) -- (q33) -- (q34) -- (q35) -- (q36) -- (q37) -- (q38) -- (q39) -- (q40) -- (q41) -- (q42) -- (q43) -- (q44) -- (q45) -- (q46) -- (q47) -- (q48) -- (q49) -- (q50) -- (q51) -- (q52) -- (q53) -- (q54) -- (q55) -- (q56) -- (q57) -- (q58) -- (q59) -- (q60) -- (q61) -- (q62) -- (q63) -- (q64) -- (q65) -- (q66) -- (q67) -- (q68) -- (q69) -- (q70) -- (q71) -- (q72) -- (q73) -- (q74) -- (q75) -- (q76) -- (q77) -- (q78) -- (q79) -- (q80) -- (q81) -- (q82) -- (q83) -- (q84) -- (q85) -- (q86) -- (q87) -- (q88) -- (q89) -- (q90) -- (q91) -- (q92) -- (q93) -- (q94) -- (q95) -- (q96) -- (q97) -- (q98) -- (q99) -- (q100) -- (q101) -- (q102) -- (q103) -- (q104) -- (q105) -- (q106) -- (q107) -- (q108) -- (q109) -- (q110) -- (q111) -- (q112) -- (q113) -- (q114) -- (q115) -- (q116) -- (q117) -- (q118) -- (q119) -- (q120) -- (q121) -- (q122) -- (q123) -- (q124) -- (q125) -- (q126) -- (q127) -- (q128) -- (q129) -- (q130) -- (q131) -- (q132) -- (q133) -- (q134) -- (q135) -- (q136) -- (q137) -- (q138) -- (q139) -- (q140) -- (q141) -- (q142) -- (q143) -- (q144) -- (q145) -- (q146) -- (q147) -- (q148) -- (q149) -- (q150) -- (q151) -- (q152) -- (q153) -- (q154) -- (q155) -- (q156) -- (q157) -- (q158) -- (q159) -- (q160) -- (q161) -- (q162) -- (q163) -- (q164) -- (q165) -- (q166) -- (q167) -- (q168) -- (q169) -- (q170) -- (q171) -- (q172) -- (q173) -- (q174) -- (q175) -- (q176) -- (q177) -- (q178) -- (q179) -- (q180) -- (q181) -- (q182) -- (q183) -- (q184) -- (q185) -- (q186) -- (q187) -- (q188) -- (q189) -- (q190) -- (q191) -- (q192) -- (q193) -- (q194) -- (q195) -- (q196) -- (q197) -- (q198) -- (q199);

    \node[below=2pt] at (-1.876, -1.994) {$\tilde x_i$};

    \draw[stealth-] (-2.2, -2.4) to[bend right] (-1.4,-3) node[right] {$B_i'$};

    \draw[stealth-, ForestGreen!50!black] (-4, -0.25) to[bend left] (-4.5,0.5) node[above] {$\partial\Omega$};
    
    \node at (-1.6,-0.4) {$B_0'$};

    \draw[thick, ForestGreen!50!Black]  (p0) -- (p1) -- (p2) -- (p3) -- (p4) -- (p5) -- (p6) -- (p7) -- (p8) -- (p9) -- (p10) -- (p11) -- (p12) -- (p13) -- (p14) -- (p15) -- (p16) -- (p17) -- (p18) -- (p19) -- (p20) -- (p21) -- (p22) -- (p23) -- (p24) -- (p25) -- (p26) -- (p27) -- (p28) -- (p29) -- (p30) -- (p31) -- (p32) -- (p33) -- (p34) -- (p35) -- (p36) -- (p37) -- (p38) -- (p39) -- (p40) -- (p41) -- (p42) -- (p43) -- (p44) -- (p45) -- (p46) -- (p47) -- (p48) -- (p49) -- (p50) -- (p51) -- (p52) -- (p53) -- (p54) -- (p55) -- (p56) -- (p57) -- (p58) -- (p59) -- (p60) -- (p61) -- (p62) -- (p63) -- (p64) -- (p65) -- (p66) -- (p67) -- (p68) -- (p69) -- (p70) -- (p71) -- (p72) -- (p73) -- (p74) -- (p75) -- (p76) -- (p77) -- (p78) -- (p79) -- (p80) -- (p81) -- (p82) -- (p83) -- (p84) -- (p85) -- (p86) -- (p87) -- (p88) -- (p89) -- (p90) -- (p91) -- (p92) -- (p93) -- (p94) -- (p95) -- (p96) -- (p97) -- (p98) -- (p99) -- (p100) -- (p101) -- (p102) -- (p103) -- (p104) -- (p105) -- (p106) -- (p107) -- (p108) -- (p109) -- (p110) -- (p111) -- (p112) -- (p113) -- (p114) -- (p115) -- (p116) -- (p117) -- (p118) -- (p119) -- (p120) -- (p121) -- (p122) -- (p123) -- (p124) -- (p125) -- (p126) -- (p127) -- (p128) -- (p129) -- (p130) -- (p131) -- (p132) -- (p133) -- (p134) -- (p135) -- (p136) -- (p137) -- (p138) -- (p139) -- (p140) -- (p141) -- (p142) -- (p143) -- (p144) -- (p145) -- (p146) -- (p147) -- (p148) -- (p149) -- (p150) -- (p151) -- (p152) -- (p153) -- (p154) -- (p155) -- (p156) -- (p157) -- (p158) -- (p159) -- (p160) -- (p161) -- (p162) -- (p163) -- (p164) -- (p165) -- (p166) -- (p167) -- (p168) -- (p169) -- (p170) -- (p171) -- (p172) -- (p173) -- (p174) -- (p175) -- (p176) -- (p177) -- (p178) -- (p179) -- (p180) -- (p181) -- (p182) -- (p183) -- (p184) -- (p185) -- (p186) -- (p187) -- (p188) -- (p189) -- (p190) -- (p191) -- (p192) -- (p193) -- (p194) -- (p195) -- (p196) -- (p197) -- (p198) -- (p199);
\end{tikzpicture}
        \caption{Representação visual dos conjuntos em (\ref{eq:cobertura2}). Note que $\{B_i'\}_{i=1}^N$ é uma cobertura finita para o conjunto $\Omega$.\\Fonte: Autoral.}
        \label{fig:coberturafronteirab0}
    \end{figure}


    Seja $\{\phi_i\}_{i=0}^N$ uma partição da unidade suave subordinada aos abertos $\{B'_i\}_{i=0}^N$ e defina
    \begin{equation}
        \bar u = \sum_{i=0}^N \phi_i \bar u_i,
    \end{equation}
    onde $\bar u_i$ está associada a $B'_i$ e $\bar u_0 = u$ (pois $\sum_{i=0}^N \phi_i = 1$). Deste modo, obtemos a desigualdade
    \begin{equation} \label{eq:xmmx}
        \Vert \bar u \Vert_{\cW^{1,p}(\bR^n)} \leqslant c \Vert u \Vert_{\cW^{1,p}(\Omega)}.
    \end{equation}
    Com efeito,
    \[
        \Vert \bar u \Vert^p_{\cW^{1,p}(\bR^n)} = \sum_{|\alpha| \leqslant 1} \Vert D^\alpha \bar u \Vert^p_{\cL^p(\bR^n)}
        % = \sum_{|\alpha| \leqslant 1} \left\Vert \sum_{i=0}^N D^\alpha(\phi_i \bar u_i) \right\Vert _{\cL^p(\bR^n)}^p 
        \leqslant \sum_{|\alpha| \leqslant 1} \sum_{i=0}^N \Vert D^\alpha (\phi_i \bar u_i) \Vert_{\cL^p(\bR)}^p,
    \]
    Logo, utilizando a regra de Leibniz (ver Teorema \ref{thm:propriedades-derivada-fraca}), inferimos
    \[
        \Vert \bar u \Vert_{\cW^{1,p}(\bR^n)}^p \leqslant \sum_{|\alpha| \leqslant 1} \sum_{i=0}^{N} \sum_{\sigma \leqslant \alpha} \binom{\alpha}{\sigma} \Vert D^\sigma \phi_i D^{\alpha - \sigma} \bar u_i \Vert_{\cL^p(\bR^n)}^p.
    \]
    Como $\supp D^\sigma \phi_i \subseteq \supp \phi_i \subseteq B'_i$, então o suporte de $D^\sigma \phi_i$ também é compacto (pois $B'_i$ é limitado), sendo assim $\max |D^\sigma \phi_i|$ existe em $B'_i$.
    Portanto, vale a desigualdade
    \[
        \begin{aligned}
            \Vert \bar u \Vert_{\cL^p(\bR^n)}^p &\leqslant c\sum_{|\alpha| \leqslant 1} \sum_{i=0}^{N} \sum_{\sigma \leqslant \alpha} \binom{\alpha}{\sigma} \Vert D^{\alpha - \sigma} \bar u_i \Vert_{\cL^p(B'_i)}^p \leqslant c\sum_{|\alpha| \leqslant 1} \sum_{i=0}^{N} \sum_{\sigma \leqslant \alpha} \binom{\alpha}{\sigma} \Vert \bar u_i \Vert_{\cW^{1,p}(B'_i)}^p,
        \end{aligned}
    \]
    onde $c$ é uma constante que depende de $B'_i$.
    Por fim, utilizando (\ref{eq:desigualdadeWO}) e o Teorema de Mudança de Variaveis (ver \ref{thm:mudanca-de-variaveis}), chegamos a
    \[
         \Vert \bar u \Vert^p_{\cW^{1,p}(\bR^n)} \leqslant c\sum_{|\alpha| \leqslant 1} \sum_{i=0}^{N} \sum_{\sigma \leqslant \alpha} \binom{\alpha}{\sigma} \Vert \bar u_i \Vert_{\cW^{1,p}(B'_i)}^p \leqslant c\Vert u_i \Vert_{\cW^{1,p}(\Omega)}^p \leqslant c \Vert u \Vert_{\cW^{1,p}(\Omega)}^p,
    \]
    onde $c$ depende de $B'_i$, $p$ e $N$.
    Portanto, deduzimos que
    \begin{equation} \label{eq:Mm}
        \Vert \bar u \Vert_{\cW^{1,p}(\bR^n)} \leqslant c \Vert u \Vert_{\cW^{1,p}(\Omega)}.
    \end{equation}
    Isto prova (\ref{eq:xmmx}).

    Como, por hipótese, $\Omega \Subset \Omega'$, então $\supp \bar u \subseteq \Omega'$, (pois $\supp \phi_i \subseteq B_i'$), diminuindo $B'_i$ se necessário.
    Defina também $\bar E: \cC^\infty(\overline\Omega) \to \cC^\infty(\bR^n)$ dada por $\bar E u = \bar u \in \cC^{\infty}(\bR^n)$.
    Temos que $\bar E$ é linear, pois \ref{eq:barux}, se $x \in B^+$, temos que
    \[
        \bar E(u + \lambda v)(x) = (\overline{u + \lambda v})(x) = (u + \lambda v)(x) = u(x) + \lambda v(x) = \bar u(x) + \lambda \bar v(x).
    \]
    Por outro lado, se $x \in B^-$
    \[
        \begin{aligned}
            \bar E(u + \lambda v)(x) = (\overline{u + \lambda v})(x) &= -3(u + \lambda v)(x_1,\dots,x_{n-1},-x_n) + 4(u + \lambda v)(x_1,\dots,x_{n-1},-\tfrac{x_n}{2})\\
            &= -3 u(x_1,\dots,x_{n-1},-x_n) + 4 u(x_1,\dots,x_{n-1},-\tfrac{x_n}{2})\\ &\quad\;+ \lambda\big[-\!3 v(x_1,\dots,x_{n-1},-x_n) + 4v(x_1,\dots,x_{n-1},-\tfrac{x_2}{2}) \big]\\[5pt]
            &= \bar u(x) + \lambda \bar v(x).
        \end{aligned}
    \]
    Além disso, é válido que
    \[
        \Vert \bar E u \Vert_{\cW^{1,p}(\bR^n)} = \Vert \bar u \Vert_{\cW^{1,p}(\bR^n)} \leqslant c \Vert u \Vert_{\cW^{1,p}(\Omega)}.
    \]
    e $\supp \bar E u  = \supp \bar u \subseteq \Omega'.$
    Sendo assim, definimos $E : \cW^{1,p}(\Omega) \to \cW^{1,p}(\bR^n)$ por
    \[
        Eu = \lim \bar u_k,
    \]
    onde $(u_k)$ é uma sequência de funções em $\cC^\infty(\overline\Omega)$ que converge para $u$ em $\cW^{1,p}(\Omega)$ (sabemos que essa sequência existe pois mostramos no Teorema \ref{thm:aprox3} que $\cC^\infty(\overline\Omega)$ é denso em $\cW^{1,p}(\Omega)$) e pela linearidade do limite, $E$ é linear.
    Podemos afirmar que o limite existe em $\cW^{1,p}(\bR^n)$, já que, por (\ref{eq:Mm}), vale
    \[
        \Vert \bar u_k - \bar u_\ell \Vert_{\cW^{1,p}(\bR^n)} \leqslant c \Vert u_k - u_\ell \Vert_{\cW^{1,p}(\Omega)} \to 0,
    \]
    quando $k,\ell \to \infty$.
    Logo $(\bar u_k)$ é de Cauchy em $\cW^{1,p}(\bR^n)$.
    Como $\cW^{1,p}(\bR^n)$ é completo (ver Teorema \ref{thm:sobolev-completo}), deduzimos que $\lim \bar u_k$ existe em $\cW^{1,p}(\bR^n)$.
    Além disso, é verdade que
    \[
        \begin{aligned}
            \Vert Eu - u \Vert_{\cW^{1,p}(\Omega)} &\leqslant \Vert Eu - \bar u_k \Vert_{\cW^{1,p}(\Omega)} + \Vert \bar u_k - u_k \Vert_{\cW^{1,p}(\Omega)} + \Vert u_k - u\Vert_{\cW^{1,p}(\Omega)}\\
            &\leqslant \Vert Eu - \bar u_k \Vert_{\cW^{1,p}(\bR^n)} + \Vert u_k - u \Vert_{\cW^{1,p}(\Omega)} \to 0,
        \end{aligned}
    \]
    quando $k \to \infty$,
    pois $u_k \to u$ em $\cW^{1,p}(\Omega)$, quando $k \to \infty$ e $\bar u_k = u_k$ em $\Omega$.
    Portanto, $Eu = u$ qtp em $\Omega$, provando o item \textbf{(a)}.

    Para verificar o ítem \textbf{(b)}, lembre que, por (\ref{eq:Mm}), $\supp \bar u_k \subseteq \Omega'$.
    Dessa forma, $\supp Eu \subseteq \Omega'$, já que
    \[
        \{ \lim u_k = Eu \neq 0\} \subseteq \{ \bar u_k \neq 0 \} \implies \supp Eu \subseteq \supp \bar u_k \subseteq \Omega'
    \]

    Por fim, para mostrar o item \textbf{(c)}, note que $E$ é um operador limitado, pois
    \[
        \Vert Eu \Vert_{\cW^{1,p}(\bR^n)} = \Vert \lim \bar u_k \Vert_{\cW^{1,p}(\bR^n)} = \lim \Vert \bar u_k \Vert_{\cW^{1,p}(\bR^n)} \leqslant c \lim \Vert u_k \Vert_{\cW^{1,p}(\Omega)} = c \Vert u \Vert_{\cW^{1,p}(\Omega)}.
    \]
    Portanto,
    \[
        \Vert Eu \Vert_{\cW^{1,p}(\bR^n)} \leqslant c \Vert u \Vert_{\cW^{1,p}(\Omega)}.
    \]
\end{prf}

\section{Traços}

Em alguns casos, no estudo de equações diferenciais parciais, é necessário impor condições de contorno na fronteira de um domínio. Para isso, precisamos entender o que significa restringir uma função em $\cW^{1,p}(\Omega)$ à fronteira de $\Omega$. Nesta seção, veremos como isso pode ser feito por meio do operador traço.

\begin{tbox} \label{thm:traco-1}
    Seja $\Omega$ um aberto limitado, com fronteira de classe $\cC^1$. Então, existe um operador linear limitado $T : \cW^{1,p}(\Omega) \to \cL^p(\partial \Omega)$, com $1 \leqslant p < \infty$, tal que
    \begin{enumerate}[leftmargin=*, label=\textbf{(\alph*)}]
        \item $Tu = u \big|_{\partial \Omega}$ se $u \in \cW^{1,p}(\Omega) \cap \cC(\Omega)$;
        \item $\Vert Tu \Vert_{\cL^p(\partial\Omega)} \leqslant c \Vert u \Vert_{W^{1,p}(\Omega)}$, onde $c$ depende apenas de $p$ e $\Omega$.
    \end{enumerate}
\end{tbox} 
\begin{prf}
    Incialmente suponha que $u \in \cC^1(\overline{\Omega})$. Da mesma forma que foi feito no Teorema \ref{thm:extensao}, considere $\tilde x \in \partial\Omega$ e suponhamos que $\partial\Omega$ está contido no plano $\{x_n = 0\}$ perto de $\tilde x$.
    Sejam $B = B(\tilde x, r)$ (e defina $B^+$ e $B^-$ como em (\ref{eq:reflexao})) e $\widehat B = B(\tilde x, \sfrac{r}{2})$,
    e considere $\xi \in \cC^{\infty}_c(B)$ de forma que $\xi \geqslant 0$ em $B$ e $\xi \equiv 1$ em $\widehat B$, e denote $\Gamma_{\tilde x} = \partial \Omega \cap \widehat B$ (ver Figura \ref{fig:tracos}) e $x' = (x_1,\dots,x_{n-1}) \in \bR^{n-1}$.
    Note que, utilizando integração por partes (ver Teorema \ref{thm:integracao-por-partes}), obtemos
    \[
        \int_{B^+} \dfrac{\partial (\xi |u|^p)}{\partial x_n}\,dx = \int_{\partial B^+} \xi |u|^p \nu_n dx' - \int_{B^+} \xi |u|^p \frac{\partial}{\partial x_n} \big( 1 \big) = \int_{\partial B^+} \xi |u|^p \, \nu_n dx'
    \]
    onde $\nu$ é o vetor normal unitário que aponta para baixo em $\partial B^+$ (pois, o pontos de $\partial B^+$ que não estão em $\partial \Omega$ estão em $\partial B$ e, consequentemente, $\xi$ os anula.), isto é $\nu =  (0,\dots,0,-1)$. Sendo assim, $\nu_n = -1$ e 
    \[
        \int_{B^+} \dfrac{\partial (\xi |u|^p)}{\partial x_n} \,dx = -\int_{\partial B^+} \xi |u|^p \,dx'.
    \]
    pois $\supp \xi |u|^p \subseteq \sup \xi \subseteq B$.
    Dessa forma, como $\Gamma_{\tilde x} \subseteq \hat B$ e $\xi(x) = 1$ para todo $x \in \hat B$, concluímos que
    \[
        \int_{\Gamma_{\tilde x}} |u|^p \,dx' = \int_{\Gamma_{\tilde x}} \xi|u|^p \,dx' \leqslant \int_{\partial B^+} \xi|u|^p \,dx' = -\int_{B^+} \dfrac{\partial (\xi |u|^p)}{\partial x_n} \,dx.
    \]
    Calculando a derivada acima, obtemos
    \[
        \int_{\Gamma_{\tilde x}} |u|^p \,dx' \leqslant - \int_{B^+} \Big[ \dfrac{\partial \xi}{\partial x_n} |u|^p + p|u|^{p-1} \sgn\,u\,\dfrac{\partial u}{\partial x_n} \xi \Big] \,dx \leqslant c \int_{B^+} \Big[ |u|^p + |u|^{p-1} \Vert Du \Vert \Big] \,dx,
    \]
    onde utilizamos o fato de $\xi$ e suas derivadas parciais terem suporte compacto para a última desigualdade. Por fim utilizando a Desigualdade de Young (ver Teorema \ref{thm:desigualdade-de-young}), resulta em
    \begin{equation} \label{eq:umaisdu}
        \int_{\Gamma_{\tilde x}} |u|^p \,dx' \leqslant c \int_{B^+} \bigg[ |u|^p + \frac{|u|^{(p-1)p'}}{p'} + \frac{\Vert Du \Vert^p}{p} \bigg] \,dx \leqslant c \int_{B^+} |u|^p + \Vert Du \Vert^p \,dx,
    \end{equation}
    onde $\frac{1}{p} + \frac{1}{p'} = 1$.

    \begin{figure}
        \centering
        \begin{tikzpicture}[scale=1.75]
        
            \draw[thick, dashed, fill=tangerine, fill opacity=0.2, draw=black!70] (1.5,0) circle (1.25);
            \draw[thick, dashed, fill=tangerine, fill opacity=0.3, draw=black!70] (1.5,0) circle (0.625);
        
            \draw[Blue, thick] (0.25,0) to (2.75,0);
            \node[above=-1pt, Blue] at (2.4,0) {$\partial \Omega$};
        
            \draw[tangerine!50!BrickRed, thick, dash pattern=on 4.5pt off 2pt] (0.875,0) to (2.125,0);
            \node[above=-1pt, tangerine!50!BrickRed] at (1.1875,0) {$\Gamma_{\tilde x}$};
        
            \filldraw (1.5,0) circle (0.03) node[below=1pt] {$\tilde x$};
        
            \node[above] at (1.5,0.625) {$\hat B$};
            \node[above] at (1.5,1.25) {$B$};


            \draw[stealth-] (1.7,0.2) to[in=180, out=90, looseness=1] (2.2,1.2) node[right] {$\xi \equiv 1$};
        \end{tikzpicture}
        \caption{}
        \label{fig:tracos}
    \end{figure}

    Caso $\partial \Omega$ não esteja necessariamente contido em $\{x_n =0 \}$ perto de $\tilde x$, considere o homeomorfismo $\Phi$  com inversa $\Psi$ da demonstração do Teorema \ref{thm:extensao} (ver (\ref{eq:Phi}) e (\ref{eq:Psi})) e defina $u' \equiv u \circ \Psi$ (i.e., $u = u' \circ \Phi$). Logo, utilizando o Teorema de Mudança de Variáveis (ver Teorema \ref{thm:mudanca-de-variaveis}), inferimos
    \[
        \int_{\Gamma_{\tilde x}} |u'|^p \,dy' = \int_{\Gamma_{\tilde x}} |u \circ \Psi| \, dy' = \int_{\Psi(\Gamma_{\tilde x})} |u|^p \,dx'
    \]
    e
    \[
        \int_{B^+} \big[ |u'(y)|^p + \Vert Du'(y) \Vert^p \big] \,dy \leqslant c\int_{\Psi(B^+)} \big[ |u(x)|^p + \Vert Du(x) \Vert^p \big] \,dx,
    \]
    onde $c$ é uma constante que depende de $\gamma$ que surge devido a regra da cadeia. Dessa forma, por (\ref{eq:umaisdu}) e pelo Teorema de Mudança de Variáveis (ver \ref{thm:mudanca-de-variaveis}) , podemos escrever
    \[
        \begin{aligned}
            \Vert u \Vert^p_{\cL^p(\Psi(\Gamma_{\tilde x}))} = \int_{\Psi(\Gamma_{\tilde x})} |u|^p \,dx' = \int_{\Gamma_{\tilde x}} |u'|^p \,dy' &\leqslant c\int_{B^+} \big[ |u'|^p + \Vert Du' \Vert^p \big]\,dy\\ 
            &\leqslant c \int_{\Psi(B^+)} \big[ |u|^p + \Vert Du \Vert^p \big] \,dx = \Vert u \Vert_{\cW^{1,p}(\Psi(B^+))}^p.
        \end{aligned}
    \]
    Como $\Psi(B^+) \subseteq \Omega$, obtemos 
    \[
        \Vert u \Vert_{\cL^p(\Psi(\Gamma_{\tilde x}))} \leqslant c \Vert u \Vert_{\cW^{1,p}(\Omega)}.
    \]

    Como $\partial\Omega$ é compacto (pois $\Omega$ é limitado) e $\partial \Omega \subseteq \bigcup_{\tilde x \in \partial \Omega} \Psi(\Gamma_{\tilde x})$ (lembre que $\Gamma_{\tilde x}$ é um aberto em $\partial\Omega$ e que $\Psi$ é um homeomorfismo) , pelo Teorema de Heine-Borel, existe uma quantidade finita de abertos (em $\partial \Omega$) $\Psi(\Gamma_i)$ tal que,
    \[
        \partial\Omega \subseteq \bigcup_{i=1}^N \Psi(\Gamma_i)
    \]
    e
    \begin{equation} \label{eq:1111}
        \Vert u \Vert_{\cL^p(\Psi(\Gamma_i))} \leqslant c \Vert u \Vert_{\cW^{1,p}(\Omega)}.
    \end{equation}
    Considere uma partição da unidade $\{\phi_i\}_{i=1}^N$ subordinada a cobertura $\{\Psi(\Gamma_i)\}_{i=1}^N$, e note que $u = \sum_{i=1}^N \phi_i u$ (pois $\sum_{i=1}^N \phi_i$).

    Defina $\widetilde T : \cC^1(\overline \Omega) \to \cL^p(\partial\Omega)$ por $\widetilde T u = u |_{\partial \Omega}$.
    Observe, que por definição, $T$ é linear.
    Dessa forma, podemos escrever, por (\ref{eq:1111}), que
    \begin{equation} \label{eq:2222}
        \begin{aligned}
            \Vert \widetilde Tu \Vert_{\cL^p(\partial \Omega)}^p = \int_{\partial\Omega} | \widetilde T u|^p \,dx = \int_{\partial\Omega} |u|^p \,dx &\leqslant c \sum_{i=1}^{N} \int_{\partial \Omega} \phi_i^p |u|^p \,dx\\ 
            &= c \sum_{i=1}^N \int_{\Psi(\Gamma_i)} |u|^p \,dx\\ 
            &= c\sum_{i=1}^N \Vert u \Vert_{\cL^p(\Psi(\Gamma_i))}^p \leqslant c\Vert u \Vert_{\cW^{1,p}(\Omega)}^p,
        \end{aligned}
    \end{equation}
    onde $c$ é uma constante que depende de $p$ e $\Omega$. Isto mostra que $\widetilde T$ é um operador limitado.
    Por fim, defina $T : \cW^{1,p}(\Omega) \to \cL^p(\partial \Omega)$ por (\ref{eq:2222})
    \[
        T u = \lim \widetilde T u_k \text{ em } \cL^p(\partial \Omega),
    \]
    onde $(u_k)$ é uma sequência de funções em $\cC^{\infty}(\overline\Omega)$ que converge para $u$ em $\cW^{1,p}(\Omega)$ (essa sequência existe pelo Teorema \ref{thm:aprox3}). 
    Primeiramente, veja que $T$ é linear por definição.
    Observe também que, por (\ref{eq:2222})
    \[
        \Vert Tu \Vert_{\cL^p(\partial \Omega)} = \Vert \lim \widetilde T u_k \Vert_{\cL^p(\partial \Omega)} = \lim \Vert \widetilde T u_k \Vert_{\cL^p(\partial \Omega)} \leqslant c \lim \Vert u_k \Vert_{\cW^{1,p}(\Omega)} = c \Vert u \Vert_{\cW^{1,p}(\Omega)},
    \]
    para todo $u \in \cW^{1,p}(\Omega)$. Isto prova \textbf{(b)}.

    Por fim, se $u \in \cW^{1,p}(\Omega) \cap \cC(\Omega)$, então, pela demonstração do Teorema \ref{thm:aprox3}, temos que $u_k \to u$ uniformemente sobre $\overline\Omega$, quando $k \to \infty$.
    Por outro lado, como
    \[
        \Vert u_k |_{\partial\Omega} - Tu \Vert_{\cL^p(\partial \Omega)} = \Vert \widetilde T u_k - Tu \Vert_{\cL^p(\partial\Omega)} \to 0,
    \]
    quando $k \to \infty$, então passando a uma subsequência (se necessário), temos, pelo Teorema \ref{thm:teorema-do-brezis} que $u_k \to Tu$ qtp em $\partial \Omega$.
    Mas como $\partial \Omega \subseteq \overline\Omega$, segue que $Tu = u$ qtp em $\partial \Omega$ (por unicidade do limite), isto é, $Tu = u|_{\partial\Omega}$ se $u \in \cW^{1,p}(\Omega) \cap \cC(\Omega)$, provando o item \textbf{(a)}.
\end{prf}

O teorema a seguir é uma caracterização do espaço de Sobolev $\cW^{1,p}_0(\Omega)$, por meio do operador traço.

\begin{tbox}[Funções traço zero em $\cW^{1,p}$] \label{thm:traco-2}
    Sejam $\Omega$ um aberto limitado, com fronteira de classe $\cC^1$ e $u \in \cW^{1,p}(\Omega)$, com $1 \leqslant p < \infty$. Então,
    \[
        u \in \cW_0^{1,p}(\Omega) \iff Tu = 0 \text{ sobre } \partial \Omega,
    \]
    onde $\cW_0^{1,p}(\Omega)$ é o fecho de $\cC^{\infty}_c(\Omega)$ em $\cW^{1,p}(\Omega)$.
\end{tbox}
\begin{prf}
    Suponha que $u \in \cW_0^{1,p}(\Omega)$. Por definição, existe uma sequência de funções $(u_k) \subseteq \cC^{\infty}_c(\Omega)$ tal que
    \[
        \Vert u_k - u \Vert_{\cW^{1,p}(\Omega)} \to 0,
    \]
    quando $k \to \infty$. Como $u_k$ tem suporte compacto e é suave em $\Omega$, então $u_k$ se anula em $\partial \Omega$ (pois $T$ é linear) para todo $k \in \bN$. Sendo assim, $Tu_k = 0$ sobre $\partial\Omega$, para todo $k \in \bN$, e como $T : \cW^{1,p}(\Omega) \to \cL^p(\partial \Omega)$ é um operador linear limitado, então
    \[
        0 = \lim \Vert Tu_k - Tu \Vert_{\cL^p(\partial \Omega)} = \Vert Tu \Vert_{\cL^p(\partial \Omega)}.
    \]
    Portanto, $Tu =0$ qtp sobre $\partial \Omega$.

    Reciprocamente, suponha que $Tu = 0$ em $\partial \Omega$. Utilizando partições da unidade e o homeomorfismo $\Phi$ que planifica $\partial \Omega$, como foi feito anteriormente (ver (\ref{eq:Phi})), podemos supor que
    \begin{equation} \label{eq:hh}
        \begin{aligned}
            &u \in \cW^{1,p}(\bR^n_+);\\
            &u \text{ tem suporte compacto em } \overline{\bR^n_+};\\
            &Tu = 0 \text{ em } \partial \bR^n_+ = \bR^{n-1},
        \end{aligned}
    \end{equation}
    onde $\bR^n_+$ denota o semiplano superior de $\bR^n$.
    Como $u \in \cW^{1,p}(\bR^n_+)$ então existe um sequência de funções $(u_k) \subseteq \cC^{\infty}(\overline{\bR^n_+}) \subseteq \cC^1(\overline{\bR^n_+})$ tal que $u_k \to u$ em $\cW^{1,p}(\bR^n_+)$, quando $k \to \infty$ (ver Teorema \ref{thm:aprox3}). Como $(u_k) \subseteq \cW^{1,p}(\bR^n_+) \cap \cC^\infty(\bR^n_+)$, então $T u_k = u_k |_{\partial \bR^n_+} = u_k |_{\bR^{n-1}}$.
    Assim, por (\ref{eq:hh}) e usando que $T$ é limitado, temos que
    \begin{equation} \label{eq:nanna}
        0 = \lim \Vert Tu_k - Tu \Vert_{\cL^p(\partial \bR^n)} = \lim \Vert Tu_k - Tu \Vert_{\cL^p(\bR^{n-1})} = \lim \Vert Tu_k \Vert_{\cL^p(\bR^{n-1})}.
    \end{equation}
    Isto é, $T u_k = u_k |_{\bR^n} \to 0$ em $\cL^p(\bR^{n-1})$, quando $k \to \infty$.

    Dito isso, seja $x' \in \bR^{n-1}$ e $x_n \geqslant 0$.
    Pelo Teorema Fundamental do Cálculo, podemos escrever $u_k$ da seguinte forma:
    \[
        u_k(x'\!,x_n) = u_k(x'\!,0) + \int_0^{x_n} \dfrac{\partial u_k}{\partial x_n} (x'\!,t) \,dt
    \]
    e, consequentemente,
    \[
        |u_k(x'\!,x_n)|^p \leqslant c \left(  |u_k(x'\!,0)|^p + \left(  \int_0^{x_n} \left|\dfrac{\partial u_k}{\partial x_n} (x'\!,t)\right| \,dt\right)^{\!p\,} \right),
    \]
    onde a constante $c$ depende de $p$ (aqui, utlizamos a desigualdade $(a + b)^p \leqslant 2^p (a^p + b^p)$). Integrando ambos os lados sobre $\bR^{n-1}$, obtemos
    \[
        \int_{\bR^{n-1}} |u_k(x',x_n)|^p \,dx' \leqslant c \left( \int_{\bR^{n-1}} |u_k(x',0)|^p \,dx' + \int_{\bR^{n-1}} \left(  \int_0^{x_n} \Vert Du_k(x',t) \Vert \,dt\right)^{\!p\,} dx' \right)
    \]
    \[
        \leqslant  c \left( \int_{\bR^{n-1}} |u_k(x',0)|^p \,dx' + \int_{\bR^{n-1}} \left( \int_0^{x_n} \Vert Du_k (x',t) \Vert^p \,dt \right)\left( \int_0^{x_n} 1^{p'} \,dt \right)^{p/p'} dx'\right),
    \]
    onde utilizamos a desigualdade de Hölder (ver Teorema \ref{thm:pre-desigualdade-de-holder}), com $\frac{1}{p}+ \frac{1}{p'} = 1$.
    Como $p/p' = p-1$, tem-se, pelo Teorema de Fubini (ver Teorema \ref{thm:fubini}), que
    \[
        \int_{\bR^{n-1}} |u_k(x',x_n)|^p \,dx' \leqslant c \left( \int_{\bR^{n-1}} |u_k(x',0)|^p \,dx' + x_n^{p-1} \int_{0}^{x_n} \int_{\bR^{n-1}} \Vert Du_k (x',t) \Vert^p  \,dx'dt \right).
    \]
    Por fim, como
    \[
        \Vert u_k |_{\bR^{n-1}} \Vert_{\cL^p(\bR^{n-1})} = \Vert Tu_k \Vert_{\cL^p(\bR^{n-1})} \to 0,
    \]
    quando $k \to \infty$ (ver \ref{thm:traco-1}), e utlizando o Teorema da Convergência Dominada (ver Teorema \ref{thm:teorema-da-convergencia-dominada}), concluímos. passando ao limite na desigualdade anterior, quando $k \to \infty$, que
    \begin{equation} \label{eq:uuuuu}
        \int_{\bR^{n-1}} |u(x',x_n)|^p \,dx' \leqslant  c x_n^{p-1}\int_0^{x_n} \int_{\bR^{n-1}} \Vert Du(x',t) \Vert^p \,dx'dt,
    \end{equation}
    pois $u_k \to u$ em $\cW^{1,p}(\bR^n_+)$, quando $k \to \infty$, o que implica em $u_k \to u$ e $Du_k \to Du$ em $\cL^p(\bR^n_+)$, quando $k \to \infty$, isso implica em $u_k(x) \to u(x)$ e $Du_k(x) \to u(x)$ qtp em $\overline{\bR^n_+}$, quando $k \to \infty$ (ver Teorema \ref{thm:teorema-do-brezis} e (\ref{eq:hh})).
    
    \begin{figure}
        \centering
        \begin{tikzpicture}
            \draw[thick, -stealth] (0,0) to (6,0);
            \filldraw (0,0) circle (0.035) node[below=2pt] {0};
            \filldraw (2,0) circle (0.035) node[below=2pt] {1};
            \filldraw (4,0) circle (0.035) node[below=2pt] {2};
            \node[below=2pt] at (6,0) {$\bR_+$};
        
            \node[above=2pt] at (1,0) {$\xi \equiv 1$};
            \node[above=2pt] at (3,0) {$0 \leq \xi \leq 1$};
            \node[above=2pt] at (5,0) {$\xi \equiv 0$};
        \end{tikzpicture}
        \caption{$\xi$ definido em (\ref{eq:xi}).}
        \label{fig:xi}
    \end{figure}
    Seja $\xi \in \cC^{\infty}_c(\bR_+)$ tal que
    \begin{equation} \label{eq:xi}
        \begin{aligned}
            &\xi \equiv 1 \text{ em } [0,1];\\
            &\xi \equiv 0 \text{ em } \bR_+ \setminus [0,2];\\
            &0 \leqslant \xi \leqslant 1,
        \end{aligned}
    \end{equation}
    (ver Figura \ref{fig:xi}) e defina $\xi_k(x) = \xi(kx_n)$ e $v_k(x) = u(x) (1 - \xi_k(x))$ para todo $x \in \bR^n_+$ e $k \in \bN$. Dessa forma, podemos escrever
    \[
        \begin{aligned}
            &\dfrac{\partial v_k}{\partial x_n}(x) = (1 - \xi_k(x))\dfrac{\partial u}{\partial x_n}(x) - ku\xi'(kx_n) \;\text{ e}\\
            &\dfrac{\partial v_k}{\partial x'}(x) = (1-\xi_k(x))\dfrac{\partial u}{\partial x'}(x).
        \end{aligned}
    \]
    Consequentemente, as seguintes desigualdades valem:
    \[
        \begin{aligned}
            \int_{\bR^n_+} \Vert Dv_k - Du \Vert^p \,dx &\leqslant c \int_{\bR^n_+} \left( \left\Vert \dfrac{\partial v_k}{\partial x'} - \dfrac{\partial u}{\partial x'} \right\Vert + \left| \dfrac{\partial v_k}{\partial x_n} - \dfrac{\partial u}{\partial x_n} \right| \right)^p dx\\
            &\leqslant c  \int_{\bR^n_+} \left( \xi_k\left\Vert \dfrac{\partial u}{\partial x'} \right\Vert + \xi_k \left| \dfrac{\partial u}{\partial x_n}\right| + k |u| |\xi'|  \right)^p dx,\\ 
        \end{aligned}
    \]
    onde $c$ é uma constante que depende de $p$ e $n$. Deste modo, chegamos a
    \begin{equation} \label{eq:asxa}
        \int_{\bR^n_+} \Vert Dv_k - Du \Vert^p \,dx \leqslant c \int_{\bR^n_+} \xi_k^p \Vert Du \Vert^p dx + ck^p \int_{\bR^n_+} |\xi'|^p |u|^p dx,
    \end{equation}
    onde utilizamos o fato de $(a + b)^p \leqslant 2^p (a^p + b^p)$.
    Observe que $\supp \xi' \subseteq \supp \xi \subseteq [0,2]$. Isto nos diz que $\xi'(kx_n) = 0$ quando $kx_n > 2$ ou $x_n > 2/k$.
    Logo, podemos escrever a última desigualdade como
    \begin{equation}
        \int_{\bR^n_+} \Vert Dv_k - Du \Vert^p \,dx \leqslant c \int_{\bR^{n}_+} \xi_k^p \Vert Du \Vert^p \,dx + ck^p \int_{0}^{\frac{2}{k}} \int_{\bR^{n-1}} |u|^p \,dx'dx_n,
    \end{equation}
    onde estimamos $\xi'$ pelo seu máximo em $[0,2]$.
    Note que
    \begin{equation} \label{eq:plus1}
        \begin{aligned}
            \int_{\bR^{n}_+} |\xi(kx_n)|^p \Vert Du \Vert^p \,dx &= \int_{0}^{\frac{2}{k}} |\xi(kx_n)|^p \int_{\bR^{n-1}} \Vert Du \Vert^p \,dx' dx_n\\
            &\leqslant c \int_{0}^{\frac{2}{k}} \int_{\bR^{n-1}} \Vert Du \Vert^p dx' dx_n \longrightarrow 0,
        \end{aligned}
    \end{equation}
    quando $k \to \infty$, onde estimamos $\xi$ pelo seu máximo em $[0,2]$ e usamos que $u \in \cW^{1,p}(\Omega)$ 
    Além disso, utlizando (\ref{eq:uuuuu}), podemos ecrever
    \begin{equation} \label{eq:plus2}
        \begin{aligned}
            ck^p \int_{0}^{\frac{2}{k}} \int_{\bR^{n-1}} |u|^p dx'dt &\leqslant ck^p \int_{0}^{\frac{2}{k}} t^{p-1} \int_{0}^{t} \int_{\bR^{n-1}} \Vert Du(x'\!,s) \Vert^p \,dx'\!ds dt\\ 
            &\leqslant ck^p\int_0^{\frac{2}{k}} t^{p-1} \int_0^{\frac{2}{k}}  \int_{\bR^{n-1}} \Vert Du(x'\!,s) \Vert dx'\!dsdt\\
            % &\leqslant \left( \int_{0}^\frac{2}{k} t^{p-1} \,dt \right) \left( \int_0^\frac{2}{k} \int_{\bR^{n-1}} \Vert Du(x'\!,s) \Vert^p dx'ds \right)\\
            &\leqslant c \int_0^{\frac{2}{k}} \int_{\bR^{n-1}} \Vert Du(x'\!,s) \Vert^p dx'ds \longrightarrow 0,
        \end{aligned}
    \end{equation}
    quando $k \to \infty$, pois $u \in \cW^{1,p}(\Omega)$.
    Portanto, por (\ref{eq:asxa}) (\ref{eq:plus1}) e (\ref{eq:plus2}), deduzimos que
    \[
        \int_{\bR^n_+} \Vert Dv_k - Du \Vert^p \,dx \to 0,
    \]
    quando $k \to \infty$,
    isto é, $\Vert Dv_k - Du \Vert_{\cL^p(\bR^n_+)} \to 0$ quando $k \to \infty$.

    Por fim, pela definição de $v_k$, inferimos que
    \[
        \Vert v_k - u \Vert_{\cL^p(\bR^n_+)}^p = \int_{\bR^n_+} |v_k - u|^p \,dx = \int_{\bR^n_+} \xi^p_k|u|^p \,dx.
    \]
    De forma análoga ao que foi feito anteriormente, chegamos a
    \begin{equation} \label{eq:plus3}
        \Vert v_k - u \Vert_{\cL^p(\bR^n_+)}^p = \int_{0}^{\frac{2}{k}} \xi(kx_n)^p \int_{\bR^{n-1}} |u(x)|^p \,dx'dx_n \leqslant c \int_{0}^{\frac{2}{k}} \int_{\bR^{n-1}} |u(x)|^p \, dx'dx_n \to 0,
    \end{equation}
    quando $k \to \infty$. Dessa forma,
    \[
        \Vert v_k - u \Vert_{\cW^{1,p}(\bR^n_+)}^p \leqslant \Vert v_k - u \Vert_{\cL^{p}(\bR^n_+)}^p + \Vert Dv_k - Du \Vert_{\cL^{p}(\bR^n_+)}^p \to 0.
    \]

    Defina $u_k = \eta_{\frac{1}{k}} * v_k$. Assim, $u_k \in \cC^{\infty}_c(\bR^n_+)$, para $k$ suficientemente grande (ver Teorema \ref{thm:aprox1}) e 
    \[
        \Vert u_k - u \Vert_{\cW^{1,p}(\bR^n_+)} \leqslant \Vert u_k - v_k \Vert_{\cW^{1,p}(\bR^n_+)} + \Vert v_k - u \Vert_{\cW^{1,p}(\bR^n_+)} \to 0,
    \]
    quando $k \to \infty$ (ver (\ref{eq:plus3})).
    Portanto, $u \in \cW^{1,p}_0(\bR^n_+)$.
\end{prf}

\section{Desigualdades de Sobolev}

Nosso objetivo nessa seção é descobrir formas de incorporar espaços de Sobolev em outros espaços.

Dividiremos o estudo dessas desigualdades em dois casos: $1 \leqslant p < n$ e $n < p \leqslant \infty$.
O caso $p = n$ não será apresentado nesse texto, aos interessados consultar \cite{evans-pde}, p.p. 275.

\subsection{Desigualdade de Gagliardo-Nirenberg-Sobolev}

Seja $1 \leqslant p < n$.
Queremos saber se é possível obter uma desigualdade do tipo\footnote{
    Lembrando que $D : \bR^n \to \bR^n$ representa o gradiente fraco, quando calculamos a norma do gradiente em $\cL^p(\Omega)$ estamos calculando a norma $\cL^p$ da norma Euclidiana do gradiente.
}
\begin{equation} \label{eq:quase-gns}
    \Vert u \Vert_{\cL^q(\bR^n)} \leqslant c \Vert Du \Vert_{\cL^p(\bR^n)},
\end{equation}
onde $c$ é uma constante positiva, $1 \leqslant q < \infty$ e $u \in \cC^\infty_c(\bR^n)$ é não nula, de forma que $c$ e $q$ não dependam de $u$.

Primeiramente, vamos mostrar que se uma desigualdade do tipo (\ref{eq:quase-gns}) é válida, o valor de $q$ não é arbitrário, mas sim admite uma forma especifica.
Para isso, seja $u \in \cC^\infty_c(\bR^n)$ não nula e $\lambda > 0$.
Sendo assim, definimos
\[
    u_\lambda(x) := u(\lambda x).
\]
Aplicando (\ref{eq:quase-gns}) a $u_\lambda$, obtemos
\begin{equation} \label{eq:ulgns}
    \Vert u_\lambda \Vert_{\cL^q(\bR^n)} \leqslant c \Vert Du_\lambda \Vert_{\cL^p(\bR^n)}.
\end{equation}
Note que, pelo Teorema de Mudança de Variáveis (ver Teorema \ref{thm:mudanca-de-variaveis}), podemos escrever
\[
    \Vert u_\lambda \Vert_{\cL^q(\bR^n)}^q = \int_{\bR^n} |u_\lambda|^q \,dx = \int_{\bR^n} |u(\lambda x)|^p \,dx = \frac{1}{\lambda^n} \int_{\bR^n} |u(y)|^q \,dy = \frac{1}{\lambda^n}\Vert u \Vert_{\cL^q(\bR^n)}^q
\]
e
\[
    \begin{aligned}
        \Vert Du_\lambda \Vert_{\cL^p(\bR^n)}^p = \int_{\bR^n} \Vert Du_\lambda \Vert^p \,dx &= \int_{\bR^n} \Vert D(u(\lambda x)) \Vert^p \,dx \\
        &= \int_{\bR^n} \Vert \lambda Du(\lambda x) \Vert^p \,dx = \frac{\lambda^p}{\lambda^n} \int_{\bR^n} \Vert Du(y) \Vert^p \,dy = \frac{\lambda^p}{\lambda^n}\Vert Du \Vert_{\cL^p(\bR^n)}^p.
    \end{aligned}
\]
Utilizando essas duas igualdades acima em (\ref{eq:ulgns}), observamos que
\[
    \frac{1}{\lambda^{\frac{n}{q}}} \Vert u \Vert_{\cL^q(\bR^n)} \leqslant c \frac{\lambda}{\lambda^{\frac{n}{p}}}\Vert Du \Vert_{\cL^p(\bR^n)},
\]
a qual podemos reescrever como
\begin{equation}
    \Vert u \Vert_{\cL^q(\bR^n)} \leqslant c\lambda^{1 - \frac{n}{p}  + \frac{n}{q}}\Vert Du \Vert_{\cL^p(\bR^n)}.
\end{equation}
Observe que se $1 - \frac{n}{p} + \frac{n}{q} > 0$, obtemos uma contradição quando $\lambda \to 0$, pois isso implicaria em $\Vert u \Vert_{\cL^q(\bR^n)} = 0$,
que só acontece se $u = 0$ (o que é uma contradição).
De forma análoga se $1 - \frac{n}{p} + \frac{n}{q} < 0$ obtemos uma contradição quando $\lambda \to \infty$.
Sendo assim, para que a igualdade (\ref{eq:quase-gns}) seja válida, precisamos que
\[
    1 - \frac{n}{p} + \frac{n}{q} = 0,
\]
ou seja,
\[
    q = \frac{np}{n - p}.
\]

Isso motiva a definição abaixo.
\begin{dbox}
    Se $1 \leqslant p < n$, o expoente conjugado de Sobolev de $p$ é dado por
    \[
        p^* = \frac{np}{n - p}.
    \]
\end{dbox}

Os cálculos no início da seção mostram que a desigualdade (\ref{eq:quase-gns}) somente é válida quando $q = p^*$. O resultado abaixo mostra que de fato a desigualdade é verídica.

\begin{tbox}[Desigualdade de Gagliardo-Nirenberg-Sobolev] Seja $1 \leqslant p < n$. Então, existe uma constante $c$, que depende apenas de $p$ e $n$, tal que
\begin{equation} \label{eq:gns}
    \Vert u \Vert_{\cL^{p^*}(\bR^n)} \leqslant c \Vert Du \Vert_{\cL^p(\bR^n)},
\end{equation}
para toda função $u \in \cC^1_c(\bR^n)$.
\end{tbox}
\begin{prf}
    Consideremos dois casos

    \underline{Caso 1:} Considere que $p = 1$ (ou seja, $p^* = \frac{n}{n-1}$).\\
    Como, por hipótese, $u$ tem suporte compacto, temos, pelo Teorema Fundamental do Cálculo, que
    \[
        u(x) = \int_{\m\infty}^{x_i} \dfrac{\partial u}{\partial x_i}(x_1,\dots,y_i,\dots,x_n) \, dy_i,
    \]
    e assim,
    \[
        |u(x)| \leqslant \int_{\m\infty}^{x_i} \left|\dfrac{\partial u}{\partial x_i}(x_1,\dots,y_i,\dots,x_n)\right| \, dy_i  \leqslant \int_{\m\infty}^{\infty} \Vert Du(x_1,\dots,y_i,\dots,x_n) \Vert \,dy_i.
    \]
    Elevando ambos os lados da desigualdade acima a $\frac{1}{n-1}$ e passando ao produtório de $1$ até $n$, obtemos
    \[
        |u(x)|^{\frac{1}{n-1}} \leqslant \prod_{i=1}^n \left( \int_{\m\infty}^{\infty} \Vert Du(x_1,\dots,y_i,\dots,x_n) \Vert \,dy_i \right)^{\frac{1}{n-1}}.
    \]
    Denotando $(x_1,\dots,y_i,\dots,x_n)$ por $X_i$ e integrando ambos os lados da desigualdade acima, em relação a $x_1$, de $-\infty$ a $\infty$, chegamos a
    \[
        \begin{aligned}
            \int_{\m\infty}^{\infty} |u(x)|^{\frac{1}{n-1}} \,dx_1 &\leqslant \int_{\m\infty}^{\infty} \prod_{i=1}^n \left( \int_{\m\infty}^{\infty} \Vert Du(X_i) \Vert \,dy_i \right)^{\frac{1}{n-1}}  dx_1\\ 
            &= \int_{\m\infty}^{\infty} \left( \int_{\m\infty}^{\infty} \Vert Du(X_1    ) \Vert \,dy_1 \right)^{\frac{1}{n-1}}  \prod_{i=2}^n \left(\int_{\m\infty}^{\infty} \Vert Du(X_i) \Vert \, dy_i\right)^{\frac{1}{n-1}} dx_1.
        \end{aligned}
    \]
    Porém, $Du(X_1)$ não depende de $x_1$, então a sua integral é constante em relação a $x_1$. Sendo assim,
    \[
        \int_{\m\infty}^{\infty} |u(x)|^{\frac{1}{n-1}} \,dx_1 \leqslant \left( \int_{\m\infty}^{\infty} \Vert Du(X_1)\Vert \,dy_1 \right)^{\frac{1}{n-1}}\int_{\m\infty}^{\infty}   \prod_{i=2}^n \left(\int_{\m\infty}^{\infty} \Vert Du(X_i) \Vert \, dy_i\right)^{\frac{1}{n-1}} dx_1.
    \]
    Por fim, utilizando a Desigualdade de Hölder Generalizada (ver Teorema \ref{thm:pre-desigualdade-de-holder-gen}) e o Teorema de Fubini (ver Teorema \ref{thm:fubini}), a desigualdade acima se torna
    \[
        \int_{\m\infty}^{\infty} |u(x)|^{\frac{1}{n-1}} \,dx_1 \leqslant \left( \int_{\m\infty}^{\infty} \Vert Du(X_1)\Vert \,dy_1 \right)^{\frac{1}{n-1}}\prod_{i=2}^n \left(\int_{\m\infty}^{\infty}   \int_{\m\infty}^{\infty} \Vert Du(X_i) \Vert \, dx_1dy_i\right)^{\frac{1}{n-1}}.
    \]
    Agora integrando a desigualdade acima em relação a $x_2$ de $-\infty$ a $\infty$, obtemos
    \[
        \int_{\m\infty}^{\infty}\int_{\m\infty}^{\infty} |u(x)|^{\frac{1}{n-1}} \,dx_1dx_2 \leqslant \int_{\m\infty}^{\infty}\!\!\left( \int_{\m\infty}^{\infty} \Vert Du(X_1)\Vert \,dy_1 \right)^{\frac{1}{n-1}}\prod_{i=2}^n \left(\int_{\m\infty}^{\infty} \!\int_{\m\infty}^{\infty} \Vert Du(X_i) \Vert \, dx_1dy_i\right)^{\frac{1}{n-1}} dx_2.
    \]
    Por conseguinte, resulta
    \[
        \int_{\m\infty}^{\infty}\int_{\m\infty}^{\infty} |u(x)|^{\frac{1}{n-1}} \,dx_1dx_2 \leqslant \int_{\m\infty}^{\infty} I_1^{\frac{1}{n-1}}I_2^{\frac{1}{n-1}}\prod_{i=3}^n I_i^{\frac{1}{n-1}} dx_2,
    \]
    onde
    \[
        I_1 = \int_{\m\infty}^{\infty} \Vert Du(X_1)\Vert \,dy_1 \text{ e } I_i = \int_{\m\infty}^{\infty} \!\int_{\m\infty}^{\infty} \Vert Du(X_i) \Vert \, dx_1dy_i \quad(i = 2,\dots,n).
    \]
    Porém, $I_2$ é constante em relação a $x_2$, então
    \[
        \int_{\m\infty}^{\infty}\int_{\m\infty}^{\infty} |u(x)|^{\frac{1}{n-1}} \,dx_1dx_2 \leqslant I_2^{\frac{1}{n-1}}\int_{\m\infty}^{\infty} I_1^{\frac{1}{n-1}}\prod_{i=3}^n I_i^{\frac{1}{n-1}} dx_2.
    \]
    Novamente, utilizando a Desigualdade de Hölder Generalizada e o Teorema de Fubini (ver Teoremas \ref{thm:pre-desigualdade-de-holder-gen} e \ref{thm:fubini}), inferimos
    {\small
    \[
        \begin{aligned}
            \int_{\m\infty}^{\infty}\int_{\m\infty}^{\infty} |u(x)|^{\frac{n}{n-1}} \,dx_1dx_2 \leqslant &\left( \int_{\m\infty}^{\infty} \int_{\m\infty}^{\infty} \Vert Du(X_1) \Vert \,dy_1 dx_2 \right)^{\frac{1}{n-1}}\\ &\left( \int_{\m\infty}^{\infty} \int_{\m\infty}^{\infty} \Vert Du(X_2) \Vert \,dx_1 dy_2 \right)^{\frac{1}{n-1}} \prod_{i=3}^n \left( \int_{\m\infty}^{\infty}\int_{\m\infty}^{\infty}\int_{\m\infty}^{\infty} \Vert Du(X_i) \Vert \,dx_1dx_2dy_i \right)^{\frac{1}{n-1}} \!\!.
        \end{aligned}
    \]\!}
    Indutivamente, repetindo esse processo de integração, chegamos a
    \[
        \begin{aligned}
            \Vert u \Vert_{\cL^{p^*}(\bR^n)}^{p^*} &= \int_{\bR^n} |u|^{\frac{n}{n-1}} \, dx \\
            &\leqslant \prod_{i=1}^n \left( \int_{\m\infty}^{\infty} \cdots \int_{\m\infty}^\infty \Vert Du \Vert dx_1\dots dy_i \dots dx_n \right)^{\frac{1}{n-1}}
            = \left(\int_{\bR^n} \Vert Du \Vert\,dx\right)^{\frac{n}{n-1}} = \Vert Du \Vert_{\cL^1(\bR^n)}^{p^*}.
        \end{aligned}
    \]
    Ou seja,
    \begin{equation} \label{eq:desigualdadegnss}
        \Vert u \Vert_{\cL^{p^*}(\bR^n)} \leqslant\Vert Du \Vert_{\cL^1(\bR^n)},
    \end{equation}
    como queriamos mostrar.

    \underline{Caso 2:} Assuma que $1 < p < n$.\\
    Considere a função $|u|^\gamma$, com $\gamma > 1$ a ser escolhido a seguir. Utilizando a desigualdade obtida no caso $p = 1$, podemos escrever
    \[
        \left( \int_{\bR^n} |u|^{\frac{\gamma n}{n - 1}}  \, dx\right)^{\frac{n-1}{n}} = \left( \int_{\bR^n} \Big[ |u|^\gamma \Big]^{\frac{n}{n-1}} \,dx \right)^{\frac{n-1}{n}} \leqslant \int_{\bR^n} \Vert D(|u|^{\gamma}) \Vert \,dx = \gamma \int_{\bR^n} |u|^{\gamma-1} \Vert Du \Vert \,dx.
    \]
    Utilizando a desigualdade de Hölder (ver Teorema \ref{thm:pre-desigualdade-de-holder}) na última integral, obtemos
    \begin{equation} \label{eq:bnbn}
        \left( \int_{\bR^n} |u|^{\frac{\gamma n}{n - 1}}  \, dx\right)^{\frac{n-1}{n}} \leqslant \gamma\left( \int_{\bR^n} |u|^{(\gamma-1)\frac{p}{p-1}} \,dx \right)^{\frac{p-1}{p}} \left( \int_{\bR^n} \Vert Du \Vert^p \,dx \right)^{\frac{1}{p}}.
    \end{equation}
    Escolhendo $\gamma$ de forma que $\displaystyle\frac{\gamma n}{n - 1} = (\gamma -1)\frac{p}{p-1}$, isto é,
    \[
        \gamma = \frac{p(n-1)}{n-p}.
    \]
    Nesse caso, $\displaystyle\frac{\gamma n}{n-1} = p^*$. Sendo assim, por (\ref{eq:bnbn}) podemos escrever
    \[
        \Vert u \Vert_{\cL^{p^*}(\bR^n)} = \left( \int_{\bR^n} |u|^{p^*} \,dx \right)^{\frac{1}{p^*}} \leqslant c \left( \int_{\bR^n} \Vert Du \Vert^p \,dx\right)^{\frac{1}{p}} = \Vert Du \Vert_{\cL^p(\bR^n)},
    \]
    finalizando a demonstração.
\end{prf}

\obs Note que o suporte compato é necessário, como exemplo tome a função $u(x) = 1$, para todo $x \in \bR^n$. Dessa forma $\Vert Du \Vert \equiv 0$. Consequentemente,
\[
    \Vert u \Vert_{\cL^{p^*}(\bR^n)} \leqslant 0 \implies u \equiv 0,
\]
que é uma contradição.

Agora, mostraremos que $\cW^{1,p}(\Omega) \subseteq \cL^{p^*}(\Omega)$, sobre certas condições de $p$ e $\Omega$.

\begin{tbox} \label{thm:desigualdade-teorema-2}
    Sejam $\Omega$ um aberto limitado, com fronteira de classe $\cC^1$ e $u \in \cW^{1,p}(\Omega)$. Então. $u \in \cL^{p^*}(\Omega)$ e, além disso,
    \[
        \Vert u \Vert_{\cL^{p^*}(\Omega)} \leqslant c \Vert u \Vert_{\cW^{1,p}(\Omega)},
    \]
    onde $c > 0$ é uma constante que depende apenas de $n$, $p$ e $\Omega$.
\end{tbox}
\begin{prf}
    Utilizando o Teorema \ref{thm:extensao}, podemos considerar a extensão de $u$ $Eu = \bar u$ tal que
    \begin{equation} \label{eq:man}
        \begin{aligned}
            &\bar u = u \text{ qtp em } \Omega;\\
            &\bar u \text{ tem suporte compacto};\\
            &\Vert \bar u \Vert_{\cW^{1,p}(\bR^n)} \leqslant c\Vert u \Vert_{\cW^{1,p}(\Omega)}.
        \end{aligned}
    \end{equation}
    Como $\bar u$ tem suporte compacto, sabemos que existe uma sequência $(u_k)$, dada por $\eta_{\frac{1}{k}} * \bar u$, tal que
    \begin{equation} \label{eq:mna}
        \Vert u_k - \bar u \Vert_{\cW^{1,p}(\Omega)} \to 0,
    \end{equation}
    quando $k \to \infty$,
    e para $k$ grande $u_k \in \cC^{\infty}_c(\bR^n)$ (ver Teorema \ref{thm:aprox1}).
    Pela Desigaldade de Gagliardo-Nirenberg-Sobolev (\ref{eq:gns}), concluímos que
    \begin{equation} \label{eq:iiii}
        \Vert u_k - u_\ell \Vert_{\cL^{p^*}(\bR^n)} \leqslant c \Vert Du_k - Du_\ell \Vert_{\cL^p(\bR^n)},
    \end{equation}
    para todo $k, \ell \in \bN$.
    Note que, por (\ref{eq:mna}), temos que
    \begin{equation} \label{eq:jjj}
        \Vert Du_k - D\bar u \Vert_{\cL^p(\bR^n)} \leqslant \Vert u_k - \bar u \Vert_{\cW^{1,p}(\bR^n)} \to 0, 
    \end{equation}
    quando $k \to \infty$.
    Isso mostra que $(Du_k)$ é convergente (e portanto, de Cauchy) em $\cL^p(\bR^n)$.
    Consequentemente, por (\ref{eq:iiii}), observamos que $(u_k)$ é de Cauchy em $\cL^{p^*}(\bR^n)$, o qual é um espaço completo (ver Teorema \ref{thm:lp-completo-pre}). Logo, existe $v \in \cL^{p^*}(\bR^n)$ tal que $u_k \to v$ em $\cL^{p^*}(\bR^n)$. Portanto, pelo Teorema \ref{thm:teorema-do-brezis}, a menos de uma subsequência $u_k(x) \to v(x)$ qtp em $\bR^n$, quando $k \to \infty$. Analogamente, por (\ref{eq:mna}) $u_k(x) \to \bar u(x)$ qtp em $\bR^n$, quando $k \to \infty$, desde que
    \begin{equation} \label{eq:jjjj}
        \Vert u_k - \bar u \Vert_{\cL^p(\bR^n)} \leqslant \Vert u_k - \bar u \Vert_{\cW^{1,p}(\bR^n)} \to 0,
    \end{equation}
    quando $k \to \infty$. Com isso, $v(x) = \bar u(x)$ qtp em $\bR^n$.
    Dessa forma,
    \[
        \Vert u_k - \bar u \Vert_{\cL^{p^*}(\bR^n)} \to 0,
    \]
    quando $k \to \infty$.
    A Desigualdade de Gagliardo-Nirenberg-Sobolev também implica em
    \[  
        \Vert u_k \Vert_{\cL^{p^*}(\bR^n)} \leqslant c \Vert Du_k \Vert_{\cL^p(\bR^n)},
    \]
    para todo $k \in \bN$.
    Passando ao limite, quando $k \to \infty$, obtemos, por (\ref{eq:jjj}) e (\ref{eq:jjjj}), que
    \[
        \Vert \bar u \Vert_{\cL^{p^*}(\bR^n)} \leqslant c \Vert D \bar u \Vert_{\cL^p(\bR^n)}.
    \]
    Essa desigualdade finaliza a demonstração, já que, por (\ref{eq:man}), vale
    \[
        \Vert u \Vert_{\cL^{p^*}(\Omega)} = \Vert \bar u \Vert_{\cL^{p^*}(\Omega)} \leqslant \Vert \bar u \Vert_{\cL^{p^*}(\bR^n)} \leqslant c \Vert D \bar u \Vert_{\cL^p(\bR^n)} \leqslant c \Vert \bar u \Vert_{\cW^{1,p}(\bR^n)} \leqslant c \Vert u \Vert_{\cW^{1,p}(\Omega)},
    \]
    como queriamos mostrar.
\end{prf}

Agora, vamos provar uma desigualdade de Sobolev que generaliza a famosa desigualdade de Poincaré.

\begin{tbox} \label{thm:poincaregen}
    Sejam $\Omega$ um aberto limitado e $u \in \cW^{1,p}_0(\Omega)$, com $1 \leqslant p < n$, então a desigualdade
    \begin{equation} \label{eq:poincaregen}
        \Vert u \Vert_{\cL^q(\Omega)} \leqslant c \Vert Du \Vert_{\cL^p(\Omega)},
    \end{equation}
    é válida para $1 \leqslant q \leqslant p^*$ e $c$ é uma constante que depende de $p, q$ e $n$.
\end{tbox}
\begin{prf} ~

    \underline{Caso 1:} Assuma $q = p^*$.\\
    Como $u \in \cW^{1,p}_0(\Omega)$, conseguimos uma sequência de funções $(u_k)$ em $\cC^{\infty}_c(\Omega)$ tal que $u_k \to u$ em $\cW^{1,p}(\Omega)$, quando $k \to \infty$.
    Além disso, podemos estender cada $u_k$ para ser $0$ em $\bR^n \setminus \Omega$.
    Sendo assim, aplicamos a Desigualdade de Gagliardo-Nirenberg-Sobolev (\ref{eq:gns}) para obter
    \begin{equation} \label{eq:passarolimite}
        \Vert u_k \Vert_{\cL^{p*}(\Omega)} \leqslant c \Vert Du_k \Vert_{\cL^p(\Omega)},
    \end{equation}
    para todo $k \in \bN$.
    Como $\Vert u_k - u \Vert_{\cW^{1,p}(\Omega)} \to 0$, quando $k \to \infty$, segue que $\Vert u_k - u \Vert_{\cL^p(\Omega)}$ e $\Vert Du_k - Du \Vert_{\cL^p(\Omega)} \to 0$, quando $k \to \infty$.
    Sendo assim, pelo Teorema \ref{thm:teorema-do-brezis}, passando a uma subsequência (se necessário), chegamos a
    \begin{equation} \label{eq:uku}
        u_k(x) \to u(x) \text{ qtp em } \Omega.
    \end{equation}
    Por outro lado, pela Desigualdade de Gagliardo-Nirenberg-Sobolev (\ref{eq:gns}), inferimos que
    \[
        \Vert u_k - u_\ell \Vert_{\cL^{p^*}(\Omega)} \leqslant c \Vert Du_k - Du_\ell \Vert_{\cL^p(\Omega)} \to 0,
    \]
    quando $k,\ell \to \infty$, pois $(Du_k)$ é convergente (em particular, de Cauchy) em $\cL^p(\Omega)$. Isto nos diz que $(u_k)$ é de Cauchy em $\cL^{p^*}(\Omega)$, que é um espaço completo (ver Teorema \ref{thm:lp-completo-pre}). Dito isso, existe $v \in \cL^{p^*}(\Omega)$ tal que $u_k \to v$ em $\cL^{p^*}(\Omega)$.
    Novamente, utilizando o Teorema \ref{thm:teorema-do-brezis}, passando a uma subsequência (se necessário), segue que
    \begin{equation} \label{eq:ukv}
        u_k(x) \to v(x) \text{ qtp em } \Omega.
    \end{equation}
    Logo, por (\ref{eq:uku}) e (\ref{eq:ukv}), $v \equiv u$ qtp em $\Omega$.
    Dessa forma,
    \[
        \Vert u_k - u \Vert_{\cL^{p*}(\Omega)} \to 0,
    \]
    quando $k \to \infty$.
    Passando ao limite em (\ref{eq:passarolimite}), quando $k\to\infty$, chegamos a
    \[
        \Vert u \Vert_{\cL^{p^*}}(\Omega) \leqslant c \Vert Du \Vert_{\cL^p(\Omega)}.
    \]
    Provando o caso em que $q = p^*$. 
    
    \underline{Caso 2:} Assuma que $1 < q < p^*$.\\
    Como $\Omega$ é limitado, temos, pelo Teorema \ref{thm:omega-limitado} e por (\ref{eq:gns}), que
    \[
        \Vert u \Vert_{\cL^q(\Omega)} \leqslant c\Vert u \Vert_{\cL^{p^*}(\Omega)} \leqslant c \Vert Du \Vert_{\cL^p(\Omega)}.
    \]
    Portanto, a desigualdade (\ref{eq:poincaregen}) é válida para todo $1 \leqslant p < n$ e $1 \leqslant q \leqslant p^*$.
\end{prf}

Um caso particular da desigualdade (\ref{eq:poincaregen}) é a Desigualdade de Poincaré que será apresentada no resultado abaixo.

\begin{cbox}[Desigualdade de Poincaré] \label{cl:poincare}
   Sejam $\Omega$ um aberto limitado e $u \in \cW^{1,p}_0(\Omega)$, com $1 \leqslant p \leqslant \infty$. Então, a desigualdade
   \[
        \Vert u \Vert_{\cL^p(\Omega)} \leqslant c\Vert Du \Vert_{\cL^p(\Omega)}
   \]
   é válida,
   onde $c$ é uma constante que depende de $p, q$ e $n$.
\end{cbox}
\begin{prf}
    Primeiramente considere $1 \leqslant p < n$.
    Por definição, $1 \leqslant p < p^*$. Além disso, pelo Teorema \ref{thm:desigualdade-teorema-2}, concluímos que
    \[
        \Vert u \Vert_{\cL^p(\Omega)} \leqslant c \Vert Du \Vert_{\cL^p(\Omega)}.
    \]

    Agora considere que $n \leqslant p < \infty$, e $1 \leqslant s < n$ tal que $p < s^*$ (isto é possível pois $s^* \to \infty$ se $s \to n^-$).
    Note que, pelo Teorema \ref{thm:omega-limitado}, podemos escrever
    \[
        \Vert u \Vert_{\cL^p(\Omega)} \leqslant \Vert u \Vert_{\cL^{s^*}(\Omega)} \leqslant c \Vert Du \Vert_{\cL^s(\Omega)} \leqslant c \Vert Du \Vert_{\cL^p(\Omega)},
    \]
    já que $1 \leqslant p < s^*$, $s < p$ e $\Omega$ é limitado.
    Por fim, considere $p = \infty$.
    Pelo Teorema \ref{thm:poincaregen}, temos que
    \[
        \Vert u \Vert_{\cL^{q^*}(\Omega)} \leqslant c \Vert Du \Vert_{\cL^{q}(\Omega)} \leqslant c \Vert Du \Vert_{\cL^\infty(\Omega)},
    \]
    para todo $1 \leqslant q < n$.
    Passando ao limite, quando $q \to n^-$, temos que $q^* \to \infty$ e, consequentemente pelo Teorema \ref{thm:omega-limitado}, $\Vert u \Vert_{\cL^\infty(\Omega)} \leqslant c \Vert Du \Vert_{\cL^\infty(\Omega)}$.
    Portanto,
    \[
        \Vert u \Vert_{\cL^p(\Omega)} \leqslant c\Vert Du \Vert_{\cL^p(\Omega)},
    \]
    para todo $1 \leqslant p \leqslant \infty$.
\end{prf}

Podemos mostrar que $\Vert u \Vert_{\cW_0^{1,p}(\Omega)} := \Vert Du \Vert_{\cL^p(\Omega)}$ é uma norma em $\cW^{1,p}_0(\Omega)$, equivalente a norma usual dos espaços de Sobolev $\Vert u \Vert_{\cW^{1,p}(\Omega)}$. Com efeito, pelo Corolário \ref{cl:poincare}, temos que
\[
    \Vert u \Vert_{\cW^{1,p}(\Omega)} = \Vert u \Vert_{\cL^p(\Omega)}^p + \Vert Du \Vert_{\cL^p(\Omega)}^p \leqslant c \Vert Du \Vert_{\cL^p(\Omega)}^p + \Vert Du \Vert_{\cL^p(\Omega)}^p \leqslant c \Vert Du \Vert_{\cL^p(\Omega)} = c \Vert u \Vert_{\cW^{1,p}_0(\Omega)}.
\]
para todo $u \in \cW^{1,p}_0(\Omega)$.
Por outro lado, é verdade que
\[
    \Vert u \Vert_{\cW^{1,p}_0(\Omega)} = \Vert Du \Vert_{\cL^p(\Omega)} \leqslant \Vert u \Vert_{\cW^{1,p}(\Omega)},
\]
para todo $u \in \cW^{1,p}_0(\Omega)$.
Portanto, as normas $\Vert \cdot \Vert_{\cW^{1,p}(\Omega)}$ e $\Vert \cdot \Vert_{\cW^{1,p}_0(\Omega)}$ são equivalentes em $\cW^{1,p}_0(\Omega)$.

Além disso, pelo Teorema \ref{thm:poincaregen} e pelo Corolário \ref{cl:poincare}, podemos qescrever 
\[
    \cW^{1,p}_0(\Omega) \hookrightarrow \cL^q(\Omega) \quad (1 \leqslant q \leqslant p^*) \;\text{ e }\; \cW^{1,s}_0(\Omega) \hookrightarrow \cL^{s^*}(\Omega) \quad (1 \leqslant s \leqslant \infty).
\]
Isto é o mesmo que dizer que $\cW^{1,p}_0(\Omega) \subseteq \cL^q(\Omega)$ com $1 \leqslant q \leqslant p^*$ e $1 \leqslant p < n$, $\cW^{1,s}(\Omega) \subseteq \cL^{s^*}(\Omega)$, com $1 \leqslant s \leqslant \infty$ e o operador inclusão $i$ (em cada um dos casos) é um operador linear limitado.

\subsection{Desigualdade de Morrey}

A próxima classe de desigualdades realiza uma conexão entre os espaços de Hölder (que veremos a definição a seguir) e os espaços de Sobolev.

\begin{dbox}
    Uma função $u : \Omega \to \bR$ é dita ser Hölder contínua, com expoente $\gamma \in (0,1]$, quando
    \[
        |u(x) - u(y)| \leqslant c \Vert x - y \Vert^\gamma,
    \]
    para todo $x,y \in \Omega$. Além disso, denotamos o espaço dessas funções por $\cC^{0,\gamma}(\overline\Omega)$.
\end{dbox}

\begin{dbox}
    Se $u : \Omega \to \bR$ é uma função contínua e limitada, escrevemos
    \[
        \Vert u \Vert_{\cC^{0,\gamma}(\overline\Omega)} = \Vert u \Vert_{\cC(\overline\Omega)} + [u]_{\cC^{0,\gamma}(\overline\Omega)},
    \]
    onde
    \[
        \Vert u \Vert_{\cC(\overline\Omega)} = \sup_{x \in \Omega} |u(x)| \;\text{ e }\; [u]_{\cC^{0,\gamma}(\overline\Omega)} = \sup_{\substack{x,y \in \Omega\\x \neq y}} \left\{ \frac{|u(x) - u(y)|}{\Vert x - y \Vert^\gamma} \right\},
    \]
    para denotar a norma de $u$ no espaço de Hölder $\cC^{0,\gamma}(\overline\Omega)$.
\end{dbox}

Com essas definições, estamos prontos para enunciar e demonstrar a Desigualdade de Morrey.

\begin{tbox}[Desigualdade de Morrey] \label{thm:holdersobolev1}
    Sejam $u \in \cC^1(\bR^n)$, $n < p \leqslant \infty$ e $\gamma = 1 - \frac{n}{p}$. Então,
    \[
        \Vert u \Vert_{\cC^{0,\gamma}(\bR^n)} \leqslant c \Vert u \Vert_{\cW^{1,p}(\bR^n)},
    \]
    onde $c$ é uma constante que depende apenas de $p$ e $n$.
\end{tbox}
\begin{prf}
    Primeiramente, escolha uma bola $B(x,r) \subseteq \bR^n$.
    Afirmamos que existe uma constante $c > 0$ dependendo apenas de $n$ tal que\footnote{A integral
    \[
        \sint_{B(x,r)} f\,dy := \frac{1}{\sigma(n)r^n}\int_{B(x,r)} f \,dy
    \]
    onde $\sigma(n)r^n$ é o volume da esfera $n$-dimensional, representa a média da função $f$ sobre $B(x,r)$.}
    \begin{equation} \label{eq:desigualdade-uyux}
        \sint_{B(x,r)} |u(y) - u(x)| \,dy \leqslant c \int_{B(x,r)} \frac{\Vert Du(y) \Vert}{\Vert y - x \Vert^{n-1}} \,dy.
    \end{equation}
    Com efeito, fixando $w \in \partial B(0,1)$ e $0 < s < r$, segue, do Teorema Fundamental do Cálculo e da Regra da cadeia, que
    \[
        |u(x + sw) - u(x)| = \left| \int_0^s \frac{du}{dt} (x + tw) \,dt \right| = \left|\int_0^s Du(x + tw) \cdot w \,dt \right| \leqslant \int_0^s \Vert Du(x + tw) \Vert \,dt,
    \]
    onde utilizamos a Desigualdade de Cauchy-Schwarz e o fato de $\Vert w \Vert = 1$. Logo, integrando ambos os lados da desigualdade acima sobre $\partial B(0,1)$ e aplicando o Teorema de Fubini (ver Teorema \ref{thm:fubini}), obtemos
    \[
        \begin{aligned}
            \int_{\partial B(0,1)} |u(x + sw) - u(x)|\,dS_w &\leqslant  \int_0^s \int_{\partial B(0,1)} \Vert Du(x + tw) \Vert \,dS_wdt\\ 
            &= \int_0^s \int_{\partial B(0,1)} \Vert Du(x + tw) \Vert \frac{t^{n-1}}{t^{n-1}} \,dS_wdt.
        \end{aligned}
    \]
    Seja $y = x + tw$, de forma que $t = \Vert x - y \Vert$. Assim, por meio de coordenadas polares e mudança de variáveis (ver Teoremas \ref{thm:coordenadas-polares} e \ref{thm:mudanca-de-variaveis}) obtemos
    \[
        \int_{\partial B(0,1)} |u(x + sw) - u(x)|\,dS_w \leqslant \int_0^s \int_{B(x,t)} \frac{\Vert Du(y) \Vert}{\Vert x - y \Vert^{n-1}} \,dS_ydt =  \int_{B(x,s)} \frac{\Vert Du(y) \Vert}{\Vert x - y \Vert^{n-1}} \,dy
    \]
    e, como $s < r$, tem-se
    \[
        \int_{\partial B(0,1)} |u(x + sw) - u(x)|\,dS_w \leqslant   \int_{B(x,r)} \frac{\Vert Du(y) \Vert}{\Vert x - y \Vert^{n-1}} \,dy.
    \]
    Multiplicando a equação acima por $s^{n-1}$ e integrando de $0$ a $r$, com respeito a $s$, chegamos a
    \[
        \int_0^r s^{n-1} \int_{\partial B(0,1)} |u(x + sw) - u(x)| \,dS_wds \leqslant \int_0^r s^{n-1} \int_{B(x,r)} \frac{\Vert Du(y) \Vert}{\Vert x - y \Vert^{n-1}} \,dyds.
    \]
    Fazendo a mudança de variáveis $y = x + sw$, obtemos
    \[
        \int_0^r \int_{\partial B(x,s)} |u(y) - u(x)|\, dS_yds \leqslant \left( \int_{B(x,r)} \frac{\Vert Du(y) \Vert}{\Vert x - y \Vert^{n-1}} \,dy \right) \bigg( \int_0^r s^{n-1} ds \bigg).
    \]
    Utilizando coordenadas polares (ver Teorema \ref{thm:coordenadas-polares}) no lado esquerdo e resolvendo a última integral do lado direito da desigualdade acima, segue que
    \[
        \int_{B(x,r)} |u(y) - u(x)|\,dy \leqslant \frac{r^n}{n} \int_{B(x,r)} \frac{\Vert Du(y) \Vert}{\Vert x - y \Vert^{n-1}} \,dy.
    \]
    Por fim, dividindo ambos os lados por $\sigma(n) r^n$ (volume da $n$-esfera de raio $r$), temos
    \[
        \sint_{B(x,r)} |u(y) - u(x)| \,dy \leqslant \frac{1}{n\sigma(n)} \int_{B(x,r)} \frac{\Vert Du(y) \Vert}{\Vert x-y \Vert^{n-1}} \,dy.
    \]
    Isso prova (\ref{eq:desigualdade-uyux}).

    Agora, fixe $x \in \bR^n$. Note que, podemos escrever
    \[
        |u(x)| = \frac{|u(x)|}{\sigma(n)}\int_{B(x,1)} dy = \sint_{B(x,1)} |u(x)| \,dy.
    \]
    Dito isso, deduzimos que
    \begin{equation} \label{eq:ux}
        |u(x)| \leqslant \sint_{B(x,1)} |u(x) - u(y)| \,dy + \sint_{B(x,1)} |u(y)|\,dy.
    \end{equation}
    Observe que, pela desigualdade de Hölder (ver Teorema \ref{thm:pre-desigualdade-de-holder}), inferimos
    \begin{equation} \label{eq:uylp}
        \begin{aligned}
            \sint_{B(x,1)} |u(y)| \,dy &= \frac{1}{\sigma(n)} \int_{B(x,1)} |u(y)| \,dy\\ 
            &\leqslant \frac{1}{\sigma(n)} \left( \int_{B(x,1)} |u(y)|^p\,dy \right)^{\frac{1}{p}} \left( \int_{B(x,1)} 1^{p'} \,dy \right)^{\frac{1}{p'}} \leqslant c \Vert u \Vert_{\cL^p(\bR^n)},
        \end{aligned}
    \end{equation}
    onde $p$ e $p'$ são expoentes conjugados.
    Além disso, utilizando a desigualdade de Hölder novamente, chegamos a
    \begin{equation} \label{eq:llk}
        \int_{B(x,1)} \frac{\Vert Du(y) \Vert}{\Vert x - y \Vert^{n-1}} \,dy \leqslant \left( \int_{B(x,1)} \Vert Du \Vert^p \,dy \right)^{\frac{1}{p}} \left( \int_{B(x,1)} \frac{dy}{\Vert x - y \Vert^{(n-1)p'}} \right)^{\frac{1}{p'}},
    \end{equation}
    onde $p$ e $p'$ são expoentes conjugados e a última integral é finita.
    De fato, utilizando coordenadas polares (ver Teorema \ref{thm:coordenadas-polares}), concluímos que
    \[
        \begin{aligned}
            \int_{B(x,1)} \frac{dy}{\Vert x - y \Vert^{(n-1)p'}} &= \int_0^1 \int_{\partial B(x,r)} \frac{1}{r^{(n-1)p'}} dS_rdr\\
            &= n\sigma(n)\int_0^1 r^{(n-1)(1-p')} \,dr = n \sigma(n)  \frac{p-1}{p-n} < \infty,
        \end{aligned}
    \]
    pois $n < p$.
    Dito isso, por (\ref{eq:llk}), segue que
    \begin{equation} \label{eq:duxy}
        \int_{B(x,1)} \frac{\Vert Du(y) \Vert}{\Vert x - y \Vert^{n-1}} \,dy \leqslant c\Vert Du \Vert_{\cL^p(\bR^n)}.
    \end{equation}
    Dessa forma, por (\ref{eq:desigualdade-uyux}), (\ref{eq:ux}), (\ref{eq:uylp}) e (\ref{eq:duxy}), deduzimos que
    \[
        |u(x)| \leqslant c \Vert u \Vert_{\cW^{1,p}(\bR^n)}.
    \]
    Como $x$ é arbitrário, também obtemos
    \begin{equation} \label{eq:normac}
        \Vert u \Vert_{\cC(\bR^n)} = \sup_{x \in \bR^n} |u(x)| \leqslant c \Vert u \Vert_{\cW^{1,p}(\bR^n)}.
    \end{equation}

    \begin{figure}
        \centering
        \begin{tikzpicture}
            \def\CA{(0,0) circle (2)}
            \def\CB{(2,0) circle (2)}
            \draw[dashed, thick, black!45] \CA \CB;
            \begin{scope}
                \clip \CA;
                \fill[tangerine!40] \CB;
            \end{scope}     
            \begin{scope}
                \clip \CA;
                \draw[tangerine, dashed, very thick] \CB;
            \end{scope}
            \begin{scope}
                \clip \CB;
                \draw[tangerine, dashed, very thick] \CA;
            \end{scope}    
    
            \node[above=2pt] at (0,2) {$B(x,r)$};
            \node[above=2pt] at (2,2) {$B(y,r)$};
    
            \draw (0,0) to node[above] {$r$} (2,0);
    
            \filldraw (0,0) circle (0.05) node[below=2pt] {$x$};
            \filldraw (2,0) circle (0.05) node[below=2pt] {$y$};
        \end{tikzpicture}
        \caption{$B = B(x,r) \cap B(y,r)$.\\
        Fonte: Autoral.}
        \label{fig:B}
    \end{figure}
    Agora, considere $x, y \in \bR^n$ e denote $r = \Vert x - y \Vert$. Portando, seja $B = B(x,r) \cap B(y,r)$ (ver Figura \ref{fig:B}), sendo assim
    \begin{equation} \label{eq:abcd}
        |u(x) - u(y)| = \sint_{B} |u(x) - u(y)| \,dz \leqslant \sint_{B} |u(x) - u(z)| \,dz + \sint_B |u(y) - u(z)| \,dz.
    \end{equation}
    Calculando a primeira integral acima, obtemos, por (\ref{eq:desigualdade-uyux}), que
    \[
        \sint_B |u(x) - u(z)| \,dz \leqslant \sint_{B(x,r)} |u(z) - u(x)| \,dz \leqslant c \int_{B(x,r)} \frac{\Vert Du(z) \Vert}{\Vert z - x \Vert^{n-1}} \,dz.
    \]
    Analogamente a (\ref{eq:duxy}), inferimos que
    \[
        \sint_B |u(x) - u(z)| \,dz \leqslant cr^{1 - \frac{n}{p}} \Vert Du \Vert_{\cL^p(\bR^n)}.
    \]
    Da mesma forma, segue que
    \[
        \sint_B |u(y) - u(z)| \,dz \leqslant cr^{1 - \frac{n}{p}} \Vert Du \Vert_{\cL^p(\bR^n)}.
    \]

    Sendo assim, por (\ref{eq:abcd}), podemos escrever
    \[
        |u(x) - u(y)| \leqslant c \Vert x - y \Vert^{1 - \frac{n}{p}} \Vert Du \Vert_{\cL^p(\bR^n)}.
    \]
    Isso mostra que
    \begin{equation} \label{eq:seminormau}
        [u]_{\cC^{0,\gamma}(\bR^n)} = \sup_{\substack{x,y \in \bR^n\\x \neq y}} \left\{ \frac{|u(x) - u(y)|}{\Vert x - y \Vert^{\gamma}} \right\} \leqslant c \Vert Du \Vert_{\cL^p(\bR^n)}.
    \end{equation}
    Portanto, por (\ref{eq:normac}) e (\ref{eq:seminormau}), concluímos que
    \[
        \Vert u \Vert_{\cC^{0,\gamma}(\bR^n)} \leqslant c \Vert u \Vert_{\cW^{1,p}(\bR^n)}.
    \]
\end{prf}

Para demonstrar a próxima desigualdade de Sobolev, precisamos do seguinte resultado

\begin{tbox} \label{thm:holder-completo}
    O espaço de Hölder $(\cC^{0,\gamma}(\overline\Omega), \Vert \cdot \Vert_{\cC^{0,\gamma}(\overline\Omega)})$ é um espaço de Banach.
\end{tbox}
\begin{prf}
    Ver \cite{irina-holder.spaces}, p.p. 5.
\end{prf}

\begin{tbox}
    Seja $\Omega$ um aberto limitado, com $\partial\Omega$ de classe $\cC^1$.
    Considere $u \in \cW^{1,p}(\Omega)$ com $n < p \leqslant \infty$.
    Então, $u$ tem uma versão (coincide com $u$ qtp em $\Omega$) contínua $u^* \in \cC^{0,\gamma}(\overline\Omega)$, com $\gamma = 1 - \frac{n}{p}$, tal que
    \[
        \Vert u^*  \Vert_{\cC^{0,\gamma}(\overline\Omega)} \leqslant \Vert u \Vert_{\cW^{1,p}(\Omega)},
    \]
    onde $c$ é um constante que depende de $n$, $p$ e $\Omega$.
\end{tbox}
\begin{prf}
    Utilizando o Teorema \ref{thm:extensao}, podemos considerar a extensão de $u$, $Eu = \bar u$, tal que
    \begin{equation} \label{eq:oll}
        \begin{aligned}
            &\bar u = u \text{ qtp em } \Omega;\\
            &\bar u \text{ tem suporte compacto};\\
            &\Vert \bar u \Vert_{\cW^{1,p}(\bR^n)} \leqslant \Vert u \Vert_{\cW^{1,p}(\Omega)}.
        \end{aligned}
    \end{equation}
    Além disso, como $\supp \bar u$ é compacto, temos, pelo Teorema \ref{thm:aprox1}, que existe uma sequência $(u_k) \subseteq \cC^\infty_c(\bR^n) \subseteq \cC^1(\bR^n)$, para $k$ suficientemente grande, tal que
    \begin{equation} \label{eq:oll1}
        \Vert u_k - \bar u \Vert_{\cW^{1,p}(\bR^n)} \to 0,
    \end{equation}
    quando $k \to \infty$
    Além disso, pelo Teorema \ref{thm:holdersobolev1}, podemos escrever
    \[
        \Vert u_k -u_l \Vert_{\cC^{0,\gamma}(\bR^n)} \leqslant c \Vert u_k - u_l \Vert_{\cW^{1,p}(\bR^n)},
    \]
    para todo $k, \ell \in \bN$.
    Isto nos diz que $(u_k)$ é de Cauchy em $\cC^{0,\gamma}(\bR^n)$ (pois é de Cauchy em $\cW^{1,p}(\bR^n)$).
    Como $\cC^{0,\gamma}(\bR^n)$ é completo (ver Teorema \ref{thm:holder-completo}), existe uma função $u^* \in \cC^{0,\gamma}(\bR^n)$ tal que
    \begin{equation} \label{eq:oll2}
        \Vert u_k - u^* \Vert_{\cC^{0,\gamma}(\bR^n)} \to 0,
    \end{equation}
    quando $k \to \infty$
    Note que $u^* = \bar u$ qtp em $\bR^n$ (análogo ao que foi feito na demonstração do Teorema \ref{thm:desigualdade-teorema-2}) (por (\ref{eq:oll1}) e (\ref{eq:oll2})) e $u = \bar u$ qtp em $\Omega$ (por (\ref{eq:oll})).
    Logo, $u = u^*$ qtp em $\Omega$, ou seja, $u^*$ é uma versão contínua de $u$.
    O Teorema \ref{thm:holdersobolev1} também implica em
    \[
        \Vert u_k \Vert_{\cC^{0,\gamma}(\bR^n)} \leqslant c \Vert u_k \Vert_{\cW^{1,p}(\bR^n)},
    \]
    para todo $k \in \bN$.
    Passando ao limite, quando $k \to \infty$, chegamos, por (\ref{eq:oll1}) e (\ref{eq:oll2}), a
    \[
        \Vert u^* \Vert_{\cC^{0,\gamma}(\bR^n)} \leqslant c \Vert \bar u \Vert_{\cW^{1,p}(\bR^n)}.
    \]
    Por fim, por (\ref{eq:oll}), é verdade que
    \[
        \Vert u^* \Vert_{\cC^{0,\gamma}(\overline \Omega)} \leqslant \Vert u^* \Vert_{\cC^{0,\gamma}(\bR^n)} \leqslant c \Vert \bar u \Vert_{\cW^{1,p}(\bR^n)} \leqslant c \Vert u \Vert_{\cW^{1,p}(\Omega)}.
    \]
    Portanto, concluímos que
    \[
        \Vert u^* \Vert_{\cC^{0,\gamma}(\overline\Omega)} \leqslant c \Vert u \Vert_{\cW^{1,p}(\Omega)},
    \]
    como era desejado.
\end{prf}

\subsection{Desigualdades gerais de Sobolev}

Agora, vamos explorar algumas desigualdades em espaços de Sobolev com derivadas (fracas) de ordem superior.

\begin{tbox} \label{thm:geral-1}
    Sejam $\Omega$ um aberto limitado com fronteira de classe $\cC^1$ e $u \in \cW^{k,p}(\Omega)$.
    Se $kp < n$, então $u \in \cL^q(\Omega)$, onde
    \[
        \frac{1}{q} = \frac{1}{p} - \frac{k}{n}.
    \]
    Além disso, a desigualdade
    \[
        \Vert u \Vert_{\cL^q(\Omega)} \leqslant c \Vert u \Vert_{\cW^{k,p}(\Omega)}
    \]
    é valida, onde $c$ depende apenas de $k, p, n$ e $\Omega$.
\end{tbox}
\begin{prf}
    Como $u \in \cW^{k,p}(\Omega)$, com $1 \leqslant p \leqslant kp < n$, temos que $D^\alpha u \in \cL^p(\Omega)$, para todo multi-índice $\alpha$ com $|\alpha| \leqslant k$. 
    Como $\Omega$ é limitado e $\partial\Omega$ é de classe $\cC^1$, pelo Teorema \ref{thm:extensao}, podemos considerar uma extenão $\overline{D^\beta u}$ de $D^\beta u$ com $|\beta| \leqslant k -1$. 
    Utilizando a Desigualdade de Gagliardo-Nirenberg-Sobolev (\ref{eq:gns}), obtemos
    \[
        \Vert \overline{D^\beta u} \Vert_{\cL^{p^*}(\bR^n)} \leqslant c \Vert D (\overline{D^\beta u}) \Vert_{\cL^p(\bR^ n)}.
    \]
    Dessa forma, como $\overline{D^\beta u} = D^\beta u$ qtp em $\Omega$, podemos escrever
    \[
        \Vert D^\beta u \Vert_{\cL^{p^*}(\Omega)} = \Vert \overline{D^\beta u} \Vert_{\cL^{p*}(\Omega)} \leqslant \Vert \overline{D^\beta u} \Vert_{\cL^{p*}(\bR^n)} \leqslant c \Vert D (\overline{D^\beta}u) \Vert_{\cL^p(\bR^ n)} \leqslant c \Vert \overline{D^\beta u} \Vert_{\cW^{1,p}(\Omega)}.
    \]
    Porém a extensão acima é um operador linear limitado. Mais precisamente, temos que $\Vert \bar v \Vert_{\cW^{1,p}(\bR^n)} \leqslant c \Vert v \Vert_{\cW^{1,p}(\Omega)}$, para todo $v \in \cW^{1,p}(\Omega)$. Dito isso, inferimos que
    \[
        \Vert D^\beta u \Vert_{\cL^{p^*}(\Omega)} \leqslant c \Vert D^\beta u \Vert_{\cW^{1,p}(\Omega)} \leqslant c \Vert u \Vert_{\cW^{k,p}(\Omega)}.
    \]
    Dessa forma, $u \in \cW^{k-1,p^*}(\Omega)$ e $\Vert u \Vert_{\cW^{k-1,p^*}(\Omega)} \leqslant c \Vert u \Vert_{\cW^{k,p}(\Omega)}$.
    De forma análoga, inferimos que $u \in \cW^{k-2,p^{**}}(\Omega)$, onde
    \[
        \frac{1}{p^{**}} = \frac{1}{p^*} - \frac{1}{n} = \frac{1}{p} - \frac{1}{n} - \frac{1}{n} = \frac{1}{p} - \frac{2}{n}.
    \]
    Indutivamente, chegamos a $u \in \cW^{0,q}(\Omega) = \cL^q(\Omega)$, com $\frac{1}{q} = \frac{1}{p} - \frac{k}{n}$, e
    \[
        \Vert u \Vert_{\cW^{k,p}(\Omega)} \geqslant c \Vert u \Vert_{\cW^{k-1,p^*}(\Omega)} \geqslant c \Vert u \Vert_{\cW^{k-2,p^{**}}(\Omega)} \geqslant \cdots\geqslant c \Vert u \Vert_{\cW^{0,q}(\Omega)} = c\Vert u \Vert_{\cL^q(\Omega)}.
    \]
    Portanto,
    \[
        \Vert u \Vert_{\cL^q(\Omega)} \leqslant c \Vert u \Vert_{\cW^{k,p}(\Omega)},
    \]
    desde que $\frac{1}{q} = \frac{1}{p} - \frac{k}{n}$.
\end{prf}

Vejamos agora uma generalização dos espaços de Hölder $\cC^{0,\gamma}(\overline \Omega)$.

\begin{dbox}
    O espaço de Hölder $\cC^{k,\gamma}(\overline\Omega)$ é formado pelas funções $u \in \cC^k(\Omega) \cap \cC^{0,\gamma}(\overline\Omega)$. Esse espaço é munido da norma
    \[
        \Vert u \Vert_{\cC^{k,\gamma}(\overline\Omega)} := \sum_{|\alpha| \leqslant k} \Vert D^{\alpha}u \Vert_{\cC(\overline\Omega)} + \sum_{|\alpha| = k} [D^\alpha u]_{\cC^{0,\gamma}(\overline\Omega)}
    \] 
\end{dbox}

O próximo resultado estabelece uma relação entre os espaços de Hölder definidos acima e os espaços de Sobolev.

\begin{tbox} \label{thm:geral2}
    Sejam $\Omega$ um aberto limitado com fronteira de classe $\cC^1$, e $u \in \cW^{k,p}(\Omega)$. 
    Se $kp > n$, então $u \in \cC^{k - \ell - 1,\gamma}(\overline\Omega)$, onde $\ell = \left\lfloor \frac{n}{p} \right\rfloor$ e
    \[
        \gamma = \left\lfloor \frac{n}{p} \right\rfloor + 1 - \frac{n}{p}
    \]
    se $\frac{n}{p}$ não é um inteiro e $\gamma \in (0,1)$ se $\frac{n}{p}$ é um inteiro.
    Além disso a desigualdade
    \begin{equation} \label{eq:geral2}
        \Vert u \Vert_{\cC^{k - \ell - 1,\gamma}(\overline\Omega)} \leqslant c \Vert u \Vert_{\cW^{k,p}(\Omega)}
    \end{equation}
    é valida, onde $c$ depende apenas de $k, p, n, \gamma$ e $\Omega$.
\end{tbox}
\begin{prf}
    Suponha que $\frac{n}{p}$ não é um inteiro (isto é, $\gamma = \left\lfloor \frac{n}{p} \right\rfloor + 1 - \frac{n}{p}$). Então, como visto na demonstração do Teorema \ref{thm:geral-1}, temos que $u \in \cW^{k-\ell,r}(\Omega)$ quando
    \begin{equation} \label{eq:111}
        \frac{1}{r} = \frac{1}{p} - \frac{\ell}{n},
    \end{equation}
    desde que $\ell p < n$ (veja que $r = p^{\overbrace{**\cdots*}^{\ell \text{ vezes}}}$). Sendo assim, $\ell$ é um inteiro tal que
    \begin{equation} \label{eq:222}
        \ell < \frac{n}{p} < \ell + 1,
    \end{equation}
    isto é, $\ell = \left\lfloor \frac{n}{p} \right\rfloor$.
    Consequentemente, temos por (\ref{eq:111}) e (\ref{eq:222}), que
    \begin{equation}
        r = \frac{pn}{n-p\ell} > n,
    \end{equation}
    pois $n - pl > 0$. Além disso, $D^\alpha u \in \cW^{1,r}(\Omega)$, para todo $|\alpha| \leqslant k - \ell -1$. Dito isso, pelo Teorema \ref{thm:extensao}, seja $\overline{D^\alpha u} \in \cW^{1,r}(\bR^n)$ uma extensão de $D^\alpha u$.
    Assim, pela Desigualdade de Morrey (ver Teorema \ref{thm:holdersobolev1}), podemos escrever
    \begin{equation} \label{eq:xx}
        \begin{aligned}
            \Vert D^\alpha u \Vert_{\cC^{0,\gamma}(\overline\Omega)} = \Vert \overline{D^\alpha u} \Vert_{\cC^{0,\gamma}(\overline\Omega)} &\leqslant \Vert \overline{D^\alpha u} \Vert_{\cC^{0,\gamma}(\bR^n)} \\
            &\leqslant c \Vert \overline{D^\alpha u} \Vert_{\cW^{1,r}(\bR^n)}\leqslant c \Vert D^\alpha u \Vert_{\cW^{1,r}(\Omega)} \leqslant c \Vert u \Vert_{\cW^{k-\ell,r}(\Omega)}.
        \end{aligned}
    \end{equation}
    pois $\overline{D^\alpha u} = D^\alpha u$ qtp em $\Omega$, $\supp \overline{D^\alpha u}$ é compacto, $\gamma = p + 1 - \frac{n}{p} = 1 - \frac{n}{r}$ e $\cC^\infty(\bR^n)$ é denso em $\cW^{k,p}(\bR^n)$ (pelo Teorema \ref{thm:aprox3}).
    % Isso mostra que $D^\alpha u \in \cC^{0,\gamma}(\overline\Omega)$, pois $u \in \cW^{k,\ell}(\Omega)$
    % Portanto, $u \in \cC^{k-\ell-1,\gamma}(\overline\Omega)$ e por (\ref{eq:xx}), inferimos 
    Portanto, por (\ref{eq:xx}), inferimos
    \[
        \begin{aligned}
            \Vert u \Vert_{\cC^{k-\ell-1,\gamma}(\overline\Omega)} &\leqslant \sum_{|\alpha| \leqslant k - \ell - 1} \left( \Vert D^\alpha u \Vert_{\cC(\overline\Omega)} + [D^\alpha u]_{\cC^{0,\gamma}(\overline\Omega)} \right)\\
            &\leqslant \sum_{|\alpha| \leqslant k - \ell - 1} \!\!\!2 \Vert D^\alpha u  \Vert_{\cC^{0,\gamma}(\overline\Omega)} \leqslant \sum_{|\alpha| \leqslant k - \ell - 1} \!\!\!c \Vert u  \Vert_{\cW^{k-\ell,r}(\Omega)} \leqslant c \Vert u \Vert_{\cW^{k-\ell,r} (\Omega)} \leqslant c \Vert u \Vert_{\cW^{k,p}(\Omega)},
        \end{aligned}
    \]
    (ver demonstração do Teorema \ref{thm:geral-1}). Isto prova (\ref{eq:geral2}).

    Por fim, suponha que $\frac{n}{p}$ é um inteiro. Seja $\ell = \left\lfloor \frac{n}{p} \right\rfloor - 1 = \frac{n}{p} - 1$.
    Como anteriormente, temos que $u \in \cW^{k-\ell,r}(\Omega)$ (pois $\Omega$ é limitado), desde que (\ref{eq:111}) seja satisfeito, onde
    \[
        r = \frac{pn}{n - p\ell} = n \geqslant p.
    \]
    pois $\frac{n}{p}$ é um inteiro positivo.
    Isto nos diz que $D^\alpha u \in \cL^r(\Omega)$ para todo $|\alpha| \leqslant k - \ell$.
    Sejam $n < q < \infty$ e $1 \leqslant s = \frac{nq}{n + q} < n$. Dessa forma, $s^* = q$ e pela Desigualdade de Gagliardo-Nirenberg-Sobolev (\ref{eq:gns})  e por um argumento de densidade, temos, considerando uma extensão $\overline{D^\alpha u} \in \cL^q(\bR^n)$, que
    \[
        \begin{aligned}
            \Vert D^\alpha u \Vert_{\cL^q(\Omega)} = \Vert \overline{D^\alpha u}  \Vert_{\cL^q(\Omega)} \leqslant \Vert \overline{D^\alpha u} \Vert_{\cL^q(\bR^n)} &\leqslant c \Vert D \overline{D^\alpha u} \Vert_{\cL^s(\bR^n)}\\ &\leqslant c \Vert D D^\alpha u \Vert_{\cL^s(\Omega)} \leqslant c\Vert D D^\alpha u \Vert_{\cL^r(\Omega)} < \infty,
        \end{aligned}
    \]
    para todo $|\alpha| \leqslant k - \ell - 1$, pois $u \in \cW^{k-\ell,r}(\Omega)$, $s < r$ e $\Omega$ é limitado. Por isso,
    \[
        \Vert D^\alpha u \Vert_{\cL^q(\Omega)} \leqslant c \Vert u \Vert_{\cW^{k-\ell,r}(\Omega)} \leqslant c \Vert u \Vert_{\cW^{k,p}(\Omega)},
    \]
    para todo $|\alpha| \leqslant k - \ell - 1$. Isto nos diz que $D^\alpha u \in \cL^q(\Omega)$, para todo $|\alpha| \leqslant k - \ell - 1 = k - \frac{n}{p}$.
    Como $n < q < \infty$, utilizando a Desigualdade de Morrey (ver Teorema \ref{thm:holdersobolev1}), encontramos
    \[
        \Vert D^\alpha u \Vert_{\cC^{0,1-\frac{n}{q}}(\overline\Omega)} = \Vert \overline{D^\alpha u} \Vert_{\cC^{0,1-\frac{n}{q}}(\overline\Omega)} \leqslant \Vert \overline{D^\alpha u} \Vert_{\cC^{0,1 - \frac{n}{q}}(\bR^n)} \leqslant c \Vert \overline{D^\alpha u} \Vert_{\cW^{1,q}(\bR^n)} \leqslant c \Vert D^\alpha u \Vert_{\cW^{1,q}(\Omega)},
    \]
    para todo $|\alpha| \leqslant k - \ell - 2$ (por um argumento de densidade).
    Assim, tomando qualquer $\gamma \in (0,1)$, $q = \frac{n}{1 - \gamma}$ e $s_\gamma = \frac{np}{(1- \gamma)p + n}$, obtemos
    \[
        \Vert u \Vert_{\cC^{k-\ell-1,\gamma}(\overline\Omega)} \leqslant c\Vert u \Vert_{\cW^{k-\ell-1,q}(\Omega)} = c \Vert u \Vert_{\cW^{k-\frac{n}{p},q}(\Omega)} \leqslant c \Vert u \Vert_{\cW^{k,s_\gamma}(\Omega)} \leqslant c \Vert u \Vert_{\cW^{k,p}(\Omega)},
    \]
    pois $n < q < \infty$ e $\frac{1}{q} = \frac{1}{s_\gamma} - \frac{n/p}{n}$ (como na demonstração do Teorema \ref{thm:geral-1}, já que $q = s_\gamma^{\overbrace{**\cdots*}^{n/p \text{ vezes}}}$).
\end{prf}

\section{Compacidade}

Nessa seção, vamos estudar mergulhos compactos dos espaços de Sobolev em espaços de Lebesgue.
Para isso, precisamos entender entender a diferença entre um mergulho contínuo, como os que foram trabalhados na seção anterior, e um mergulho compacto.

\begin{dbox}
    Sejam $X, Y$ espaços de Banach com $X \subseteq Y$. Dizemos que $X$ está compactamente mergulhado em $Y$ e denotamos $X \doublehookrightarrow Y$,
    se para todo $x \in X$, tem-se
    \[
        \Vert x \Vert_{Y} \leqslant c \Vert x \Vert_X
    \]
    e se toda sequência limitada em $X$ é precompacta em $Y$, isto é, existe uma subsequência que converge em $Y$.
\end{dbox}

\obs Se um mergulho satisfaz apenas a primeira propiedade, dizemos que $X$ está continuamente mergulhado em $Y$ e denotamos por $X \hookrightarrow Y$. Na seção anterior vimos os seguintes mergulhos contínuos.
\begin{itemize}
    \item $\cW^{1,p}_0(\Omega) \hookrightarrow \cL^q(\Omega)$, com $1 \leqslant q \leqslant p^*$ e $1 \leqslant p < n$, pelo Teorema \ref{thm:poincaregen};
    \item $\cW^{1,p}_0(\Omega) \hookrightarrow \cL^{p^*}(\Omega)$, com $1 \leqslant s \leqslant \infty$, pelo Corolário \ref{cl:poincare};
    \item $\cW^{k,p}(\Omega) \hookrightarrow \cL^q(\Omega)$, com $kp < n$ e $\frac{1}{q} = \frac{1}{p} + \frac{k}{n}$, pelo Teorema \ref{thm:geral-1};
    \item $\cW^{k,p}(\Omega) \hookrightarrow \cC^{k-\ell-1\gamma}(\overline\Omega)$, com $kp > n$ e $\ell = \left\lfloor \frac{n}{p} \right\rfloor$, pelo Teorema \ref{thm:geral2}.
\end{itemize}

\begin{tbox}[Teorema de Rellich-Kondrachov] \label{thm:compacidade}
    Seja $\Omega \subseteq \bR^n$ um aberto limitado com fronteira de classe $\cC^1$.
    Então,
    \[
        \cW^{1,p}(\Omega) \doublehookrightarrow \cL^q(\Omega),
    \]
    com $1 \leqslant p < n$ e $1 \leqslant q < p^*$.
\end{tbox}
% \begin{prf}
%     Seja $1 \leqslant q < p^*$ fixo.
%     Como $\Omega$ é limitado, segue que $\Vert u \Vert_{\cL^q(\Omega)} \leqslant c \Vert u \Vert_{\cL^{p^*}(\Omega)}$ e pelo Teorema \ref{thm:desigualdade-teorema-2} temos $\Vert u \Vert_{\cL^{p^*}(\Omega)} \leqslant c \Vert u \Vert_{\cW^{1,p}(\Omega)}$. Logo $\cW^{1,p}(\Omega) \subseteq \cL^q(\Omega)$ e $\Vert u \Vert_{\cL^p(\Omega)} \leqslant c \Vert u \Vert_{\cW^{1,p}(\Omega)}$.
%     Resta mostrar que se $(u_k)_{k = 1}^\infty$ é uma sequência limitada em $\cW^{1,p}(\Omega)$, então existe uma subsequência $(u_{k_j})_{j=1}^\infty$ que converge em $\cL^q(\Omega)$.

%     Pelo Teorema de Extensão, podemos supor sem perda de generalidade que $\Omega = \bR^n$ e que para todo $k \in \bN$, $u_k$ tem suporte compacto em algum aberto limitado $V \subseteq \bR^n$. 
%     Também podemos supor que
%     \begin{equation} \label{eq:supfinito}
%         \sup_k \Vert u_k \Vert_{\cW^{1,p}(V)} < \infty
%     \end{equation}
%     poisa sequência é limitada

%     Primeiramente vamos estudar as funções suavizadas $u_k^\varepsilon = \eta_\varepsilon * u_k$ onde $\eta_\varepsilon$ é a função molificadora vista na Seção \ref{sec:aproximacoes}. Também podemos supor que para todo $k \in \bN$ e $\varepsilon > 0$ as funções $u_k^\varepsilon$ tem suporte em $V$. Afirmamos que
%     \begin{equation} \label{eq:convergenciaemlq}
%         \Vert u^\varepsilon_k - u_k \Vert_{\cL^q(V)} \to 0
%     \end{equation}
%     uniformemente em $k$, quando $\varepsilon \to 0$.
%     Com efeito, note que se $u_k$ é suave, então
%     \[
%         u_k^\varepsilon(x) - u_k(x) = \int_{B(0,\varepsilon)} \eta_\varepsilon(\tau) u_k (x - \tau) \,d\tau - u_k(x)  = \int_{B(0,\varepsilon)} \frac{1}{\varepsilon^n} \eta \left( \frac{\tau}{\varepsilon} \right) u_k (x - \tau) \,d\tau - u_k(x)
%     \]
%     Fazendo a substituição $\tau = \varepsilon y$ e lembrando que $\int_{B(0,1)} \eta(y) \,dy = 1$, obtemos
%     \[
%         u_k^\varepsilon(x) - u_k(x) = \int_{B(0,1)} \eta(y) \left( u_k(x - \varepsilon y) - u_k(x) \right) \,dy = \int_{B(0,1)} \eta(y) \int_0^1 \frac{d}{dt}u_k(x - \varepsilon t y) \,dt dy
%     \]
%     Para calcular essa derivada, seja $g(t) = x - \varepsilon t y$. Pela Regra da Cadeia temos que
%     \[
%         \frac{d}{dt} (u_k \circ g) = \sum_{j=1}^n \dfrac{\partial u_k}{\partial x_j} (g(t)) \, g_j'(t) = -\varepsilon\sum_{j=1}^{n} \dfrac{\partial u_k}{\partial x_j}(g(t)) \,y_j = -\varepsilon Du_k(x - \varepsilon ty) \cdot y
%     \]
%     Sendo assim
%     \[
%         u_k^\varepsilon(x) - u_k(x) = -\varepsilon \int_{B(0,1)} \eta(y) \int_0^1 Du_k(x - \varepsilon ty) \cdot y \,dtdy.
%     \]
%     Daí, passando o módulo em ambos os lados e Utilizando a Desigualdade de Hölder (Cauchy-Schwarz) obtemos
%     \[
%         |u_k^\varepsilon(x) - u_k(x)| \leqslant \varepsilon \int_{B(0,1)} \eta(y) \int_0^1 \Vert Du_k(x - \varepsilon ty) \Vert \,dtdy
%     \]
%     e integrando ambos os lados sobre $V$
%     \[
%         \begin{aligned}
%             \int_{V} |u_k^\varepsilon(x) - u_k(x)|\,dx &\leqslant \int_{V} \varepsilon \int_{B(0,1)} \eta(y) \int_0^1 \Vert Du_k(x - \varepsilon ty) \Vert \,dtdydx\\
%             &\leqslant \varepsilon \int_{B(0,1)} \eta(y) \int_0^1 \int_{V} \Vert Du_k(x - \varepsilon ty) \Vert \,dxdtdy \leqslant \varepsilon \int_{V} \Vert Du_k(z) \Vert    \,dz.
%         \end{aligned}
%     \]
%     Isto é
%     \[
%         \Vert u_k^\varepsilon - u_k \Vert_{\cL^1(V)} \leqslant \varepsilon \Vert Du_k \Vert_{\cL^1(V)} \leqslant \varepsilon c \Vert Du_k \Vert_{\cL^p(V)} \leqslant \varepsilon c \Vert u_k \Vert_{\cW^{1,p}(V)}
%     \]
%     Akém disso, utilizando (\ref{eq:supfinito}) e o Teorema da Convergência Dominada obtemos que
%     \begin{equation} \label{eq:eml1converge}
%         \Vert u_k^\varepsilon - u_k \Vert_{\cL^1(V)} \to 0
%     \end{equation}
%     quando $\varepsilon \to 0$ uniformemente em $k$. Como $1 \leqslant q < p^*$, podemos utilizar a Desigualdade de Interpolação das normas $\cL^p$
%     \[
%         \Vert u_k^\varepsilon - u_k \Vert_{\cL^q(V)} \leqslant \Vert u_k^\varepsilon - u_k \Vert_{\cL^1(V)}^\theta \Vert u_k^\varepsilon - u_k \Vert_{\cL^{p^*}(V)}^{1- \theta}
%     \]
%     onde $\frac{1}{q} = \theta + \frac{(1 - \theta)}{p^*}$ e $0 < \theta < 1$.
%     Ademais, por (\ref{eq:supfinito}) e pela Desigualdade de Gagliardo-Nirenberg-Sobolev, $\Vert u_k^\varepsilon - u_k \Vert_{\cL^{p^*}(V)}^{1- \theta}$ é finito. De fato
%     \[
%         \Vert u_k^\varepsilon - u_k \Vert_{\cL^{p^*}(V)}^{1 - \theta} \leqslant c \Vert Du_k^\varepsilon - Du_k \Vert_{\cL^p(V)}^{1 - \theta} \leqslant c \Vert u_k^\varepsilon - u_k \Vert_{\cW^{1,p}(V)}^{1- \theta} < \infty. 
%     \]
%     Assim por (\ref{eq:eml1converge})
%     \[
%         \Vert u_k^\varepsilon - u_k \Vert_{\cL^q(V)} \leqslant c \Vert u_k^\varepsilon - u_k \Vert_{\cL^1(V)}^\theta \to 0
%     \]
%     uniformemente em $k$, quando $\varepsilon \to 0$. Como era desejado.

%     Agora, afirmamos que, para cada $\varepsilon > 0$ fixo, a sequência $(u_k^\varepsilon)_{k=1}^\infty$ é uniformemente limitada e equicontínua.
%     Com efeito, se $x \in \bR^n$
%     \begin{equation} \label{eq:ufin}
%         |u_k^\varepsilon(x)| \leqslant \int_{B(x,\varepsilon)} \eta_\varepsilon (x - y) |u_k(y)| \,dy \leqslant \Vert \eta_\varepsilon \Vert_{\cL^\infty(\bR^n)} \Vert u_k \Vert_{\cL^1(V)} \leqslant \frac{c}{\varepsilon^n} < \infty
%     \end{equation}
%     onde por (\ref{eq:supfinito}) $c$ não depende de $k$. De forma análoga, mostramos que
%     \begin{equation} \label{eq:Dufin}
%         \Vert Du_k^\varepsilon(x) \Vert  \leqslant \frac{c}{\varepsilon^{n+1}} < \infty.
%     \end{equation}
%     Isso prova que $(u_k^\varepsilon)_{k=1}^\infty \subseteq \cW^{1,p}(V)$ é uniformemente limitada pois mostramos que $\Vert u_k^\varepsilon \Vert_{\cW^{1,p}(\Omega)}\leqslant M$ onde $M > 0$ não depende de $k$.
%     Ademais, $(u_k^\varepsilon)_{k=1}^\infty$ é equicontínua pois, dado $\tilde\varepsilon > 0$ existe $\delta < \tilde\varepsilon / L$ tal que, pela Desigualdade do Valor Médio
%     \[
%         \Vert x - y \Vert \leqslant \delta \implies |u_k^\varepsilon(x) - u_k^\varepsilon(y)| \leqslant L \Vert x - y \Vert < L \delta < \tilde\varepsilon.
%     \]
%     onde $L = \sup_{x \in V} \Vert Du_k(x) \Vert$, que existe por (\ref{eq:supfinito}) e (\ref{eq:Dufin}) e não depende de $k$ e $x$.

%     Agora seja $\delta > 0$ fixo. Mostremos que existe uma subsequência $(u_{k_j})_{j=1}^\infty \subseteq (u_k)_{k=1}^\infty$ tal que
%     \begin{equation} \label{eq:limsup1}
%         \limsup_{j,\ell \to \infty} \Vert u_{k_j} - u_{k_\ell} \Vert_{\cL^q(V)} \leqslant \delta.
%     \end{equation}
%     De fato, por (\ref{eq:convergenciaemlq}), conseguimos um valor de $\varepsilon > 0$ suficientemente pequeno tal que
%     \[
%         \Vert u_k^\varepsilon - u_k \Vert_{\cL^q(V)} \leqslant \frac{\delta}{2}
%     \]
%     para todo $k \in \bN$
%     Além disso, sabemos que para todo $k \in \bN$ e $\varepsilon > 0$ as funções $u_k$ e $u_k^\varepsilon$ tem suporte em um aberto limitado fixo $V \subseteq \bR^n$, podemos utilizar o fato da sequência $(u_k^\varepsilon)_{k=1}^\infty$ ser equicontínua e uniformemente limitada junto do Critério de Compacidade de Arzelà-Ascoli para obter uma subsequência $(u_{k_j}^\varepsilon)_{j=1}^\infty \subseteq (u_k^\varepsilon)_{k=1}^\infty$ que converge uniformemente em $V$.
%     Em particular
%     \[
%         \limsup_{j,\ell \to \infty} \Vert u_{k_j}^\varepsilon - u_{k_\ell}^\varepsilon \Vert_{\cL^q(V)} = 0.
%     \]
%     Dessa forma
%     \[
%         \begin{aligned}
%             \limsup_{j,\ell \to \infty} \Vert u_{k_j} - u_{k_\ell} \Vert_{\cL^q(V)} &\leqslant \limsup_{j,\ell \to \infty} \Vert u_{k_j} - u_{k_j}^\varepsilon \Vert_{\cL^q(V)} \\
%             &+ \limsup_{j,\ell \to \infty} \Vert u_{k_j}^\varepsilon - u_{k_\ell}^\varepsilon \Vert_{\cL^q(V)} + \limsup_{j,\ell \to \infty} \Vert u_{k_\ell}^\varepsilon - u_{k_\ell} \Vert_{\cL^q(V)} \leqslant \delta.
%         \end{aligned}
%     \]
%     como era desejado.
%     Por fim, escolhendo $\delta = 1,\frac{1}{2},\dots$ em (\ref{eq:limsup1}), conseguimos extrair uma subsequência $(u_{k_j})_{j=1}^\infty \subseteq (u_k)_{k=1}^\infty$ que satisfaz
%     \[
%         \limsup_{j,\ell \to \infty} \Vert u_{k_j} - u_{k_\ell} \Vert_{\cL^q(V)} = 0
%     \]
%     \textcolor{red}{por que isso é suficiente?}
% \end{prf}

\begin{prf}
    Seja $1 \leqslant q < p^*$ fixo.
    Como $\Omega$ é limitado, temos, pelo Teorema \ref{thm:omega-limitado}, que $\Vert u \Vert_{\cL^q(\Omega)} \leqslant c \Vert u \Vert_{\cL^{p^*}(\Omega)}$
    e, pelo Teorema \ref{thm:desigualdade-teorema-2}, segue que
    $\Vert u \Vert_{\cL^{p^*}(\Omega)} \leqslant c \Vert u \Vert_{\cW^{1,p}(\Omega)}$.
    Logo, $\cW^{1,p}(\Omega) \subseteq \cL^q(\Omega)$ e
    \[
        \Vert u \Vert_{\cL^q(\Omega)} \leqslant c \Vert u \Vert_{\cW^{1,p}(\Omega)}.
    \]
    Resta mostrar que se $(u_k)_{k=1}^\infty$ é uma sequência limitada em $\cW^{1,p}(\Omega)$, então existe uma subsequência $(u_{k_j})_{j=1}^\infty$ que converge em $\cL^q(\Omega)$.
    
    Como $(u_k)_{k=1}^\infty$ é limitada, existe $M > 0$ tal que $\Vert u_k \Vert_{\cW^{1,p}(\Omega)} \leqslant M$, para todo $k \in \bN$.
    Observe que, pelo Teorema \ref{thm:extensao}, temos que existe uma sequência $(\bar u_k)_{k=1}^\infty \subseteq \cW^{1,p}(\bR^n)$ tal que, para todo $k \in \bN$, $\bar u_k = u_k$ qtp em $\Omega$, $\bar u_k$ tem suporte compacto e $\Vert \bar u_k \Vert_{\cW^{1,p}(\bR^n)} \leqslant c \Vert u_k \Vert_{\cW^{1,p}(\Omega)}$.
    Dessa forma, inferimos
    \[
        \Vert \bar u_k \Vert_{\cW^{1,p}(\bR^n)} \leqslant c \Vert u_k \Vert_{\cW^{1,p}(\Omega)} \leqslant cM,
    \]
    para todo $k \in \bN$.
    Logo, $(\bar u_k)_{k=1}^\infty \subseteq \cW^{1,p}(\bR^n)$ é limitada.
    Com isso
    \begin{equation} \label{eq:supfinito}
        \sup_{k} \Vert \bar u_k \Vert_{\cW^{1,p}(\bR^n)} \leqslant cM.
    \end{equation}

    Primeiramente, vamos estudar as funções suavizadas $\bar u_k^\varepsilon = \eta_\varepsilon * \bar u_k$ (para $\varepsilon > 0$), onde $\eta_\varepsilon$ é a função molificadora vista na Seção \ref{sec:aproximacoes}.
    Também podemos supor que, para todo $k \in \bN$ e $\varepsilon > 0$, as funções $\bar u_k^\varepsilon$ tem suporte em um aberto limitado $V$.
    Afirmamos que
    \[
        \Vert \bar u_k^\varepsilon - \bar u_k \Vert_{\cL^q(V)} \to 0,
    \]
    uniformemente em $k$, quando $\varepsilon \to 0$.
    Com efeito, considere $v_k \in \cC^{\infty}_c(\bR^n)$ e $v_k^\varepsilon = \eta_\varepsilon * v_k$ (para $\varepsilon > 0$ pequeno).
    Dessa forma, podemos escrever
    \[
        v_k^\varepsilon(x) - v_k(x) = \int_{B(0,\varepsilon)} \eta_\varepsilon(\tau) v_k (x - \tau) \,d\tau - v_k(x)  = \int_{B(0,\varepsilon)} \frac{1}{\varepsilon^n} \eta \left( \frac{\tau}{\varepsilon} \right) v_k (x - \tau) \,d\tau - v_k(x).
    \]
    Fazendo a substituição $\tau = \varepsilon y$ e lembrando que $\int_{B(0,1)} \eta(y) \,dy = 1$, obtemos, pelo Teorema Fundamental do Cálculo, que
    \[
        v_k^\varepsilon(x) - v_k(x) = \int_{B(0,1)} \eta(y) \left( v_k(x - \varepsilon y) - v_k(x) \right) \,dy = \int_{B(0,1)} \eta(y) \int_0^1 \frac{d}{dt}v_k(x - \varepsilon t y) \,dt dy.
    \]
    Para calcular a derivada acima, defina $g(t) = x - \varepsilon t y$. Pela regra da cadeia temos que
    \[
        \frac{d}{dt} (v_k \circ g)(t) = \sum_{j=1}^n \dfrac{\partial v_k}{\partial x_j} (g(t)) \, g_j'(t) = -\varepsilon\sum_{j=1}^{n} \dfrac{\partial v_k}{\partial x_j}(g(t)) \,y_j = -\varepsilon Dv_k(x - \varepsilon ty) \cdot y.
    \]
    Sendo assim, é verdade que
    \[
        v_k^\varepsilon(x) - v_k(x) = -\varepsilon \int_{B(0,1)} \eta(y) \int_0^1 Dv_k(x - \varepsilon ty) \cdot y \,dtdy.
    \]
    Logo, passando ao módulo em ambos os lados da igualdade acima e utilizando a desigualdade de Cauchy-Schwarz, deduzimos que
    \[
        |v_k^\varepsilon(x) - v_k(x)| \leqslant \varepsilon \int_{B(0,1)} \eta(y) \int_0^1 \Vert Dv_k(x - \varepsilon ty) \Vert \,dtdy.
    \]
    Integrando ambos os lados da desigualdade acima sobre $V$, e aplicando o Teorema de Fubini e o Teorema de Mudança de Variáveis (ver Teoremas \ref{thm:fubini} e \ref{thm:mudanca-de-variaveis}), chegamos a
    \[
        \begin{aligned}
            \int_{V} |v_k^\varepsilon(x) - v_k(x)|\,dx &\leqslant \int_{V} \varepsilon \int_{B(0,1)} \eta(y) \int_0^1 \Vert Dv_k(x - \varepsilon ty) \Vert \,dtdydx\\
            &\leqslant \varepsilon \int_{B(0,1)} \eta(y) \int_0^1 \int_{V} \Vert Dv_k(x - \varepsilon ty) \Vert \,dxdtdy \leqslant \varepsilon \int_{V} \Vert Dv_k(z) \Vert    \,dz,
        \end{aligned}
    \]
    (basta escolher $z = x - \varepsilon t y$).
    Isto é,
    \begin{equation} \label{eq:vekvek}
        \Vert v_k^\varepsilon - v_k \Vert_{\cL^1(V)} \leqslant \varepsilon \Vert Dv_k \Vert_{\cL^1(V)}
    \end{equation}
    para todo $k \in \bN$. Como $\bar u_k \in \cW^{1,p}(V) $ (pois $\bar u_k \in \cW^{1,p}(\bR^n)$), então existe uma sequência $(v^k_\ell) \subseteq \cC^{\infty}_c(V)$, dada por $v^k_\ell = \eta_{\frac{1}{\ell}} * \bar{u}_k$, tal que
    \[
        \Vert v^k_\ell - \bar{u}_k \Vert_{\cW^{1,p}(V)} \to 0,
    \]
    quando $\ell \to \infty$, para cada $k \in \bN$.
    Sendo assim, como $p \geqslant 1$ e $V$ é limitado (ver Teorema \ref{thm:omega-limitado}) concluímos que
    \begin{equation} \label{eq:xl}
        \Vert v^k_\ell - \bar{u}_k \Vert_{\cL^1(V)} \leqslant c \Vert v^k_\ell - \bar{u}_k \Vert_{\cL^p(V)} \leqslant c \Vert v^k_\ell - \bar{u}_k \Vert_{\cW^{1,p}(\Omega)} \to 0,
    \end{equation}
    e, de forma análoga,
    \begin{equation} \label{eq:xn}
        \Vert Dv^k_\ell - D\bar{u}_k \Vert_{\cL^1(V)}\leqslant c \Vert Dv^k_\ell - D\bar{u}_k \Vert_{\cL^p(V)} \leqslant c \Vert v^k_\ell - \bar{u}_k \Vert_{\cW^{1,p}(\Omega)} \to 0,
    \end{equation}
    ambos limites tomados quando $\ell \to \infty$. Dessa forma, inferimos, por (\ref{eq:xn}), que
    \begin{equation} \label{eq:xm}
        \Vert v^{k\varepsilon}_\ell - \bar u_k^{\varepsilon} \Vert_{\cL^1(V)} = \Vert \eta_\varepsilon * v^k_\ell - \eta_\varepsilon * \bar u_k \Vert_{\cL^1(V)} = \Vert \eta_\varepsilon  * (v^k_\ell - \bar u_k) \Vert_{\cL^1(V)} \leqslant \Vert v^k_\ell - \bar u_k \Vert_{\cL^1(V)} \to 0,
    \end{equation}
    quando $\ell \to \infty$, onde a última desigualdade vem de (\ref{eq:assss}).
    Por (\ref{eq:vekvek}), segue que
    \[
        \Vert v^{k\varepsilon}_\ell - v^k_\ell \Vert_{\cL^1(V)} \leqslant \varepsilon \Vert D v^k_\ell \Vert_{\cL^1(V)},
    \]
    para todo $k,\ell \in \bN$. Passando ao limite quando $\ell \to \infty$, obtemos por (\ref{eq:xl}), (\ref{eq:xn}) e (\ref{eq:xm}), que
    \[
        \Vert \bar u_k^\varepsilon - \bar u_k \Vert_{\cL^1(V)} \leqslant \varepsilon \Vert D\bar u_k \Vert_{\cL^1(V)},
    \]
    para todo $k \in \bN$. Assim, por (\ref{eq:supfinito}), concluímos que
    \[
        \Vert \bar u_k^\varepsilon - \bar u_k \Vert_{\cL^1(V)} \leqslant \varepsilon \Vert D\bar u_k \Vert_{\cL^1(V)} \leqslant \varepsilon c \Vert D\bar u_k \Vert_{\cL^p(V)} \leqslant \varepsilon c \Vert \bar u_k \Vert_{\cW^{1,p}(V)} \leqslant \varepsilon c M,
    \]
    para todo $k \in \bN$ (pois $p \geqslant 1$ e $V$ é limitado (ver Teorema \ref{thm:omega-limitado})). Consequentemente,
    \begin{equation} \label{eq:eml1converge2}
        \Vert \bar u_k^\varepsilon - \bar u_k \Vert_{\cL^1 (V)} \to 0,
    \end{equation}
    uniformemente em $k$, quando $\varepsilon \to 0$. 
    
    Como $1 \leqslant q < p^*$, podemos utilizar a Desigualdade de Interpolação (ver Teorema \ref{thm:desigualdade-de-inteporlacao}) para obter
    \[
        \Vert \bar u_k^\varepsilon - \bar u_k \Vert_{\cL^q(V)} \leqslant \Vert \bar u_k^\varepsilon - \bar u_k \Vert_{\cL^1(V)}^\theta \Vert \bar u_k^\varepsilon - \bar u_k \Vert_{\cL^{p^*}(V)}^{1- \theta},
    \]
    onde $\frac{1}{q} = \theta + \frac{(1 - \theta)}{p^*}$ e $0 < \theta < 1$.
    Ademais, por (\ref{eq:supfinito}), pela Desigualdade de Gagliardo-Nirenberg-Sobolev (\ref{eq:gns}) e por um argumento de densidade, deduzimos que $\Vert \bar u_k^\varepsilon - \bar u_k \Vert_{\cL^{p^*}(V)}^{1- \theta}$ é finito para todo $k \in\bN$. De fato, é verdade que
    \begin{equation} \label{eq:idk}
        \begin{aligned}
            \Vert \bar u_k^\varepsilon - \bar u_k \Vert_{\cL^{p^*}(V)}^{1 - \theta} &\leqslant c \Vert D\bar u_k^\varepsilon - D\bar u_k \Vert_{\cL^p(V)}^{1 - \theta}\\ 
            &\leqslant c \Vert \bar u_k^\varepsilon - \bar u_k \Vert_{\cW^{1,p}(V)}^{1- \theta} \leqslant c \big[\Vert \bar u_k^\varepsilon \Vert_{\cW^{1,p}(V)} + \Vert \bar u_k \Vert_{\cW^{1,p}(V)} \big]^{1-\theta} < cM.
        \end{aligned}
    \end{equation}
    Assim por (\ref{eq:eml1converge2}), deduzimos que
    \[
        \Vert \bar u_k^\varepsilon - \bar u_k \Vert_{\cL^q(V)} \leqslant c \Vert \bar u_k^\varepsilon - \bar u_k \Vert_{\cL^1(V)}^\theta \to 0,
    \]
    uniformemente em $k$, quando $\varepsilon \to 0$.

    Agora, afirmamos que, para cada $\varepsilon > 0$ fixo, a sequência $(\bar u_k^\varepsilon)_{k=1}^\infty$ é uniformemente limitada e equicontínua.
    Com efeito, se $x \in \bR^n$, tem-se pela desigualdade de Hölder (ver Teorema \ref{thm:pre-desigualdade-de-holder}) e por (\ref{eq:supfinito}), que
    \begin{equation} \label{eq:ufin}
        \begin{aligned}
            |\bar u_k^\varepsilon(x)| \leqslant \int_{B(x,\varepsilon)} \eta_\varepsilon (x - y) |\bar u_k(y)| \,dy &\leqslant \Vert \eta_\varepsilon \Vert_{\cL^\infty(\bR^n)} \Vert \bar u_k \Vert_{\cL^1(V)}\\ 
            &\leqslant \frac{c}{\varepsilon^n} \Vert \eta \Vert_{\cL^\infty(\bR^n)} \Vert \bar u_k \Vert_{\cL^p(V)} \leqslant \frac{cM}{\varepsilon^n} < \infty,
        \end{aligned}
    \end{equation}
    onde por (\ref{eq:supfinito}) $c$ não depende de $k$ (lembre que $\supp \bar u_k \subseteq V$, $p \geqslant 1$ e $V$ é limitado). De forma análoga, mostramos que
    \begin{equation} \label{eq:Dufin}
        \Vert D\bar u_k^\varepsilon(x) \Vert  \leqslant \frac{c}{\varepsilon^{n+1}} < \infty.
    \end{equation}
    Deste modo, $(\bar u_k^\varepsilon)_{k=1}^\infty$ é uniformemente limitada pois, mostramos que $\Vert \bar u_k^\varepsilon \Vert_{\cW^{1,p}(\Omega)}\leqslant M$ onde $M > 0$ não depende de $k$.
    Ademais, $(\bar u_k^\varepsilon)_{k=1}^\infty$ é equicontínua, pois, dado $\tilde\varepsilon > 0$ existe $\delta < \tilde\varepsilon / L$ tal que, pela Desigualdade do Valor Médio, encontramos
    \[
        \Vert x - y \Vert < \delta \implies |\bar u_k^\varepsilon(x) - \bar u_k^\varepsilon(y)| \leqslant L \Vert x - y \Vert < L \delta < \tilde\varepsilon,
    \]
    onde $L = \sup_{x \in \bR^n} \Vert D\bar u_k(x) \Vert$, o qual existe por (\ref{eq:supfinito}) e (\ref{eq:Dufin}) e não depende de $k$ e $x$.

    Por fim, afirmamos que existe uma subsequência $(\bar u_{k_j})_{j=1}^\infty$ de $(\bar u_k)_{k=1}^\infty$, tal que
    \[
        \Vert \bar u_{k_j} - \bar u_{k_\ell} \Vert_{\cL^q(V)} \to 0,
    \]
    quando $j,\ell \to \infty$.
    Com efeito, sabemos que
    \[
        \Vert \bar u_k^\varepsilon - \bar u_k \Vert_{\cL^q(V)} \to 0,
    \]
    uniformemente em $k$, quando $\varepsilon \to 0$ (ver (\ref{eq:idk})).
    Logo, para todo $\delta > 0$, existe $\varepsilon_0 > 0$ suficientemente pequeno tal que
    \begin{equation} \label{eq:cross1}
        \Vert \bar u_k^{\varepsilon_0} - \bar u_k \Vert_{\cL^q(V)} \leqslant \frac{\delta}{3},
    \end{equation}
    para todo $k \in \bN$.
    Por outro lado, $\supp \bar u_k, \supp \bar u_k^{\varepsilon_0} \subseteq V$. Além disso, provamos que $(\bar u_k^{\varepsilon_0})_{k=1}^\infty$ é uniformemente limitada e equicontínua.
    Logo, pelo Teorema de Arzelà-Ascoli (ver Teorema \ref{thm:arzela}), existe uma subsequência $(\bar u_{k_j}^{\varepsilon_0})_{j=1}^\infty$ de $(\bar u_k^{\varepsilon_0})_{k=1}^\infty$ que converge uniformente sobre $V$ (fora de $V$, $\bar u_k$ é nula). Em particular, esta subsequência é de Cauchy sobre $V$.
    Portanto,
    \[
        |\bar u_{k_j}^{\varepsilon_0}(x) - \bar u_{k_\ell}(x)| \to 0,
    \]
    uniformemente em $V$, quando $j,\ell \to \infty$.
    Logo, pelo Teorema da Convergência Dominada (ver Teorema \ref{thm:teorema-da-convergencia-dominada}), temos que
    \[
        \lim_{j,\ell \to \infty} \Vert \bar u_{k_j}^{\varepsilon_0} - \bar u_{k_\ell}^{\varepsilon_0} \Vert_{\cL^q(V)}^q = \lim_{j,\ell \to \infty} \int_V | \bar u_{k_j}^{\varepsilon_0}(x) - \bar u_{k_\ell}^{\varepsilon_0}(x)|^q \,dx = 0,
    \]
    pois $| \bar u_{k_j}^{\varepsilon_0}(x) - \bar u_{k_\ell}^{\varepsilon_0}(x)|^q \leqslant 2 cM \varepsilon_0^{-n} \Vert \eta \Vert_{\cL^1(\bR^n)} \in \cL^1(V)$ (lembre que $V$ é limitado).
    Consequentemente, para $j,\ell > 0$ suficientemente grandes, concluímos que
    \begin{equation} \label{eq:cross2}
        \Vert \bar u_{k_j}^{\varepsilon_0} - \bar u_{k_\ell}^{\varepsilon_0} \Vert_{\cL^q(V)} \leqslant \frac{\delta}{3}.
    \end{equation}
    Dessa forma, por (\ref{eq:cross1}) e (\ref{eq:cross2}), podemos escrever
    \[
        \Vert \bar u_{k_j} - \bar u_{k_\ell} \Vert_{\cL^q(V)} \leqslant \Vert \bar u_{k_j} - \bar u_{k_j}^{\varepsilon_0} \Vert_{\cL^q(V)} + \Vert \bar u_{k_j}^{\varepsilon_0} - \bar u_{k_\ell}^{\varepsilon_0} \Vert_{\cL^q(V)} + \Vert \bar u_{k_\ell}^{\varepsilon_0} - \bar u_{k_\ell} \Vert_{\cL^q(V)} \leqslant \delta,
    \]
    para $j, \ell \in \bN$ suficientemente grandes.
    Isto nos diz que $(\bar u_{k_j})_{j=1}^\infty$ é uma sequência de Cauchy em $\cL^q(\bR^n)$ (pois $\supp \bar u_{k_j} \subseteq V$).
    Dito isso, chegamos a
    \[
        \Vert u_{k_j} - u_{k_\ell} \Vert_{\cL^q(\Omega)} = \Vert \bar u_{k_j} - \bar u_{k_\ell} \Vert_{\cL^q(\Omega)} \leqslant \Vert \bar u_{k_j} - \bar u_{k_\ell} \Vert_{\cL^q(\bR^n)} \to 0,
    \]
    quando $j,\ell \to \infty$.
    Como $\cL^q(\Omega)$ é um espaço completo (ver Teorema \ref{thm:lp-completo-pre}), então $(u_{k_j})_{j=1}^\infty$ é uma sequência convergente em $\cL^q(\Omega)$, como era desejado.
\end{prf}

\obs Vimos no Teorema \ref{thm:compacidade} que $\cW^{1,p}(\Omega) \doublehookrightarrow \cL^q(\Omega)$, para todo $1 \leqslant q < p^*$ (se $\Omega$ é limitado com fronteira de classe $\cC^1$). Como $p^* > p$, então, $\cW^{1,p}(\Omega) \doublehookrightarrow \cL^p(\Omega)$, com $1 \leqslant p < n$ (se $\Omega$ é limitada e $\partial\Omega$ é de classe $\cC^1$).
Na verdade, temos que
\[
    \cW^{1,p}(\Omega) \doublehookrightarrow \cL^p(\Omega),
\]
para todo $1 \leqslant p \leqslant \infty$.
Com efeito, se $n < p \leqslant \infty$, então, pela desigualdade de Morrey (ver Teorema \ref{thm:holdersobolev1}), temos que, dada uma função $u \in \cW^{1,p}(\Omega)$, podemos escrever
\[
    \Vert \bar u \Vert_{\cC^{0,1-\frac{n}{p}}(\bR^n)} \leqslant c \Vert \bar u \Vert_{\cW^{1,p}(\bR^n)},
\]
onde $\bar u \in \cW^{1,p}(\bR^n)$ é uma extensão de $u$. Logo,
\[
    \sup_{x \in \bR^n} |\bar u(x)| + \sup_{\substack{x,y \in \Omega\\x \neq y}} \left\{ \frac{|\bar u(x) - \bar u(y)|}{\Vert x - y \Vert^\gamma} \right\} \leqslant c \Vert \bar u \Vert_{\cW^{1,p}(\bR^n)},
\]
isso implica em
\[
    |\bar u(x)| \leqslant c \Vert u \Vert_{\cW^{1,p}(\bR^n)},
\]
para todo $x \in \bR^n$ e
\[
    |\bar u(x) - \bar u(y)| \leqslant c \Vert x - y \Vert^\gamma \Vert \bar u \Vert_{\cW^{1,p}(\bR^n)},
\]
para todo $x, y \in \bR^n$.
Assim sendo, seja $(u_k) \subseteq \cW^{1,p}(\Omega)$ uma sequência limitada, isto é, $\Vert u_k \Vert_{\cW^{1,p}(\Omega)} \leqslant c$, para todo $k \in \bN$.
Logo,
\[
    \Vert \bar u_k \Vert_{\cW^{1,p}(\bR^n)} \leqslant c \Vert u_k \Vert_{\cW^{1,p}(\Omega)} \leqslant c,
\]
onde $\bar u_k \in \cW^{1,p}(\bR^n)$ é uma extensão de $u_k$, para todo $k \in \bN$.
Assim,
\[
    |\bar u_k(x)| \leqslant c \Vert \bar u_k \Vert_{\cW^{1,p}(\bR^n)} \leqslant c,
\]
para todo $x \in \bR^n$, e
\[
    | \bar u_k(x) - \bar u_k(y)| \leqslant c \Vert x - y \Vert^\gamma \Vert u_k \Vert_{\cW^{1,p}(\bR^n)} \leqslant c \Vert x - y \Vert^{\gamma},
\]
para todo $x,y \in \bR^n$.
Pelo Teorema de Arzelà-Ascoli (ver Teorema \ref{thm:arzela}), existe $(\bar u_{k_\ell})_{\ell=1}^\infty$ subsequência de $(\bar u_k)_{k=1}^\infty$, tal que
\[
    |\bar u_{k_\ell}(x) - v(x)| \to 0,
\]
quando $\ell \to \infty$, para todo $x \in K \subseteq \bR^n$ compacto.
Por outro lado, pelo Teorema da convergência dominada (ver Teorema \ref{thm:teorema-da-convergencia-dominada}), obtemos
\[
    \Vert \bar u_{k_j} - \bar u_{k_\ell} \Vert_{\cL^p(\bR^n)} = \int_{V} |\bar u_{k_j}(x) - \bar u_{k_\ell}(x)|^p \,dx \to 0,
\]
quando $j,\ell \to \infty$.
Como $\cL^p(\bR^n)$ é um espaço completo (ver Teorema \ref{thm:lp-completo-pre}), então existe $v \in \cL^p(\bR^n)$, tal que
\[
    \Vert \bar u_{k_\ell} - v \Vert_{\cL^p(\bR^n)} \to 0,
\]
quando $\ell \to \infty$, para todo $n < p \leqslant \infty$.
Por fim,
\[
    \Vert u_{k_\ell} - v \Vert_{\cL^p(\Omega)} = \Vert \bar u_{k_\ell} - v \Vert_{\cL^p(\Omega)} \leqslant \Vert \bar u_{k_\ell} - v \Vert_{\cL^p(\bR^n)} \to 0.
\]
Isto é,
\[
    \cW^{1,p}(\Omega) \doublehookrightarrow \cL^p(\Omega),
\]
para todo $1 \leqslant p \leqslant \infty$, $p \neq n$, onde $\Omega$ é limitado com fronteira de classe $\cC^1$.

Agora, seja $1 \leqslant s < n$ tal que $1 \leqslant n < s^*$ (isto implica em $s^* \to \infty$ quando $s \to n^-$).
Pelo Teorema \ref{thm:compacidade}
\[
    \cW^{1,s}(\Omega) \doublehookrightarrow \cL^n(\Omega).
\]
Seja $(u_k) \subseteq \cW^{1,n}(\Omega)$ limitada, isto é, $\Vert u_k \Vert_{\cW^{1,n}(\Omega)} \leqslant c$, para todo $k \in \bN$.
Logo,
\[
    \Vert u_k \Vert_{\cW^{1,s}(\Omega)} \leqslant \Vert u_k \Vert_{\cW^{1,n}(\Omega)} \leqslant c,
\]
para todo $k \in \bN$.
Assim sendo, existe uma subsequência $(u_{k_\ell})_{\ell=1}^\infty$ de $(u_k)_{k=1}^\infty$ tal que
\[
    \Vert u_{k_\ell} - u \Vert_{\cL^n(\Omega)} \to 0,
\]
quando $k \to \infty$, para alguma função $u \in \cL^n(\Omega)$.
Consequentemente, $\cW^{1,n}(\Omega) \doublehookrightarrow \cL^n(\Omega)$ e portanto
\[
    \cW^{1,p}(\Omega) \doublehookrightarrow \cL^p(\Omega),
\]
para todo $1 \leqslant p \leqslant \infty$.
\chapter{Algumas aplicações dos Espaços de Sobolev} \label{ch:aplicacoes}

\section{Preliminares}
\subsection{Desigualdades}

O primeiro preliminar para esse capítulo é a Desigualdade de Gagliardo-Nirenberg, iremos apresentar o caso geral abaixo, mas utilizaremos apenas alguns casos particulares que serão mencionados após o teorema.
\begin{tbox}[Desigualdade de Gagliardo-Nirenberg]
    Sejam $1 \leqslant q \leqslant \infty$, $\ell,k \in \bN$, com $\ell < k$, e satisfazendo uma das hipóteses abaixo:
    \[
        \left\{ \begin{aligned}
            &\,r = 1\\
            &\frac{\ell}{k} \leqslant \theta \leqslant 1
        \end{aligned} \right.
        \quad
        \text{ou}
        \quad
        \left\{ \begin{aligned}
            &1 < r < \infty\\[4pt]
            &k - \ell - \frac{n}{r} \in \bN\\
            &\frac{\ell}{k} \leqslant \theta < 1.
        \end{aligned} \right.
    \]
    Além disso, se
    \[
        \frac{1}{p} = \frac{\ell}{n} + \theta \left( \frac{1}{r} - \frac{k}{n} \right) + \frac{1 - \theta}{q},
    \]
    então existe uma constante positiva $c$ que não depende de $u$ tal que
    \[
        \Vert D^\ell u \Vert_{\cL^p(\bR^n)} \leqslant c \Vert D^k u \Vert_{\cL^r(\bR^n)}^\theta \Vert u \Vert_{\cL^q(\bR^n)}^{1-\theta},
    \]
    para toda função $u \in \cL^q(\bR^n) \cap \cW^{k,r}(\bR^n)$.  
\end{tbox}
\begin{prf}
    O artigo \cite{fiorenza-gns} é destinado a demonstração dessa desigualdade.
\end{prf}

A desigualdade de Ladyzhenskaya é um caso particular da desigualdade de Gagliardo-Nirenberg apresentada acima. Mais precisamente, quando $\ell = 0$, $k = 1$, $p = 4$, $q = r = 2$ e $\theta = \frac{3}{4}$, obtemos
\begin{equation} \label{eq:gagliardoL4}
    \Vert u \Vert_{\cL^4(\bR^n)} \leqslant c \Vert u \Vert_{\cL^2(\bR^n)}^{\frac{1}{4}} \Vert Du \Vert_{\cL^2(\bR^n)}^{\frac{3}{4}}.
\end{equation}
Outras formas da desigualdade de Gagliardo-Nirenberg que serão utilizadas são as seguintes:
\begin{equation} \label{eq:gagliardoinf}
    \Vert u \Vert_{\cL^\infty(\bR^3)} \leqslant c \Vert u \Vert_{\cL^2(\bR^3)}^{\frac{1}{4}} \Vert D^2 u \Vert_{\cL^2(\bR^3)}^{\frac{3}{4}},
\end{equation}
onde $\ell = 0$, $k = 2$, $p = \infty$, $q = r = 2$ e $\theta = \frac{3}{4}$, e
\begin{equation} \label{eq:gagliardoDu}
    \Vert Du \Vert_{\cL^2(\bR^3)} \leqslant \Vert u \Vert_{\cL^2(\bR^3)}^{\frac{1}{2}} \Vert D^2 u \Vert_{\cL^2(\bR^3)}^{\frac{1}{2}},
\end{equation}
onde $\ell = 1$, $k = p = q = r = 2$ e $\theta = \frac{1}{2}$.

\subsection{Transformada de Fourier}

A transformada de Fourier é uma ferramenta indispensável para o estudo de Equações Diferenciais Parciais. Aqui ,apresentaremos as definições básicas e algumas propriedades elementares que serão utlizadas ao decorrer do texto.

\begin{dbox}
    Seja $f \in \cL^1(\bR^n)$, definimos a transformada de Fourier de $f$ por
    \[
        \cF[f](\omega) = \hat f(\omega) := \frac{1}{(2\pi)^{\frac{n}{2}}} \int_{\bR^n} e^{-i x \cdot \omega} f(x) \,dx
    \]
    e a transformada inversa de $f$ por
    \[
        \cF^{-1}[f](x) =  \check f(x) = \frac{1}{(2\pi)^{\frac{n}{2}}} \int_{\bR^n} e^{i x \cdot \omega} f(\omega) \,d\omega.
    \]
\end{dbox}

Como $|e^{\pm ix \cdot \omega}| = 1$ e $f \in \cL^1(\bR^n)$, as integrais acima convergem para todo $x \in \bR^n$ (ou $\omega \in \bR^n$ no caso da transformada inversa).

Vejamos agora um resultado que relaciona as normas $\cL^2$ da transformada de Fourier e sua inversa.

\begin{tbox}[Identidade de Planchael] \label{thm:norma-transformada}
    Seja $f \in \cL^1(\bR^n) \cap \cL^2(\bR^n)$, então $\hat f, \check f \in \cL^2(\bR^n)$ e
    \[
        \Vert f \Vert_{\cL^2(\bR^n)} = \Vert \hat f \Vert_{\cL^2(\bR^n)} = \Vert \check f \Vert_{\cL^2(\bR^n)}.
    \]
\end{tbox}
\begin{prf}
    Ver \cite{evans-pde}, p.p. 183.
\end{prf}

O teorema abaixo explora as propriedades elementares da transformada de Fourier.

\begin{tbox}[Propiedades da transformada de Fourier] \label{thm:propriedades-transformada}
    Sejam $u, v \in \cL^2(\bR^n)$. Então, são válidas as seguintes igualdades:
    \begin{enumerate}[leftmargin=*, label=\textbf{(\alph*)}]
        \item $\widehat{\lambda u + v} = \lambda \hat u + \hat v$;
        \item $\left\langle u, v\right\rangle _{\cL^2(\bR^n)} = \left\langle \hat u, \hat v\right\rangle _{\cL^2(\bR^n)}$;
        \item $\widehat{D^\alpha u} = (i\omega)^\alpha \hat u$, para todo multi-índice $\alpha$ tal que $D^\alpha u \in \cL^2(\bR^n);$\footnotemark
        \item $\widehat{u * v} = (2\pi)^{\frac{n}{2}} \hat u \hat v$;
        \item $\widehat{uv} = (2\pi)^{-\frac{n}{2}} (\hat u * \hat v)$.
    \end{enumerate}
\end{tbox}
\begin{prf}
    Ver \cite{evans-pde} p.p. 185.
\end{prf}

\footnotetext{Se $\omega \in \bR^n$ e $\alpha \in \bN^n$ um multi-índice então, $\omega^\alpha = \omega_1^{\alpha_1}\omega_2^{\alpha_2} \cdots \omega_n^{\alpha_n}$.}

Os exemplos abaixo serão úteis para definir um operador que será utilizado extensivamente nesse capítulo.

\begin{ex}[Transformada da derivada temporal]
    Considere uma função $u : \bR^n \times \bR \to \bR$ e sua derivada temporal. Dessa forma, podemos escrever
    \[
        \widehat{\frac{\partial u}{\partial t}} = \frac{1}{(2\pi)^\frac{n}{2}} \int_{\bR^n}  e^{-i \omega \cdot x} \frac{\partial u}{\partial t}(x,t) \,dx = \frac{\partial }{\partial t} \left[ \frac{1}{(2\pi)^{\frac{n}{2}}} \int_{\bR^n} e^{-i \omega \cdot x} u(x,t) \, dx \right] = \frac{\partial \hat u}{\partial t},
    \]
    onde a penúltima igualdade é válida pois $e^{-i \omega \cdot x}$ não depende de $t$. 
\end{ex}

\begin{ex}[Derivada do Laplaciano]
    Lembre que dada uma função $u : \bR^n \to \bR$, o Laplaciano de $u$ é dado por
    \[
        \Delta u = \frac{\partial^2 u}{\partial x_1^2} + \cdots + \frac{\partial^2 u}{\partial x_n^2}.
    \]
    Dito isso, utlizando a transformada da derivada vista no item \textbf{(c)} do Teorema \ref{thm:propriedades-transformada}, podemos escrever
    \[
        \widehat{\frac{\partial^2 u}{\partial x_j^2}} = -\omega_j^2 \hat u.
    \]
    Dessa forma, chegamos a
    \[
        \widehat{\Delta u} = -(\omega_1^2 + \cdots + \omega_n^2) \hat u = - \Vert \omega \Vert^2 \hat u.
    \]
\end{ex}

\subsection{Semigrupo do calor}

Agora, introduziremos o conceito de semigrupo, que será utlizado para explorar propriedades das soluções de Leray para as equações de Navier-Stokes.

\begin{dbox}
    Seja $X$ um espaço de Banach. Uma família de operadores lineares limitados $(T(t))_{t \geqslant 0}$, onde $T : [0,\infty) \to X$, é um semigrupo se
    \begin{enumerate}
        \item $T(0) = I_X$, onde $I_X : X \to X$ é o operador identidade;
        \item $T(t + s) = T(t)T(s)$,  para todo $t, s \geqslant 0$.
    \end{enumerate}
\end{dbox}

Nesse trabalho, será importante conhecer o semigrupo do calor, o qual provém da solução da equação do calor
\begin{align}
    \bu_t - \nu \Delta\bu = 0,&\;\text{ em } \bR^n \times (0,\infty); \label{eq:calor}\\
    \bu = \bv, &\;\text{ em } \bR^n \times \{0\} \label{eq:calor-inicial}.
\end{align} 
Para encontrar a solução do sistema acima, utlizaremos a transformada de Fourier.
Dito isso, aplicando a transformada de Fourier em (\ref{eq:calor}) e (\ref{eq:calor-inicial}), obtemos
\[
    \left\{
        \begin{aligned}
        \hat\bu_t - \nu \Vert \omega \Vert^2 \bu = 0, &\quad t > 0;\\
        \hat\bu = \hat\bv, &\quad t = 0.
        \end{aligned}
    \right.
\]
Nesse caso, temporariamente ignoramos as variáveis espaciais e trabalhamos apenas no domínino do tempo, sendo assim basta resolver a Equação Diferencial Ordinária (EDO) acima e depois aplicar a transformada de Fourier inversa.
Com efeito, esta EDO tem solução dada por
\[
    \hat\bu = \hat\bv e^{-\nu t \Vert \omega \Vert^2}.
\]
Dessa forma podemos aplicar a transformada de Fourier inversa para obter
\begin{equation} \label{eq:ls}
    \bu = \frac{\bv * F}{(2\pi)^{\frac{n}{2}}},
\end{equation}
onde $F$ é a transformada de Fourier inversa de $e^{-\nu t \Vert \omega \Vert^2}$. Ou seja,
\begin{equation} \label{eq:ssss}
    F(x,t) = \frac{1}{(2\pi)^\frac{n}{2}}\int_{\bR^n} e^{i \omega \cdot x} e^{-\nu t \Vert \omega \Vert^2} \,d\omega =  \frac{1}{(2\pi)^\frac{n}{2}}\prod_{k=1}^n \int_{\m\infty}^{\infty} e^{i \omega_k x_k - \nu t\omega_k^2} \,d\omega_k.
\end{equation}
Para resolver essa integral, primeiramente precisamos completar o quadrado no expoente.
Deste modo, podemos escrever
\[
    \begin{aligned}
        \int_{\m\infty}^{\infty} e^{i \omega_k x_k - \nu t\omega_k^2} \,d\omega_k &= \int_{\m\infty}^{\infty} \exp \left(-\nu t\omega_k^2 + i x_k\omega_k + \frac{x_k^2}{4\nu t} - \frac{x_k^2}{4\nu t}\right) \, d\omega_k\\
        &= \int_{\m\infty}^\infty \exp\left( -\frac{x_k^2}{4\nu t} \right) \exp \left( - \left( \sqrt{\nu t}\omega_k - \frac{ix_k}{2\sqrt{\nu\tau}} \right)^2 \right) \,d\omega_k.
    \end{aligned}
\] 
Fazendo a substituição $s_k = \sqrt{\nu t}\omega_k - \frac{ix_k}{2\nu t}$, obtemos $ds_k = \sqrt{\nu t} \,d\omega_k$ e, consequentemente,
\begin{equation} \label{eq:SSSSS}
    \int_{\m\infty}^{\infty} e^{i \omega_k x_k - \nu t\omega_k^2} \,d\omega_k = \frac{1}{(\nu t)^{\frac{1}{2}}} e^{\frac{-x_k^2}{4\nu t}} \int_{\m\infty}^\infty e^{-s_k^2} \,ds_k = \left( \frac{\pi}{\nu t} \right)^{\frac{1}{2}} e^{-\frac{x_k^2}{4\nu t}}.
\end{equation}
Sendo assim, por (\ref{eq:ssss}) e (\ref{eq:SSSSS}), chegamos a
\[
    F(x,t)  =  \frac{1}{(2\pi)^\frac{n}{2}}\prod_{k=1}^n \left( \frac{\pi}{\nu t} \right)^{\frac{1}{2}} e^{-\frac{x_k^2}{4\nu t}} = \frac{1}{(2\nu t)^\frac{n}{2}} e^{^{-\frac{\Vert x \Vert^2}{4\nu t}}}.
\]
Portanto, por (\ref{eq:ls}), concluímos que
\[
    \bu (x,t) = \frac{1}{(4\pi\nu t)^{\frac{n}{2}}} \int_{\bR^n} e^{-\frac{\Vert x - y \Vert^2}{4 \nu t}} \bv(y) \,dy.
\]

O semigrupo do calor, denotado por $e^{\nu \Delta \tau}$, com $\tau \geqslant 0$, é uma família de operadores dada por
\[
    e^{\nu \Delta \tau} \bv = \frac{\bv * E(\cdot,\tau)}{(4\pi \nu \tau)^{\frac{n}{2}}},
\]
onde $E(x,\tau) = e^{-\frac{\Vert x \Vert^2}{4 \nu \tau}}$.

O restante dessa subseção será destinado a estudar algumas propriedades que serão utlizadas neste trabalho.

\begin{pbox}[Priopriedades do semigrupo do calor] \label{thm:propriedades-semi-grupo-calor}
    Considere o semigrupo do calor $(e^{\nu\Delta\tau})_{\tau \geqslant 0}$, então são válidas as seguintes igualdades:
    \begin{enumerate}[leftmargin=*, label=\textbf{(\alph*)}]
        \item $e^{\nu\Delta\tau}(\lambda u + v) = \lambda e^{\nu\Delta\tau} u + e^{\nu\Delta\tau} v$;
        \item $D^\alpha (e^{\nu\Delta\tau} u) = e^{\nu\Delta\tau} (D^\alpha u)$;
        \item $\widehat{e^{\nu\Delta \tau} u} = \hat u e^{-\nu \tau \Vert \omega \Vert^2}$.
        \item $e^{\nu\Delta \tau}$ é um operador limitado na norma $\Vert  \cdot\Vert_{\cL^2(\bR^3)}$, para todo $\tau \geqslant 0$;
    \end{enumerate}
\end{pbox} 
\begin{prf}~

    \textbf{(a)} Segue do fato da convolução ser um operador linear.

    \textbf{(b)} Análogo ao que foi mostrado no Teorema \ref{thm:aprox1}.

    \textbf{(c)} Segue do Teorema \ref{thm:propriedades-transformada} \textbf{(d)}, já que
    \[
        e^{\nu \Delta \tau} u = \frac{u * E(\cdot,\tau)}{(4\pi \nu \tau)^{\frac{n}{2}}},
    \]
    implica em
    \begin{equation} \label{eq:euhat}
        \widehat{e^{\nu \Delta \tau} u} = \frac{(2\pi)^{\frac{n}{2}} \hat u \hat E(\cdot,\tau)}{(4\pi\nu\tau)^{\frac{n}{2}}}.
    \end{equation}
    Sendo assim, resta calcular a transformada de Fourier de $E(x,\tau) = e^{-\frac{\Vert x \Vert^2}{4\nu\tau}}$.
    Com efeito, denotando $\frac{1}{4\nu\tau}$ por $c$, obtemos
    \begin{equation} \label{eq:Ehat}
        \hat{E}(\omega,\tau) = \frac{1}{(2\pi)^{\frac{n}{2}}}\int_{\bR^n} e^{-i\omega\cdot x} e^{-\frac{\Vert x \Vert^2}{4\nu\tau}} \,dx = \frac{1}{(2\pi)^{\frac{n}{2}}} \prod_{k=1}^n \int_{\m\infty}^\infty e^{-i\omega_k x_k - cx_k^2} \,dx_k.
    \end{equation}
    De forma análoga ao que foi feito quando resolvemos a equação do calor, podemos completar o quadrado no expoente. Dessa forma, concluímos que
    \[
        \int_{\m\infty}^\infty e^{-i\omega_k x_k - cx_k^2} \,dx_k = e^{-\frac{\omega_k^2}{4c}}\int_{\m\infty}^\infty e^{-c\left( x_k + \frac{i\omega_k}{2c} \right)^2} \,dx_k. 
    \]
    Fazendo a substituição $s_k = x_k + \frac{i\omega_k}{2c}$, obtemos $ds_k = dx_k$. Logo, podemos escrever
    \[
        \int_{\m\infty}^\infty e^{-i\omega_k x_k - cx_k^2} \,dx_k = e^{-\frac{\omega_k^2}{4c}}\int_{\m\infty}^\infty e^{-cs_k^2} \,ds_k = \left( \frac{\pi}{c} \right)^{\frac{1}{2}} e^{-\frac{\omega_k^2}{4c}}. 
    \]
    Sendo assim, por (\ref{eq:Ehat}) e lembrando que $c = \frac{1}{4\nu\tau}$, segue que
    \[
        \hat E(\omega,\tau) = \frac{1}{(2\pi)^{\frac{n}{2}}} \prod_{k=1}^n (4\pi \nu \tau)^{\frac{1}{2}} e^{-\nu \tau \omega_k^2} = \frac{(4\pi\nu\tau)^{\frac{n}{2}}}{(2\pi)^{\frac{n}{2}}} e^{-\nu\tau \Vert \omega \Vert^2},
    \]
    e, por (\ref{eq:euhat}), concluímos que
    \[
        \widehat{e^{\nu\Delta \tau} u} = \hat u e^{-\nu\tau \Vert \omega \Vert^2}.
    \]

    \textbf{(d)} Basta utilizar a desigualdade de Young para convoluções (ver Teorema \ref{thm:desigualde-de-young-para-convolucoes}), já que
    \[
        \Vert e^{\nu\Delta\tau} u \Vert_{\cL^2(\bR^3)} = c \Vert u * E(\cdot,\tau) \Vert_{\cL^2(\bR^3)} \leqslant c \Vert u \Vert_{\cL^2(\bR^3)} \Vert E(\cdot,\tau) \Vert_{\cL^1(\bR^3)}.
    \]
    Isto prova que $e^{\nu\Delta\tau}$ é um operador limitado em $\cL^2$ para todo $\tau \geqslant 0$, pois $E(\cdot,\tau)$ é integrável em $\cL^1(\bR^3)$.
\end{prf}

O próximo resultado mostra uma estimativa da norma $\cL^2$ das derivadas parciais do semigrupo do calor.

\begin{pbox}
    Sejam $1 \leqslant r \leqslant 2$ e $u \in \cL^2(\bR^n)$, com $n \in \bN$, então
    \begin{equation} \label{eq:estimativa-util}
        \Vert D^{\alpha} \big( e^{\nu\Delta\tau} u \big) \Vert_{\cL^2(\bR^n)} \leqslant c(n,k) (\nu\tau)^{-\frac{n}{2}\left( \frac{1}{r} - \frac{1}{2} \right) - \frac{k}{2}} \Vert u \Vert_{\cL^r(\bR^n)},
    \end{equation} 
    onde $k = |\alpha|$.
\end{pbox}
\begin{prf}
    Ver \cite{lorenz-navier.stokes}, p.p. 32.
\end{prf}

\subsection{Notação}

Introduziremos a notação que será utliizada ao decorrer deste capítulo.   

Letras em negrito representam vetores $n$-dimensionais $\bu = (u_1,\dots,u_n)$ (na maioria dos casos $n = 3$),
a $k$-ésima derivada parcial é denotada por $D_k$ (ou $D^{e_k}$), enquanto $D^k$ representa o gradiente $k$-dimensional.
Também vale ressaltar a definição da norma $\cL^p(\bR^n)$ das funções vetoriais:
\[
    \Vert \bu \Vert_{\cL^p(\bR^n)} = \left( \sum_{i=1}^n \Vert u_i \Vert_{\cL^p(\bR^n)}^p \right)^{\frac{1}{p}},
\]
\[
    \Vert D\bu \Vert_{\cL^p(\bR^n)} = \left( \sum_{j=1}^n\sum_{i=1}^n \Vert D_ju_i \Vert_{\cL^p(\bR^n)}^p \right)^{\frac{1}{p}}
\]
e mais geralmente,
\[
    \Vert D^k\bu \Vert_{\cL^p(\bR^n)} = \left( \sum_{j_1 = 1}^n \cdots \sum_{j_k  =1}^{n}\sum_{i=1}^n \Vert D_{j_1} \cdots D_{j_k}u_i \Vert_{\cL^p(\bR^n)}^p \right)^{\frac{1}{p}}.
\]

\section{Decaimentos das soluções de Leray}

\subsection{Introdução e contexto histórico} \label{sec:intro}

\begin{figure}
    \centering  
    \includegraphics[height=3cm]{leray.jpg}
    \caption{Jean Leray (1906 -- 1998)}
\end{figure}

Em 1934, no artigo \textit{``Sur le mouvement d'un liquide visqueux emplassement l'espace''} (ver \cite{leray-fluid}) Leray construiu soluções fracas de energia finita\footnote{i.e., $\Vert \bu(\cdot,t) \Vert_{\cL^2(\bR^3)} < \infty$.}
\begin{equation} \label{eq:solucao-leray}
    \bu(\cdot,t) \in \cL^\infty\big([0,\infty), \cL^2_\sigma(\bR^3)\big) \cap \cC_w\left([0,\infty), \cL^2(\bR^3)\right) \cap \cL^2\big([0,\infty), \dot H^1(\bR^3)\big)\footnotemark
\end{equation}
para as seguintes equações de Navier-Stokes em $\bR^3$: 
\footnotetext{
    As definições desses espaços são dadas abaixo:
    \begin{itemize}[label=$\cdot$]
        \item $L^\infty\big([0,\infty), L^2_\sigma(\bR^3)\big)$ é o espaço das funções $\bu(\cdot,t) : [0,\infty) \to L^2(\bR^3)$ tal que $\nabla \cdot \bu =0$ e $\Vert \bu(\cdot,t) \Vert_{L^2(\bR^3)} < \infty$.
        \item $\cC_w\left([0,\infty), L^2(\bR^3)\right)$ é o espaço das funções $\bu(\cdot,t) : [0,\infty) \to L^2(\bR^3)$ fracamente contínuas.
        \item $L^2\big([0,\infty), \dot H^1(\bR^3)\big)$ é o espaço das funções $\bu(\cdot,t) : [0,\infty) \to \dot H^1(\bR^3)$ tal que
        \[
            \int_0^\infty \Vert \bu(\cdot,t) \Vert_{\dot H^1(\bR^3)}^2 \,dt < \infty.
        \]
    \end{itemize}
    \vspace{3pt}
}
\begin{equation} \label{eq:navierstokes}
    \left\{
        \begin{aligned}
        &\bu_t + \bu \cdot \nabla \bu + \nabla p = \nu \Delta\bu;\\
        &\nabla\cdot\bu = 0;\\
        &\bu(\cdot,0) = \bu_0 \in \cL^2_\sigma(\bR^3),
        \end{aligned}
    \right.
\end{equation}
onde $\nu > 0$ é constante e $\nabla \cdot \bu = \sum_{j=1}^3 \frac{\partial u_j}{\partial x_j}$.
Estas soluções são tais que $\Vert \bu(\cdot,t) - \bu_0 \Vert_{\cL^2(\bR^3)} \to 0$, quando $t \to 0^+$, e satisfazem a desigualdade de energia abaixo:
\begin{equation} \label{eq:desigualdade-de-energia}
    \Vert \bu(\cdot,t) \Vert_{\cL^2(\bR^3)}^2 \leqslant \Vert \bu(\cdot,t) \Vert_{L^2(\bR^3)}^2 + 2\nu \int_0^t \Vert D\bu(\cdot,s) \Vert_{\cL^2(\bR^3)}^2 \,ds \leqslant \Vert \bu_0 \Vert_{\cL^2(\bR^3)}^2,
\end{equation}
para todo $t > 0$.
A unicidade desas soluções ainda é um problema em aberto; porém, no mesmo artigo Leray mostrou que existe um instante de tempo $T_{**}$ tal que a solução $\bu$ se torna suave em $\bR^3 \times [T_{**}, \infty)$ e $\bu(\cdot,t) \in \cL^{\infty}_{\loc}\big( [T_{**}, \infty), H^k(\bR^3) \big)$\footnote{i.e., $\Vert \bu(\cdot,t) \Vert_{H^k(K)} < \infty$, onde $K \subseteq [T_{**},\infty)$ é compacto.} para cada $k \geqslant 0$.
Um problema importante que foi deixado em aberto por Leray no final de seu artigo diz respeito ao decaimento de energia em $L^2$ da solução de (\ref{eq:navierstokes}). Matematicamente, isto significa entender o que acontece com $\Vert \bu(\cdot,t) \Vert_{\cL^2(\bR^3)}$, quando $t \to \infty$. Leray suspeitava que
\[
    \Vert \bu(\cdot,t) \Vert_{\cL^2(\bR^3)} \to 0,
\]
quando $t \to \infty$. Uma demonstração para esse fato será apresentada no Teorema \ref{thm:problema-leray}.

Uma outra forma de estudar propriedades das soluções das equações de Navier-Stokes (\ref{eq:navierstokes}) é a partir das soluções $\bv(\cdot,t)$ do problema linearizado
\begin{equation} \label{eq:navier-stokes-linearizado}
    \left\{
        \begin{aligned}
        &\bv_t = \nu \Delta \bv;\\
        &\bv(\cdot,t_0) = \bu(\cdot,t_0),
    \end{aligned}
    \right.
\end{equation}
com $t \geqslant t_0 \geqslant 0$. Aqui, 
\[
    \bv(\cdot,t) = e^{\nu \Delta (t-t_0)} \bu(\cdot,t_0),
\]
onde $e^{\nu \Delta (t-t_0)}$ é o semigrupo do calor visto nos preliminares.
Com essas soluções, é possível estudar algumas estimativas de decaimento como
\[
    \Vert \bv(\cdot,t) \Vert_{\cL^2(\bR^n)} \to 0,
\]
\[
    t^{\frac{n}{4}}\Vert \bv(\cdot,t) \Vert_{\cL^\infty(\bR^n)} \to 0,
\]
quando $t \to \infty$
Uma outra pergunta importante (que não será trabalhada aqui) é sobre o erro ou diferença da solução de Leray e da solução do problema linearizado.
Essa pergunta foi respondida por Weigner (em \cite{wiegner-decay}), onde foi provado que
\[
    t^{\frac{n}{4} - \frac{1}{2}} \Vert \bu(\cdot,t) - e^{\nu \Delta (t-t_0)} \bu(\cdot,t_0) \Vert_{\cL^2(\bR^3)} \to 0,
\]
quando $t \to \infty$.

\subsection{Resultados auxiliares}

Tomando uma função molificadora $\eta \in \cC^{\infty}_c(\bR^3)$ e sua versão escalada $\eta_\delta$ (vista no Capítulo \ref{ch:sobolev}, Seção \ref{sec:aproximacoes}) definimos $\bar\bu_{0,\delta} = \eta_\delta * \bu_0$, introduzimos $\bu_\delta, p_\delta \in \cC^{\infty}\big( \bR^3 \times [0,\infty) \big)$ como a solução única do problema regularizado
\begin{equation} \label{eq:navier-stokes-regularizado}
    \left\{
        \begin{aligned}
        &\partial_t \bu_\delta + \bar \bu_\delta \cdot \nabla \bu_\delta  + \nabla p_\delta= \nu \Delta \bu_\delta;\\
        &\bu_\delta(\cdot,0) = \bar \bu_{0,\delta},
        \end{aligned}
    \right.
\end{equation}
onde $\bar\bu_\delta = \eta_\delta * \bu_\delta$. Em seu artigo, Leray mostrou que existe uma sequência apropriada $\delta' \to 0$ tal que conseguimos seguinte a convergência fraca\footnote{Dado um espaço de Banach $X$, dizemos que uma sequência $(x_n) \subseteq X$ converge fracamente para $x \in X$ ($x_n \rightharpoonup x$) se $f(x_n) \to f(x)$, para todo $f \in X'$, onde $X'$ é o espaço de todos funcionais lineares limitados de $X$ em $\bR$.} em $\cL^2(\bR^3)$:
\begin{equation} \label{eq:b1}
    \bu_{\delta'} \rightharpoonup \bu,
\end{equation}
para todo $t \geqslant 0$, onde $\bu(\cdot,t)$ apresentada em (\ref{eq:solucao-leray}) é contínua no instante $t = 0$.
Além disso, a desigualdade de energia (\ref{eq:desigualdade-de-energia}) é satisfeita para todo $t \geqslant 0$ e, em particular,
\begin{equation} \label{eq:2.3}
    \int_0^\infty \Vert D\bu_\delta(\cdot,t) \Vert_{\cL^2(\bR^3)}^2 \,dt \leqslant \frac{1}{2\nu} \Vert \bu_{0} \Vert_{\cL^2(\bR^3)}.
\end{equation}
Outros resultados importantes se referem à projeção de Helmholtz\footnote{A projeção de Helmholtz é uma forma de escrever um campo vetorial $F$ como $F = G + H$, onde $G, H$ são campos vetoriais tais que $\nabla \cdot G = 0$ e $H = \nabla \Phi$ para alguma função $\Phi : \bR^n \to \bR$.} de $-\bu(\cdot,t) \cdot \nabla \bu(\cdot,t)$ em $\cL^2_\sigma(\bR^3)$ isto é, o campo $\BQ(\cdot,t) \in \cL^2_\sigma(\bR^3)$ dado por
\begin{equation} \label{eq:defQ}
    \BQ(\cdot,t) := -\bu(\cdot,t) \cdot \nabla \bu(\cdot,t) - \nabla p(\cdot,t),
\end{equation}
para todo $t > 0$.
A seguir, estudaremos algumas estimativas para $\BQ(\cdot,t)$.

\begin{pbox} \label{pr:nsei}
    Para quase todo $s > 0$ (e para todo $s \geqslant T_{**}$), tem-se
    \[
        \Vert e^{\nu\Delta(t-s)}\BQ(\cdot,s) \Vert_{\cL^2(\bR^3)} \leqslant c \nu^{-\frac{3}{4}} (t-s)^{-\frac{3}{4}} \Vert \bu(\cdot,s) \Vert_{\cL^2(\bR^3)} \Vert D\bu(\cdot,s) \Vert_{\cL^2(\bR^3)},
    \]
    para todo $t > s$, onde $c$ é uma constante positiva.
\end{pbox}
\begin{prf}
    Seja $\hat f = \cF[f]$ a transformada de Fourier de uma função $f \in \cL^1(\bR^3)$, dada por
    \[
        \hat f (\omega) = (2\pi)^{-\frac{3}{2}} \int_{\bR^3} e^{-i\omega \cdot x} f(x) \, dx.
    \]
    Dada $\bv(\cdot,s) \in \cL^1 (\bR^3) \cap \cL^2(\bR^3)$ arbitrária, obtemos, pelo Teorema \ref{thm:norma-transformada}, que
    \[
        \Vert e^{\nu\Delta(t-s)} \bv(\cdot,s) \Vert_{\cL^2(\bR^3)}^2 = \Vert \cF [e^{\nu\Delta(t-s)} \bv(\cdot,s)] \Vert_{\cL^2(\bR^3)}^2 = \int_{\bR^3} e^{-2\nu \Vert \omega \Vert^2 (t-s)}\Vert \hat\bv(\omega,s) \Vert^2 \,d\omega,
    \]
    onde utlizamos o resultado sobre a transformada de Fourier do semigrupo do calor (ver Teorema \ref{thm:propriedades-semi-grupo-calor} \textbf{(c)}). Além disso, $\Vert \hat \bv(\omega,s) \Vert^2 \leqslant \Vert \hat\bv(\cdot,s) \Vert_\infty = \sup\{ \Vert \hat\bv(\omega,s) \Vert \,; \omega \in \bR^3 \}$. Logo,
    \[
        \Vert e^{\nu\Delta(t-s)} \bv(\cdot,s) \Vert_{\cL^2(\bR^3)}^2 \leqslant \Vert \hat\bv(\cdot,s) \Vert_\infty \int_{\bR^3} e^{-2\nu (t-s) \Vert \omega \Vert^2} d\omega,
    \]
    onde a integral do lado direito é uma Gaussiana, cujo resultado é $c \nu^{-\frac{3}{2}} (t - s)^{-\frac{3}{2}}$. Portanto, podemos escrever
    \begin{equation} \label{eq:2.7}
        \Vert e^{\nu\Delta(t-s)} \bv(\cdot,s) \Vert_{\cL^2(\bR^3)}\leqslant c \nu^{-\frac{3}{4}} (t-s)^{-\frac{3}{4}} \Vert \hat\bv(\cdot,s) \Vert_\infty.
    \end{equation}
    O resultado que queremos, é uma aplicação direta de (\ref{eq:2.7}) com $\bv(\cdot,s) = \BQ(\cdot,s)$, o restante da demonstração será dedicado à estimativa do valor de $\Vert \hat\BQ(\cdot,s) \Vert_\infty$. 
    
    Note que utilizando a derivada da transformada de Fourier, temos que $\cF[D^\alpha f] = (i\omega)^\alpha \hat f$ (ver Teorema \ref{thm:propriedades-transformada}), sendo assim se $\alpha = e_j$, para algum $j = 1,2,3$, $\cF [D^{e_j} f] = i\omega_j \hat f$. Dito isso,
    \[
        \cF[\nabla p(\cdot,s)] = i \hat p(\omega,s) \omega,
    \] 
    e $\omega \cdot \hat\BQ = 0$, pois, por (\ref{eq:defQ}) e (\ref{eq:navierstokes}), $\BQ = \bu_t - \nu \Delta \bu$ e 
    \[
        \nabla \cdot \BQ = (\nabla \cdot \bu)_t - \nu \Delta (\nabla \cdot \bu) = 0,
    \]
    já que $\nabla \cdot \bu = 0$. Dessa forma,
    \[
        0 = \widehat{\nabla \cdot \BQ} = \sum_{j=1}^3 \widehat{\frac{\partial Q_j}{\partial x_j}} = \sum_{j=1}^3 i \omega_j \hat Q_j,
    \] 
    ou seja, $\omega \cdot \hat\BQ = 0$. Além disso, pela definição de $\BQ(\cdot,s)$ (ver (\ref{eq:defQ})), temos que
    \[
        \hat\BQ(\omega,s) + \cF[\nabla p(\cdot,s)](\omega) = -\cF[\bu(\cdot,s) \cdot \nabla\bu(\cdot,s)](\omega).
    \]
    Com isso, fazendo o produto interno por $\hat\BQ$ em ambos os lados da igualdade acima, obtemos
    \[
        \hat\BQ(\omega,s) \cdot \hat\BQ(\omega,s) + i \hat p (\omega,s) \omega \cdot \hat\BQ(\omega,s) =  -\cF[\bu(\cdot,s) \cdot \nabla\bu(\cdot,s)](\omega) \cdot \hat\BQ(\omega,s)
    \]
    Deste modo, pela desigualde de Cauchy-Schwarz, podemos escrever,
    \[
        \Vert \hat\BQ(\omega,s) \Vert^2 \leqslant \Vert \cF [\bu(\cdot,s) \cdot \nabla \bu(\cdot,s)] \Vert \Vert \hat\BQ \Vert.
    \]
    Ou seja,
    \[
        \Vert \hat\BQ(\omega,s) \Vert \leqslant \Vert \cF[\bu(\cdot,s) \cdot \nabla\bu(\cdot,s)](\omega) \Vert.
    \]
    Isso nos diz que
    \begin{equation} \label{eq:2.9}
        \Vert \hat\BQ (\cdot,s) \Vert_\infty \leqslant \Vert \cF [\bu \cdot \nabla \bu](\cdot,s) \Vert_\infty.
    \end{equation}
    Por outro lado, para $i = 1,2,3$, é verdade que
    \[
        | \cF[\bu (\cdot,s) \cdot \nabla u_i(\cdot,s)](\omega)| \leqslant \sum_{j=1}^{3} | \cF[u_j (\cdot,s) D_j\, u_i(\cdot,s)](\omega)| \leqslant (2\pi)^{-\frac{3}{2}} \sum_{j=1}^3 \Vert u_j(\cdot,s) D_j\,u_i(\cdot,s) \Vert_{\cL^1(\bR^3)},
    \]
    para todo $\omega \in \bR^3$.
    Por fim, novamente utilizando a Desigualdade de Hölder (Cauchy-Schwarz) e a definição das normas, chegamos a
    \[
        | \cF[\bu (\cdot,s) \cdot \nabla u_j(\cdot,s)](\omega)| \leqslant c \Vert \bu(\cdot,s) \Vert_{\cL^2(\bR^3)} \Vert \nabla u_j(\cdot,s) \Vert_{\cL^2(\bR^3)},
    \]
    para todo $\omega \in \bR^3$.
    Isto mostra que
    \begin{equation} \label{eq:2.10}
        \Vert \cF[\bu\cdot \nabla \bu] (\cdot,s) \Vert_\infty \leqslant c\Vert \bu(\cdot,s) \Vert_{\cL^2(\bR^3)} \Vert D \bu(\cdot,s) \Vert_{\cL^2(\bR^3)}.
    \end{equation}
    Dito isso, por (\ref{eq:2.7}), (\ref{eq:2.9}) e (\ref{eq:2.10}), inferimos que
    \[
        \Vert e^{\nu\Delta(t-s)}\BQ(\cdot,s) \Vert_{\cL^2(\bR^3)} \leqslant c \nu^{-\frac{3}{4}} (t-s)^{-\frac{3}{4}} \Vert \bu(\cdot,s) \Vert_{\cL^2(\bR^3)} \Vert D\bu(\cdot,s) \Vert_{\cL^2(\bR^3)},
    \]
    como queriamos mostrar.
\end{prf}

Repetindo o argumento utilizado na demonstração acima para as soluções do problema regularizado (\ref{eq:navier-stokes-regularizado}), obtemos
\begin{equation} \label{eq:2.11}
    \Vert e^{\nu\Delta(t-s)} \BQ_\delta \Vert_{\cL^2(\bR^3)} \leqslant c \nu^{-\frac{3}{4}}(t - s)^{-\frac{3}{4}} \Vert \bu_\delta(\cdot,s) \Vert_{\cL^2(\bR^3)} \Vert D\bu_\delta(\cdot,s) \Vert_{\cL^2(\bR^3)},
\end{equation}
onde $c$ é uma constante positíva e
\[
    \BQ_\delta(\cdot,s) = -\bar\bu_\delta(\cdot,s) \cdot \nabla \bu_\delta(\cdot,s) - \nabla p_\delta(\cdot,s).
\]

\obs Vale ressaltar que as soluções do problema regularizado também satisfazem a seguinte desigualdade de energia:
\begin{equation} \label{eq:desigualdade-de-energia-regularizado}
    \Vert \bu_\delta(\cdot,t) \Vert_{\cL^2(\bR^3)}^2 + 2\nu \int_0^t \Vert D\bu_\delta(\cdot,s) \Vert^2_{\cL^2(\bR^3)} \,ds \leqslant \Vert \bu_0 \Vert_{\cL^2(\bR^3)}^2,
\end{equation}
para todo $t > 0$.

O resultado a seguir é uma generalização da Proprosição \ref{pr:nsei}.

\begin{pbox} \label{pr:DaQ}
    Para quase todo $s > 0$ (e todo $s \geqslant T_{**}$), tem-se
    \[
        \Vert D^{\alpha} \big( e^{\nu\Delta(t-s)}\BQ(\cdot,s)\big) \Vert_{\cL^2(\bR^3)} \leqslant c(k) \nu^{-\left( \frac{k}{2} + \frac{3}{4} \right)} (t - s)^{-\left( \frac{k}{2} + \frac{3}{4}\right)} \Vert \bu(\cdot,s) \Vert_{\cL^2(\bR^3)} \Vert D\bu(\cdot,s) \Vert_{\cL^2(\bR^3)},
    \]
    para todo $t > s$, onde $k = |\alpha|$ e $c(k)$ depende apenas de $k$.
\end{pbox}
\begin{prf}
    Por (\ref{eq:estimativa-util}), obtemos
    \[
        \begin{aligned}
            \Vert D^\alpha [e^{\nu \Delta (t - s)} \BQ(\cdot,s)] \Vert_{\cL^2(\bR^3)} &= \Vert D^\alpha \big[e^{\nu \Delta (t - s)/2} [ e^{\nu \Delta (t - s)/2} \BQ(\cdot,s) ] \big] \Vert_{\cL^2(\bR^3)}\\ 
            &\leqslant c(k) \nu^{-\frac{k}{2}} (t - s)^{-\frac{k}{2}} \Vert e^{\nu\Delta(t-s)/2} \BQ(\cdot,s) \Vert_{\cL^2(\bR^3)},
        \end{aligned}
    \]
    e, pela Proposição \ref{pr:nsei}, podemos escrever
    \[
        \Vert D^\alpha [e^{\nu \Delta (t - s)/2} \BQ(\cdot,s)] \Vert_{\cL^2(\bR^3)} \leqslant c(k) \nu^{-\left( \frac{k}{2} + \frac{3}{4} \right)} (t - s)^{-\left( \frac{k}{2} + \frac{3}{4} \right)} \Vert \bu(\cdot,s) \Vert_{\cL^2(\bR^3)} \Vert D\bu(\cdot,s) \Vert_{\cL^2(\bR^3)}.
    \]
\end{prf}

A seguir, mostremos que o gradiente da solução de Leray se torna decrescente a partir de um instante de tempo.

\begin{pbox} \label{pr:tstar}
    Seja $\bu(\cdot,t)$ uma solução de Leray para (\ref{eq:navierstokes}). Então existe $t_{**} > T_{**}$ (com $t_{**}$ dependendo da solução $\bu$) suficientemente grande tal que $\Vert D\bu(\cdot,t) \Vert_{\cL^2(\bR^3)}$ é uma funçao monotonicamente decrescente de $t$ no intervalo $[t_{**}, \infty)$. (rever)
\end{pbox}
\begin{prf}
    Sejam $t_0 \geqslant T_{**}$ e $t > t_0$.
    Aplicando a $k$-ésima derivada parcial $D_k$ à primeira equação de (\ref{eq:navierstokes}), realizando o produto escalar com $D_k \bu$, integrando em $\bR^3 \times [t_0,t]$, obtemos
    \begin{equation} \label{eq:A}
        \begin{aligned}
            \int_{\bR^3} \int_{t_0}^t D_k \bu_s \cdot D_k \bu \,dsdx &+ \int_{\bR^3} \int_{t_0}^t D_k (\bu \cdot \nabla \bu) \cdot D_k \bu \,dsdx\\ 
            &+ \int_{\bR^3} \int_{t_0}^t D_k (\nabla p) \cdot D_k\bu \,dsdx = \nu \int_{\bR^3} \int_{t_0}^t D_k (\Delta \bu) \cdot D_k \bu \,dsdx
        \end{aligned}
    \end{equation}
    Vamos analisar cada integral separadamente.
    Note que
    \[
        D_k \bu_t \cdot D_k \bu = \frac{1}{2} \frac{d}{dt} \Vert D_k \bu \Vert^2.
    \]
    Dessa forma 
    \[
        \int_{t_0}^t D_k \bu_s \cdot D_k \bu \,dsdx = \int_{t_0}^t \frac{1}{2} \frac{d}{ds} \Vert D_k \bu(\cdot,s) \Vert^2 \,dx =  \frac{1}{2} \big[ \Vert D_k \bu(\cdot,t) \Vert^2 -  \Vert D_k \bu(\cdot,t_0) \Vert^2 \big] .
    \]
    Além disso, a segunda integral em (\ref{eq:A}) pode ser estudada da seguinte forma
    \begin{align}
            \int_{\bR^3} D_k (\bu \cdot \nabla \bu) \cdot D_k \bu \,dx &= \sum_{i,j=1}^3 \int_{\bR^3} D_k (u_i D_i u_i) D_k u_j \,dx \label{eq:xx1}\\
            &= \sum_{i,j=1}^3 \int_{\bR^3} (D_k u_i) (D_i u_j)(D_k u_j) \,dx + \sum_{i,j=1}^3 \int_{\bR^3} u_i (D_k D_i u_j) (D_k u_j)\,dx. \nonumber
    \end{align}
    Por outro lado, utilizando integração por partes (ver Teorema \ref{thm:integracao-por-partes}) inferimos que
    \[
        \begin{aligned}
            \sum_{i,j=1}^3 \int_{\bR^3} u_i (D_k D_i u_j)(D_k u_j) \,dx &= -\sum_{i,j=1}^3 \int_{\bR^3} (D_i u_i) (D_k u_j)^2 \,dx - \sum_{i,j=1}^{3} \int_{\bR^3} u_i (D_k u_j) (D_k D_i u_j) \,dx\\
            &=-\sum_{j=1}^3 \int_{\bR^3} (\nabla \cdot \bu) (D_k u_j)^2 \,dx - \sum_{i,j=1}^3 \int_{\bR^3} u_i D(D_k D_i u_j) (D_k u_j) \,dx\\
            &=- \sum_{i,j=1}^3 \int_{\bR^3} u_i D(D_k D_i u_j) (D_k u_j) \,dx,
        \end{aligned}
    \]
    pois $\nabla \cdot \bu = 0$. Ou seja
    \begin{equation} \label{eq:x2}
        \sum_{i=1}^3 \sum_{j=1}^{3} \int_{\bR^3} u_i (D_k D_i u_j) (D_k u_j) \,dx = 0.
    \end{equation}
    Análogamente, podemos escrever
    \begin{align}
        \sum_{i,j=1}^3 \int_{\bR^3} (D_k u_i) (D_i u_j) (D_k u_j) \,dx &= -\sum_{j=1}^{3} \int_{\bR^3} D_k (\nabla \cdot \bu) u_j (D_k u_j) \,dx - \sum_{i,j=1}^3 \int_{\bR^3} (D_k u_j) I_k D_i D_k u_j \,dx \nonumber\\
        &= - \sum_{i,j=1}^3 \int_{\bR^3} (D_k u_j) u_k D_i D_k u_j \,dx. \label{eq:x3}
    \end{align}
    Substituindo (\ref{eq:x2}) e (\ref{eq:x3}) em (\ref{eq:xx1}), chegamos a
    \[
        \int_{\bR^3} D_k(\bu \cdot \nabla \bu) \,dx = - \sum_{i,j=1}^{3} \int_{\bR^3} u_j (D_k u_i) (D_i D_k u_j) \,dx.
    \]
    A terceira integral em (\ref{eq:A}) é nula, já que
    \[
        \int_{\bR^3} D_k (\nabla p) \cdot D_k \bu \,dx = \sum_{j=1}^{3} \int_{\bR^3} D_k D_j p D_k u_j \,dx,
    \]
    e, utlizando integração por partes (ver Teorema \ref{thm:integracao-por-partes}), obtemos
    \[
        \int_{\bR^3} D_k \nabla p \cdot D_k \bu \,dx = -\sum_{j=1}^3 \int_{\bR^3} D_k p D_j D_k  u_j \,dx = - \int_{\bR^3} D_k p D_k \left( \bu \cdot \nabla \bu \right) \,dx = 0,
    \]
    pois novamente $\nabla \cdot \bu = 0$.
    Por fim, a última integral em (\ref{eq:A}) pode ser escrita da seguinte forma:
    \[
        \int_{\bR^3} \Delta D_k \bu \cdot D_k \bu \,dx = \sum_{i,j=1}^3 \int_{\bR^3} D_j^2 D_k u_i D_k u_i 
    \]
    novamente utilizando integração por partes, concluimos que
    \[
        \begin{aligned}
            \int_{\bR^3} \Delta D_k \bu \cdot D_k \bu \,dx &= -\sum_{i,j=1}^3 \int_{\bR^3} (D_jD_k u_i)(D_jD_k u_i)\\ 
            &= -\sum_{i,j=1}^3  \int_{\bR^3} (D_j D_k u_i)^2 \,dx = -\!\!\int_{\bR^3} \Vert D_kD\bu \Vert^2 \,dx
        \end{aligned}
    \]
    Dito isso, voltando para (\ref{eq:A}) e somando em $1 \leqslant k \leqslant 3$, segue que
    \[
        \begin{aligned}
            \frac{1}{2} \sum_{k=1}^3 \int_{\bR^3} \Vert D_k \bu(\cdot,t) \Vert^2 - \Vert D_k \bu(\cdot,t_0) \Vert^2 \,dx  &+ \nu \sum_{k=1}^3 \int_{t_0}^t \int_{\bR^3} \Vert D_kD\bu \Vert^2 \,dx ds\\
        &= \sum_{i=1}^3 \sum_{j=1}^3\sum_{k=1}^3 \int_{t_0}^{t}\int_{\bR^3} u_j D_k u_i D_i D_k u_j \,dxds.
        \end{aligned}
    \]
    Lembrando da definição das normas vistas nos preliminares, multiplicando ambos os lados por $2$ e reorganizando os termos, deduzimos que
    \begin{equation} \label{eq:y2}
        \begin{aligned}
            \Vert D\bu(\cdot,t) \Vert_{\cL^2(\bR^3)}^2 &+ 2\nu \int_{t_0}^t \Vert D^2 \bu(\cdot,t) \Vert_{\cL^2(\bR^3)}^2 \,ds\\ &= \Vert D\bu(\cdot,t_0) \Vert_{\cL^2(\bR^3)}^2 + 2 \sum_{i=1}^3 \sum_{j=1}^3\sum_{k=1}^3 \int_{t_0}^t \int_{\bR^3} u_i D_k u_j D_j D_k u_i \,dxds.
        \end{aligned}
    \end{equation}
    Vamos análisar o último termo da equação acima separadamente.
    Note que utilizando a desigualdade de Hölder (ver Teorema \ref{thm:pre-desigualdade-de-holder}) e as definições das normas, temos que
    \[
        \begin{aligned}
            \sum_{i=1}^3\sum_{j=1}^3\sum_{k=1}^3 \int_{\bR^3} u_i (D_k u_j)(D_k D_k u_i)\,dx &= \sum_{j=1}^3 \sum_{k=1}^3 \int_{\bR^3} (D_k u_j) \left( \sum_{i=1}^3 u_i (D_j D_k u_i) \right) \,dx\\
            &\leqslant \sum_{j=1}^3 \sum_{k=1}^{3} \int_{\bR^3} (D_k u_j) \left( \sum_{i=1}^3 u_i^2      \right)^{\frac{1}{2}} \left( \sum_{i=1}^3 (D_j D_k u_i)^2 \right)^{\frac{1}{2}} \,dx\\
            &=\sum_{j=1}^3\sum_{k=1}^{3} \int_{\bR^3} (D_k u_j) \Vert \bu \Vert \Vert D_j D_k \bu \Vert \,dx.
        \end{aligned}
    \]
    Novamente utilizando a desigualdade de Hölder, obtemos
    \[
        \begin{aligned}
            \sum_{i=1}^3\sum_{j=1}^3\sum_{k=1}^3 \int_{\bR^3} u_i (D_k u_j)(D_k D_k u_i)\,dx
            &\leqslant \Vert \bu \Vert_{\cL^\infty(\bR^3)} \sum_{j=1}^3 \int_{\bR^3} \left( \sum_{k=1}^3 |D_k u_j|^2 \right)^{\frac{1}{2}} \left( \sum_{k=1}^3 \Vert D_j D_k \bu \Vert \right)^{\frac{1}{2}} \,dx\\
            &\leqslant \Vert \bu \Vert_{\cL^\infty(\bR^3)} \sum_{j=1} \int_{\bR^3} \Vert Du_j \Vert \Vert D_j D\bu \Vert \,dx.
        \end{aligned}
    \]
    Utilizando a desigualdade de Hölder mais uma vez, segue que
    \[
        \begin{aligned}
            \sum_{i=1}^3\sum_{j=1}^3\sum_{k=1}^3 \int_{\bR^3} u_i (D_k u_j)(D_k D_k u_i)\,dx &\leqslant \Vert \bu \Vert_{\cL^\infty(\bR^3)}\int_{\bR^3} \left( \sum_{j=1}^3 \Vert D u_j \Vert^2 \right)^{\frac{1}{2}} \left( \sum_{j=1}^{3} \Vert Dj D\bu \Vert^2  \right)^{\frac{1}{2}} \,dx\\
            &=\Vert \bu \Vert_{\cL^\infty(\bR^3)} \int_{\bR^3} \Vert D\bu \Vert \Vert D^2 \bu \Vert \,dx.
        \end{aligned}
    \]
    Ainda utilizando a desigualdade de Hölder, concluimos que
    \begin{equation} \label{eq:y1}
        \begin{aligned}
            \sum_{i=1}^3\sum_{j=1}^3\sum_{k=1}^3 \int_{\bR^3} u_i (D_k u_j)(D_k D_k u_i)\,dx &\leqslant \Vert \bu \Vert_{\cL^2} \left( \int_{\bR^3} \Vert D\bu \Vert^2 \right)^{\frac{1}{2}} \left( \int_{\bR^3}  \Vert D^2\bu \Vert^{2} \right)^{\frac{1}{2}}\\ 
            &= \Vert \bu \Vert_{\cL^\infty(\bR^3)} \Vert D\bu \Vert_{\cL^2(\bR^3)} \Vert D^2 \bu \Vert_{\cL^2(\bR^3)}.
        \end{aligned}
    \end{equation}
    Portanto, substituindo (\ref{eq:y1}) em (\ref{eq:y2}), chegamos a
    \[
        \begin{aligned}
            \Vert D\bu(\cdot,t) \Vert_{\cL^2(\bR^3)}^2 &+ 2\nu \int_{t_0}^t \Vert D^2 \bu(\cdot,t) \Vert_{\cL^2(\bR^3)}^2 \,ds\\ &\leqslant \Vert D\bu(\cdot,t_0) \Vert_{\cL^2(\bR^3)}^2 + 2 \Vert \bu \Vert_{\cL^\infty(\bR^3)} \Vert D\bu \Vert_{\cL^2(\bR^3)} \Vert D^2 \bu \Vert_{\cL^2(\bR^3)},
        \end{aligned}
    \]
    que pela Desigualdade de Gagliardo-Nirenberg (\ref{eq:gagliardoinf}) em $\Vert \bu(\cdot,s) \Vert_{\infty}$, podemos escrever como:
    \[
        \begin{aligned}
            \Vert D\bu(\cdot,t) \Vert_{\cL^2(\bR^3)}^2 &+ 2\nu \int_{t_0}^t \Vert D^2 \bu(\cdot,t) \Vert_{\cL^2(\bR^3)}^2 \,ds\\ &\leqslant \Vert D\bu(\cdot,t_0) \Vert_{\cL^2(\bR^3)}^2 + 2 \! \int_{t_0}^t \Vert \bu(\cdot,s) \Vert_{\cL^2(\bR^3)}^{\frac{1}{2}} \Vert D\bu(\cdot,s) \Vert_{\cL^2(\bR^3)}^{\frac{1}{2}} \Vert D^2\bu(\cdot,s) \Vert_{\cL^2(\bR^3)}^2 \,ds.
        \end{aligned}
    \]
    Consequentemente, por (\ref{eq:desigualdade-de-energia}), podemos escrever
    \begin{align} 
        \Vert D\bu(\cdot,t) \Vert_{\cL^2(\bR^3)}^2 &+ 2 \nu \int_{t_0}^t \Vert D^2 \bu(\cdot,s) \Vert_{\cL^2(\bR^3)}^2 \,ds \label{eq:maisumadesigualdade}\\ &\leqslant \Vert D\bu(\cdot,t_0) \Vert_{\cL^2(\bR^3)}^2 + 2 \! \int_{t_0}^t \big[ \Vert \bu_0 \Vert_{\cL^2(\bR^3)} \Vert D\bu(\cdot,s) \Vert_{\cL^2(\bR^3)} \big]^\frac{1}{2} \Vert D^2\bu(\cdot,s) \Vert_{\cL^2(\bR^3)}^2, \nonumber
    \end{align}
    para todo $t \geqslant t_0$. 

    Seja $t_0 \geqslant t_*$ (onde $t_*$ é dado em (??)) tal que
    \begin{equation} \label{eq:MM1}
        \Vert \bu_0 \Vert_{\cL^2(\bR^3)} \Vert D\bu(\cdot,t_0) \Vert_{\cL^2(\bR^3)} \leqslant \nu^2.
    \end{equation}
    Caso contrário,
    \[
        \Vert \bu_0 \Vert_{\cL^2(\bR^3)} \Vert D\bu(\cdot,s) \Vert_{\cL^2(\bR^3)} \geqslant \nu^2,
    \]
    para todo $s \geqslant t_*$.
    Integrando em $[t_*,t]$, com $t \geqslant t_*$, teriamos
    \[
        \Vert \bu_0 \Vert_{\cL^2(\bR^3)} \int_{t_*}^{t} \Vert D\bu(\cdot,s) \Vert_{\cL^2(\bR^3)}^2 \geqslant \nu^4 (t - t_*),
    \]
    para todo $t \geqslant t_*$.
    Pela desigualdade de energia (\ref{eq:desigualdade-de-energia}), chegmos a
    \[
        t - t_* \leqslant \frac{\Vert \bu_0 \Vert_{\cL^2(\bR^3)}^2}{2 \nu^5},
    \]
    para todo $t \geqslant t_*$.
    Isto é um absurdo, pois estariamos dizendo que $t$ possui cota superior.
    Dessa forma, afirmamos que
    \begin{equation} \label{eq:MM2}
        \Vert \bu_0 \Vert_{\cL^2(\bR^3)} \Vert D\bu(\cdot,s) \Vert_{\cL^2(\bR^3)} < \nu^2,
    \end{equation}
    para todo $s \geqslant t_0$.
    Com efeito, se (\ref{eq:MM2}) não fosse válida, por suavidade existiria $t_1 > t_0$ tal que
    \begin{equation} \label{eq:MM3}
        \Vert \bu_0 \Vert_{\cL^2(\bR^3)} \Vert D\bu(\cdot,s) \Vert_{\cL^2(\bR^3)} < \nu^2,
    \end{equation}
    para todo $t_0 \leqslant s < t_1$, com
    \begin{equation} \label{eq:MM4}
        \Vert \bu_0 \Vert_{\cL^2(\bR^3)} \Vert D\bu(\cdot,t_1) \Vert_{\cL^2(\bR^3)} = \nu^2.
    \end{equation}
    Escolhendo $t=t_1$ em (\ref{eq:maisumadesigualdade}), temos que
    \[
        \Vert D\bu(\cdot,t_1) \Vert_{\cL^2(\bR^3)} \leqslant \Vert D\bu(\cdot,t_0) \Vert_{\cL^2(\bR^3)}.
    \]
    De fato, por (\ref{eq:MM3}) e (\ref{eq:maisumadesigualdade}), podemos escrever
    \[
        \begin{aligned}
            \Vert D\bu(\cdot,t_1) \Vert_{\cL^2(\bR^3)}^2 &+ 2\nu \int_{t_0}^{t_1} \Vert D^2\bu(\cdot,s) \Vert_{\cL^2(\bR^3)}^2 \,ds\\
            &\leqslant \Vert D\bu(\cdot,t_0) \Vert_{\cL^2(\bR^3)}^2 + 2 \int_{t_0}^{t_1} \big[ \Vert \bu_0 \Vert_{\cL^2(\bR^3)}\Vert D\bu(\cdot,s) \Vert_{\cL^2(\bR^3)} \big]^{\frac{1}{2}} \Vert D^2\bu(\cdot,s) \Vert_{\cL^2(\bR^3)}^2\\
            &\leqslant \Vert D\bu(\cdot,t_0) \Vert_{\cL^2(\bR^3)}^2 + 2\nu \int_{t_0}^{t_1} \Vert D^2\bu(\cdot,s) \Vert_{\cL^2(\bR^3)}^2 \,ds,
        \end{aligned}
    \]
    o que resulta em
    \[
        \Vert D\bu(\cdot,t_1) \Vert_{\cL^2(\bR^3)} \leqslant \Vert D\bu(\cdot,t_0) \Vert_{\cL^2(\bR^3)},
    \]
    como era desejado. Então por (\ref{MM4}) e (\ref{eq:MM1}), concluimos que
    \[
        \nu^2 = \Vert \bu_0 \Vert_{\cL^2(\bR^3)} \Vert D\bu(\cdot,t_1) \Vert_{\cL^2(\bR^3)} \leqslant \Vert \bu_0 \Vert_{\cL^2(\bR^3)} \Vert D\bu(\cdot,t_0) \Vert_{\cL^2(\bR^3)} < \nu^2,
    \]
    o que é uma contradição! Dessa forma, (\ref{eq:MM3}) é válida.
    Por (\textcolor{red}{4.18}) e (\textcolor{red}{4.19}), inferimos que
    \[
        \begin{aligned}
            \Vert D\bu(\cdot,t) \Vert_{\cL^2(\bR^3)}^2 &+ 2\nu \int_{t_0}^t \Vert D^2 \bu(\cdot,s) \Vert_{\cL^2(\bR^3)}^2 \,ds\\
            &\leqslant \Vert D\bu(\cdot,t_0) \Vert_{\cL^2(\bR^3)}^2 + 2\nu \int_{t_0}^t \Vert D^2 \bu(\cdot,s) \Vert_{\cL^2(\bR^3)} \,ds,
        \end{aligned}
    \]
    pra todo $t \geqslant t_0$. Com isso,
    \begin{equation} \label{eq:MM5}
        \Vert D\bu(\cdot,t) \Vert_{\cL^2(\bR^3)} \leqslant \Vert D\bu(\cdot,t_0) \Vert_{\cL^2(\bR^3)},
    \end{equation}
    para todo $t \geqslant t_0$.
    Pela prova de (\textcolor{red}{4.18}) e por (\ref{eq:MM5}), deduzimos que
    \[
        \begin{aligned}
            \Vert D\bu(\cdot,t) \Vert_{\cL^2(\bR^3)}^2 & \leqslant \Vert D\bu(\cdot,t) \Vert_{\cL^2(\bR^3)}^2 + 2 \gamma \int_{t_2}^t \Vert D^2\bu(\cdot,s) \Vert_{\cL^2(\bR^3)}^2 \,ds \leqslant \Vert D\bu(\cdot,t_2) \Vert_{\cL^2(\bR^3)}^2,
        \end{aligned}
    \]
    para todo $t \geqslant t_2 \geqslant t_0$, onde $\gamma = \nu - \Vert \bu_0 \Vert_{\cL^2(\bR^3)}^{\frac{1}{2}} \Vert D\bu(\cdot,t_0) \Vert_{\cL^2(\bR^3)}^{\frac{1}{2}} > 0$ (ver (??)) é um constnte.
    Portanto, escolhendo $t_{**} = t_2$, finalizamos a demonstração, pois mostrmos que
    \[
        \Vert D\bu(\cdot,t) \Vert_{\cL^2(\bR^3)} \leqslant \Vert D\bu(\cdot,t_2) \Vert_{\cL^2(\bR^3)},
    \]
    para todo $t \geqslant t_{**}$.
\end{prf}

A proposição abaixo utiliza a solução do problema linearizado (\ref{eq:navier-stokes-linearizado}) para estabelecer uma estimativa importante para o prosseguimento do nosso texto.

\begin{pbox} \label{pr:dav}
    Seja $\bu(\cdot,t)$ solução de Leray para (\ref{eq:navierstokes}).
    Dados $\tilde t_0 > t_0 > 0$, tem-se
    \[
        \Vert D^\alpha \bv(\cdot,t) - D^\alpha \tilde \bv(\cdot,t)\Vert_{\cL^2(\bR^3)} \leqslant c(k) \nu^{-\left( \frac{5}{4} + \frac{k}{2} \right)} \Vert \bu_0 \Vert_{\cL^2(\bR^3)}^2 (\tilde t_0 - t_0)^{\frac{1}{2}} (t - \tilde t_0)^{-\left( \frac{3}{4} + \frac{k}{2} \right)},
    \]
    para todo $t > \tilde t_0$, onde $\bv(\cdot,t) = e^{\nu\Delta(t-t_0)} \bu(\cdot,t_0)$, $\tilde\bv(\cdot,t) = e^{\nu\Delta(t-\tilde t_0)} \bu(\cdot,\tilde t_0)$ e $k = |\alpha|$.
\end{pbox}
\begin{prf}
    Primeiramente, reescrevemos $\bv(\cdot,t)$ como
    \[
        \bv(\cdot,t) = e^{\nu\Delta(t-t_0)} \left[\bu(\cdot,t_0) - \bu_\delta(\cdot,t_0)\right] + e^{\nu\Delta(t-t_0)}\bu_\delta(\cdot,t_0),
    \]
    onde $t > t_0$ e $\bu_\delta(\cdot,t)$ é dada em (\ref{eq:navier-stokes-regularizado}).
    Ademais, temos que
    \begin{equation} \label{eq:5}
        \bu_\delta(\cdot,t_0) = e^{\nu\Delta t_0}\bar\bu_{0,\delta} + \int_0^{t_0} e^{\nu\Delta(t_0 - s)}\BQ_\delta(\cdot,s)\,ds.
    \end{equation}
    Com efeito, considere a equação
    \[
        \partial_t \bu_\delta = \BQ_\delta + \nu\Delta\bu_\delta,
    \]
    onde $\BQ_\delta = - \bar\bu_\delta \cdot \nabla \bu_\delta - \nabla p_\delta$ e $\bar{\bu}_\delta = \eta_\delta * \bar{\bu}_\delta$.
    Aplique o semigrupo do calor $e^{\nu\Delta(t_0 - s)}$ em ambos os lados e integre sobre $[0,t_0]$ para obter
    \begin{equation} \label{eq:4}
        \int_0^{t_0} e^{\nu\Delta(t_0 - s)} \partial_s \bu_\delta \,ds = \int_0^{t_0} e^{\nu\Delta(t_0 - s)}\BQ_\delta \,ds + \nu\int_0^{t_0} e^{\nu\Delta(t_0 - s)} \Delta \bu_\delta \, ds.
    \end{equation}
    Além disso, note que
    \[
        \partial_s \left[ e^{\nu\Delta(t_0-s)}\bu_\delta \right] = -\nu e^{\nu\Delta(t_0 -s)}\Delta\bu_\delta + e^{\nu\Delta(t_0 - s)} \partial_s \bu_\delta,
    \]
    ou seja,
    \[
        e^{\nu\Delta(t_0 - s)} \partial_s \bu_\delta = \partial_s \left[ e^{\nu\Delta(t_0-s)}\bu_\delta \right] + \nu e^{\nu\Delta(t_0 -s)}\Delta\bu_\delta.
    \]
    Dessa forma, (\ref{eq:4}) se torna
    \[
        \int_0^{t_0}  \partial_s \left[ e^{\nu\Delta(t_0-s)}\bu_\delta \right] \,ds + \nu \int_0^{t_0} e^{\nu\Delta(t_0 -s)}\Delta\bu_\delta \,ds = \int_0^{t_0} e^{\nu\Delta(t_0 - s)}\BQ_\delta \,ds + \nu\int_0^{t_0} e^{\nu\Delta(t_0 - s)} \Delta \bu_\delta \,ds,
    \]
    isto é,
    \[
        \bu_\delta(\cdot,t_0) - e^{\nu \Delta t_0} \bar\bu_{0,\delta} = \int_0^{t_0} e^{\nu\Delta(t_0 - s)}\BQ_\delta \,ds,
    \]
    por (\ref{eq:navierstokes}). Isto prova (\ref{eq:5}).
    Dito isso, é verdade que
    \[
        \bv(\cdot,t) = e^{\nu\Delta(t - t_0)} \left[ \bu(\cdot,t_0) - \bu_\delta(\cdot,t_0) \right] + e^{\nu\Delta t}\bar\bu_{0,\delta} + \int_0^{t_0} e^{\nu\Delta(t-s)} \BQ_\delta(\cdot,s)\,ds,
    \]
    para todo $t>t_0$. De forma análoga, deduzimos que
    \[
        \tilde\bv(\cdot,t) = e^{\nu\Delta(t - \tilde t_0)} \left[ \bu(\cdot,\tilde t_0) - \bu_\delta(\cdot,\tilde t_0) \right] + e^{\nu\Delta t}\bar\bu_{0,\delta} + \int_0^{\tilde t_0} e^{\nu\Delta(t-s)} \BQ_\delta(\cdot,s)\,ds.
    \]
    para todo $t > \tilde t_0$.
    Dessa forma, podemos escrever
    \[
        \begin{aligned}
            D^\alpha\tilde\bv(\cdot,t) -  D^\alpha\bv(\cdot,t) &= D^\alpha\left( e^{\nu\Delta(t-\tilde t_0)} \big[\bu(\cdot,\tilde t_0) - \bu_\delta(\cdot,\tilde t_0)\big] \right)\\ 
            &- D^\alpha\left( e^{\nu\Delta(t-t_0)} \big[\bu(\cdot, t_0) - \bu_\delta(\cdot, t_0)\big] \right)
            + D^\alpha\int_{t_0}^{\tilde t_0} e^{\nu\Delta(t-s)}\BQ_{\delta}(\cdot,s) \,ds.
        \end{aligned}
    \]
    Agora, considere que $K \subseteq \bR^3$ é um compacto qualquer. Portanto, para cada $t > \tilde t_0$ e $\delta > 0$, infere-se
    \begin{equation} \label{eq:acima}
        \Vert D^\alpha\tilde\bv(\cdot,t) -  D^\alpha\bv(\cdot,t) \Vert_{\cL^2(K)} \leqslant J_{\alpha,\delta}(t) + \int_{t_0}^{\tilde t_0} \Vert D^\alpha\big[ e^{\nu\Delta(t-s)}\BQ_{\delta}(\cdot,s)\big] \Vert_{\cL^2(K)}\,ds,
    \end{equation}
    onde
    \[
        \begin{aligned}
            J_{\alpha,\delta}(t) = \big\Vert D^\alpha\big( e^{\nu\Delta(t-\tilde t_0)} \big[\bu(\cdot,\tilde t_0) &- \bu_\delta(\cdot,\tilde t_0)\big] \big) \big\Vert _{\cL^2(K)} \\
            &+ \big\Vert D^\alpha\big( e^{\nu\Delta(t-t_0)} \big[\bu(\cdot, t_0) - \bu_\delta(\cdot, t_0)\big] \big) \big\Vert _{\cL^2(K)}.
        \end{aligned}
    \]
    Utilizando a Proposição \ref{pr:DaQ} e (\ref{eq:2.11}), temos que
    \[
        \Vert D^\alpha\tilde\bv(\cdot,t) -  D^\alpha\bv(\cdot,t) \Vert_{\cL^2(K)} \leqslant J_{\alpha,\delta}(t) + c(k) \nu^{-\frac{k}{2}}\int_{t_0}^{\tilde t_0} (t - s)^{-\frac{k}{2}} \Vert e^{\nu\Delta(t-s)/2}\BQ_{\delta}(\cdot,s) \Vert_{\cL^2(\bR^3)} \,ds
    \]
    \begin{equation} \label{eq:3.3}
        \leqslant J_{\alpha,\delta}(t) + c(k) \nu^{-\left( \frac{k}{2} + \frac{3}{4} \right)} (t - \tilde t_0)^{-\frac{k}{2}} \int_{t_0}^{\tilde t_0} (t-s)^{-\frac{3}{4}} \Vert \bu_\delta(\cdot,s) \Vert_{\cL^2(\bR^3)} \Vert D\bu_\delta(\cdot,s) \Vert_{\cL^2(\bR^3)} \,ds.
    \end{equation}
    Observe que a integral acima pode ser simplificada, já que $s \leqslant \tilde t_0$ implica em $(t - s)^{-\frac{3}{4}} \leqslant (t - \tilde t_0)^{-\frac{3}{4}}$ e pela desigualdade de energia (\ref{eq:desigualdade-de-energia-regularizado}), $\Vert \bu_\delta(\cdot,s) \Vert_{\cL^2(\bR^3)} \leqslant \Vert \bu_0 \Vert_{\cL^2(\bR^3)}$. Sendo assim, podemos escrever
    \begin{equation} \label{eq:C}
        \int_{t_0}^{\tilde t_0} (t-s)^{-\frac{3}{4}} \Vert \bu_\delta(\cdot,s) \Vert_{\cL^2(\bR^3)} \Vert D\bu_\delta(\cdot,s) \Vert_{\cL^2(\bR^3)} \,ds \leqslant (t - \tilde t_0)^{-\frac{3}{4}} \Vert \bu_0 \Vert_{\cL^2(\bR^3)} \int_{t_0}^{\tilde t_0} \Vert D\bu_\delta(\cdot,s) \Vert_{\cL^2(\bR^3)}\,ds.
    \end{equation}
    Por outro lado, pela desigualdade de Hölder e novamente pela desigualdade de energia (\ref{eq:desigualdade-de-energia-regularizado}), segue que
    \begin{equation} \label{eq:B}
        \int_{t_0}^{\tilde t_0} \Vert D\bu_\delta(\cdot,s) \Vert_{\cL^2(\bR^3)}\,ds \leqslant \left( \int_{t_0}^{\tilde t_0} \Vert D\bu_\delta \Vert^2_{\cL^2(\bR^3)} \,ds \right)^{\frac{1}{2}} \left( \int_{t_0}^{\tilde t_0} ds \right)^{\frac{1}{2}} \leqslant \frac{(\tilde t_0 - t_0)^{\frac{1}{2}}}{(2\nu)^{\frac{1}{2}}} \Vert \bu_0 \Vert_{\cL^2(\bR^3)}.
    \end{equation}
    Portanto, por (\ref{eq:C}) e (\ref{eq:B}), obtemos
    \[
        \Vert D^\alpha\tilde\bv(\cdot,t) -  D^\alpha\bv(\cdot,t) \Vert_{\cL^2(K)} \leqslant J_{\alpha,\delta}(t) + c(k) \nu^{-\left( \frac{k}{2} + \frac{5}{4} \right)}(t-\tilde t_0)^{-\left( \frac{k}{2} + \frac{3}{4} \right)} (\tilde t_0 - t_0)^{\frac{1}{2}} \Vert \bu_0 \Vert_{\cL^2(\bR^3)}^2.
    \]
    Tomando $\delta = \delta'$ (ver (\ref{eq:b1})), temos que $J_{\alpha,\delta}(t)\to0$, quando $\delta \to 0$, pois, dados $\sigma,\tau >0$, é verdade que
    \begin{equation} \label{eq:Daendt}
        \Vert D^\alpha (e^{\nu\Delta\tau}[\bu(\cdot,\sigma) -  \bu_{\delta}(\cdot,\sigma)]) \Vert_{\cL^2(K)} \to 0,
    \end{equation}
    quando $\delta \to 0$. De fato, denotando $\Phi_\delta(\cdot,\tau) = D^\alpha \big( e^{\nu \Delta \tau} [\bu(\cdot,\sigma) - \bu_\delta(\cdot,\sigma)] \big)$, tem-se
    \[
        \Phi_\delta(\cdot,\tau) =  H_\alpha(\cdot,\tau) * [\bu(\cdot,\sigma) - \bu_\delta(\cdot,\sigma)],
    \]
    onde $H_\alpha(\cdot,\tau) = D^\alpha E(\cdot,\tau) \in \cL^1(\bR^3) \cap \cL^\infty(\bR^3)$, com $E(x,\tau) = e^{-\frac{\Vert x \Vert^2}{4 \nu \tau}}$, não depende de $\delta$. Como $\bu(\cdot,\sigma) - \bu_{\delta}(\cdot,\sigma) \rightharpoonup 0$, quando $\delta \to 0$ (ver (\ref{eq:b1})), segue que $\Phi_{\delta}(x,\tau) \to 0$ para cada $x \in \bR^3$, quando $\delta \to 0$.
    Por outro lado, as seguintes estimativas valem  pela desigualdade de Hölder e mudança de variáveis:
    \[
        \begin{aligned}
            |\Phi_\delta(x,t)| &\leqslant \int_{\bR^3} | H_\alpha(x-y,\tau) [\bu(y,\sigma) - \bu_\delta(y,\sigma)]| \,dy\\
            &\leqslant \left( \int_{\bR^3} |H_\alpha(x-y,\tau)|^2 \,dy \right)^{\frac{1}{2}} \left( \int_{\bR^3} |\bu(y,\sigma) - \bu_\delta(y,\sigma)|^2 \,dy \right)^{\frac{1}{2}}\\[4pt] 
            &= \Vert H_\alpha(\cdot,\tau) \Vert_{\cL^2(\bR^3)} \Vert \bu(\cdot,\sigma) - \bu_\delta(\cdot,\sigma) \Vert_{\cL^2(\bR^3)}.
        \end{aligned}
    \]
    Utilizando a desigualdade triangular e (\ref{eq:desigualdade-de-energia-regularizado}), obtemos
    \[
        |\Phi_\delta(x,\tau)| \leqslant 2 \Vert H_\alpha(\cdot,\tau) \Vert_{\cL^2(\bR^3)} \Vert \bu_0 \Vert_{\cL^2(\bR^3)},
    \]
    para todo $x \in \bR^3$. Pelo Teorema da Convergência Dominada (pois $2 \Vert H_\alpha(\cdot,\tau) \Vert_{\cL^2(\bR^3)} \Vert \bu_0 \Vert_{\cL^2(\bR^3)}$ é constante em relação a $x$ e $K$ é limitado, logo pertence a $\cL^1(K)$), segue que $\Vert \Phi_{\delta}(\cdot,t) \Vert_{\cL^2(K)} \to 0$, quando $\delta\to 0$ (pelo fato de $K$ ser compacto). Isto prova (\ref{eq:Daendt}).
    Dito isso, fazendo $\delta \to 0$ em (\ref{eq:3.3}), inferimos que
    \[
        \Vert D^\alpha \tilde\bv(\cdot,t) - D^\alpha \bv(\cdot,t) \Vert_{\cL^2(K)} \leqslant c(k) \nu^{-\left( \frac{k}{2} + \frac{5}{4} \right)} \Vert \bu_0 \Vert^2_{\cL^2(\bR^3)} (\tilde t_0 - t_0)^{\frac{1}{2}} (t - \tilde t_0)^{-\left( \frac{k}{2} + \frac{3}{4} \right)},
    \]
    para todo $t_0 > \tilde t_0$. Tomando o supremo de todos $K \subseteq \bR^3$ compactos, obtemos a desigualdade desejada.
\end{prf}

\subsection{Decaimentos da solução de Leray}

O restante dessa seção será destinado a estudar estimativas de decaimento para as soluções de Leray.

\begin{tbox} \label{thm:Duto0}
    Seja $\bu(\cdot,t)$ uma solução de Leray para (\ref{eq:navierstokes}); então,
    \begin{equation} \label{eq:tDuto0}
        t^{\frac{1}{2}} \Vert D\bu(\cdot,t) \Vert_{\cL^2(\bR^3)} \to 0,
    \end{equation}
    quando $t \to \infty$.
\end{tbox}
\begin{prf}
    Seja $t_{**}$ o instante de tempo encontrado na Proposição \ref{pr:tstar}. Suponha que (\ref{eq:tDuto0}) é falsa, nesse caso existem uma sequência crescente $t_\ell \to \infty$ (com $t_\ell \geqslant t_{**}$ e $t_\ell \geqslant 2 t_{\ell -1}$ para todo $\ell \in \bN$) e um $\varepsilon > 0$ tais que
    \[
        t_\ell \Vert D\bu(\cdot,t_\ell) \Vert_{\cL^2(\bR^3)}^2 \geqslant \varepsilon,
    \]
    para todo $\ell \in \bN$.
    Em particular, é verdade, pela Proposição \ref{pr:tstar}, que
    \begin{equation} \label{eq:ppp}
        \int_{t_{\ell-1}}^{t_\ell} \Vert D\bu(\cdot,s) \Vert_{\cL^2(\bR^3)}^2 \,ds \geqslant (t_\ell - t_{\ell-1}) \Vert D\bu(\cdot,t_\ell) \Vert_{\cL^2(\bR^3)}^2 \geqslant \frac{1}{2}t_\ell \Vert D\bu(\cdot,t_\ell) \Vert_{\cL^2(\bR^3)}^2 \geqslant \frac{1}{2}\varepsilon,
    \end{equation}
    para todo $\ell \in \bN$, o que contradiz a desigualdade de energia (\ref{eq:2.3}), pois
    \[
        \int_0^\infty \Vert D\bu(\cdot,s) \Vert_{\cL^2(\bR^3)}^2 \,ds < \infty,
    \]
    e passando ao limite em (\ref{eq:ppp}), quando $\ell \to \infty$, encontramos $\varepsilon \leqslant 0$.
    Portanto, a afirmação do teorema é válida.
\end{prf}

O teorema abaixo foi conjecturado por Leray no final do seu artigo \cite{leray-fluid}, e foi somente resolvido 50 anos depois por Kato (ver \cite{kato-navier.stokes}). Uma simples demonstração será apresentada abaixo utilizando conceitos que já eram conhecidos em 1934.

\begin{tbox}[Solução do problema clássico de Leray] \label{thm:problema-leray}
    Seja $\bu(\cdot,t)$ uma solução de Leray para (\ref{eq:navierstokes}), então
    \[
        \Vert \bu(\cdot,t) \Vert_{\cL^2(\bR^3)}\to 0,
    \]
    quando $t \to \infty$.
\end{tbox}
\begin{prf}
    Seja $t_{**}$ como definido na Proposição \ref{pr:tstar}.
    Dado $\varepsilon > 0$, tomemos $t_0 \geqslant t_{**}$ suficientemente grande tal que, pelo Teorema \ref{thm:Duto0}, podemos inferir
    \begin{equation} \label{eq:xxx}
        t^{\frac{1}{2}} \Vert D\bu(\cdot,t) \Vert_{\cL^2(\bR^3)} \leqslant \varepsilon,
    \end{equation}
    para todo $t \geqslant t_0$.
    Como $\bu(\cdot,t)$ é suave em $[t_0,\infty)$ (ver Seção \ref{sec:intro}), obtemos, de forma análoga ao que foi feito na demonstração da Proposição \ref{pr:dav}, que
    \begin{equation} \label{eq:2212}
        \bu(\cdot,t) = e^{\nu\Delta(t - t_0)} \bu(\cdot,t_0) + \int_{t_0}^{t} e^{\nu\Delta(t-s)}\BQ(\cdot,s) \, ds,
    \end{equation}
    para todo $t \geqslant t_0$
    Dito isso, utlizando a Proposição \ref{pr:nsei}, podemos escrever
    \[
        \begin{aligned}
            \Vert \bu(\cdot,t) \Vert_{\cL^2(\bR^3)} &\leqslant \Vert e^{\nu\Delta(t - t_0)} \bu(\cdot,t_0) \Vert_{\cL^2(\bR^3)} + \int_{t_0}^t \Vert e^{\nu\Delta(t-s)} \BQ(\cdot,s) \Vert_{\cL^2(\bR^3)}\,ds\\
            &\leqslant \Vert e^{\nu\Delta(t - t_0)} \bu(\cdot,t_0) \Vert_{\cL^2(\bR^3)} + c \nu^{-\frac{3}{4}} \int_{t_0}^t (t - s)^{-\frac{3}{4}} \Vert \bu(\cdot,s) \Vert_{\cL^2(\bR^3)} \Vert D\bu(\cdot,s) \Vert_{\cL^2(\bR^3)} \,ds.
        \end{aligned}
    \]
    Pela desigualdade de energia (\ref{eq:desigualdade-de-energia}), concluímos que
    \[
        \Vert \bu(\cdot,t) \Vert_{\cL^2(\bR^3)} \leqslant\Vert e^{\nu\Delta(t - t_0)} \bu(\cdot,t_0) \Vert_{\cL^2(\bR^3)} + c \nu^{-\frac{3}{4}} \Vert \bu_0 \Vert_{\cL^2(\bR^3)} \int_{t_0}^t (t - s)^{-\frac{3}{4}}\Vert D\bu(\cdot,s) \Vert_{\cL^2(\bR^3)} \,ds,
    \]
    para todo $t \geqslant t_0$.
    Por (\ref{eq:xxx}), deduzimos que
    \[
        \Vert \bu(\cdot,t) \Vert_{\cL^2(\bR^3)} \leqslant \Vert e^{\nu\Delta(t - t_0)} \bu(\cdot,t_0) \Vert_{\cL^2(\bR^3)} + c \nu^{-\frac{3}{4}} \Vert \bu_0 \Vert_{\cL^2(\bR^3)}\, \varepsilon \! \int_{t_0}^t (t - s)^{-\frac{3}{4}} s^{-\frac{1}{2}} \,ds.
    \]
    Note que
    \[
        \int_{t_0}^t (t - s)^{-\frac{3}{4}}s^{-\frac{1}{2}}\,ds \leqslant c,
    \]
    para todo $t \geqslant t_0 + 1$. Sendo assim, é verdade que
    \[
        \Vert \bu(\cdot,t) \Vert_{\cL^2(\bR^3)} \leqslant \Vert e^{\nu\Delta(t - t_0)} \bu(\cdot,t_0) \Vert_{\cL^2(\bR^3)} + c\varepsilon\nu^{-\frac{3}{4}} \Vert \bu_0 \Vert_{\cL^2(\bR^3)}.
    \]
    Por outro lado, pelo Teorema \ref{thm:norma-transformada}, sabemos que
    \[
        \begin{aligned}
            \Vert e^{\nu \Delta (t-t_0)} \bu(\cdot,t_0) \Vert_{\cL^2(\bR^3)}^2 &= \Vert \cF [ e^{\nu \Delta (t-t_0) }\bu(\cdot,t_0)] \Vert_{\cL^2(\bR^3)}^2\\ &= \int_{\bR^3} \Vert \cF[ e^{\nu \Delta (t-t_0)} \bu(\omega,t_0)]\Vert^2 \,d\omega = \int_{\bR^3} e^{-2\nu (t-t_0) \Vert \omega \Vert^2} \Vert \hat\bu (\omega,t_0) \Vert^2 \,d\omega.
        \end{aligned}
    \]
    Como $e^{-2\nu (t-t_0) \Vert \omega \Vert^2} \Vert \hat\bu(\omega,t_0) \Vert^2 \leqslant \Vert \hat\bu(\omega,t_0) \Vert^2 \in \cL^1(\bR^3)$, então, pelo Teorema da Convergência Dominada (ver Teorema \ref{thm:teorema-da-convergencia-dominada}), inferimos
    \[
        \Vert e^{\nu \Delta (t-t_0)} \bu(\cdot,t_0) \Vert_{\cL^2(\bR^3)} \to 0,
    \]
    quando $t \to \infty$. Dito isso, concluímos que
    \[
        \Vert \bu(\cdot,t) \Vert_{\cL^2(\bR^3)} \leqslant (1 + c \nu^{-\frac{3}{4}} \Vert \bu_0 \Vert_{\cL^2(\bR^3)}) \,\varepsilon,
    \]
    para todo $t,t_0 > 0$ suficientemente grandes.
    Como $\varepsilon > 0$ é arbitrário, isso mostra que
    \[
        \Vert \bu(\cdot,t) \Vert_{\cL^2(\bR^3)} \to 0,
    \]
    quando $t \to \infty$.
\end{prf}

Note que os Teoremas \ref{thm:Duto0} e \ref{thm:problema-leray} juntos mostram que temos o decaimento de energia na norma do espaço de Sobolev $H^1(\bR^3)$, já que $\Vert \bu(\cdot,t) \Vert_{H^1(\bR^3)}^2 = \Vert \bu(\cdot,t) \Vert_{\cL^2(\bR^3)}^2 + \Vert D\bu(\cdot,t) \Vert_{\cL^2(\bR^3)}^2$. Mais precisamente,
\[
    \Vert \bu(\cdot,t) \Vert_{H^1(\bR^3)}^2 = \Vert \bu(\cdot,t) \Vert_{\cL^2(\bR^3)}^2 + t^{-1} \big[ t^{\frac{1}{2}} \Vert D\bu(\cdot,t) \Vert_{\cL^2(\bR^3)}\big]^2 \to 0,
\]
quando $t \to \infty$.

A demonstração do lema abaixo é bastante extensa, porém é de extrema importância para o último resultado dessa seção.

\begin{lbox} \label{lm:uk}
    Para cada $k \geqslant 0$ inteiro, denotando $U_k(t) := t^{\frac{k}{2}} \Vert D^k \bu(\cdot,t) \Vert_{\cL^2(\bR^3)}$, tem-se
    \[
        U_k \in \cL^\infty([T_{**}, \infty)).
    \]
\end{lbox}
\begin{prf}
    O caso $k = 0$ segue da desigualdade de energia (\ref{eq:desigualdade-de-energia}), pois
    \[
        U_0(t) =t^0 \Vert D^0\bu(\cdot,t) \Vert_{\cL^2(\bR^3)} = \Vert \bu(\cdot,t) \Vert_{\cL^2(\bR^3)} \leqslant \Vert \bu_0 \Vert_{\cL^2(\bR^3)},
    \]
    para todo $t > 0$ (em particular, para todo $t \geqslant T_{**}$). Logo, $U_0 \in \cL^{\infty}([T_{**}, \infty))$.

    O caso $k = 1$ também já foi provado anteriormente. Com efeito,
    pelo Teorema \ref{thm:Duto0}, temos que $U_1(t) \to 0$, quando $t \to \infty$. Isso é suficiente para mostrar que $U_1 \in \cL^\infty([T_{**}, \infty))$, pois $|U_1(t)| \leqslant 1$, para todo $t$ suficientemente grande.

    Dito isso, resta provar a afirmaçao para $k \geqslant 2$ inteiro. De forma análoga ao que foi feito na demonstração da Proposição \ref{pr:dav}, dado $t_0 \geqslant T_{**}$, podemos escrever $\bu(\cdot,t)$ da seguinte forma:
    \[
        \bu(\cdot,t) = e^{\nu \Delta(t-t_0)} \bu(\cdot,t_0) + \int_{t_0}^{t} e^{\nu \Delta (t-s)} \BQ(\cdot,s) \,ds,
    \]
    para todo $t \geqslant t_0$.
    Dessa forma, para cada multi-índice $\alpha$, temos que
    \begin{equation} \label{eq:aaa}
        \Vert D^\alpha \bu(\cdot,t) \Vert_{\cL^2(\bR^3)} \leqslant \Vert D^\alpha [ e^{\nu \Delta(t-t_0)} \bu(\cdot,t_0)] \Vert_{\cL^2(\bR^3)} + \int_{t_0}^t \Vert D^\alpha [e^{\nu \Delta (t-s)} \BQ(\cdot,s)] \Vert_{\cL^2(\bR^3)} \,ds,
    \end{equation}
    para todo $t \geqslant t_0$.
    Defina $U^\alpha(t) := t^{\frac{k}{2}} \Vert D^\alpha \bu(\cdot,t) \Vert$, com $k = |\alpha|$.
    Por (\ref{eq:aaa}), segue que
    \[
        U^\alpha(t) \leqslant I_1(\alpha,t) + I_2(\alpha,t) + J_\alpha(t),
    \]
    onde
    \[
        I_1(\alpha,t) = t^{\frac{k}{2}}\Vert D^\alpha [ e^{\nu \Delta(t-t_0)} \bu(\cdot,t_0)] \Vert_{\cL^2(\bR^3)},
    \]
    \[
        I_2(\alpha,t) = t^{\frac{k}{2}}\int_{t_0}^{t'} \Vert D^\alpha [e^{\nu \Delta (t-s)} \BQ(\cdot,s)] \Vert_{\cL^2(\bR^3)} \,ds,
    \]
    \begin{equation} \label{eq:Jalfa}
        J_\alpha(t) = t^{\frac{k}{2}}\int_{t'}^t \Vert D^\alpha [e^{\nu \Delta (t-s)} \BQ(\cdot,s)] \Vert_{\cL^2(\bR^3)} \,ds,
    \end{equation}
    com $t' = \frac{t_0 + t}{2}$. Conseguimos estimar $I_1(\alpha, t)$ de forma direta.
    Com efeito, por (\ref{eq:estimativa-util}) e pela desigualdade de energia (\ref{eq:desigualdade-de-energia}), inferimos que
    \begin{equation} \label{eq:BB2}
        |I_1(\alpha,t)| \leqslant c(k,\nu) \Vert \bu(\cdot,t_0) \Vert_{\cL^2(\bR^3)} (t - t_0)^{-\frac{k}{2}} t^{\frac{k}{2}} \leqslant c(k, \nu) \Vert \bu_0 \Vert_{\cL^2(\bR^3)} (t - t_0)^{-\frac{k}{2}} t^{\frac{k}{2}}.
    \end{equation}
    Porém, note que
    \begin{equation} \label{eq:xxxx}
        (t - t_0)^{-\frac{k}{2}} t^{\frac{k}{2}} = \left( \frac{t}{t - t_0} \right)^{\! \frac{k}{2}} = \left( 1 + \frac{t_0}{t - t_0} \right)^{\! \frac{k}{2}} \leqslant (1 + t_0)^{\frac{k}{2}},
    \end{equation}
    pois $\frac{t_0}{t - t_0} \leqslant t_0$, para todo $t \geqslant t_0 + 1$. Dessa forma, podemos escrever
    \begin{equation} \label{eq:W1}
        |I_1(\alpha,t)| \leqslant c(k,\nu,t_0,\bu_0),
    \end{equation}
    para todo $t \geqslant t_0 + 1$. Por outro lado, utlizando a Proposição \ref{pr:DaQ}, deduzimos
    \begin{equation} \label{eq:opp}
        \begin{aligned}
            |I_2(\alpha,t)| &\leqslant c(k,\nu) t^{\frac{k}{2}} \! \int_{t_0}^{t'} (t-s)^{-\left( \frac{k}{2} + \frac{3}{4} \right)} \Vert \bu(\cdot,s) \Vert_{\cL^2(\bR^3)} \Vert D\bu(\cdot,s) \Vert_{\cL^2(\bR^3)} \,ds\\
            &\leqslant c(k,\nu)t^{\frac{k}{2}} \!\int_{t_0}^{t'} (t-s)^{-\left( \frac{k}{2} + \frac{3}{4} \right)} \Vert \bu(\cdot,s) \Vert_{\cL^2(\bR^3)} \, s^{-\frac{1}{2}} \left[ s^{\frac{1}{2}} \Vert D\bu(\cdot,s) \Vert_{\cL^2(\bR^3)} \right] \,ds.
        \end{aligned}
    \end{equation}
    Sabemos que $U_1 \in \cL^\infty([T_{**},\infty))$; assim, existe uma constante $M_1$ tal que
    \begin{equation} \label{eq:C1}
        |U_1(s)| = s^{\frac{1}{2}} \Vert D\bu(\cdot,s) \Vert_{\cL^2(\bR^3)} \leqslant M_1 \text{ qtp em } [T_{**},\infty).
    \end{equation}
    Dessa forma, também utlizando a desigualdade de energia (\ref{eq:desigualdade-de-energia}) e o fato de $s \leqslant t'$ na integral de (\ref{eq:opp}), o que implica em $(t-s)^{-\left( \frac{k}{2} + \frac{3}{4} \right)} \leqslant \left( \frac{t - t_0}{2} \right)^{-\left( \frac{k}{2} + \frac{3}{4} \right)}$, concluímos que
    \[
        |I_2(\alpha,t)|\leqslant c(k,\nu) M_1 \Vert \bu_0 \Vert_{\cL^2(\bR^3)}\, t^{\frac{k}{2}} (t-t_0)^{-\left(\frac{k}{2} + \frac{3}{4} \right)} \int_{t_0}^{t'} s^{-\frac{1}{2}} \,ds.
    \]
    Sendo assim, por (\ref{eq:xxxx}), chegamos a
    \begin{equation} \label{eq:B3}
        |I_2(\alpha,t)|\leqslant c(k,\nu) \Vert \bu_0 \Vert_{\cL^2(\bR^3)} (t-t_0)^{-\frac{3}{4}} (t + t_0)^{\frac{1}{2}}.
    \end{equation}
    Por outro lado,
    \[
        (t - t_0)^{-\frac{3}{4}} (t + t_0)^{\frac{1}{2}} = (t - t_0)^{-\frac{1}{4}} (t - t_0)^{-\frac{1}{2}} (t + t_0)^{\frac{1}{2}} \leqslant \left( \frac{t + t_0}{t- t_0} \right)^{\frac{1}{2}} \leqslant (2t_0 + 1)^{\frac{1}{2}},
    \]
    para todo $t \geqslant t_0 + 1$.
    Por isso, podemos escrever
    \begin{equation} \label{eq:W2}
        |I_2(\alpha,t)|\leqslant c(k,\nu, t_0, M_1, \bu_0).
    \end{equation}
    para todo $t \geqslant t_0 + 1$.
    Assim, obtemos
    \begin{equation} \label{eq:4.10}
        U^\alpha(t) \leqslant c(k,\nu,t_0,M_1,\bu_0) + J_\alpha(t),
    \end{equation}
    para todo $t \geqslant t_0 + 1$.
    Logo, ainda resta estimar $J_\alpha(t)$, porém não conseguimos estimar $J_\alpha(t)$ de forma geral para todo $\alpha$, então faremos essa estimativa para o caso em que $k =|\alpha| = 2$ e depois utilizaremos indução sobre $k$ para mostrar uma estimativa para este termo.
    Com efeito, considerando $D^\alpha = D_iD_\ell$ (consequentemente, denotando $J_\alpha(t)$ por $J_{i\ell}(t)$), temos, por (\ref{eq:estimativa-util}), que
    \[
        J_{i\ell}(t) = t\int_{t'}^t \Vert D_i [ e^{\nu \Delta (t-t_0)} D_\ell \BQ (\cdot,s) ] \Vert_{\cL^2(\bR^3)} \,ds \leqslant c(\nu) \, t \int_{t'}^t (t - s)^{-\frac{7}{8}} \Vert D_\ell \BQ(\cdot,s) \Vert_{\cL^{\frac{4}{3}}(\bR^3)} \,ds.
    \]
    Para estimar $\Vert D_\ell \BQ(\cdot,s) \Vert_{\cL^{\frac{4}{3}}(\bR^3)}$
    \textcolor{red}{
        Note que $D_\ell\BQ = - D_\ell [ \bu \cdot \nabla \bu] - D_\ell \nabla p = - D_\ell [\bu \cdot \nabla \bu] - \nabla q_\ell$ onde $q_\ell = D_\ell p$. Aplicando o divergente nessa equação, temos que
        \[
            -\Delta q_\ell = \mathrm{div} ( D_\ell [ \bu \cdot \nabla \bu]).
        \]
        Aplicando a teoria de Calderon-Zygmund, temos para cada $1 < r < \infty$
        \[
            \Vert \nabla q_\ell (\cdot,t) \Vert_{\cL^r(\bR^n)} \leqslant c(r,n) \Vert D_{\ell} [\bu(\cdot,t) \cdot \nabla\bu (\cdot,t) ] \Vert
        \]
        o que implica em
        \[
            \Vert D_\ell \BQ(\cdot,t) \Vert_{\cL^r(\bR^n)} \leqslant c(r,n) \Vert D_\ell[\bu(\cdot,t) \cdot \nabla\bu(\cdot,t)] \Vert_{\cL^r(\bR^n)}.  
        \]
        No nosso, caso temos
        \[
            \Vert D_\ell\BQ(\cdot,s) \Vert_{\cL^\frac{4}{3}(\bR^3)} \leqslant c \Vert D_{\ell} [\bu(\cdot,s) \cdot \nabla \bu(\cdot,s)] \Vert_{\cL^{\frac{4}{3}}(\bR^3)}
        \]
    }\!Dito isso, chegamos a
    \[
        \begin{aligned}
            J_{i\ell}(t) &\leqslant c(\nu) \, t \int_{t'}^t (t - s)^{-\frac{7}{8}} \Vert D_\ell [\bu(\cdot,s) \cdot \nabla \bu(\cdot,s)] \Vert_{\cL^\frac{4}{3}(\bR^3)} \,ds\\
            &\leqslant c(\nu) \, t \int_{t'}^t (t - s)^{-\frac{7}{8}} \left( \Vert D_{\ell} \bu(\cdot,s) \cdot \nabla \bu(\cdot,s) \Vert_{\cL^\frac{4}{3}(\bR^3)} + \Vert \bu(\cdot,s) \cdot \nabla D_\ell \bu(\cdot,s) \Vert_{\cL^\frac{4}{3}(\bR^3)} \right) \,ds. 
        \end{aligned}
    \]
    Utilizando a desigualdade de Hölder, obtemos
    \[
        J_{i\ell}(t) \leqslant c(\nu) \, t \! \int_{t'}^t \! (t - s)^{-\frac{7}{8}} \Big( \Vert D\bu(\cdot,s) \Vert_{\cL^4(\bR^3)} \Vert D\bu(\cdot,s) \Vert_{\cL^2(\bR^3)}+ \Vert \bu(\cdot,s) \Vert_{\cL^4(\bR^3)} \Vert D^2\bu(\cdot,s) \Vert_{\cL^2(\bR^3)} \Big) ds,
    \]
    e, pela desigualdade de Gagliardo-Nirenberg (\ref{eq:gagliardoL4}), podemos escrever
    \[
        \begin{aligned}
            J_{i\ell}(t) \leqslant c(\nu) \, t \int_{t'}^t  (t - s)^{-\frac{7}{8}} \Big( \Vert D\bu(\cdot,s) &\Vert_{\cL^2(\bR^3)}^{\frac{5}{4}} \Vert D^2\bu(\cdot,s) \Vert_{\cL^2(\bR^3)}^{\frac{3}{4}} +\\ \Vert \bu(\cdot,s) &\Vert_{\cL^2(\bR^3)}^{\frac{1}{4}} \Vert D\bu(\cdot,s) \Vert_{\cL^2(\bR^3)}^{\frac{3}{4}} \Vert D^2\bu(\cdot,s) \Vert_{\cL^2(\bR^3)} \Big) \, ds.
        \end{aligned}
    \]
    Reescrevendo a desigualdade acima, concluímos que
    \[
        \begin{aligned}
            J_{i\ell}(t) \leqslant c(\nu) \, t \int_{t'}^t s^{-\frac{11}{8}} &(t - s)^{-\frac{7}{8}} \Big( \big(s^{\frac{1}{2}}\Vert D\bu(\cdot,s) \Vert_{\cL^2(\bR^3)} \big)^{\frac{5}{4}} \big( s\Vert D^2\bu(\cdot,s) \Vert_{\cL^2(\bR^3)}\big)^{\frac{3}{4}}\\ &+ \Vert \bu(\cdot,s) \Vert_{\cL^2(\bR^3)}^{\frac{1}{4}} \big( s^{\frac{1}{2}}\Vert D\bu(\cdot,s) \Vert_{\cL^2(\bR^3)}\big)^{\frac{3}{4}} \big(s\Vert D^2\bu(\cdot,s) \Vert_{\cL^2(\bR^3)}\big) \Big) \, ds.
        \end{aligned}
    \]
    Fazendo as substituições adequadas, por (\ref{eq:C1}) e pela desigualdade de energia (\ref{eq:desigualdade-de-energia}), segue que
    \[
        \begin{aligned}
            J_{i\ell}(t) \leqslant c(\nu) \,t M_1^{\frac{5}{4}} \left( \frac{t + t_0}{2} \right)^{-\frac{11}{8}} \int_{t'}^t &(t - s)^{-\frac{7}{8}} U_2(s)^{\frac{3}{4}} \,ds\\ 
            &+ c(\nu) \, t M_1^{\frac{3}{4}} \Vert \bu_0 \Vert^{\frac{1}{4}}_{\cL^2(\bR^3)} \left( \frac{t + t_0}{2} \right)^{-\frac{11}{8}} \int_{t'}^{t} (t - s)^{-\frac{7}{8}} U_2(s) \,ds,
        \end{aligned}
    \]
    já que $s \geqslant t$ nas integrais acima. 
    Utilizando a desigualdade de Young ($ab \leqslant a^pp^{-1} + b^q q^{-1}$, onde $p^{-1} + q^{-1} = 1$, com $a = 1$, $b = U_2(s)^{\frac{3}{4}}$, $p = 4$ e $q = \frac{4}{3}$), temos que
    \begin{equation} \label{eq:youngu2}
        U_2(s)^{\frac{3}{4}} \leqslant 1 + U_2(s).
    \end{equation}
    Dessa forma, chegamos a
    \[
        \begin{aligned}
            J_{i\ell}(t) \leqslant c(\nu, M_1) \,t (t + t_0)^{-\frac{11}{8}} \int_{t'}^t &(t - s)^{-\frac{7}{8}} \,ds + \Big( c(\nu, M_1) \,t (t + t_0)^{-\frac{11}{8}}\\ 
            &+ c(\nu, M_1, \bu_0) \, t (t + t_0)^{-\frac{11}{8}} \Big) \int_{t'}^t (t - s)^{-\frac{7}{8}} U_2(s) \,ds.
        \end{aligned}
    \]
    Para simplificar a expressão à direita acima, note que
    \begin{equation} \label{eq:mais1}
         t(t + t_0)^{-\frac{11}{8}} = t(t + t_0)^{-1} (t+t_0)^{-\frac{3}{8}} = \frac{t}{t + t_0} (t + t_0)^{-\frac{3}{8}} \leqslant (t + t_0)^{-\frac{3}{8}},
    \end{equation}
    \begin{equation} \label{eq:mais2}
        \int_{t'}^t (t-s)^{-\frac{7}{8}} \,ds = 8 \left( \frac{t - t_0}{2} \right)^{\frac{1}{8}}
    \end{equation}
    e também que
    \begin{equation} \label{eq:mais3}
        (t + t_0)^{-\frac{3}{8}} (t - t_0)^{\frac{1}{8}} = (t + t_0)^{-\frac{1}{4}} (t + t_0)^{-\frac{1}{8}} (t - t_0)^{\frac{1}{8}} = (t + t_0)^{-\frac{1}{4}} \left( \frac{t - t_0}{t+t_0} \right)^{\frac{1}{8}} \leqslant (t + t_0)^{-\frac{1}{4}}.
    \end{equation}
    Dito isso, chegamos a
    \begin{equation} \label{eq:W3}
        J_{i\ell}(t) \leqslant c(\nu, M_1) (t + t_0)^{-\frac{1}{4}} + c(\nu,M_1, \bu_0) (t+t_0)^{-\frac{3}{8}} \int_{t'}^t (t - s)^{-\frac{7}{8}} U_2(s)\,ds,
    \end{equation}
    para todo $i, \ell \in \bN$ e $t \geqslant t_0 + 1$. Dessa forma, por (\ref{eq:W1}), (\ref{eq:W2}) e (\ref{eq:W3}), inferimos
    \begin{equation} \label{eq:4.13}
        U_2(t) \leqslant c_* (k, \nu, t_0, M_1, \bu_0) + c_{**} (\nu, M_1, \bu_0)(t + t_0)^{-\frac{3}{8}} \int_{t'}^t (t-s)^{-\frac{7}{8}} U_2(s) \,ds,
    \end{equation}
    para todo $ t \geqslant t_0 + 1$.
    Considere agora, $t_2$ e $\mathbf{M}_2$ dados por
    \[
        t_2 := 1 + t_0 + 2^{16}c_{**}^4 \;\text{ e }\; \mathbf{M}_2 := \sup \{ U_2(s) : t_0 \leqslant  s \leqslant t_2\}.
    \]
    Afirmamos que
    \begin{equation} \label{eq:W5}
        U_2(t) \leqslant 2 c_* (k,\nu,t_0,M_1,\bu_0) + 16 c_{**}(\nu,M_1,\bu_0) \, \mathbf{M}_2,
    \end{equation}
    para todo $t\geqslant t_2$.
    De fato, definindo
    \[
        \mathbf{U}_2(t) := \sup\{ U_2(s): t_2 \leqslant s \leqslant t\}.
    \]
    Se $t' \geqslant t_2$, então, por (\ref{eq:4.13}), (\ref{eq:mais2}) e (\ref{eq:mais3}), deduzimos que
    \[
        \begin{aligned}
            U_2(t) &\leqslant c_* (k, \nu, t_0, M_1, \bu_0) + c_{**} (\nu,M_1, \bu_0) \, \mathbf{U}_2(t) (t + t_0)^{-\frac{3}{8}} \int_{t'}^t (t -s)^{-\frac{7}{8}} \,ds\\
            &\leqslant c_* (k, \nu, t_0, M_1, \bu_0) + 8c_{**} (\nu,M_1, \bu_0) (t + t_0)^{-\frac{1}{4}}\mathbf{U}_2(t). 
        \end{aligned}
    \]
    Porém, como $t \geqslant t_2$, temos, pela definição de $t_2$, que
    \[
        t + t_0 > t_2 > 2^{16} c_{**}^4.
    \]
    Isso implica em $2^{16} c_{**}^4 < t + t_0$, e elevando ambos os lados a $1/4$ obtemos $16 c_{**} < (t + t_0)^{\frac{1}{4}}$, o qual podemos reescrever como $8c_{**} (t + t_0)^{-\frac{1}{4}} < 1/2$. Dito isso, chegamos a
    \[
        U_2(t) < c_*(k,\nu,t_0,M_1, \bu_0) + \frac{1}{2} \mathbf{U}_2(t).
    \]

    Por outro lado, se $t'< t_2$, podemos reescrever (\ref{eq:4.13}) da seguinte forma:
    \[
        U_2(t) \leqslant c_* + c_{**} (t + t_0)^{-\frac{3}{8}} \left( \int_{t'}^{t_2} (t - s)^{-\frac{7}{8}} U_2(s) \,ds + \int_{t_2}^t (t - s)^{-\frac{7}{8}}U_2(s) \,ds \right).
    \]
    Observe que em $[t',t_2]$, é verdade que $U_2(s) \leqslant \mathbf{M}_2 \leqslant \mathbf{M}_2 + \mathbf{U}_2(s)$, por outro lado em $[t_2, t]$, temos que $U_2(s) \leqslant \mathbf{U}_2(s) \leqslant \mathbf{M}_2 + \mathbf{U}_2(s)$.
    Dessa forma
    \[
        \begin{aligned}
            U_2(t) &\leqslant c_* + c_{**} (t + t_0)^{-\frac{3}{8}} \, \mathbf{M}_2\int_{t'}^{t} (t - s)^{-\frac{7}{8}}\,ds + c_{**} (t + t_0)^{-\frac{3}{8}}\,\mathbf{U}_2(t)\int_{t'}^{t} (t - s)^{-\frac{7}{8}}\,ds\\
            &\leqslant c_* + 8c_{**} (t+ t_0)^{-\frac{1}{4}}\, \mathbf{M}_2 + 8 c_{**}(t+t_0)^{-\frac{1}{4}}\mathbf{U}_2(t).
        \end{aligned}
    \]
    Como $(t + t_0)^{-\frac{1}{4}} \leqslant 1$ (pois, $t \geqslant 1$) e $8c_{**} (t+t_0)^{-\frac{1}{4}} < \frac{1}{2}$ (como foi feito acima), segue que
    \[
        U_2(t) \leqslant c_* + 8c_{**} \mathbf{M}_2 + \frac{1}{2} \mathbf{U}_2(t),
    \]
    para todo $t \geqslant t_2$.
    Com isso,
    \[
        U_2(s) \leqslant c_* + 8c_{**} \mathbf{M}_2 + \frac{1}{2} \mathbf{U}_2(s) \leqslant c_* + 8c_{**}\mathbf{M}_2 + \frac{1}{2}\mathbf{U}_2(t),
    \]
    para todo $s \in [t_2,t]$.
    Pasando ao supremo em $[t_0,t]$ na desigualdade acima, inferimos
    \[
        \mathbf{U}_2(t) \leqslant c_* + 8c_{**} \mathbf{M}_2 + \frac{1}{2} \mathbf{U}_2(t),
    \]
    para todo $t \geqslant t_2$. Isto prova (\ref{eq:W5}).
    Ou seja, mostramos que $U_2(t)$ é limitado por uma constante que não depende de $t$ em $[t_2, \infty)$. Portanto, $U_2 \in \cL^\infty([t_2, \infty])$. Porém, sabemos que em $[T_{**}, \infty)$ e em particular, em $[T_{**}, t_2]$, $\bu(\cdot,t)$ é suave (ver \cite{leray-fluid}), o que implica em $U_2$ ser suave em um compacto, consequentemente limitada.
    Deste modo, $U_2 \in \cL^\infty([T_{**},\infty))$, como era desejado. Assim, provamos o caso $k = 2$.

    Suponha que $U_{\ell} \in \cL^\infty([T_{**},\infty))$, para todo $\ell < k$. Vamos mostrar que $U_k \in \cL^\infty([T_{**},\infty))$. Dito isso, seja $\alpha$ um multi-índice de ordem $k$.
    Denotando $D^\alpha$ por $D_i D^{\gamma}$ (onde $\gamma$ é um multi-índice de ordem $k-1$), temos, de forma análoga ao que foi feito no caso $k = 2$, que
    \[
        J_\alpha(t) = t^{\frac{k}{2}} \int_{t'}^t \Vert D^\alpha [ e^{\nu \Delta (t - s)}\BQ(\cdot,s)] \Vert_{\cL^2(\bR^3)} \,ds \leqslant c(\nu) \, t^{\frac{k}{2}} \int_{t'}^t (t - s)^{-\frac{7}{8}} \Vert D^{\gamma} \BQ(\cdot,s) \Vert_{\cL^\frac{4}{3}(\bR^3)} \,ds.
    \]
    \textcolor{red}{Além disso por Calderon-Zygmund
    \[
        \Vert D^{\gamma}\BQ(\cdot,t) \Vert_{\cL^r(\bR^n)} \leqslant c(r,n) \Vert D^{\gamma} [ \bu(\cdot,t) \cdot \nabla \bu(\cdot,t) ] \Vert_{\cL^r(\bR^n)},
    \]
    para cada $1 < r < \infty$ e qualquer multi-índice $\gamma$}. Dessa forma, encontramos
    \[
        J_\alpha(t) \leqslant c(\nu)\, t^{\frac{k}{2}} \int_{t'}^t (t-s)^{-\frac{7}{8}} \Vert D^{\gamma} [\bu(\cdot,s) \cdot \nabla \bu(\cdot,s)] \Vert_{\cL^\frac{4}{3}(\bR^3)}\,ds.
    \]
    Utilizando a regra de Leibniz (ver Teorema \ref{thm:propriedades-derivada-fraca}), obtemos
    \[
        J_\alpha(t) \leqslant c(\nu) t^{\frac{k}{2}} \sum_{\sigma \leqslant \gamma} \int_{t'}^t (t - s)^{-\frac{7}{8}} \Vert D^{\sigma} \bu(\cdot,s) \cdot \nabla D^{\gamma - \sigma} \bu(\cdot,s) \Vert_{\cL^{\frac{4}{3}}(\bR^3)}
    \]
    e, utilizando a Desigualdade de Hölder (ver Teorema \ref{thm:pre-desigualdade-de-holder}), concluímos que
    \[
        J_\alpha(t) \leqslant c(\nu) \sum_{\ell = 0}^{k-1} t^{\frac{k}{2}} \int_{t'}^t (t- s)^{-\frac{7}{8}} \Vert D^\ell \bu(\cdot,s) \Vert_{\cL^4(\bR^3)} \Vert D^{k-\ell} \bu(\cdot,s) \Vert_{\cL^2(\bR^3)}\,ds.
    \]
    Deste modo, pela desigualdade de Gagliardo-Nirenberg (\ref{eq:gagliardoL4}), segue que
    \[
        J_\alpha(t) \leqslant c(\nu) \sum_{\ell=0}^{k-1} t^{\frac{k}{2}} \int_{t'}^t ( t- s)^{-\frac{7}{8}} \Vert D^\ell \bu(\cdot,s) \Vert_{\cL^2(\bR^3)}^{\frac{1}{4}} \Vert D^{\ell + 1} \bu(\cdot,s) \Vert_{\cL^2(\bR^3)}^{\frac{3}{4}} \Vert D^{k-\ell} \bu(\cdot,s) \Vert_{\cL^2(\bR^3)} \,ds.
    \]
    Escrevendo $J_\alpha(t)$ como a soma $J_1(t) + J_2(t) + J_3(t)$, onde
    \[
        J_1(t) = c(\nu) t^{\frac{k}{2}} \int_{t'}^t ( t- s)^{-\frac{7}{8}} \Vert \bu(\cdot,s) \Vert_{\cL^2(\bR^3)}^{\frac{1}{4}} \Vert D \bu(\cdot,s) \Vert_{\cL^2(\bR^3)}^{\frac{3}{4}} \Vert D^{k} \bu(\cdot,s) \Vert_{\cL^2(\bR^3)} \,ds,
    \]
    \[
        J_2(t) = c(\nu) \sum_{\ell=1}^{k-2} t^{\frac{k}{2}} \int_{t'}^t ( t- s)^{-\frac{7}{8}} \Vert D^\ell \bu(\cdot,s) \Vert_{\cL^2(\bR^3)}^{\frac{1}{4}} \Vert D^{\ell + 1} \bu(\cdot,s) \Vert_{\cL^2(\bR^3)}^{\frac{3}{4}} \Vert D^{k-\ell} \bu(\cdot,s) \Vert_{\cL^2(\bR^3)} \,ds,
    \]
    \[
        J_3(t) = c(\nu) t^{\frac{k}{2}} \int_{t'}^t ( t- s)^{-\frac{7}{8}} \Vert D^{k-1} \bu(\cdot,s) \Vert_{\cL^2(\bR^3)}^{\frac{1}{4}} \Vert D^{k} \bu(\cdot,s) \Vert_{\cL^2(\bR^3)}^{\frac{3}{4}} \Vert D \bu(\cdot,s) \Vert_{\cL^2(\bR^3)} \,ds.
    \]
    Como sabemos que $\Vert D^q \bu(\cdot,s) \Vert_{\cL^2(\bR^3)} = s^{-\frac{q}{2}} U_q(s) \leqslant s^{-\frac{q}{2}} M_q$, para todo $q < k$ (por hipótese de indução) e $\Vert D^k\bu(\cdot,s) \Vert_{\cL^2(\bR^3)} = s^{-\frac{k}{2}} U_k(s)$, obtemos
    \begin{equation} \label{eq:t1}
        \begin{aligned}
            J_1(t) &\leqslant c(\nu) \, M_1^{\frac{3}{4}} \Vert \bu_0 \Vert_{\cL^2(\bR^3)}^{\frac{1}{4}} \, t^{\frac{k}{2}} \int_{t'}^t (t - s)^{-\frac{7}{8}} s^{-\left( \frac{3}{8} + \frac{k}{2} \right)} \,U_k(s) \,ds\\
            &\leqslant c(\nu,k) M_1^{\frac{3}{4}} \Vert \bu_0 \Vert_{\cL^2(\bR^3)}^{\frac{1}{4}} (t + t_0)^{-\frac{3}{8}} \int_{t'}^t (t- s)^{-\frac{7}{8}} U_k(s) \,ds, 
        \end{aligned}
    \end{equation}
    onde utilizamos o fato de
    \begin{equation} \label{eq:BB1}
        \begin{aligned}
            t^\frac{k}{2}s^{-\left( \frac{3}{8} + \frac{k}{2} \right)} \leqslant t^\frac{k}{2}\left( \frac{t + t_0}{2} \right)^{-\left( \frac{3}{8} + \frac{k}{2} \right)} &= c(k)\,t^{\frac{k}{2}} (t + t_0)^{-\left( \frac{3}{8} + \frac{k}{2} \right)} \\
            &= c(k)\left( \frac{t}{t + t_0} \right)^{\frac{k}{2}} (t + t_0)^{-\frac{3}{8}} \leqslant c(k) (t + t_0)^{-\frac{3}{8}},
        \end{aligned}
    \end{equation}
    para a última desigualdade.
    Analogamente, deduzimos, por (\ref{eq:BB1}), (\ref{eq:mais2}) e (\ref{eq:mais3}) que
    \begin{equation} \label{eq:t2}
        \begin{aligned}
            J_2(t) &\leqslant c(\nu) \sum_{\ell=1}^{k-2} M_{\ell}^{\frac{1}{4}} M_{\ell + 1}^{\frac{3}{4}} M_{k-\ell} \, t^{\frac{k}{2}} \int_{t'}^t (t - s)^{-\frac{7}{8}} s^{-\left( \frac{3}{8} + \frac{k}{2} \right)} \,ds\\
            &\leqslant c(\nu,k) \sum_{\ell=1}^{k-2} M_{\ell}^{\frac{1}{4}} M_{\ell + 1}^{\frac{3}{4}} M_{k-\ell} \, (t + t_0)^{-\frac{3}{8}} \int_{t'}^t (t - s)^{-\frac{7}{8}} \,ds\\
            &\leqslant c(\nu,k) \sum_{\ell=1}^{k-2} M_{\ell}^{\frac{1}{4}} M_{\ell + 1}^{\frac{3}{4}} M_{k-\ell} \, (t + t_0)^{-\frac{1}{4}},
        \end{aligned}
    \end{equation}
    e também
    \begin{equation} \label{eq:t3}
        \begin{aligned}
            J_3(t) &\leqslant c(\nu) M_1 M_{k-1}^{\frac{1}{4}} \, t^{\frac{k}{2}} \int_{t'}^{t} (t - s)^{-\frac{7}{8}} s^{-\left( \frac{3}{8} + \frac{k}{2} \right)} U_k(s)^{\frac{3}{4}} \,ds\\
            &\leqslant c(\nu,k) M_1 M_{k-1}^{\frac{1}{4}} (t + t_0)^{-\frac{3}{8}} \int_{t'}^t (t - s)^{-\frac{7}{8}} U_k(s)^{\frac{3}{4}}\,ds.
        \end{aligned}
    \end{equation}
    Lembrando que $U_k(s)^{\frac{3}{4}} \leqslant 1 + U_k(s)$ (análogo à (\ref{eq:youngu2})), segue, por (\ref{eq:4.10}), (\ref{eq:t1}), (\ref{eq:t2}) e (\ref{eq:t3}), que
    \[
        U^\alpha(t) \leqslant c(k,\nu, t_0, \bu_0, M_1,\dots,M_{k-1}) + c(k,\nu,\bu_0, M_1) (t + t_0)^{-\frac{3}{8}} \int_{t'}^t (t-s)^{-\frac{7}{8}} U_k(s)\,ds,
    \]
    onde $U^\alpha(t) = t^{\frac{k}{2}} \Vert D^\alpha \bu(\cdot,t) \Vert_{\cL^2(\bR^3)}$.
    Portanto, chegamos a
    \[
        U_k(t) \leqslant c_*(k,\nu, t_0, \bu_0, M_1,\dots,M_{k-1}) + c_{**}(k,\nu,\bu_0, M_1) (t + t_0)^{-\frac{3}{8}} \int_{t'}^t (t-s)^{-\frac{7}{8}} U_k(s)\,ds,
    \]
    para todo $t \geqslant t_0 + 1$.
    De forma análoga ao caso $k = 2$, definimos $t_k$ e $\mathbf{M}_k$ por
    \[
        t_k := 1 + t_0 + 2^{16}c_{**}^4 \;\text{ e }\; \mathbf{M}_k := \sup \{U_k(s) \,; t_0 \leqslant s \leqslant t_k\}.
    \]
    Sendo assim, afirmamos novamente que
    \begin{equation} \label{eq:B6}
        U_k(t) \leqslant 2c_*(k,\nu,t_0,\bu_0,M_1,\dots,M_{k-1}) + 16 c_{**} (k,\nu,\bu_0,M_1) \mathbf{M}_k,
    \end{equation}
    para todo $t \geqslant t_k$.
    Essa demonstração é exatamente a mesma que foi feita para o caso $k = 2$.
    Ou seja, temos que $U_k$ é limitada por uma constante que não depende de $t$ em $[t_k,\infty)$, como $U_k$ é limitada em $[T_{**}, \infty)$, segue que $U_k \in \cL^\infty([T_{**}, \infty))$, como era desejado.
\end{prf}

O resultado abaixo é uma generalização dos Teoremas \ref{thm:Duto0} e \ref{thm:problema-leray}.

\begin{tbox} \label{thm:leraygen}
    Seja $\bu(\cdot,t)$ uma solução de Leray para (\ref{eq:navierstokes}), então, para todo $k \geqslant 0$ inteiro tem-se
    \[
        t^{\frac{k}{2}} \Vert D^k\bu(\cdot,t) \Vert_{\cL^2(\bR^3)} \to 0,
    \]
    quando $t \to \infty$.
\end{tbox}
\begin{prf}
    Os Teoremas \ref{thm:problema-leray} e \ref{thm:Duto0} mostram os casos $k = 0$ e $k = 1$, respectivamente. Sendo assim, considere multi-índice $\alpha$ com $|\alpha| = k \geqslant 2$ inteiro e $t_0 \geqslant T_{**}$.
    Por \ref{eq:2212}, obtemos
    \[
        \begin{aligned}
            \Vert D^\alpha \bu(\cdot,t) \Vert_{\cL^2(\bR^3)} \leqslant \Vert &D^\alpha[e^{\nu \Delta(t-t_0)} \bu(\cdot,t_0)] \Vert_{\cL^2(\bR^3)}\\  &+ \int_{t_0}^{t'} \Vert D^\alpha[e^{\nu \Delta (t-s)} \BQ(\cdot,s)] \Vert_{\cL^2(\bR^3)} \,ds
        + \int_{t'}^{t} \Vert D^\alpha[e^{\nu \Delta (t-s)} \BQ(\cdot,s)] \Vert_{\cL^2(\bR^3)} \,ds,
        \end{aligned}
    \]
    para todo $t \geqslant t_0$.
    Na demonstração do Lema \ref{lm:uk} (ver (\ref{eq:BB2}) e (\ref{eq:xxxx})), vimos que dado um multi-índice $\alpha$ com $|\alpha| = k$, podemos escrever
    \begin{equation} \label{eq:T1}
        t^{\frac{k}{2}}\Vert D^\alpha [e^{\nu\Delta(t-t_0)}\bu(\cdot,t_0)] \Vert_{\cL^2(\bR^3)} \leqslant c \left( 1 + \frac{t_0}{t - t_0} \right)^{\frac{k}{2}} \to 0,
    \end{equation}
    quando $t \to \infty$.
    Por outro lado, temos por (\ref{eq:B3}) que
    \[
        t^{\frac{k}{2}}\int_{t_0}^{t'} \Vert D^\alpha [e^{\nu\Delta (t-s)}\BQ(\cdot,s)] \Vert_{\cL^2(\bR^3)} \,ds \leqslant c(k,\nu) \Vert \bu_0 \Vert_{\cL^2(\bR^3)} (t - t_0)^{-\frac{3}{4}} (t + t_0)^{\frac{1}{2}},
    \]
    para todo multi-índice $\alpha$ com $|\alpha| = k$.
    Note que, quando $t \to \infty$, o termo $(t - t_0)^{-\frac{3}{4}}$ decai na ordem de $t^{\frac{3}{4}}$, equanto $(t + t_0)^{\frac{1}{2}}$ cresce na ordem de $t^{\frac{1}{2}}$.
    Dessa forma, como $\frac{3}{4} > \frac{1}{2}$, concluímos que
    \begin{equation} \label{eq:T2}
        \int_{t_0}^{t'} \Vert D^\alpha [e^{\nu\Delta (t-s)}\BQ(\cdot,s)] \Vert_{\cL^2(\bR^3)} \,ds  \to 0,
    \end{equation}
    quando $t \to \infty$. 
    Por fim, temos, por (\ref{eq:Jalfa}), (\ref{eq:t1}), (\ref{eq:t2}) e (\ref{eq:t3}), que
    \begin{equation} \label{eq:T3}
        \begin{aligned}
            t^{\frac{k}{2}} \int_{t'}^t \Vert D^\alpha[e^{\nu\Delta(t-s)}\BQ(\cdot,s)] \Vert_{\cL^2(\bR^3)} \,ds \leqslant c (t + t_0)^{-\frac{1}{4}} \to 0, \\
            % &\leqslant c(k,\nu) \left( \sum_{\ell=1}^k M_\ell^{\frac{1}{4}} M_{\ell+1}^{\frac{3}{4}} M_{k-\ell} \right) (t + t_0)^{-\frac{1}{4}} \to 0
        \end{aligned}
    \end{equation}
    quando $t \to \infty$, para todo multi-índice $\alpha$ com $|\alpha| \leqslant k$. Logo, por (\ref{eq:T1}), (\ref{eq:T2}) e (\ref{eq:T3}), dado um multi-índice $\alpha$ com $|\alpha| = k$, temos que
    \[
        t^{\frac{k}{2}} \Vert D^\alpha \bu(\cdot,t) \Vert_{\cL^2(\bR^3)} \to 0,
    \]
    quando $t \to \infty$.
    Portanto, chegamos a
    \[
        t^{\frac{k}{2}} \Vert D^k \bu(\cdot,t) \Vert_{\cL^2(\bR^3)} = \left( \sum_{\ell_1=1}^{3} \cdots \sum_{\ell_k=1}^3 t^{\frac{k}{2}} \Vert D_{\ell_1} \cdots D_{\ell_k} \bu(\cdot,t) \Vert_{\cL^2(\bR^3)}^2 \right)^{\frac{1}{2}} \to 0,
    \]
    quando $t \to \infty$, pois, $D_{\ell_1} \cdots D_{\ell_k} = D^\alpha$ para algum multi-índice $\alpha$ com $|\alpha| = k$.
\end{prf}

O Teorema \ref{thm:leraygen} mostra um decaimento para as soluções de Leray na norma do espaço $H^k(\bR^n) = \cW^{k,2}(\bR^n)$, já que
\[
    \Vert \bu(\cdot,t) \Vert_{H^k(\bR^n)}^2 = \sum_{\ell = 0}^k \Vert D^\ell\bu(\cdot,t) \Vert_{\cL^2(\bR^n)}^2.
\]
Mais precisamente
\[
    \Vert \bu(\cdot,t) \Vert_{H^k(\bR^n)}^2 = \sum_{\ell = 0}^k \big[ t^{-\frac{\ell}{2}} t^{\frac{\ell}{2}} \Vert D^\ell \bu(\cdot,t) \Vert_{\cL^2(\bR^n)} \big]^2 = \sum_{\ell = 0}^k t^{-\ell} \big[ t^{\frac{\ell}{2}} \Vert D^\ell \bu(\cdot,t) \Vert_{\cL^2(\bR^n)} \big]^2 \to 0,
\]
quando $t \to \infty$.

\section{Problema de Dirichlet} \label{sec:dirichlet}

Nessa seção, retornamos ao problema visto na motivação do capítulo anterior.

O ponto principal dessa seção é o Teorema de Lax-Milgram, o qual será utilizado para mostrar a existência de soluções fracas para o problema de Dirichet.
Para demonstrá-lo, será necessário apresentar alguns resultados de análise funcional
\begin{tbox} \label{thm:hilbert1}
    Seja $f : X \to \bR$ um operador linear limitado, onde $X$ é um espaço vetorial normado. Então, são válidas as seguintes afirmações:
    \begin{enumerate}[leftmargin=*, label=\textbf{(\alph*)}]
        \item $f$ é continuo;
        \item $\ker f$ é fechado, onde $\ker f = \{x \in X \,; f(x) = 0\}$ é o núcleo de $f$.
    \end{enumerate}
\end{tbox}
\begin{prf}
    Ver \cite{kreyszig-functional.analysis}, p.p. 97 e 98.
\end{prf}

Um outro resultado resultado importante diz respoeito à soma direta de subespaços.

\begin{tbox} \label{thm:hilbert2}
    Se $E$ é um subespaço fechado de um espaço de Hilbert $H$, então
    \[
        H = E \oplus E^\perp.
    \]
\end{tbox}
\begin{prf}
    Ver \cite{cesar-analise.funcional}, p.p. 129.
\end{prf}

O Teorema da representação de Hilbert é o resultado mais importante para a demonstração do Teorema de Lax-Milgram, este será enunciado e demonstrado abaixo.

\begin{tbox}[Teorema da representação de Riesz para espaços de Hilbert] \label{thm:representacao-riesz}
    Sejam $H$ um espaço de Hilbert e $f : H \to \bR$ um funcional linear limitado.
    Então, existe um único $v \in H$ tal que
    \[
        f(u) = \left\langle u, v\right\rangle,
    \]
    para todo $u \in H$.
    Além disso, $\Vert f \Vert = \Vert v \Vert$.
\end{tbox}
\begin{prf}
    Primeiramente, mostremos a existência de $v \in H$ tal que
    \[
        f(u) = \left\langle u,v \right\rangle,
    \]
    para todo $u \in H$.
    Seja $f : H \to \bR$ um funcional linear limitado não nulo (se $f \equiv 0$, basta escolher $v = 0$), então existe $z \in H$ tal que $f(z) \neq 0$. 
    Como $f$ é linear, $z$ deve ser não nulo. Ou seja, $\ker f \neq H$.
    Como $f$ é limitado, segue, pelo Teorema \ref{thm:hilbert1} \textbf{(b)}, que $\ker f$ é fechado. Sendo assim, pelo Teorema \ref{thm:hilbert2}, $H = \ker f \oplus \ker f ^\perp$.
    Portanto, $\ker f ^\perp \neq \{0\}$ (pois, caso contrário, $\ker f = H$).
    Dito isso, seja $w \in \ker f ^\perp$ não nulo, então, para cada $u \in H$, $f(u) w - f(w)u \in \ker f$; já que,
    \[
        f(f(u)w - f(w)u) = f(u) f(w) - f(w)f(u) = 0.
    \]
    Dessa forma, é verdade que
    \[
        0 = \left\langle f(u)w - f(w)u, w \right\rangle = f(u) \left\langle w,w \right\rangle - f(w) \left\langle u,w \right\rangle,
    \]
    ou seja,
    \[
        f(u) = \left\langle u, \frac{f(w)}{\Vert w \Vert^2}w \right\rangle.
    \]
    Escolhendo $v := \dfrac{f(w)}{\Vert w \Vert^2} w \in H$ (note que $v$ não depende de $u$), temos que existe $v \in H$ tal que $f(u) = \left\langle u,v \right\rangle$, para todo $u \in H$.

    Para mostrar que $v$ é único, suponha que exista $\tilde v \in H$ tal que $f(u) = \left\langle u, \tilde v \right\rangle$, para todo $u \in H$.
    Dito isso, podemos escrever
    \[
        \left\langle u, v \right\rangle = f(u) = \left\langle u,\tilde v \right\rangle,
    \] 
    para todo $u \in H$. Logo, é verdade que
    \[
        \left\langle u , v - \tilde v \right\rangle = 0,
    \]
    para todo $u \in H$, Em particular, se $u = v - \tilde v \in H$, temos $\Vert v - \tilde v \Vert^2 = 0$, ou seja, $v = \tilde v$. Portanto, $v$ é único.

    Por fim, resta mostrar que $\Vert f \Vert = \Vert v \Vert$.
    Com efeito, note que $f(v) = \left\langle v, v \right\rangle = \Vert v \Vert^2$. Ou seja,
    \[
        \Vert v \Vert^2 = f(v) \leqslant |f(v)| \leqslant \Vert f \Vert \Vert v \Vert.
    \]
    Como $v$ é não nulo, temos $\Vert v \Vert \leqslant \Vert f \Vert$.
    Por outro lado, pela desigualdade de Cauchy-Schwarz, inferimos
    \[
        |f(u)| = |\left\langle u,v \right\rangle| \leqslant \Vert u \Vert \Vert v \Vert,
    \]
    para todo $u \in H$,
    o que implica em $\Vert f \Vert \leqslant \Vert v \Vert$.
    Portanto, vale a igualdade:
    \[
        \Vert f \Vert = \Vert v \Vert.
    \]
\end{prf}

Agora, estamos prontos para enunciar e demostrar o Teorema de Lax-Milgram.

\begin{tbox}[Teorema de Lax-Milgram] \label{thm:lax-milgram}
    Sejam $H$ um espaço de Hilbert, $B : H \times H \to \bR$ uma aplicação bilinear tal que existem $\alpha, \beta > 0$ tais que
    \begin{enumerate}
        \item $| B(u,v)| \leqslant \alpha \Vert u \Vert \Vert v \Vert$, para todo $u,v \in H$, i.e., a forma bilinear é limitada;
        \item $\beta \Vert u \Vert^2 \leqslant B(u,u)$, para todo $u \in H$, i.e., a forma bilinear é coerciva,
    \end{enumerate}
    e $f : H \to \bR$ um funcional linear limitado. Então, existe um único $v \in H$ tal que
    \[
        B(u,v) = f(u),
    \]
    para todo $u \in H$.
\end{tbox}
\begin{prf}
    Essa demonstração sera dividida em seis passos.

    \underline{Passo 1:} Mostremos que existe uma aplicação $A : H \to H$ tal que $B(u,v) = \left\langle u, Av \right\rangle$, para todo $u,v \in H$.

    Para cada $v \in H$ defina $g_v : H \to \bR$ por
    \[
        g_v(u) = B(u,v),
    \]
    para todo $u \in H$.
    Claramente $g_v$ é linear (pois, $B$ é uma forma bilinear).
    Além disso, $g_v$ é limitada, já que
    \[
        |g_v(u)| = |B(u,v)| \leqslant \alpha \Vert u \Vert \Vert v \Vert,
    \]
    para todo $u \in H$, onde utilizamos o fato de $B$ ser limitada (ver item 1).
    Sendo assim,
    \[
        \Vert g_v \Vert \leqslant \alpha \Vert v \Vert.
    \]
    Logo, $g_v$ é limitada.
    Dessa forma, pelo Teorema da representação de Riesz (ver Teorema \ref{thm:representacao-riesz}), existe um único $w_v \in H$ tal que
    \[
        B(u,v) = g_v(u) = \left\langle u, w_v \right\rangle,
    \]
    para todo $u \in H$.
    Agora, seja $A : H \to H$ dado por $Av = w_v$.
    Dessa forma, podemos escrever
    \begin{equation} \label{eq:anmm}
        B(u,v) = g_v(u) = \left\langle u,w_v \right\rangle = \left\langle u, Av \right\rangle,
    \end{equation}
    para todo $u,v \in H$.

    \underline{Passo 2:} Afirmamos que $A$ é um operador linear e limitado.

    Sejam $x, y \in H$ e $\lambda \in \bR$.
    Sendo assim, por (\ref{eq:anmm}), obtemos
    \[
        \left\langle u, A(\lambda x + y) \right\rangle = B(u, \lambda x + y) = \lambda B(u,x) + B(u,y) = \lambda \left\langle u, Ax \right\rangle + \left\langle u, Ay \right\rangle = \left\langle u, \lambda Ax + Ay \right\rangle,
    \]
    para tudo $u \in H$,
    pois $B$ é uma forma bilinear.
    Com isso, $A(\lambda x + y) = \lambda Ax + Ay$, para todo $x,y \in H$ e $\lambda \in \bR$.
    Logo, $A$ é linear.
    Além disso, dado $v \in H$, inferimos, por (\ref{eq:anmm}), que
    \[
        \Vert Av \Vert^2 = \left\langle Av, Av \right\rangle = B(Av, v) \leqslant |B(Av, v)| \leqslant \alpha \Vert Av \Vert \Vert v \Vert,
    \]
    onde utlizamos o fato de $B$ ser uma forma bilinear limitada (ver item 1). Deste modo, a seguinte desigualdade é válida:
    \[
        \Vert Av \Vert \leqslant \alpha \Vert v \Vert,
    \]
    para todo $v \in H$.
    Portanto, $A$ é limitado.

    \underline{Passo 3:} É verdade que $A$ é injetivo e $\mathrm{Im}A$ é fechado.

    Sabemos que existe $\beta > 0$ tal que $\beta \Vert u \Vert^2 \leqslant |B(u,u)|$, para todo $u \in H$ (ver item 2). Dessa forma, por (\ref{eq:anmm}), e pela desigualdade de Cauchy-Schwarz, concluímos que
    \[
        \beta \Vert u \Vert^2 \leqslant |B(u,u)| = | \left\langle u, Au \right\rangle | \leqslant \Vert u \Vert \Vert Au \Vert,
    \]
    para todo $u \in H$, ou seja,
    \begin{equation} \label{eq:uau}
        \beta \Vert u \Vert \leqslant \Vert Au \Vert,
    \end{equation}
    para todo $u \in H$.
    Seja $u \in \ker A$, então $Au = 0$. Logo, por (\ref{eq:uau}), podemos escrever
    \[
        \beta \Vert u \Vert \leqslant \Vert Au \Vert = 0,
    \]
    o que implica em $u = 0$.
    Portanto, $A$ é injetiva.

    Além disso, seja $(Au_k) \subseteq \mathrm{Im} A$ tal que $A u_k \to v \in H$, quando $k \to \infty$.
    Mostremos que $v \in \mathrm{Im}A$.
    Com efeito, $(Au_k)$ é de Cauchy (porque converge). Dito isso, por (\ref{eq:uau}) e pela linearidade de $A$, tem-se que
    \[
        \beta \Vert u_k - u_\ell \Vert \leqslant \Vert Au_k - Au_\ell \Vert \to 0,
    \]
    para todo $k, \ell \in \bN$.
    Ou seja, $(u_k) \subseteq H$ é de Cauchy.
    Como $H$ é completo, existe $u_0 \in H$ tal que $u_k \to u_0$, quando $k \to \infty$. Como $A$ é limitado, e consequentemente continuo, (ver Teorema \ref{thm:hilbert1} \textbf{(a)}), assim sendo, $Au_k \to Au_0$, quando $k \to \infty$.
    Pela unicidade do limite, deduzimos que $v = Au_0 \in \mathrm{Im}A$.
    Portanto, $\mathrm{Im}A$ é fechado.

    \underline{Passo 4:} Vamos mostrar que $A$ é sobrejetivo.

    Sabemos, pelo Teorema \ref{thm:hilbert2}, que $H = \mathrm{Im}A \oplus \mathrm{Im}A^\perp$ (pois, $\mathrm{Im}A$ é fechado). Mostremos que $\mathrm{Im}A^\perp = \{0\}$.
    Com efeito, se $w \in \mathrm{Im}A^\perp$, temos, pelo item 2 e por (\ref{eq:anmm}), que
    \[
        \beta \Vert w \Vert^2 \leqslant B(w,w) \leqslant | B(w,w) | = | \left\langle w, Aw \right\rangle | = 0,
    \]
    pois, $Aw \in \mathrm{Im}A$ e $w \in \mathrm{Im}A^\perp$.
    Logo, $w = 0$, o que implica em $H = \mathrm{Im}A$.
    Portanto, $A$ é sobrejetiva.

    \underline{Passo 5:} É verdade que existe $v \in H$ tal que $f(u) = B(u,v)$, para todo $u \in H$.

    Como $f$ é um funcional linear limitado em $H$, temos, pelo Teorema da representação de Riesz (ver Teorema \ref{thm:representacao-riesz}), que existe $z \in H$, tal que $f(u) = \left\langle u, z \right\rangle$, para todo $u \in H$.
    Como $A$ é uma bijeção, existe um único $v \in H$ tal que $z = Av$. Dessa forma, podemos escrever
    \begin{equation} \label{eq:annn}
        f(u) = \left\langle u,z \right\rangle = \left\langle u, Av \right\rangle = B(u,v),
    \end{equation}
    para todo $u \in H$.

    \underline{Passo 6:} Vamos mostrar que $v$ em (\ref{eq:annn}) é único.

    Suponha que existe $\tilde v \in H$ tal que $f(u) = B(u, \tilde v)$, para todo $u \in H$.
    Dessa forma, as igualdades abaixo são válidas:
    \[
        B(u,v) = f(u) = B(u,\tilde v),
    \]
    para todo $u \in H$.
    Por isso, pela bilinearidade de B, segue que $B(u, v - \tilde v) = 0$m para todo $u \in H$.
    Dito isso, pelo item 2, chegamos a
    \[
        \beta \Vert v - \tilde v \Vert^2 \leqslant | B(v - \tilde v, v - \tilde v) | = 0.
    \]
    Sendo assim, $v = \tilde v$.
    Portanto, $v$ é único.
\end{prf}

A seguir, mostraremos a existência de soluções fracas para um sistema de equações diferenciais parciais, através do Teorema de Lax-Milgram.

\noindent\textbf{Aplicação do Teorema de Lax-Milgram}\textbf{.} Considere o problema de Dirichlet
\begin{equation} \label{eq:problema-de-dirichlet}
    \left\{
    \begin{aligned}
        -\Delta u + u = f, &\text{ em } \Omega;\\
        u = 0, &\text{ sobre } \partial\Omega,
    \end{aligned}
    \right.
\end{equation}
onde $\Omega \subseteq \bR^n$ é um aberto limitado e $f \in \cL^2(\Omega)$.
Na motivação (ver Seção \ref{sec:motivacao}), vimos que $u$ é uma solução fraca para o problema de Dirichlet se satisfaz
\[
    \int_\Omega Du \cdot D\phi \,dx + \int_\Omega u\phi \,dx = \int_\Omega f\phi \,dx,
\]
para todo $\phi \in \cC^{\infty}_c(\Omega)$.
Porém, pela densidade das funções teste em $H^{1}_0(\Omega)$ (pois, $H_0^1(\Omega)$ é o fecho de $\cC^{\infty}_c(\Omega)$ em $H^1(\Omega)$), podemos dizer que $u$ é uma solução fraca do problema de Dirichlet, se esta satisfaz a igualdade
\[
    \int_\Omega Du \cdot Dv \,dx + \int_\Omega uv \,dx = \int_\Omega fv \,dx,
\]
para todo $v \in H_0^1(\Omega)$\footnote{Vale ressaltar que $H_0^1(\Omega) = \cW^{1,2}_0(\Omega)$ é um espaço com produto interno
\[
    \left\langle u, v \right\rangle _{H^1_0(\Omega)} := \sum_{|\alpha| \leqslant 1}\int_\Omega D^\alpha u D^\alpha v \,dx.
\]
Mais precisamente, pelo Teorema \ref{thm:sobolev-completo}, $H^1_0(\Omega)$ é um espaço de Hilbert.
}.

Defina a forma $B : H^1_0(\Omega) \times H^1_0(\Omega) \to \bR$ por
\[
    B(u,v)=\int_\Omega Du \cdot Dv \,dx + \int_{\Omega} uv \,dx,
\]
para todo $u, v \in H^1_0(\Omega)$,
e o funcional $\varphi : H^1_0(\Omega) \to \bR$ por
\[
    \varphi(v) = \int_\Omega fv \,dx,
\]
para todo $v \in H^1_0(\Omega)$.
Note que $B$ é uma forma bilinear limitada e coerciva. Com efeito, a bilinearidade de $B$ segue do fato do gradiente fraco $D$ e a integral serem operadores lineares. Além disso, é verdade que
\[
    B(u,u) = \int_\Omega Du \cdot Du \,dx + \int_\Omega u^2 \,dx
    = \int_\Omega \Vert Du \Vert^2 \,dx + \int_\Omega |u|^2 \,dx 
    = \Vert u \Vert_{H^1_0(\Omega)}^2.
\]
Logo, $B$ é coercivo e, utilizando a Desigualdade de Hölder (ver Teorema \ref{thm:pre-desigualdade-de-holder}), segue que
\[
    \begin{aligned}
        |B(u,v)| &\leqslant \int_\Omega |Du \cdot Dv | \,dx + \int_\Omega |u v | \,dx\\ &\leqslant \int_\Omega \Vert Du \Vert \Vert Dv \Vert \,dx + \int_\Omega |u| |v| \,dx \leqslant \Vert Du \Vert_{\cL^2(\Omega)}\Vert Dv \Vert_{\cL^2(\Omega)} + \Vert u \Vert_{\cL^2(\Omega)}\Vert v \Vert_{\cL^2(\Omega)},
    \end{aligned}
\]
para todo $u, v \in H^1_0(\Omega)$
Porém, sabemos que $\Vert u \Vert_{\cL^2(\Omega)}, \Vert Du \Vert_{\cL^2(\Omega)} \leqslant \Vert u \Vert_{H^1_0(\Omega)}$.
Dito isso, obtemos
\[
    |B(u,v)| \leqslant c \Vert u \Vert_{H^1_0(\Omega)} \Vert v \Vert_{H^1_0(\Omega)},
\]
para todo $u, v \in H^1_0(\Omega)$. Logo, $B$ é limitado.

Por fim, temos que $\varphi$ é um funcional linear limitado.
De fato, a linearidade de $\varphi$ segue da distributividade do produto e da linearidade da integral.
Por outro lado, pela Desigualdade de Hölder (ver Teorema \ref{thm:pre-desigualdade-de-holder}), podemos escrever
\[
    |\varphi(v)| \leqslant \int_\Omega |f| |v| \,dx \leqslant \Vert f \Vert_{\cL^2(\Omega)} \Vert v \Vert_{\cL^2(\Omega)} \leqslant \Vert f \Vert_{\cL^2(\Omega)} \Vert v \Vert_{H^1_0(\Omega)},
\]
para todo $v \in H^1_0(\Omega)$, onde $\Vert f \Vert_{\cL^2(\Omega)} < \infty$, pois, por hípotese, $f \in \cL^2(\Omega)$.

Dessa forma, pelo Teorema de Lax-Milgram (ver Teorema \ref{thm:lax-milgram}), existe um único $u \in H^1_0(\Omega)$ tal que
\[
    B(u,v) = \varphi(v),
\]
para todo $v \in H^1_0(\Omega)$.
Portanto, $u$ é a única solução fraca para o problema de Dirichlet.

\nocite{*}
\printbibliography

\end{document}
